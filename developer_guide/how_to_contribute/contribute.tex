% =============================================================
\section{How to contribute to Meso-NH sources}
\label{sec:contribute}
% =============================================================
Meso-NH is build upon many contribution since 30 years. As the code is growing and shared with increasing models and contributors, a number of coding norms must be followed. At each pack release, a new tag is attached to its name (such as MNH-V5-5-0, MNH-V5-6-1 etc). The integration of contributions to a pack depends on its type described in a first part. Coding norms and guidance are described in a second part.
\begin{importantblock}
Your contribution must be accessible in the MNH-ladev repository for integration.
See section \ref{sec:dev_sec} for more info
\end{importantblock}
\subsection{Integration workflow and guidelines}
Meso-NH packs VX-X-X can be divided into two categories :
\begin{itemize}
    \item developement/major release-pack : the first and second X of VX-X (e.g. 5.4, 5.5, 5.6 etc)
    \item bugfix release-pack : the third X of VX-X-X (e.g. 5.4.1, 5.4.2, 5.4.3...)
\end{itemize}
A major release contains several scientific and technical changes and numerous bugfixes/commits. The time frequency of major release is about one over 12 to 18 months depending on the amount of contributors. \\
A bugfix release contains a sufficient number of minor commits/bugfixes to justify a new release, or a hot bugfix that may impact a large number of users. The time frequency of bugfix can vary to a few days up to 12 months depending on the stability of the current pack release (usually a few months). \\

At the integrator's demand\footnote{Contribution request is sent on mesonh@meteo.fr}, contributors are invited 1 to 3 months in advance to prepare their git branches on MNH-ladev repo. Depending on the number of potential contribution and their content (only minors or major), the integrator may asks to only fetch bugfixes or main developpement contributions. \\

\subsubsection{Bugfix release-pack}
In the case of a bugfix release-pack, please follow these guidelines :
\begin{itemize}
    \item do not forget to base your work on the last version of the master's branch : use git pull
    \item only commits with bugfixes are asked. Cherry-picking is possible, please prepare a list of commits hash to share.
\end{itemize}
Please note that contributions from entire branches with major developments would be declined and postponed to the next call of contributions for a major release. \\

\subsubsection{Major release-pack}

In the case of a major release-pack, please follow these guidelines :
\begin{itemize}
 \item do not forget to merge the master's branch into your work before sharing your branch : use git pull

 \item if your development is a new feature, you will be asked to share at least one new test case related to your work with python3 plots showing the interests variables related to your new work. This is important to us to track new bugs in case of non-wanted modifications of a part of your work. 
 \item if your development is a major modification of a current part of the code, you will be asked to share at least one new test case with python3 plots showing the interests variables related to your new work; and to have tested your branch on 2 other cases that is impacted by your development. Test cases used for integration is listed in section \ref{sec:run_ktest_examples}. 
 \item you will be asked to provide a contribution to the scientific and user's guides 
 \item your work is published in a peer-reviewd paper (recommended)    
\end{itemize}

\subsection{Coding guidelines}
\subsubsection{General}
These guidelines apply mostly to every FORTRAN sources in src/ MNH, SURFEX, PHYEX, LIB.

\paragraph{File structure}
\begin{itemize}
    \item Check if the function you code is already coded
    \item Avoid CONTAINS routines, create a new fortran routine file
    \item For new file, keep the common structure with an updated statements for the LICENCE and documentation
\end{itemize}

\paragraph{Code ergonomy}
\begin{itemize}
    \item Maximum 132 characters per line, use \&
    \item Code in CAPITAL characters
    \item Comments in lower-case characters
    \item No blank line, use !
    \item Remove debugging PRINTs and WRITE before committing
    \item Comment your code !
\end{itemize}

\paragraph{Clean code}
\begin{itemize}
    \item Remove debugging PRINTs and WRITE before committing
    \item Remove unused local variables
    \item Remove unused dummy variables
    \item Remove unused module variables (USE MODD\_ ...)
    \item Select the variables used : USE MODD\_TOTO, ONLY : MYVAR
\end{itemize}

\paragraph{Variables}
\begin{itemize}
    \item Variables names must follow the DOCTOR norm (see hereafter)
    \item Variables names must be consistent through subroutines (a variable must be easily found with grep)
    \item All variables must be declared, use IMPLICIT NONE
    \item Allocatables must be deallocated
    \item Pointer must be initialized by NULL()
\end{itemize}

\begin{center}
    \begin{tabular}{|c|c|c|c|c|c|}
    \hline
Type / Status & INTEGER    & REAL      & LOGICAL     & CHARACTER      & TYPE    \\ \hline
Global        & N          & X          & L & C              & T \\
        &          &        &  (not LP)  &              &  (not TP,TS,TZ) \\ \hline
Dummy argument& K          & P & O           & H              & TP               \\
        &          &        &  (not PP)  &              &   \\

\hline
Local         & I  & Z  & G (not GS)  & Y  & TZ               \\
        &          &  (not IS)      &  (not ZS)  &     (not YS, YP)         &   \\

\hline
Loop control  & J & -          & -           & -              & -                \\
  & (not JP) &           &            &               &                 \\

\hline
\end{tabular}
\end{center}
\paragraph{Reproductibility}
\begin{itemize}
    \item Use the parallelizaded version of basic functions (functions ended by \_ll such as MAX\_ll or SUM3D\_ll)
    \item Avoid anticipated exit of a loop with EXIT, CYCLE, RETURN statements
\end{itemize}

\subsubsection{Extra rules for PHYEX}
The externalized atmospheric physics PHYEX has extra rules in order to comply with all the models using PHYEX (AROME, HARMONIE-AROME, etc). These rules are more strict in order to transform automatically the code for GPU applications. \\
The general idea behind these rules is that all the physics can be run with arrays written in one physical dimension (the vertical axis). The fortran raw code is written in 2D or 3D in a way that automatic functions (e.g. written in python) can read the fortran code and transform it to another fortran code that can be run on any type of GPUs.

The previous general rules applies to PHYEX. The following extra rules apply on PHYEX/ :

\paragraph{Variables}
\begin{itemize}
    \item Do not use allocatables
    \item Dimensions of dummy argument arrays must be explicit : no (:,:), use the structure D\%
    \item No variables from modules can be imported except variables declared with the PARAMETER attribute. Put the variable in a type received by the subroutine interface
    \item Use loop index JIJ for computation on horizontal dimensions
    \item Use loop index JL on KSIZE microphysics computation point
    \item Horizontal dimensions arrays are packed into one dimension : instead of A(D\%NIT, D\%NJT, D\%NKT), use A(D\%NIJT, D\%NKT) where D\%NIT, D\%NJT, D\%NKT are physical dimensions in x, y, z directions and D\%NIJT = D\%NIT*D\%NJT
\end{itemize}

\paragraph{Subroutines}
\begin{itemize}
    \item Do not use functions returning arrays, use subroutines
    \item avoid CONTAINS subroutines, if really needed, the local arrays of the subroutines must have different names than the hosting subroutine or than other contained subroutines
\end{itemize}

\paragraph{Statements}
\begin{itemize}
    \item all calculation on arrays must show explicit dimensions. Instead of A = B + C, write : A(:,:) = B(:,:) + C(:,:) even for initialization
    \item Do not use nested WHERE, convert it to DO...IF...
    \item \textcolor{red}{temporary} Do not use ANY, COUNT functions on arrays of horizontal dimensions
     \item \textcolor{red}{temporary} no (:) on TYPE\%VAR
     \item compilation keys must be avoided. If really needed, the statements betwen ifdef and else must not split a statement
\end{itemize}