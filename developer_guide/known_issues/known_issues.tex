% =============================================================
\section{Known issues}
\label{sec:known_issues}
% =============================================================

% ~~~~~~~~~~~~~~~~~~~~~~~~~~~~~~~~~~~~~~~~~~~~~~~~~~~~~~~~~~~~~
\subsection{Segmentation violation : problem of stack size limit}
% ~~~~~~~~~~~~~~~~~~~~~~~~~~~~~~~~~~~~~~~~~~~~~~~~~~~~~~~~~~~~~

When running the examples coming with the Meso-NH package, if you obtain a ``segmentation violation'' error it is probably a problem with the stack size limit on your computer ... Check this limit with the command :
\begin{bashcode}
 ulimit -s
\end{bashcode}

The limit is given in Kbytes (KB) and is often 8192 KB, this mean only 8 MB for array in stack memory. It's a very low value. We recommend you to defined it to ``unlimited'' in your environment (``.bashrc'' or ``.profile'') or in your script used to launch your executables like this :
\begin{bashcode}
ulimit -s unlimited
\end{bashcode}

% ~~~~~~~~~~~~~~~~~~~~~~~~~~~~~~~~~~~~~~~~~~~~~~~~~~~~~~~~~~~~~
\subsection{Bug with ``ifort 10.0.xxx'' compilers}
% ~~~~~~~~~~~~~~~~~~~~~~~~~~~~~~~~~~~~~~~~~~~~~~~~~~~~~~~~~~~~~

Some routines do not compile with the version ``10.0.XXX'' of ``ifort'' (at least until 10.0.023), solution is to upgrade to ifort ``10.1.015''. \\

The ifort ``10.1.008'' has also some bug in the generation of optimized vectorized code resulting in unpredictable flaoting point error. Solution is to compile the routine with problem with ``-O1'' option or upgrade to at least ``10.1.015'' version.