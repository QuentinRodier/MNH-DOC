\chapter{The Meso-NH files}
  
\section{The F90 namelists}

All the information required  to  perform a given step of a numerical
experiment are provided by
different files including a NAMELIST set. 
Thus, the Meso-NH user can change the value of the parameters without
any compilation (and therefore save computer time). These files provide the way 
for the Meso-NH user to interact with the numerical code and finally, they
 contain the identification
cards of the different steps of the numerical experiment. 


These NAMELISTs are Fortran 90
NAMELISTs, which obey to strict writing rules (Metcalf and Reid 1993): no
comment is allowed inside the namelists, no empty namelist can be written (it
gives a Fortran execution error). 

The information are written in the following form:

\begin{verbatim}
 &NAM_LUNITn  CINIFILE = "     FMFILE.1               " /
 &NAM_CONFn  LUSERV = T, LUSERC = F, LUSERR = F, LUSERI = F, LUSERS = F,
 LUSERG = F, LUSERH = F, NSV = 0 /
\end{verbatim}


 {\tt \&NAM\_LUNITn}
is the name of the first namelist of this file, the $/$ character indicates the 
end of the list of information. These namelist  parameters are defined by VAR = VALUE and
these prescriptions are separated from each others by a comma and a blank 
character. Note that you can use more than one line to give a namelist, but in
this case it is imperative to let a blank character at the end of each line.
 
The Meso-NH user does not need to prescribe all the parameters of a namelist,
the missing information are taken equal to the default values
written in the Fortran code. For example, the second namelist in the previous 
example can be written as:
\begin{verbatim}
 &NAM_CONFn  LUSERV = T /
\end{verbatim}
because the other variables of {\tt \&NAM\_CONFn} are set to the default values.
\\

In order to clearly separate what can be modified for a given step of a
numerical experiment, we affect a different namelist file name for each step.

\begin{itemize}
\item
To PREpare a Meso-NH file containing PhysioGraphical Data
\subitem  ${ \large \Longrightarrow } $ file PRE\_PGD1.nam 
\item
To PREpare a Meso-NH file with PhysioGraphical Data in conformity
\subitem  ${ \large \Longrightarrow } $ file PRE\_NEST\_PGD1.nam 
\item
To PREpare a ZOOMed Meso-NH file with PhysioGraphical Data
\subitem  ${ \large \Longrightarrow } $ file PRE\_ZOOM1.nam 
\item
To PREpare an initial Meso-NH file for an IDEAlized case study 
\subitem  ${ \large \Longrightarrow } $ file PRE\_IDEA1.nam 
\item
To PREpare an initial Meso-NH file for an REAL case study 
\subitem  ${ \large \Longrightarrow } $ file PRE\_REAL1.nam 
\item
To SPAWN a Meso-NH file into another one with better horizontal resolution
\subitem  ${ \large \Longrightarrow } $ file SPAWN1.nam 
\item
To EXecute a simulation SEGment for the $n ^{th} $ model
\subitem  ${ \large \Longrightarrow } $ file EXSEGn.nam
\item
To compute DIAGnostics after a simulation
\subitem  ${ \large \Longrightarrow } $ file DIAG1.nam
\end{itemize}


 Because the grid-nesting technique requires the simultanous presence
of multiple models in the computer memory, free-parameters 
must be fixed for every model. This is performed by associating a namelist 
file per model, this explains why the namelist are suffixed by a number 1 or n 
just above.


The different parameters present in these files are all given in this book 
 and more
details on the description of a given parameter are present in the code itself.


\section{The Meso-NH files}

% A Meso-NH FM-file is a set of 2 files, (which can be sticked together via the cpio UNIX
% command):
% \begin{itemize}
% \item
%  a \underline{descriptive part} ( {\tt .des}) containing 
% information about the file generation and its description (in ASCII
% characters)
% \item
% a \underline{binary part} ( {\tt .lfi}) where the fields are stored. The structure of this
%  file is a direct access  type file, written and read by routines developped in
% M\'et\'eo-France (Fischer, 1994) based on {\sc lfi} routines (Clochard, 1989), 
% which can be used on a lot of different computers.
% \end{itemize}

Meso-NH writes files by group of 2:
\begin{itemize}
\item
 a \underline{descriptive part} ({\tt .des}) containing 
information about the file generation and its description (in ASCII
characters)
\item
a \underline{binary part} where the fields are stored. Two different fileformat are available:
  \begin{itemize}
  \item NetCDF (files with a {\tt .nc} suffix). This is the recommanded fileformat and is the default since Meso-NH 5.4.0.
        NetCDF is a self-describing, machine-independent fileformat.
        In Meso-NH, these files are mostly compliant with the CF conventions (version 1.7) and the COMODO extensions (version 1.4).
  \item LFI  (files with a {\tt .lfi} suffix). This is the historic fileformat for Meso-NH. Its structure
        is a direct access type file, written and read by routines developped by
        M\'et\'eo-France (Fischer, 1994) based on {\sc LFI} routines (Clochard, 1989), 
        which can be used on a lot of different computers.
  \end{itemize}
\end{itemize}

These binary files are used to store all the data necessary to run any step 
of a numerical experiment. Three different filetypes are taken into account in the 
Meso-NH project:
\begin{itemize}
\item
the \underline{synchronous backup file} contains all the values of all the fields allowing a
restart of the model and of some diagnostic fields desired by the Meso-NH user.
All these informations are obtained at the same instant during the simulation,
hence their synchronous name.
\item
the \underline{diachronic file} contains time series of information requested by the Meso-NH 
user. They are obtained during more than one time step of the model.
If in LFI fileformat, it is the type in which your file must be if you want to plot it with
the graphics software \tt diaprog \rm (you can convert a synchronous file
into a diachronic one with \tt conv2dia\rm ).
\item
the \underline{physiographic file} contains external information like 
orography, vegetation classes, chemical emissions, data sets, etc.
\end{itemize}

A fourth filetype is possible: output files. They contain only user-selected fields.
They are useful, for example, if you need to output frequently some data without spending too much diskspace
and too much time writing them.


\subsection{The synchronous backup file} \label{ss:synchro}

This type of file contains only information corresponding to the same instant
of the simulation, it remains open during a whole time step of the simulation,
and the writing orders can be given from any routine of the model.

\subsubsection{The descriptive part}
This part is the list of all the namelists of the EXSEG\$n.nam file.
Thus, a complete description of this part is given
with the EXSEG\$n.nam description in chapter \ref{ch:model}.

If the file has been generated during a segment of the model integration, the 
{\tt .des} part contains the different
namelists fixing the free-parameters for the dynamics and the physics 
of the Meso-NH model. This allows the user to know a large
part of the history of this file. For the namelists or variables ommited 
in the EXSEG\$n.nam file, the values are set to the default ones
 (see the tables in ch.\ref{ch:model}).

If the file is the result of the initialization programs 
({\tt PREP\_IDEAL\_CASE}, {\tt PREP\_REAL\_CASE} or {\tt SPAWNING}),
 the values of the namelists variables are the ones of the descriptive part
of the input file of the program if it does exist. Otherwise, the values are
set to the default ones, except for these that can be initialized during the
initialization program (e.g. {\tt CINIFILE} or {\tt LUSERV}).

Note that a physiographic file does not have a descriptive part.

\subsubsection{The binary part}

This type of file can be in netCDF or LFI fileformat.

% All the writings and readings of this type of files are done through LFI 
% routines. A general subroutine to read and write a Meso-NH file is given in the
% Meso-NH library, it provides a file including the fields of the previous record
% list. This Fortran library provides a way to tackle direct access binary files
% and thus a very quick access to the data stored in this file in any order.

It should be noted that additional fields can be added to these basic
information, which have been obtained at the same instant. 
 In order to be easily drawn by the Meso-NH graphic package, the
comment field must be filled, according to the following rules:
\begin{itemize}
\item 
the length of the character string is equal to 100
\item 
the type of the supplementary field must be specified:

\begin{tabular} {|| c |c|| }
\hline
\hline
type & comment field \\
\hline
3D scalar & X\_Y\_Z\_varname  (UNIT) \\
\hline
2D scalar & X\_Y\_varname  (UNIT) \\
\hline
3D vector & VX\_xvarname\_VY\_yvarname\_VZ\_zvarname  (UNIT) \\
\hline
2D scalar & VX\_xvarname\_VY\_yvarname\_VZ\_zvarname  (UNIT) \\
          & or  VX\_xvarname\_VY\_yvarname       (UNIT) \\
\hline
1D scalar & Z\_zvarname  (UNIT) \\
\hline
\hline
\end{tabular}
\end{itemize}

\subsection{The diachronic file}
A diachronic file is a file obtained during a segment of simulation or resulting of the
conversion of a synchronous file with {\tt conv2dia} for graphical purposes (available only for LFI fileformat).

The file directly obtained during the simulation has a name ended by {\tt .000},
and contains records such as averaged variables, tendencies, fluxes stored
at different times of the simulation on the whole or some parts of the domain.
Such records are obtained by requesting temporal series (\ref{ss:series}),
budgets (\ref{ss:budget}), aircraft or balloon (\ref{ss:balloon}), profiler
or station (\ref{ss:station}), or LES diagnostics (\ref{s:LESdiag}).

\subsection{The physiographic file}
It is a bidimensional MesoNH file with contains surface data such as orography,
vegetation classes, chemical emissions, etc.

See the documentation: ''Surfex user's guide'' for more
details.

\subsection{The output file}
An output file is obtained at the user request. They are written at once for a given timestep.
They contain only user-selected fields.
They are useful, for example, if you need to output frequently some data without spending too much diskspace
and too much time writing them.

In contrast to the synchronous backup files, output files does not contain enough information to restart a calculation. Furthermore, they are not associated with a descriptive file.


\section{References}
\decrefname
J. Clochard, 1989: Logiciel de Fichiers Index\'es. Direction de la
 Météorologie Nationale. Note de travail ARPEGE n$^\circ$12.

\decrefname
J. Clochard, 1991: Logiciel de Fichiers Index\'es. Direction de la
 Météorologie Nationale.  Technical report.

\decrefname
D. Gazen, 1999: Parallel IO routines. Man page on Meso-NH web site.

\decrefname
C. Fischer, 1994: File structure and content in the Meso-NH model. Meso-NH note.
