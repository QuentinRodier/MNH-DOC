\chapter{{\bf PREP\_REAL\_CASE}}\label{c:PREPREAL}
\section{Presentation}
{\bf PREP\_REAL\_CASE } program performs the change of
orography and vertical grid by interpolating horizontally and vertically for a 
GRIB file or only vertically for a MESO-NH file. The MESO-NH output file  will
 be used either for the beginning of the simulation or for coupling.\\
The main hypothesis is that hydrostatism is verified.
Therefore, if the input file is a MESO-NH file, there is a small 
loss of information.\\

What's going in and out?
\begin{itemize}
\item Input:
\begin{itemize}
\item a file containing the atmospheric 3D and surface 2D variable fields
(hereafter called atmospheric file); it can be
either
\begin{itemize}
\item a GRIB file obtained from {\bf extractecmwf} or {\bf extractarpege}
\item or a MESO-NH file (obtained with {\bf SPAWNING} for example)
\end{itemize}
In the first case, both horizontal and vertical interpolations are carried
 utn by the PREP\_REAL\_CASE. In the second case, only vertical interpolation is 
carried on by PREP\_REAL\_CASE  as the horizontal interpolation is already 
done by the spwaning (see chapter \ref{c:SPAWNING})
\item a physiographic data file (it can also be a complete MESO-NH file).
\item an optional file containing the chemical species (here after called
chemical file); it is used only if the atmospheric file is a GRIB file. 
It can be either
\begin{itemize}
\item a GRIB file obtained from {\bf extractarpege} (e.g. an file from the
Mocage french model)
\item or a MESO-NH file (obtained in a previous simulation for example)
\end{itemize}
\item the file PRE\_REAL1.nam which contains the directives for PREP\_REAL\_CASE
\end{itemize}
\item Output:
\begin{itemize}
\item the MESO-NH FM-file
\end{itemize}
\end{itemize}


\subsection{The physiographic data file}
This is a FM file, but with fewer elements than a MESO-NH file. It contains
the physiographic 2D fields. The geographic and grid data are stored on this 
file.
This file is created by the program PREP\_PGD.
It is possible to use a complete MESO-NH file, since it also contains
the physiographic fields.
If one only wants to modify the vertical grid of
a MESO-NH file, without any change to the orography, one can specify it as both  atmospheric file and physiographic files.

It contains:
\begin{itemize}
\item the definition of the projection, the horizontal domain and the 
horizontal grid
\item the physiographic fields
\end{itemize}

\subsection{The atmospheric file}
Both GRIB and FM files are self explanatory.
The physiographic data stored in it will not be saved in the output
MESONH file.

\subsection{The surface file (optional)}
Both GRIB and FM files are self explanatory. The surface fields can be read in another file than the atmospheric one.
 
\subsection{The chemical file (optional)}
Both GRIB and FM files are self explanatory. If the atmospheric file is a GRIB
file, the chemical species can be read in another file than the atmospheric one.
 
\newpage
\section{The file PRE\_REAL1.nam}
This file contains namelists with the directives to run PREP\_REAL\_CASE.
The namelists contain the names of the files and the definition of
the vertical grid.  The file can also contain a free formatted part after the vertical grid
definition namelist, where  the vertical levels can be prescribed  if this
option is chosen.
\subsection{Namelist NAM\_AERO\_CONF (aerosol initialization)}
\index{NAM\_AERO\_CONF!namelist description}
\begin{center}
\begin{tabular} {|l|l|l|}
\hline
Fortran name & Fortran type & default value\\
\hline
\hline
LORILAM      & logical       & FALSE    \\
LDUST        & logical       & FALSE    \\
LSALT        & logical       & FALSE    \\
LINITPM      & logical       & FALSE    \\
XINIRADIUSI  & real          & 0.05     \\
XINIRADIUSJ  & real          & 0.2      \\
XINISIGI     & real          & 1.8      \\
XINISIGJ     & real          & 2.0      \\
XN0IMIN      & real          & 10.      \\
XN0JMIN      & real          & 1.       \\
CRGUNIT      & character     & "NUMB"   \\
NMODE\_DST   & integer       & 3       \\
XN0MIN       & real          & 1.e3 , 1.e1 , 1.e-2 \\
XINIRADIUS   & real          & 0.044, 0.3215, 1.575 \\
XINISIG      & real          & 2.0, 1.78, 1.85 \\
CRGUNITD     & character     & "NUMB"   \\
NMODE\_SLT   & integer       & 3       \\
XN0MIN\_SLT  & real          & 1.e4 , 1.e2 , 1.e-1 \\
XINIRADIUS\_SLT & real       & 0.14, 1.125,  7.64\\
XINISIG\_SLT    & real       & 1.9, 2., 2. \\
CRGUNITS     & character     & "MASS"   \\
\hline
\end{tabular}
\end{center}

See section \ref{s:namaeropre} page \pageref{s:namaeropre} for details.
			  
			  
\subsection{Namelist NAM\_BLANK}
See section \ref{s:namblank} page \pageref{s:namblank} for details.

\subsection{Namelist NAM\_CH\_CONF ( file names)}
\index{NAM\_CH\_CONF!namelist description}

\begin{center}
\begin{tabular} {|l|l|l|}
\hline
Fortran name & Fortran type & default value\\
\hline
\hline
LUSECHAQ  & logical  & .FALSE.  \\
LUSECHIC  & logical  & .FALSE.   \\
LUSECHEM  & logical  & .FALSE.   \\

\hline
\end{tabular}
\end{center}

\begin{itemize}
\item LUSECHAQ : switch to initialize aqueous phase chemistery.
\index{LUSECHAQ!\innam{NAM\_CH\_CONF}}
\item LUSECHIC : switch to initialize ice  phase chemistery.
\index{LUSECHIC!\innam{NAM\_CH\_CONF}}
\item LUSECHEM : switch to initialize gazeous phase chemistery whithout aqueous phase chemistery
\index{LUSECHEM!\innam{NAM\_CH\_CONF}}

\end{itemize}

\subsection{Namelist NAM\_CONFIO}
See section \ref{s:namconfio} page \pageref{s:namconfio} for details.

\subsection{Namelist NAM\_CONFZ}
See section \ref{s:namconfz} page \pageref{s:namconfz} for details.

\subsection{Namelist NAM\_FILE\_NAMES ( file names)}
\index{NAM\_FILE\_NAMES!namelist description}

\begin{center}
\begin{tabular} {|l|l|l|}
\hline
Fortran name & Fortran type & default value\\
\hline
\hline
HATMFILE      & character (LEN=28) & ' '           \\
HATMFILETYPE  & character (LEN=6)  & 'MESONH'      \\
HPGDFILE      & character (LEN=28) & ' '           \\
HSURFFILE     & character (LEN=28) & ' '           \\
HSURFFILETYPE & character (LEN=6)  & 'MESONH'      \\
HCHEMFILE     & character (LEN=28) & ' '           \\
HCHEMFILETYPE & character (LEN=6)  & 'MESONH'      \\
CINIFILE      & character (LEN=28) & 'INIFILE'     \\
\hline
\end{tabular}
\end{center}

\begin{itemize}
\item HATMFILE : name of the atmospheric file.
\index{HATMFILE!\innam{NAM\_FILE\_NAMES}}
\item HATMFILETYPE : type of  atmospheric file ('GRIBEX', 'MESONH')
\index{HATMFILETYPE!\innam{NAM\_FILE\_NAMES}}
\item HPGDFILE : name of the Physiographic Data File.
\index{HPGDFILE!\innam{NAM\_FILE\_NAMES}}
\item HSURFFILE : optional name of the file containing the surface fields.
\index{HSURFFILE!\innam{NAM\_FILE\_NAMES}}
\item HSURFFILETYPE : type of surface file ('GRIBEX', 'MESONH')
\index{HSURFFILETYPE!\innam{NAM\_FILE\_NAMES}}
\item HCHEMFILE : optional name of the file containing
the chemical species if they are not in the HATMFILE or if the ones of the
HATMFILE are not used (only if HATMFILETYPE is 'GRIBEX'). The grids must
be the same as the ones of the output file (CINIFILE).
\index{HCHEMFILE!\innam{NAM\_FILE\_NAMES}}
\item HCHEMFILETYPE : type of the chemical file ('GRIBEX', 'MESONH', 'NETCDF')
\index{HCHEMFILETYPE!\innam{NAM\_FILE\_NAMES}}
\item CINIFILE : name of the MESO-NH output FM-file, used as initial
or coupling file in a MESO-NH simulation
\index{CINIFILE!\innam{NAM\_FILE\_NAMES}}
\end{itemize}
\newpage
\subsection{Namelist NAM\_HURR\_CONF (hurricane filtering and vortex bogussing)}
\index{NAM\_HURR\_CONF!namelist description}
Two PREP\_REAL\_CASE jobs are to be performed.
In a mono-model configuration, the
first job allows to remove analysed hurricane from the input GRIB fields:
filtered and interpolated fields are written in a MesoNH file.
It is used as input file for the second
PREP\_REAL\_CASE job during which the analytical vortex is added.\\

Each step (hurricane filtering and vortex bogussing) is separately invoked
within the PREP\_REAL\_CASE program:
\subitem
i) The hurricane filtering is applied on four input atmospheric GRIB fields
(HATMFILETYPE='GRIBEX'), when they are in the horizontal grid of the PGD file
and in the vertical grid of the GRIB file. The input atmospheric GRIB fields 
filtered are the two horizontal components of wind, the absolute temperature,
the humidity 
and the surface pressure reduced to ground level. Each field is decomposed 
into three parts: first, the BASic part is computed by the low-pass Barnes
filter; then the hurricane (symmetric) disturbance is computed from the 
remainder disturbance part. The initial fields are then remplaced by their 
ENVironmental part: total field minus hurricane disturbance part.
\subitem
ii) The vortex bogussing consists on a symmetric vortex added to the input
atmospheric MesoNH fields (HATMFILETYPE='MESONH'). 
The tangential wind is computed from an analytical formulation (Holland, 1980):
{\bf Mercator projection must be used} to respect hypotheses of Holland.
Then, the balanced mass field is deduced
from the thermal wind relation. The bogus of the two horizontal components of
wind and the potential temperature is added to the initial (filtered) fields.

For a grid-nesting simulation, the hurricane filtering is first applied for
the outer domain (dad model), with the program PREP\_REAL\_CASE.
The filtered fields are then horizontally interpolated for inner domains
with the program SPAWNING (see section \ref{s:spawn}). 
Then, for each inner domain, a vortex bogussing is added with the program
PREP\_REAL\_CASE.

\begin{center}
\begin{tabular} {|l|l|l|}
\hline
Fortran name & Fortran type & default value\\
\hline
\hline
LFILTERING     & logical  & .FALSE.    \\
CFILTERING     & character (LEN=5)  & 'UVT  '  \\
NK             & integer  & 50  \\
XLAMBDA        & real & 0.2     \\
XLATGUESS      & real & XUNDEF  \\
XLONGUESS      & real & XUNDEF  \\
XBOXWIND       & real & XUNDEF  \\
XRADGUESS      & real & XUNDEF  \\
NPHIL          & integer  & 24  \\
NDIAG\_FILT    & integer  & -1  \\
NLEVELR0       & integer  & 15  \\
\hline
LBOGUSSING     & logical    & .FALSE.   \\
XLATBOG        & real & XUNDEF  \\
XLONBOG        & real & XUNDEF  \\
XVTMAXSURF     & real & XUNDEF  \\
XRADWINDSURF   & real & XUNDEF  \\
XMAX           & real & 16000.  \\
XC             & real & 0.7  \\
XRHO\_Z        & real & -0.3  \\
XRHO\_ZZ       & real & 0.9   \\
XB\_0          & real & 1.65    \\
XBETA\_Z       & real & -0.5   \\
XBETA\_ZZ      & real & 0.35   \\
XANGCONV0      & real & 0.   \\
XANGCONV1000   & real & 0.   \\
XANGCONV2000   & real & 0.   \\
CDADATMFILE    & character (LEN=28) & ' '           \\
CDADBOGFILE    & character (LEN=28) & ' '           \\
\hline
\end{tabular}
\end{center}

\begin{itemize}
\item LFILTERING : Flag to filter the fields (U,V,T,Q,reduced Ps) of the 
                  atmospheric file (logical)
\index{LFILTERING!\innam{NAM\_HURR\_CONF}}
\item CFILTERING : to choose the fields to be filtered (U,V,T,reduced Ps).
\index{CFILTERING!\innam{NAM\_HURR\_CONF}}
\begin{itemize}
\item 'UVT  ' : U,V,T are filtered (default),
\item 'UVTQ  ' : U,V,T,Q are filtered (default),
\item 'UVTP ' : U,V,T and reduced PS are filtered,
\item 'UVTPQ ' : U,V,T,Q and reduced PS are filtered,
\end{itemize}
\item NK : number of points of the half-width of the window in which the 
Barnes filter is applied to compute low-pass component of a given field
\index{NK!\innam{NAM\_HURR\_CONF}}
\item XLAMBDA : a coefficient in the exponential weighting function of the 
Barnes filter
\index{XLAMBDA!\innam{NAM\_HURR\_CONF}}
\item XLATGUESS : latitude of the guessed position of the cyclone center
\index{XLATGUESS!\innam{NAM\_HURR\_CONF}}
\item XLONGUESS : longitude of the guessed position of the cyclone center
\index{XLONGUESS!\innam{NAM\_HURR\_CONF}}
\item XBOXWIND : half-width of the box inside which the dynamical center
is searched from the guessed position (km)
\index{XBOXWIND!\innam{NAM\_HURR\_CONF}}
\item XRADGUESS : guess  of the radius of the domain in which the cyclone 
will be filtered (km)
\index{XRADGUESS!\innam{NAM\_HURR\_CONF}}
\item NPHIL : number of azimuthal directions used for the cylindrical coordinates
\index{NPHIL!\innam{NAM\_HURR\_CONF}}
\item NDIAG\_FILT : allows storage of several components calculated from total fields. {\bf Be careful, the components are on the GRIB vertical grid: in diaprog,
plot them only on \_K\_ levels}. 
Then to visualize all the GRIB vertical levels, the 
number of MesoNH vertical levels must be equal or greater than the number of
levels in the input GRIB file.
\index{NDIAG\_FILT!\innam{NAM\_HURR\_CONF}}
\begin{enumerate}
\item[] 0 : total (unfiltered) fields: UT15, VT15 for wind components; TEMPTOT,
PRESTOT for absolute temperature and surface pressure, \\
        \hspace*{0.5cm} environmental (filtered) fields (total field minus 
 hurricane disturbance component): UT16, VT16, TEMPENV, PRESENV,
\item[] 0,1 : basic fields (low-pass component isolated
 by the Barnes filter): UT17, VT17, TEMPBAS, PRESBAS, 
\item[] 0,1,2 : total disturbance tangential wind component (XVTDIS).
\end{enumerate}
\item NLEVELR0 : level used to compute R0
\index{NLEVELR0!\innam{NAM\_HURR\_CONF}}
\item LBOGUSSING : Flag to switch on the addition of the bogus vortex (logical)
\index{LBOGUSSING!\innam{NAM\_HURR\_CONF}}
\item XLATBOG : latitude of the bogussed position of the analytical
 cyclone center
\index{XLATBOG!\innam{NAM\_HURR\_CONF}}
\item XLONBOG : longitude of the bogussed position of the analytical
 cyclone center
\index{XLONBOG!\innam{NAM\_HURR\_CONF}}
\item XVTMAXSURF : maximum tangential wind near the surface or about 500 m 
 altitude (m/s)
\index{XVTMAXSURF!\innam{NAM\_HURR\_CONF}}
\item XRADWINDSURF : radius of maximum wind near the surface or about 500 m 
 altitude (km)
\index{XRADWINDSURF!\innam{NAM\_HURR\_CONF}}
\item XMAX : altitude where the tangential wind vanishes (m)
\index{XMAX!\innam{NAM\_HURR\_CONF}}
\item The following variables are parameters describing tangential wind in 
Holland's law (see formulation in routine {\tt holland\_vt.f90}).
\subitem {XC :} standard coefficient for maximum tangential wind,
\index{XC!\innam{NAM\_HURR\_CONF}}
\subitem{XRHO\_Z, XRHO\_Z :} standard coefficients for radius of maximum wind,
\index{XRHO\_Z, XRHO\_Z!\innam{NAM\_HURR\_CONF}}
\subitem{XB\_0, XBETA\_Z, XBETA\_ZZ :} standard coefficients for B parameter.
\index{XB\_0,XBETA\_Z,XBETA\_ZZ!\innam{NAM\_HURR\_CONF}}
\item XANGCONV0, XANGCONV1000, XANGCONV2000 : convergence angle of wind near the
surface, at 1000m and 2000m altitude.  
\index{XANGCONV0!\innam{NAM\_HURR\_CONF}}
\index{XANGCONV1000!\innam{NAM\_HURR\_CONF}}
\index{XANGCONV2000!\innam{NAM\_HURR\_CONF}}

\item CDADATMFILE : if LBOGUSSING=.TRUE. : name of the dad of HATMFILE
\index{CDADATMFILE!\innam{NAM\_HURR\_CONF}}
\item CDADBOGFILE : if LBOGUSSING=.TRUE. : name of the dad of CINIFILE.
The program will check that CDADATMFILE and CDADBOGFILE have the same 
characteristics,
before replacing the dad name of CINIFILE by CDADBOGFILE instead of CDADATMFILE.
CDADBOGFILE must exist before running the prep\_real\_case job. 
\index{CDADBOGFILE!\innam{NAM\_HURR\_CONF}}
\end{itemize}


\subsection{Namelist NAM\_IBM\_LSF}
See section \ref{s:namibmlsf} page \pageref{s:namibmlsf} for details.

\subsection{Namelist ibm\_idea.nam}
See section \ref{s:namibmidea} page \pageref{s:namibmidea} for details.

\subsection{Namelist NAM\_REAL\_CONF  (configuration variables)}
\index{NAM\_REAL\_CONF!namelist description}

\begin{center}
\begin{tabular} {|l|l|l|}
\hline
Fortran name & Fortran type & default value\\
\hline
\hline
CEQNSYS       & character (LEN=3)  &if HATMFILETYPE = 'GRIBEX': \\
 & & \hspace*{0.5cm}'DUR' \\
              &                    & if HATMFILETYPE = 'MESONH': \\
 & & \hspace*{0.5cm} CEQNSYS value used in input MESONH file\\
CPRESOPT      & character (LEN=5)  & 'CRESI' \\
NVERB         & integer            & 1      \\
LSHIFT        & logical            & if HATMFILETYPE = 'GRIBEX':  \\
 & & \hspace*{0.5cm}.TRUE. \\
              &                    & if HATMFILETYPE = 'MESONH': \\
 & & \hspace*{0.5cm}.FALSE. \\
LDUMMY\_REAL  & logical            & .FALSE.                     \\
LRES          & logical            & .FALSE.                     \\
XRES          & real               & 1.E-07                      \\
LCOUPLING     & logical            & .FALSE.                     \\
NHALO      & integer        & 1      \\
JPHEXT     & integer        & 1      \\
\hline
\end{tabular}
\end{center}

\begin{itemize}
\item CEQNSYS : EQuatioN SYStem
\index{CEQNSYS!\innam{NAM\_REAL\_CONF}}
\begin{itemize}
\item 'LHE': Lipps-HEmler 1982
\item 'MAE': Modified Anelastic Equations
\item 'DUR': following DURran 1990 derivations
\end{itemize}

\item CPRESOPT : option for pressure solver 
('RICHA', 'CGRAD', 'CRESI', 'ZRESI').
\index{CPRESOPT!\innam{NAM\_REAL\_CONF}}

\item NVERB : verbosity level (error diagnostics are computed if
NVERB$>$4)
\index{NVERB!\innam{NAM\_REAL\_CONF}}

\item LSHIFT : flag to shift altitudes in boundary layer
\index{LSHIFT!\innam{NAM\_REAL\_CONF}}

\item LDUMMY\_REAL : flag to read dummy fields stored in the GRIB file (if you have previously run {\bf extractarpege}
asking for additional fields by modifying FULLPOS namelist or {\bf extractecmwf} by modifying MARS requests).You have to fill also a free formatted part as described in the exemple of AROME field in section \ref{i:realdummy} p.\pageref{i:realdummy}).
\index{LDUMMY\_REAL!\innam{NAM\_REAL\_CONF}}

\item LRES : flag to change the residual divergence limit
\index{LRES\innam{NAM\_REAL\_CONF}}

\item XRES : Value of the residual divergence limit
\index{XRES\innam{NAM\_REAL\_CONF}}

\item \textbf{LCOUPLING : Flag to handle the output file as coupling file instead of initial file : no surface field will be computed. (time saving and smaller file) }
\index{LCOUPLING\innam{NAM\_REAL\_CONF}}

\item
NHALO: Size of the halo for parallel distribution.
This variable is related to computer performance but has no
impact on simulation results.\\
\index{NHALO!\innam{NAM\_REAL\_CONF}}

\item
JPHEXT:  Horizontal External points number\\
JPHEXT must be equal to 3 for cyclic cases with WENO5.
\index{JPHEXT!\innam{NAM\_REAL\_CONF}}

\end{itemize}


\subsection{Namelist NAM\_VER\_GRID (vertical grid definition)}
\index{NAM\_VER\_GRID!namelist description}

The use of the THINSHELL approximation is specified in this namelist.
There are five ways to compute the vertical grid (the three first ones are
 similar to PREP\_IDEAL\_CASE):
\begin{enumerate}
\item
constant grid mesh: only the number of levels NKMAX and the grid mesh sizes 
ZDZGRD and ZDZTOP are used. These must be equal. The type of grid 
YZGRID\_TYPE is set to 'FUNCTN'.
\item
two layers are defined, with constant stretching in each of these, the grid
mesh sizes being given near the ground and at top of the model. It is possible
that the top grid size is never reached, if the number of points is not enough
for the prescribed stretchings. The type of grid 
YZGRID\_TYPE is also set to 'FUNCTN'.
\item
the levels are given by the user. The type of grid YZGRID\_TYPE is set to
'MANUAL' in the namelist, and only the number of levels NKMAX is also used in it. 
\item
The levels in the output MESONH file are the same as in the input MESONH file 
(only if the atmospheric input file is a MESONH file).
The type of grid YZGRID\_TYPE is set to 'SAMEGR' (for "same grid") and
NKMAX is not specified.
\item 
The physical levels of the output MESONH file are the same as 
the lower NKMAX physical levels in the input MESONH file 
(only if the atmospheric input file is a MESONH file).
The type of grid YZGRID\_TYPE is set to 'SAMEGR' (for "same grid") and NKMAX
is specified.
\end{enumerate}

The variables of this namelist are:

\begin{center}
\begin{tabular} {|l|l|l|}
\hline
Fortran name & Fortran type & default value\\
\hline
\hline
LTHINSHELL     & logical    & .FALSE.    \\
NKMAX          & integer    & 60     if HATMFILETYPE='GRIBEX'  \\
               & integer    & same as in input file if HATMFILETYPE='MESONH'  \\
YZGRID\_TYPE   & character (len=6) & 'FUNCTN' if HATMFILETYPE='GRIBEX'  \\
               &                   & 'SAMEGR' if HATMFILETYPE='MESONH'  \\
ZDZGRD         & real & 300 m  \\
ZDZTOP         & real & 300 m     \\
ZZMAX\_STRGRD  & real & 0 m     \\
ZSTRGRD        & real & 0  \%        \\
ZSTRTOP        & real & 0 \% \\
LSLEVE         & logical & FALSE   \\
XLEN1          & real & 7500.   \\
XLEN2          & real & 2500.   \\
\hline
\end{tabular}
\end{center}

\begin{itemize}
\item LTHINSHELL : Flag for the thinshell approximation (logical)
\index{LTHINSHELL!\innam{NAM\_VER\_GRID}}
\item NKMAX : number of points in z-direction of the required 
              physical domain. The total size of the array written in initial
file will be $NKMAX +2 JPVEXT$ ($JPVEXT$ is fixed to 1 for the present version
 of Meso-NH) 
\index{NKMAX!\innam{NAM\_VER\_GRID}}
\item   YZGRID\_TYPE : type of vertical grid definition:
\index{YZGRID\_TYPE!\innam{NAM\_VER\_GRID}}
\begin{itemize}
\item 'FUNCTN' : the levels are calculated by the program, according to the
namelist variables.
\item 'MANUAL' : the levels are written in the free-formatted part after the
namelist.
\item 'SAMEGR' : the levels are the same as those in the input file. Only
available when atmospheric input file is a MESONH file.
\end{itemize}
 
\item   ZDZGRD :   mesh length in z-direction near the ground (m)
\index{ZDZGRD!\innam{NAM\_VER\_GRID}}

\item   ZDZTOP :   mesh length in z-direction near the top of the model (m)
\index{ZDZTOP!\innam{NAM\_VER\_GRID}}

\item   ZZMAX\_STRGRD :   Altitude separating the two constant stretching layers  (m)
\index{ZZMAX\_STRGRD!\innam{NAM\_VER\_GRID}}

\item   ZSTRGRD :   Constant imposed stretching (in \%) in the lower layer
(below ZZMAX\_STRGRD)
\index{ZSTRGRD!\innam{NAM\_VER\_GRID}}

\item   ZSTRTOP :   Constant imposed stretching (in \%) in the upper layer
(above ZZMAX\_STRGRD)
\index{ZSTRTOP!\innam{NAM\_VER\_GRID}}

\item  LSLEVE \index{LSLEVE!\innam{NAM\_VER\_GRID}} : flag for Sleve 
vertical coordinate.
\item  XLEN1 \index{XLEN1!\innam{NAM\_VER\_GRID}} : decay scale for
smooth topography (in meters)
\item  XLEN2 \index{XLEN2!\innam{NAM\_VER\_GRID}} : decay scale for
smale-scale topography deviation (in meters)

\end{itemize}

\subsection{Namelists of the externalized surface for PREP\_REAL\_CASE} \label{surfex_prep}
Fore more informations, see SURFEX documentation.
\begin{itemize}
\item NAM\_PREP\_SURF\_ATM\index{NAM\_PREP\_SURF\_ATM!surfex namelist}
\item NAM\_PREP\_SEAFLUX\index{NAM\_PREP\_SEAFLUX!surfex namelist}
\item NAM\_PREP\_WATFLUX\index{NAM\_PREP\_WATFLUX!surfex namelist}
\item NAM\_PREP\_FLAKE\index{NAM\_PREP\_FLAKE!surfex namelist}
\item NAM\_PREP\_ISBA\index{NAM\_PREP\_ISBA!surfex namelist}
\item NAM\_PREP\_ISBA\_SNOW\index{NAM\_PREP\_ISBA\_SNOW!surfex namelist}
\item NAM\_PREP\_ISBA\_CARBON\index{NAM\_PREP\_ISBA\_CARBON!surfex namelist}
\item NAM\_PREP\_TEB\index{NAM\_PREP\_TEB!surfex namelist}
\item NAM\_PREP\_TEB\_SNOW\index{NAM\_PREP\_TEB\_SNOW!surfex namelist}
\item NAM\_PREP\_TEB\_GARDEN\index{NAM\_PREP\_TEB\_GARDEN!surfex namelist}
\item NAM\_PREP\_GARDEN\_SNOW\index{NAM\_PREP\_GARDEN\_SNOW!surfex namelist}
\end{itemize}


\subsection{Free formatted part : Vertical grid}

This part is optional in the file, read only if YZGRID\_TYPE='MANUAL'.
It must begin by the keyword {\bf ZHAT}
In this case (NKMAX+1) levels are written in meters in free format after the keyword, 
from ground level (generally 0) to rigid top level.

\subsection{Second free formatted part related to chemical species}:
This part is only used if you have previously run {\bf extractarpege}
with MOCAGE outputs. This part must begin by the keyword {\bf MOC2MESONH}.
Then, the list of the MesoNH chemical species, and
their corresponding GRIB code in the GRIB file, is specified as follows: 
\begin{verbatim}
MOC2MESONH
transfer mocage/RACM variables (default values)
2 # NUMBER OF OPTIONAL GRIB VARIABLES
(A4,1X,I5)
O3   180
NO2  183
\end{verbatim}

\subsection{Examples of namelist file PRE\_REAL1.nam}
\begin{itemize}
\item GRIBEX file, levels being calculated, and chemical species
\begin{verbatim}
&NAM_FILE_NAMES HATMFILE   ='ALT90101500'     , HATMFILETYPE='GRIBEX' ,
                HPGDFILE   ='PGDFILE_10km'    ,
                CINIFILE   ='CPL_example1'    /
&NAM_REAL_CONF CEQNSYS="LHE", NVERB=7 /
&NAM_VER_GRID NKMAX=60, YZGRID_TYPE='FUNCTN', ZDZGRD=50., ZDZTOP=500., 
              ZZMAX_STRGRD=3000., ZSTRGRD=2., ZSTRTOP=6. /
&NAM_BLANK
              
MOC2MESONH
transfer mocage/RACM variables 
2 # NUMBER OF OPTIONAL GRIB VARIABLES
(A4,1X,I5)
O3   180
NO2  183
\end{verbatim}
{\bf N.B.:} the mocage part is written at the end of namelist file.
\subitem

\item MESONH file and levels given manually
\begin{verbatim}
&NAM_FILE_NAMES HATMFILE   ='POI03.1.DAY01.001'  , HATMFILETYPE='MESONH' ,
                HPGDFILE   ='PGDFILE_10km'       ,
                CINIFILE   ='CPL_example2'       /
&NAM_REAL_CONF NVERB=5 /
&NAM_VER_GRID NKMAX=10, YZGRID_TYPE='MANUAL' /
ZHAT
0.
1050.
2100.
3250.
4300.
5200.
6100.
7000.
8000.
9000.
10000.
\end{verbatim}
\end{itemize}


%---------------------------------------------------------------
\section{Processing of extra fields in AROME GRIB file}
\label{i:realdummy}

AROME GRIB files obtained with extractMF contain fields which aren't read by default by PREP\_REAL\_CASE. You could want to have this extra fields in your FM file. To get this fields, you have to use LDUMMY\_REAL=T and fill a free formatted part at the end of PRE\_REAL1.nam.

This part must begin by the keyword {\bf DUMMY\_2D} (for 2D fields).
Then, the list of the MesoNH dummy fields, and their corresponding GRIB code in the GRIB file (parameter code, and if 
ambiguity with another variables type and value of level),
 is specified as follows (This example  matches with all the extra fields available in AROME GRIB file): 

\begin{verbatim}
DUMMY_2D
diagnostiques prevus
10
ACCPLUIE 62
ACCNEIGE 79
ACCGRAUPEL 78
INSPLUIE 169
INSNEIGE 64
INSGRAUPEL 63
UM05 33 105 10
VM05 34 105 10
CLSHUMI.RELA 52 105 2
CLSTEMPE 11 105 2
\end{verbatim}


%%%%%%%%%%%%%%%%%%%%%%%%%%%%%%%%%%%%%%%%%%%%%%%%%%%%%%%%%%%%%%%%%
%%%%%%%%%%%%%%%%%%%%%%%%%%%%%%%%%%%%%%%%%%%%%%%%%%%%%%%%%%%%%%%%%
%%%%%%%%%%%%%%%%%%%%%%%%%%%%%%%%%%%%%%%%%%%%%%%%%%%%%%%%%%%%%%%%%
%%%%%%%%%%%%%%%%%%%%%%%%%%%%%%%%%%%%%%%%%%%%%%%%%%%%%%%%%%%%%%%%%
%%%%%%%%%%%%%%%%%%%%%%%%%%%%%%%%%%%%%%%%%%%%%%%%%%%%%%%%%%%%%%%%%
%%%%%%%%%%%%%%%%%%%%%%%%%%%%%%%%%%%%%%%%%%%%%%%%%%%%%%%%%%%%%%%%%
%%%%%%%%%%%%%%%%%%%%%%%%%%%%%%%%%%%%%%%%%%%%%%%%%%%%%%%%%%%%%%%%%
%%%%%%%%%%%%%%%%%%%%%%%%%%%%%%%%%%%%%%%%%%%%%%%%%%%%%%%%%%%%%%%%%
\clearpage
\chapter{Horizontal interpolation from a MESO-NH file: {\bf SPAWNING}} \label{c:SPAWNING}
\label{s:spawn}
\section{Presentation}
This program performs the horizontal interpolation from one MESO-NH file
into another (respectively file 1 and file 2).
The grid of the file 2 must be exactly included in
the grid of file 1. The file 2 can be directly used
for a model run, but it contains smooth surface fields (especially the orography).
It is possible to run the model with
the two files with gridnesting interaction, since a iterative procedure
insures the gridnesting condition on the orographies.

The domain of the file 2 can be defined either:
\begin{enumerate}
\item by namelist NAM\_GRID2\_SPA specification.
\item with the domain of another FM file, which grid is coherent with the
input file. For example this file can be a PGD file created by {\bf
PREP\_PGD} with a domain defined from the domain of file 1 and the same
type of specifications as those in NAM\_GRID2\_SPA (see above).
\end{enumerate}



\section{The input SPAWN1.nam file}

\subsection{Namelist NAM\_BLANK} 
See section \ref{s:namblank} page \pageref{s:namblank} for details.
\newpage

\subsection{Namelist NAM\_GRID2\_SPA (manual definition of domain) }

\begin{center}
\begin{tabular} {|l|l|l|}
\hline
Fortran name & Fortran type & default value\\
\hline
\hline
IXOR         & integer    & 1   \\
IYOR         & integer    & 1  \\
IXSIZE       & integer    & file 1 domain  \\
IYSIZE       & integer    & file 1 domain \\
IDXRATIO     & integer    & 1   \\
IDYRATIO     & integer    & 1  \\
GBAL\_ONLY    & logical    & .FALSE.  \\
\hline
\end{tabular}
\end{center}


\begin{itemize}
\item IXOR: first point I index, according to the file 1 grid, left to 
and out of the new physical domain.
\index{IXOR!\innam{NAM\_GRID2\_SPA}}
\item IYOR: first point J index, according to the file 1 grid, under 
and out of the new physical domain.
\index{IYOR!\innam{NAM\_GRID2\_SPA}}
\item IXSIZE: number of grid points in I direction, according to file 1 grid, recovered
by the new domain. It must be equal to $2^m \times 3^n \times 5^p$ with $(m,n,p) \ge 0$
\index{IXSIZE!\innam{NAM\_GRID2\_SPA}}
\item IYSIZE: number of grid points in J direction, according to file 1 grid, recovered
by the new domain. It must be equal to $2^m \times 3^n \times 5^p$ with $(m,n,p) \ge 0$
\index{IYSIZE!\innam{NAM\_GRID2\_SPA}}
\item IDXRATIO: resolution factor in I direction between the file 1 grid and the new grid.
It must be  equal to $2^m \times 3^n \times 5^p$ with $(m,n,p) \ge 0$
\index{IDXRATIO!\innam{NAM\_GRID2\_SPA}}
\item IDYRATIO: resolution factor in J direction between the file 1 grid and the new grid.
It must be equal to $2^m \times 3^n \times 5^p$ with $(m,n,p) \ge 0$
\index{IDYRATIO!\innam{NAM\_GRID2\_SPA}}
\item GBAL\_ONLY: Flag  to enforce anelastic constraint only. The spawned file have the same characteristics as the CINIFILE one.
\index{GBAL\_ONLY!\innam{NAM\_GRID2\_SPA}}
\end{itemize}

\subsection{Namelist NAM\_LUNIT2\_SPA (file names)}

\begin{center}
\begin{tabular} {|l|l|l|}
\hline
Fortran name & Fortran type & default value\\
\hline
\hline
CINIFILE     & character (len=28)    & 'INIFILE'   \\
CINIFILEPGD  & character (len=28)    & ''   \\
YDOMAIN      & character (len=28)    & none  \\
YSPAFILE     & character (len=28)    & none \\
YSPANBR      & character (len=2)     & '00' \\
YDADINIFILE  & character (len=28)    & ''   \\
YDADSPAFILE  & character (len=28)    & ''   \\
YSONFILE     & character (len=28)    & ''   \\
\hline
\end{tabular}
\end{center}

\begin{itemize}
\item CINIFILE :
\index{CINIFILE!\innam{NAM\_LUNIT2\_SPA}} 
name of the initial FM-file 1 (father domain)
which will be used to spawn model 2.
\item CINIFILEPGD :
\index{CINIFILEPGD!\innam{NAM\_LUNIT2\_SPA}} 
name of the PGD associated to the initial FM-file 1 CINIFILE (father domain).
\item YDOMAIN : 
\index{YDOMAIN!\innam{NAM\_LUNIT2\_SPA}}
name of the file which defines the domain for model 2. If a domain file is provided for YDOMAIN, then all the information of namelist NAM\_GRID2\_SPA will be ignored. 
\item YSPAFILE : optional name of the spawned FM-file 2 (output file).
\index{YSPAFILE!\innam{NAM\_LUNIT2\_SPA}}
If the user does not specify this name,
or if YSPAFILE = CINIFILE, the code builds the spawned FM-file name as:
\subitem YSPAFILE = CINIFILE.spaYSPANBR
\subitem or YSPAFILE = CINIFILE.sprYSPANBR if YSONFILE is provided.
\item YSPANBR :
\index{YSPANBR!\innam{NAM\_LUNIT2\_SPA}} :
NumBeR which will be added to CINIFILE to generate the  
FM-file name of the SPAwned file (string of 2 characters)
\item YDADINIFILE : if GBAL\_ONLY=.TRUE. : name of the CINIFILE dad.
\index{YDADINIFILE!\innam{NAM\_LUNIT2\_SPA}}
\item YDADSPAFILE : if GBAL\_ONLY=.TRUE. : name of the YSPAFILE dad. 
\index{YDADSPAFILE!\innam{NAM\_LUNIT2\_SPA}}
Program will check that YDADINIFILE and YDADSPAFILE have the same 
characteristics,
before replacing the dad name of YSPAFILE by YDADSPAFILE instead of YDADINIFILE.
YDADSPAFILE must exist before running the spawning job. 
\item YSONFILE : optional name of a spawned FM-file (input file). 
It must have the same resolution as the spawned FM-file 2 (output file).
The fields of YSONFILE will be used at points included in the domain
defined by YDOMAIN or NAM\_GRID2\_SPA, instead of interpolated fields
 of CINIFILE.
This allows to keep finest information when defining a new finest domain to
follow atmospheric system.
\index{YSONFILE!\innam{NAM\_LUNIT2\_SPA}}
\end{itemize} 

\subsection{Namelist NAM\_SPAWN\_SURF}
\begin{center}
\begin{tabular} {|l|l|l|}
\hline
Fortran name & Fortran type & default value\\
\hline
\hline
LSPAWN\_SURF & logical   & .TRUE.   \\
\hline
\end{tabular}
\end{center}
\begin{itemize}
\item LSPAWN\_SURF :
\index{LSPAWN\_SURF!\innam{NAM\_SPAWN\_SURF}} : 
flag to  perform or not the interpolation of all the surface fields (both physiographic and prognostic fields). Note that these interpolations are performed by the {\bf externalized surface} facilities. However, no specific namelist is required for this operation.
\begin{itemize}
\item LSPAWN\_SURF = .TRUE. : the interpolations are made
\item LSPAWN\_SURF = .FALSE. : the interpolations are not made and therefore \underline{no} surface field will be present in the output spawned file.
\end{itemize}
\end{itemize}
%%%%%%%%%%%%%%%%%%%%%%%%%%%%%%%%%%%%%%%%%%%%%%%%%%%%%%%%%%%%%%%
%%%%%%%%%%%%%%%%%%%%%%%%%%%%%%%%%%%%%%%%%%%%%%%%%%%%%%%%%%%%%%%
%%%%%%%%%%%%%%%%%%%%%%%%%%%%%%%%%%%%%%%%%%%%%%%%%%%%%%%%%%%%%%%
%%%%%%%%%%%%%%%%%%%%%%%%%%%%%%%%%%%%%%%%%%%%%%%%%%%%%%%%%%%%%%%
%%%%%%%%%%%%%%%%%%%%%%%%%%%%%%%%%%%%%%%%%%%%%%%%%%%%%%%%%%%%%%%
%%%%%%%%%%%%%%%%%%%%%%%%%%%%%%%%%%%%%%%%%%%%%%%%%%%%%%%%%%%%%%%
%%%%%%%%%%%%%%%%%%%%%%%%%%%%%%%%%%%%%%%%%%%%%%%%%%%%%%%%%%%%%%%
\chapter{{\bf PREP\_SURFEX}}
\section{Presentation}
{\bf PREP\_SURFEX} performs the interpolations of surface fields from one
grid to another.
\\

What's going in and out?
\begin{itemize}
\item Input:
\begin{itemize}
\item a file containing the surface 2D variable fields
(hereafter called input file); it can be either
\begin{itemize}
\item a GRIB file obtained from {\bf extractecmwf} or {\bf extractarpege}
\item a MESO-NH file (obtained with {\bf SPAWNING} for example)
\end{itemize}
\item a physiographic data file (it can also be a complete MESO-NH file).
\item the file PRE\_REAL1.nam which contains the directives for PREP\_SURFEX
\end{itemize}
\item Output:
\begin{itemize}
\item the FM-file containing the physiographic and pronostic surface fields.
\end{itemize}
\end{itemize}


\section{The file PRE\_REAL1.nam}
This file contains namelists with the directives to run PREP\_SURFEX.
The namelists contain the names of the files.
\begin{enumerate}

\item\underline{Namelist NAM\_FILE\_NAMES}: (contains file names)
\index{NAM\_FILE\_NAMES!namelist description}

\begin{center}
\begin{tabular} {|l|l|l|}
\hline
Fortran name & Fortran type & default value\\
\hline
\hline
HATMFILE      & character (LEN=28) & ' '           \\
HATMFILETYPE  & character (LEN=6)  & 'GRIBEX'      \\
HPGDFILE      & character (LEN=28) & ' '           \\
CINIFILE      & character (LEN=28) & 'INIFILE'     \\
\hline
\end{tabular}
\end{center}

\begin{itemize}
\item HATMFILE : name of the atmospheric file (up to 28 characters)
\index{HATMFILE!\innam{NAM\_FILE\_NAMES}}
\item HATMFILETYPE : type of the atmospheric file ('GRIBEX', 'MESONH')
\index{HATMFILETYPE!\innam{NAM\_FILE\_NAMES}}
\item HPGDFILE : name of the Physiographic Data File (up to 28 characters)
\index{HPGDFILE!\innam{NAM\_FILE\_NAMES}}
\item CINIFILE : name of the MESO-NH output FM-file, used as initial
or coupling file in a MESO-NH simulation
\index{CINIFILE!\innam{NAM\_FILE\_NAMES}}
\end{itemize}

\item\underline{ externalized surface} namelists for {\bf PREP\_SURFEX}

The surface initial fields are produced by the {\bf externalized surface} facilities.
Refer to the {\bf documentation of the surface} for more details.
For {\bf PREP\_SURFEX}, you must fill the following namelists:

\begin{itemize}
\item NAM\_PREP\_SURF\_ATM
\item NAM\_PREP\_SEAFLUX (if you chose to use the SEAFLX scheme)
\item NAM\_PREP\_WATFLUX (if you chose to use the WATFLX scheme)
\item NAM\_PREP\_TEB (if you chose to use the TEB urban scheme)
\item NAM\_PREP\_ISBA (if you chose to use the ISBA scheme)
\end{itemize}

\underline{CAUTION}: 
\begin{enumerate}
\item Note that all namelists can be void, but only if the initial file name for HATMFILE you provide in namelist NAM\_FILE\_NAMES \underline{contains} the externalized surface fields.
\item If the file HATMFILE does not contain externalized surface fields, you must fill at least namelist NAM\_PREP\_SURF\_ATM (if you want to initialize the surface prognostic fields from an input file). You can also define more precisely the surface fields by using the namelists for each scheme.
\end{enumerate}

\end{enumerate}




