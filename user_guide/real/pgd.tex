\chapter{Creation of MESO-NH physiographic data file} \label{c:PGD}


\section{\bf PREP\_PGD}
The physiographic fields are averaged or interpolated on the MESO-NH
physiographic grid by the program {\bf PREP\_PGD}. 
They are stored in a FM file, called PGD file, but with fewer elements than a MESO-NH file.
With the physiographic 2D fields, the geographic and grid data
are written in this file.
Note that a MESO-NH simulation runs on the grid defined here in 
{\bf PREP\_PGD}. \\

\subsection{Namelist NAM\_CONFIO}
See section \ref{s:namconfio} page \pageref{s:namconfio} for details.

\subsection{Namelist NAM\_CONF\_PGD}
\index{NAM\_CONF\_PGD!namelist description}
\begin{center}
\begin{tabular} {|l|l|l|}
\hline
Fortran name & Fortran type & default value\\
\hline
\hline
NHALO\_MNH  & INTEGER & 1 \\
JPHEXT      & INTEGER & 1 \\
\hline
\end{tabular}
\end{center}

\begin{itemize}
\item
NHALO\_MNH: Size of the halo for parallel distribution.
This variable is related to computer performance but has no
impact on simulation results.\\
\index{NHALO\_MNH!\innam{NAM\_CONF\_PGD}}

\item
JPHEXT:  Horizontal External points number\\
JPHEXT must be equal to 3 for cyclic cases with WENO5.
\index{JPHEXT!\innam{NAM\_CONF\_PGD}}
\end{itemize}



\subsection{Namelist NAM\_PGDFILE}
\index{NAM\_PGDFILE!namelist description}

\begin{center}
\begin{tabular} {|l|l|l|}
\hline
Fortran name & Fortran type & default value\\
\hline
\hline
CPGDFILE      & character (LEN=28) & ' '           \\
NHALO         & INTEGER      & 15 \\
\hline
\end{tabular}
\end{center}

\begin{itemize}
\item CPGDFILE : name of the output Physiographic Data File
\index{CPGDFILE!\innam{NAM\_PGDFILE}}
\item NHALO :  Size of the halo for parallel distribution
\index{NHALO!\innam{NAM\_PGDFILE}}

\end{itemize}
\subsection{Namelist NAM\_ZSFILTER}
\index{NAM\_ZSFILTER!namelist description}

\begin{center}
\begin{tabular} {|l|l|l|}
\hline
Fortran name & Fortran type & default value\\
\hline
\hline
NZSFILTER       & INTEGER      & 1 \\
\hline
\end{tabular}
\end{center}

\begin{itemize}
\item NZSFILTER : number of iterations of the spatial filter applied to smooth the orography (integer, 1 iteration removes the $2\Delta x$ signal, 50\% of the $4\Delta x$
signal, 25\% of the $6\Delta x$ signal, etc\footnote{The amplitude of the
filtered signal for each wavelength $\lambda\Delta x$ 
is $\frac{1}{2}\left( cos(2\pi/\lambda) +1\right)$.}...).
\index{NZSFILTER!\innam{NAM\_ZSFILTER}}
\end{itemize}
%\newpage
\subsection{Namelists for the externalized surface}

As indicated above, further definition of surface parameters are not done by MESONH itself, but by the externalized surface included in it. For more informations see SURFEX documentation.\\

\begin{itemize}
\item NAM\_PGD\_SCHEMES\index{NAM\_PGD\_SCHEMES!surfex namelist}
\item NAM\_PGD\_GRID\index{NAM\_PGD\_GRID!surfex namelist}
\item NAM\_CONF\_PROJ\index{NAM\_CONF\_PROJ!surfex namelist}
\item NAM\_CONF\_PROJ\_GRID\index{NAM\_CONF\_PROJ\_GRID!surfex namelist}
\item NAM\_INIFILE\_CONF\_PROJ\index{NAM\_INIFILE\_CONF\_PROJ!surfex namelist}
\item NAM\_WRITE\_COVER\_TEX\index{NAM\_WRITE\_COVER\_TEX!surfex namelist}
\item NAM\_READ\_DATA\_COVER\index{NAM\_READ\_DATA\_COVER!surfex namelist}
\item NAM\_PGD\_ARRANGED\_COVER\index{NAM\_PGD\_ARRANGED\_COVER!surfex namelist}
\item NAM\_COVER\index{NAM\_COVER!surfex namelist}
\item NAM\_ECOCLIMAP2\index{NAM\_ECOCLIMAP2!surfex namelist}
\item NAM\_ZS\index{NAM\_ZS!surfex namelist}
\item NAM\_SEABATHY\index{NAM\_SEABATHY!surfex namelist}
\item NAM\_ISBA\index{NAM\_ISBA!surfex namelist}
\item NAM\_DATA\_FLAKE\index{NAM\_DATA\_FLAKE!surfex namelist}
\item NAM\_DUMMY\_PGD\index{NAM\_DUMMY\_PGD!surfex namelist}
\item NAM\_CH\_EMIS\_PGD\index{NAM\_CH\_EMIS\_PGD!surfex namelist}
\end{itemize}

\subsection{Examples of PRE\_PGD1.nam file \label{example}}

A {\bf PREP\_PGD} run where you use the data files provided by the MESO-NH team and some files of your own for the dummy fields:
\begin{verbatim}
&NAM_PGDFILE        CPGDFILE = 'PGDFILE_1' /
&NAM_PGD_GRID       CGRID = 'CONF PROJ ' /
&NAM_CONF_PROJ      XLAT0 = 45., XLON0=0., XRPK=0.7, XBETA=0. /
&NAM_CONF_PROJ_GRID NIMAX=100, NJMAX=100, XLATCEN=42.5, XLONCEN=2.5, XDX=10000. , XDY=10000. /
&NAM_INIFILE_CONF_PROJ /
&NAM_PGD_SCHEMES CNATURE='ISBA  ', CSEA='SEAFLX', CTOWN='TEB   ', CWATER='WATFLX' /
&NAM_COVER YCOVER='ecoclimap_v2', YCOVERFILETYPE='DIRECT' /
&NAM_ZS    YZS='gtopo30', YZSFILETYPE='DIRECT' /
&NAM_ISBA  YCLAY='clay_fao', YCLAYFILETYPE='DIRECT',  XUNIF_RUNOFFB=0.5 ,
           YSAND='sand_fao', YSANDFILETYPE='DIRECT' /
&NAM_DUMMY_PGD  NDUMMY_NBR         = 6
                CDUMMY_NAME(1)     = 'SST_2001062100'          
                CDUMMY_AREA(1)     = 'SEA'
                CDUMMY_ATYPE(1)    = 'ARI'
                CDUMMY_FILE(1)     = 'SSTn2001062100.dat'          
                CDUMMY_FILETYPE(1) = 'ASCLLV'
                CDUMMY_NAME(2)     = 'SST_2001062200'          
                CDUMMY_AREA(2)     = 'SEA'
                CDUMMY_ATYPE(2)    = 'ARI'
                CDUMMY_FILE(2)     = 'SSTn2001062200.dat'          
                CDUMMY_FILETYPE(2) = 'ASCLLV'
                CDUMMY_NAME(3)     = 'SST_2001062300'          
                CDUMMY_AREA(3)     = 'SEA'
                CDUMMY_ATYPE(3)    = 'ARI'
                CDUMMY_FILE(3)     = 'SSTn2001062300.dat'          
                CDUMMY_FILETYPE(3) = 'ASCLLV'
                CDUMMY_NAME(4)     = 'SST_2001062400'          
                CDUMMY_AREA(4)     = 'SEA'
                CDUMMY_ATYPE(4)    = 'ARI'
                CDUMMY_FILE(4)     = 'SSTn2001062400.dat'          
                CDUMMY_FILETYPE(4) = 'ASCLLV'
                CDUMMY_NAME(5)     = 'SST_2001062500'          
                CDUMMY_AREA(5)     = 'SEA'
                CDUMMY_ATYPE(5)    = 'ARI'
                CDUMMY_FILE(5)     = 'SSTn2001062500.dat'          
                CDUMMY_FILETYPE(5) = 'ASCLLV'
                CDUMMY_NAME(6)     = 'SST_2001062600'          
                CDUMMY_AREA(6)     = 'SEA'
                CDUMMY_ATYPE(6)    = 'ARI'
                CDUMMY_FILE(6)     = 'SSTn2001062600.dat', CDUMMY_FILETYPE(6) = 'ASCLLV'/
\end{verbatim}
\newpage
Another {\bf PREP\_PGD} run for the father PGD file, with all the namelist variables : 
\begin{verbatim}
&NAM_PGDFILE CPGDFILE='PGD_AMMA1_10km_m46_b1' /
&NAM_PGD_SCHEMES  CNATURE='ISBA' , CSEA='SEAFLX', CWATER='WATFLX', CTOWN='TEB'/
&NAM_PGD_GRID CGRID='CONF PROJ'/
&NAM_CONF_PROJ   XLAT0=13. ,XLON0=-1.5, XRPK=0.2249510543, XBETA=0 /
&NAM_CONF_PROJ_GRID  XLATCEN=13. , XLONCEN=-1.5 , NIMAX=324 ,NJMAX=240 ,
                     XDX=10000 , XDY=10000/
&NAM_COVER     YCOVER='ecoclimats_v2', YCOVERFILETYPE='DIRECT',LRM_TOWN=.FALSE. /
&NAM_ZS        YZS='gtopo30',    YZSFILETYPE='DIRECT' ,
               COROGTYPE='AVG', XENV=0./
&NAM_ISBA      NPATCH=12, CISBA='3-L',  CPHOTO='AGS', NGROUND_LAYER=3, 
               YCLAY='clay_fao', YCLAYFILETYPE='DIRECT' ,
               YSAND='sand_fao', YSANDFILETYPE='DIRECT',
               XUNIF_RUNOFFB=0.5 /
&NAM_CH_EMIS_PGD NEMIS_PGD_NBR=0 /               
&NAM_DUMMY_PGD  NDUMMY_PGD_NBR=0 /
\end{verbatim}

And for the son PGD file : 
\begin{verbatim}
&NAM_PGDFILE CPGDFILE='PGD_AMMA1_5km_m46_b1' /
&NAM_PGD_SCHEMES  CNATURE='ISBA' , CSEA='SEAFLX', CWATER='WATFLX', CTOWN='TEB'/
&NAM_PGD_GRID YINIFILE='PGD_AMMA1_10km_m46_b1',YINIFILETYPE='MESONH' /
&NAM_INIFILE_CONF_PROJ IXOR=101 ,IYOR=21 ,IXSIZE=180 ,IYSIZE=150 ,
                       IDXRATIO=2, IDYRATIO=2/
&NAM_COVER     YCOVER='ecoclimats_v2', YCOVERFILETYPE='DIRECT',LRM_TOWN=.FALSE. /
&NAM_ZS        YZS='gtopo30', YZSFILETYPE='DIRECT' ,
               COROGTYPE='AVG', XENV=0./
&NAM_ISBA      NPATCH=12, CISBA='3-L',  CPHOTO='AGS', NGROUND_LAYER=3, 
               YCLAY='clay_fao', YCLAYFILETYPE='DIRECT' ,
               YSAND='sand_fao', YSANDFILETYPE='DIRECT',
               XUNIF_RUNOFFB=0.5 /
&NAM_CH_EMIS_PGD NEMIS_PGD_NBR=0 /               
&NAM_DUMMY_PGD  NDUMMY_PGD_NBR=0 /
\end{verbatim}

%%%%%%%%%%%%%%%%%%%%%%%%%%%%%%%%%%%%%%%%%%%%%%%%%%%%%%%%%%%%%%%
\newpage
\section{Modification of PGD files for grid-nesting: {\bf
PREP\_NEST\_PGD}}

In order to run models with the gridnesting technique, a condition on
the orography must be satisfied. In the following, if file \#2 is completely
included in (and therefore in interaction during the run with) file \#1,
file \#2 will be called the SON file, and file \#1 the DAD file.
In the following, the DAD file number must be smaller than any of its SON number.

The condition on the orography is:
"the mean of orography for a SON file in the domain corresponding to the
grid mesh of its DAD file, must be equal to the orography of the DAD file
in this mesh".

Such a condition is not automatically satisfied when using enhanced
orographies.
The program {\bf PREP\_NEST\_PGD} performs post-treatments on the orographies
of up to 8 PGD files that will be used to create initialization files 
for a gridnested run. It modifies the orography of a DAD from the mean
of the orography of its (several) SON(s).



The namelist file PRE\_NEST\_PGD1.nam contains:
\begin{enumerate}
\item
\underline{Namelists NAM\_PGD{\it \bf{N}}} (where {\it \bf{N}} goes from 1 to 8):
\index{YPGD1!\innam{NAM\_PGD1}}
\index{YPGD2!\innam{NAM\_PGD2}}
\index{YPGD3!\innam{NAM\_PGD3}}
\index{YPGD4!\innam{NAM\_PGD4}}
\index{YPGD5!\innam{NAM\_PGD5}}
\index{YPGD6!\innam{NAM\_PGD6}}
\index{YPGD7!\innam{NAM\_PGD7}}
\index{YPGD8!\innam{NAM\_PGD8}}
\index{IDAD!\innam{NAM\_PGD1}}
\index{IDAD!\innam{NAM\_PGD2}}
\index{IDAD!\innam{NAM\_PGD3}}
\index{IDAD!\innam{NAM\_PGD4}}
\index{IDAD!\innam{NAM\_PGD5}}
\index{IDAD!\innam{NAM\_PGD6}}
\index{IDAD!\innam{NAM\_PGD7}}
\index{IDAD!\innam{NAM\_PGD8}}
\begin{itemize}
\item YPGD{\it \bf{N}}: name of the PGD file \#{\it \bf{N}}
\item IDAD: number of the DAD file of file \#{\it \bf{N}}.
The DAD file number IDAD must be \underline{smaller} than \#{\it \bf{N}}.
\end{itemize}
\bigskip
\item
\underline{Namelist NAM\_NEST\_PGD}:
\begin{itemize}
\item
YNEST: string of 2 characters to be added to the PGD file names to define
the corresponding output PGD file names. The input file {\it YPGD{\bf N}}
will be modified into file {\it YPGD{\bf N}}.nest{\it YNEST}
\index{YNEST!\innam{NAM\_NEST\_PGD}}
\end{itemize}

\bigskip
\item
\underline{Namelist NAM\_CONFIO}:
See section \ref{s:namconfio} page \pageref{s:namconfio} for details.

\bigskip
\item
\underline{Namelist NAM\_CONF\_NEST} :
\index{NAM\_CONF\_NEST!namelist description}
\begin{center}
\begin{tabular} {|l|l|l|}
\hline
Fortran name & Fortran type & default value\\
\hline
\hline
NHALO\_MNH  & INTEGER & 1 \\
JPHEXT      & INTEGER & 1 \\
\hline
\end{tabular}
\end{center}

\begin{itemize}
\item
NHALO\_MNH: Size of the halo for parallel distribution.
This variable is related to computer performance but has no
impact on simulation results.\\
NHALO must be equal to 3 for WENO5 cases in parallel runs
\index{NHALO!\innam{NAM\_CONF\_NEST}}

\item
JPHEXT:  Horizontal External points number\\
JPHEXT must be equal to 3 for cyclic cases with WENO5.
\index{JPHEXT!\innam{NAM\_CONF\_NEST}}
\end{itemize}

\end{enumerate}



\underline{Example of namelist PRE\_NEST\_PGD1.nam:}
\begin{verbatim}
&NAM_PGD1 YPGD1 = 'PGDFILE_1' /
&NAM_PGD2 YPGD2 = 'PGDFILE_2', IDAD = 1 /
&NAM_PGD3 YPGD3 = 'PGDFILE_3', IDAD = 1 /
&NAM_PGD4 YPGD4 = 'PGDFILE_4', IDAD = 3 /
&NAM_PGD5 YPGD5 = 'PGDFILE_5', IDAD = 2 /
&NAM_PGD6  /
&NAM_PGD7  /
&NAM_PGD8  /
&NAM_NEST_PGD YNEST = 'e1' /
\end{verbatim}

%%%%%%%%%%%%%%%%%%%%%%%%%%%%%%%%%%%%%%%%%%%%%%%%%%%%%%%%%%%%%%%

\section{Zoom of a PGD file: {\bf ZOOM\_PGD}}
The previous condition on the orography needed
when using the gridnesting technique implies that all the PGD files have to be
created (with {PREP\_PGD} and {PREP\_NEST\_PGD} programs)
{\bf before} beginning the run.
However, the user is not always sure where (and when) to initialize
the inner models.
To avoid to set exactly the domain of inner models at the PREP\_PGD step, 
one solution is to make PGD file on larger domain
and then, zoom it\footnote{This was done during the PREP\_IDEAL\_CASE or 
PREP\_REAL\_CASE step before the masdev4\_6 version.}
 on the part of the domain of interest when known
with the following program {\bf ZOOM\_PGD}. Then the output PGD file is used
as PGD file for the interpolations of atmospheric fields
with {SPAWNING} and {PREP\_REAL\_CASE} programs. 



The namelist file PRE\_ZOOM1.nam contains 2 namelists:
\begin{enumerate}
\item
\underline{Namelist NAM\_PGDFILE}: (contains file names)
\index{NAM\_PGDFILE!namelist description}

\begin{center}
\begin{tabular} {|l|l|l|}
\hline
Fortran name & Fortran type & default value\\
\hline
\hline
CPGDFILE      & character (LEN=28) & 'PGDFILE'      \\
YZOOMFILE     & character (LEN=28) & none           \\
YZOOMNBR      & character (LEN=2)  & '00'           \\
\hline
\end{tabular}
\end{center}

\begin{itemize}
\item CPGDFILE : name of the input Physiographic Data File
\index{CPGDFILE!\innam{NAM\_PGDFILE}}
\item YZOOMFILE : optional name of the zoomned FM-file (output file).
\index{YZOOMFILE!\innam{NAM\_PGDFILE}} 
If the user does not specify this name,
or if YZOOMFILE = CPGDFILE, the code builds the zoomned FM-file name as:
\subitem YZOOMFILE = CPGDFILE.zYZOOMNBR
\item YZOOMNBR :
\index{YZOOMNBR!\innam{NAM\_PGDFILE}} :
NumBeR which will be added to CPGDFILE to generate the  
name of the Zoomned FM-file (string of 2 characters).
\end{itemize}
\item
\underline{Namelist NAM\_CONFIO}
See section \ref{s:namconfio} page \pageref{s:namconfio} for details.
\item
\underline{Namelist NAM\_MESONH\_DOM}: (contains domain definition variables)
\index{NAM\_MESONH\_DOM!namelist description}

\begin{center}
\begin{tabular} {|l|l|l|l|}
\hline
Fortran name & Fortran type & default value & remarks \\
\hline
\hline
NIMAX & integer & input PGD domain & \\
NJMAX & integer & input PGD domain & \\
NXOR  & integer & NUNDEF & if NUNDEF, domain is \\ 
      &         &        & in the middle of PGD \\ 
NYOR  & integer & NUNDEF & if NUNDEF, domain is \\ 
      &         &        & in the middle of PGD \\ 
\hline
\end{tabular}
\end{center}

\begin{itemize}
\item NIMAX : number of grid points in I direction, according to input file 
grid, recovered by the new domain. NIMAX must be equal to $2^m \times 3^n \times 5^p$ with $(m,n,p) \in [0,+\infty[$
\index{NIMAX!\innam{NAM\_MESONH\_DOM}}
\item NJMAX : number of grid points in J direction, according to input file 
grid, recovered by the new domain. NJMAX must be equal to $2^m \times 3^n \times 5^p$ with $(m,n,p) \in [0,+\infty[$
\index{NJMAX!\innam{NAM\_MESONH\_DOM}}
\item NXOR : first point I index, left to and out of the new physical domain.
\index{NXOR!\innam{NAM\_MESONH\_DOM}}
\item NYOR : first point J index, under and out of the new physical domain.
\index{NYOR!\innam{NAM\_MESONH\_DOM}}
\end{itemize}

\end{enumerate}

\underline{Example of namelist PRE\_ZOOM1.nam:}
\begin{verbatim}
&NAM_PGDFILE CPGDFILE = 'PGDFILE_1.neste1' ,
             YZOOMNBR = '58'    /
&NAM_MESONH_DOM NIMAX=60, NJMAX=50,
                NXOR=5, NYOR=8      /
\end{verbatim}


