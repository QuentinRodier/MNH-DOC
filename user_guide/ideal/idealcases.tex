\chapter{Preparation of an ideal simulation : PREP\_IDEAL\_CASE}\label{c:PREPIDEAL}

\section{Overview of PREP\_IDEAL\_CASE functionalities}


The "PREP\_IDEAL\_CASE" program  prepares a MESONH file, that contains all the
parameters and fields necessary for the execution of the MESONH model.
Specifically, the grid parameters, the initial fields and the geophysical fields
are included in this file. It is possible using this program to generate
idealized fields defined by few parameters.\\ 

The generated initial conditions are produced analytically, leading to 
 quasi-1D fields or 3D fields or a single profile build with either:
\begin{itemize}
\item
layers of constant Brunt-Vaisala frequency, shear  and 
humidity  
\item a Radiosounding and ideal surface fields
\item a Radiosounding and real physiographic fields 
\item
a Radiosounding and real and ideal surface fields at the same time
\end{itemize}

For these latter cases, the initial fields may be  hydrostatically or
geostrophically  balanced or not.
For these fields to satisfy the anelastic constraint, a final correction is
applied to them.


The interaction between the PREP\_IDEAL\_CASE program and the user is made through 
the PRE\_IDEA1.nam file. The degrees of freedom  are collected in a set of
namelists,  read by this program. 

\newpage
\section {The input: the PRE\_IDEA1.nam file}

It is made  of  two parts :

\begin{itemize}
\item A namelist-part with directives for the preparation
of an idealized case (always present).  The
order of namelists is free and unset namelists can be ommited.
\item A free-formatted part describing
a vertical profile of n layers of constant moist Brunt-Vaisala frequency or
a radiosounding and sometimes the explicit list of the heights of the vertical
levels. This part can be present or absent in the other cases.

\end{itemize}

To initialize a simulation with a radiosounding and real
terrain conditions, it is necessary to perform the PREP\_PGD program 
to create a MESO-NH physiographic data file. This data file contains
 the orography and the physiographic data fields (related to the soil scheme).
It is also possible to perform a complete ideal case with ideal orography and 
non trivial surface conditions.
The user can combine the two possibilities with flags included in the namelist 
NAM\_REAL\_PGD and initialize a simulation with a real orography and
idealized homogeneous surface fields.
If a PREP\_PGD file is specified and if the flags in namelist NAM\_REAL\_PGD
are set to FALSE, homogeneous values can be imposed by the user in namelists 
from the externalized surface facility PGD (namelists NAM\_COVER and NAM\_ISBA),
else the PREP\_PGD fields are taken into account.
\\

In the following, the namelists are listed in alphabetical order.
\newpage
\subsection{Namelist NAM\_AERO\_PRE (init. aerosol scalar variables)} \label{s:namaeropre}
\index{NAM\_AERO\_PRE!namelist description}

If you initialize aerosol during PREP\_IDEAL\_CASE as for ORILAM (chemical aerosols), DUST and SEA SALT.
use the following namelist variables:
\begin{center}
\begin{tabular} {|l|l|l|}
\hline
Fortran name & Fortran type & default value \\
\hline
LORILAM      & logical       & FALSE    \\
LDUST        & logical       & FALSE    \\
LSALT        & logical       & FALSE    \\
LINITPM      & logical       & FALSE    \\
XINIRADIUSI  & real          & 0.05     \\
XINIRADIUSJ  & real          & 0.2      \\
XINISIGI     & real          & 1.8      \\
XINISIGJ     & real          & 2.0      \\
XN0IMIN      & real          & 10.      \\
XN0JMIN      & real          & 1.       \\
CRGUNIT      & character     & "NUMB"   \\
NMODE\_DST   & integer       & 3       \\
XN0MIN       & real          & 1.e3 , 1.e1 , 1.e-2 \\
XINIRADIUS   & real          & 0.044, 0.3215, 1.575 \\
XINISIG      & real          & 2.0, 1.78, 1.85 \\
CRGUNITD     & character     & "NUMB"   \\
NMODE\_SLT   & integer       & 3       \\
XN0MIN\_SLT  & real          & 1.e4 , 1.e2 , 1.e-1 \\
XINIRADIUS\_SLT & real       & 0.14, 1.125,  7.64\\
XINISIG\_SLT    & real       & 1.9, 2., 2. \\
CRGUNITS     & character     & "MASS"   \\
\hline
\end{tabular}
\end{center}

\begin{itemize}
\item LORILAM\index{LORILAM!\innam{NAM\_AERO\_PRE}}:
Flag to activate  chemical aerosol initialization (only if LCH\_INIT\_FIELD=T in NAM\_CH\_MNHCn\_PRE).
\item LDUST\index{LDUST!\innam{NAM\_AERO\_PRE}}: 
Flag to activate  passive dust initialization (3 modes).
\item LSALT\index{LSALT!\innam{NAM\_AERO\_PRE}}: 
Flag to activate  passive sea salt initialization (3 modes).
\item LINITPM\index{LINITPM!\innam{NAM\_AERO\_PRE}}:
Flag to activate  primary aerosol initialization (Black and Organic carbon) from concentration of CO (only if LORILAM=T in NAM\_CH\_MNHCn\_PRE).
\item XINIRADIUSI\index{XINIRADIUSI!\innam{NAM\_AERO\_PRE}}:
Initial mean radius of aitken mode in $\mu m$  (only if LORILAM=T \\
in NAM\_AERO\_PRE).
\item XINIRADIUSJ\index{XINIRADIUSJ!\innam{NAM\_AERO\_PRE}}:
Initial mean radius of accumulation mode in $\mu m$ (only if LORILAM=T in NAM\_AERO\_PRE).
\item XINISIGI\index{XINISIGI!\innam{NAM\_AERO\_PRE}}:
Initial standard deviation of aitken  mode (only if LORILAM=T in NAM\_AERO\_PRE).
\item XINISIGJ\index{XINISIGJ!\innam{NAM\_AERO\_PRE}}:
Initial standard deviation of accumulation  mode (only if LORILAM=T in NAM\_AERO\_PRE).
\item XN0IMIN\index{XN0IMIN!\innam{NAM\_AERO\_PRE}}:
Minimum number concentration of aitken mode (only if LORILAM=T in NAM\_AERO\_PRE).
\item XN0JMIN\index{XN0JMIN!\innam{NAM\_AERO\_PRE}}:
Minimum number concentration of accumulation mode (only if LORILAM=T in NAM\_AERO\_PRE).
\item CRGUNIT\index{CRGUNIT!\innam{NAM\_AERO\_PRE}}:
Definition of XINIRADIUSI or XINIRADIUSJ: mean radius is in mass or in number; possible values are MASS or NUMB (only if LORILAM=T in NAM\_AERO\_PRE).
\item NMODE\_DST\index{NMODE\_DST!\innam{NAM\_AERO\_PRE}}:
Number of DUST mode (between 1 and 3 and only if LDUST=T in NAM\_AERO\_PRE).
\item XN0MIN\index{XN0MIN!\innam{NAM\_AERO\_PRE}}:
Minimum number concentration of the NMODE\_DST in particles by m3 (only if LDUST=T in NAM\_AERO\_PRE).
\item XINIRADIUS\index{XINIRADIUS!\innam{NAM\_AERO\_PRE}}:
Initial mean radius of the NMODE\_DST modes in $\mu m$ (only if LDUST=T in NAM\_AERO\_PRE). 
\item XINISIG\index{XINISIG!\innam{NAM\_AERO\_PRE}}:
Initial standard deviation of the NMODE\_DST modes (only if LDUST=T in NAM\_AERO\_PRE). 
\item CRGUNITD\index{CRGUNITD!\innam{NAM\_AERO\_PRE}}:
Definition of XINIRADIUS : mean radius is in mass or in number; possible values are MASS or NUMB (only if LDUST=T in NAM\_AERO\_PRE).
\item NMODE\_SLT\index{NMODE\_SLT!\innam{NAM\_AERO\_PRE}}:
Number of SALT mode in $\mu m$ (between 1 and 3 and only if LSALT=T in NAM\_AERO\_PRE).
\item XN0MIN\_SLT\index{XN0MIN\_SLT!\innam{NAM\_AERO\_PRE}}:
Minimum number concentration of the NMODE\_SLT in particles by m3 (only if LSALT=T in NAM\_AERO\_PRE).
\item XINIRADIUS\_SLT\index{XINIRADIUS\_SLT!\innam{NAM\_AERO\_PRE}}:
Initial mean radius of the NMODE\_SLT modes (only if LSALT=T in NAM\_AERO\_PRE).
\item XINISIG\_SLT\index{XINISIG\_SLT!\innam{NAM\_AERO\_PRE}}:
Initial standard deviation of the NMODE\_SLT modes (only if LSALT=T in NAM\_AERO\_PRE).
\item CRGUNITS\index{CRGUNITS!\innam{NAM\_AERO\_PRE}}:
Definition of XINIRADIUS\_SLT : mean radius is in mass or in number; possible values are MASS or NUMB (only if LSALT=T in NAM\_AERO\_PRE).
\end{itemize}


\subsection{Namelist NAM\_BLANKn (available user variables)} \label{s:namblank}
\index{NAM\_BLANKn !namelist description}
\begin{center}
\begin{tabular} {|l|l|l|}
\hline
Fortran name & Fortran type & default value \\
\hline
XDUMMY1 .. XDUMMY8 & real                 & 0.       \\
NDUMMY1 .. NDUMMY8 & integer              & 0        \\
LDUMMY1 .. LDUMMY8 & logical              & TRUE     \\
CDUMMY1 .. CDUMMY8 & 80 characters        & ''       \\
XDUMMY             & array(real)          & 20* 0.   \\
NDUMMY             & array(integer)       & 20* 0    \\
LDUMMY             & array(logical)       & 20* TRUE \\
CDUMMY             & array(80 characters) & 20* ''   \\
\hline
\end{tabular}
\end{center}

Eight dummy, real, integer, logical and character*80 variables and
arrays of dummy, real, integer, logical and character*80 for test
and debugging  purposes are defined and passed through the namelist
read operations. None of the MesoNH routines uses any of these
variables. When a developper choses to introduce temporarily a
parameter to some subroutine, he has to introduce a USE MODD\_BLANK\_n
statement into that subroutine. Then he can use any of the variables
defined here and change them easily via the namelist input.



\subsection{Namelist NAM\_CH\_MNHCn\_PRE (init. chemistry scalar variables)}

If you initialize MNH-C using PREP\_IDEAL\_CASE,
use the following namelist variables:
\begin{center}
\begin{tabular} {|l|l|l|}
\hline
Fortran name & Fortran type & default value \\
\hline
LCH\_INIT\_FIELD  & logical       & FALSE      \\
CCHEM\_INPUT\_FILE & 80 characters & MNHC.input \\
\hline
\end{tabular}
\end{center}

\begin{itemize}

\item LCH\_INIT\_FIELD\index{LCH\_INIT\_FIELD!\innam{NAM\_CH\_MNHCn\_PRE}}: 
Flag to activate initialization subroutine CH\_INIT\_FIELD.

\item CCHEM\_INPUT\_FILE\index{CCHEM\_INPUT\_FILE!\innam{NAM\_CH\_MNHCn\_PRE}}:
 name of the general purpose input file for initialization.
 

\end{itemize}


\subsection{Namelist NAM\_CONFIO}
See section \ref{s:namconfio} page \pageref{s:namconfio} for details.


\subsection{Namelist NAM\_CONF\_PRE  (configuration variables)}
\index{NAM\_CONF\_PRE!namelist description}

\begin{center}
\begin{tabular} {|l|l|l|}
\hline
Fortran name & Fortran type & default value \\
\hline
LCARTESIAN & logical     & TRUE  \\
LPACK      & logical     & TRUE  \\
CEQNSYS    & 3 characters& 'DUR'   \\
NVERB      & integer     & 5     \\
CIDEAL     & 4 characters& 'CSTN'  \\
CZS        & 4 characters& FLAT  \\
LBOUSS     & logical     & FALSE \\
LPERTURB   & logical     & FALSE \\
LFORCING   & logical     & FALSE \\
LSHIFT     & logical     & FALSE \\
L2D\_ADV\_FRC  & logical     & FALSE \\
L2D\_REL\_FRC  & logical     & FALSE \\
NHALO      & integer        & 1      \\
JPHEXT     & integer        & 1      \\
LOCEAN     & logical        & FALSE \\
\hline
\end{tabular}
\end{center}

\begin{itemize}

\item  LCARTESIAN \index{LCARTESIAN!\innam{NAM\_CONF\_PRE}}
: Flag for cartesian geometry 
\begin{itemize}
\item .TRUE. for cartesian geometry
\item .FALSE. for conformal projection
\end{itemize}

\item LPACK \index{LPACK!\innam{NAM\_CONF\_PRE}}: Flag to compress FM file
for 1D or 2D version.

\item CEQNSYS \index{CEQNSYS!\innam{NAM\_CONF\_PRE}}: Equation system 
resolved by the MESONH model
\begin{itemize}
\item 'LHE' Lipps and HEmler anelastic system
\item 'DUR' approximated form of the DURran version of the anelastic sytem
\item 'MAE' classical Modified Anelastic Equations but with not any approximation 
in the momentum equation
\end{itemize}

\item  NVERB \index{NVERB!\innam{NAM\_CONF\_PRE}} : verbosity level
\begin{itemize}
\item  0 for minimum of prints
\item  5 for intermediate level of prints
\item  10 for maximum of prints.
\end{itemize}
If $CSURF$="EXTE" in
namelist NAM\_GRn\_PRE  \index{CSURF!\innam{NAM\_GRn\_PRE}}, NVERB=10
prints two \LaTeX \ files containing the initialisation of
surface scheme variables for each type of surface cover
(in french or in english).

\item CIDEAL  \index{CIDEAL!\innam{NAM\_CONF\_PRE}}: kind of idealized fields 
\begin{itemize}
\item 'CSTN' : Constant moist Brunt Vaisala frequency case 
\item 'RSOU' : radiosounding case
\end{itemize}


\item CZS  \index{CZS!\innam{NAM\_CONF\_PRE}} :  orography selector 
The formulae are given below in the description of the namelist  
NAM\_GRIDH\_PRE.
\begin{itemize}
\item 'FLAT' : constant XHMAX orography (zero by default)
\item 'SINE' : sine-shaped orography 
\item 'BELL' : bell-shaped orography
\item 'AGNE' : 	orography with h*a**2/(x**2+a**2) shape

\item 'DATA': discretized orography. The data describing the orography 
are given in the free format part. 
Only the orography corresponding
to the computational domain must be provided in free format. For 3D orography,
data are read like if it was a map (the first line is the Northern border and
the first data is the North-West corner) with one line per Y-axis increment.
\end{itemize}

\item LBOUSS  \index{LBOUSS!\innam{NAM\_CONF\_PRE}}: Flag for a Boussinesq version. 
\begin{itemize}
\item .TRUE. The reference anelastic state is  $\theta _{ref} =
cte = \theta _{ref} (z=0) $ and $ \rho _{ref} = cte = \rho _{ref}  (z=0) $. 
In this case, the stratification is taken into account in the Meso-NH model in
the flottability term. The typical length, on which this stratification
varies, is much greater than the domain heigth and the   $\theta _{ref}$
variation can be therefore neglected.
\item .FALSE. The reference anelastic state varies with the altitude.
\end{itemize}

\item LPERTURB  \index{LPERTURB!\innam{NAM\_CONF\_PRE}}: Flag to add a perturbation on the initially horizontally homogeneous fields.
This perturbation is not balanced. 

3 perturbation types are implemented in the routine {\it set\_perturb.f90} :
\begin{itemize}
\item 
a spherical perturbation  on the dry potential temperature  and the moisture 
fields (typical for convection initialization).
\item 
a  perturbation on the horizontal components of the wind derived
from a streamfunction (typical for large scale studies).
 This prevents the wind from becoming divergent. 
\item 
a  perturbation on the dry potential temperature field at the first mass level
near the ground, corresponding to a white noise (uniform amplitude in the
spectral space) (typical for Large Eddy Simulations initialization)  
\end{itemize}
When set to .TRUE., the parameters for the exact definition of the perturbation can
be set in the namelist NAM\_PERT\_PRE or sometimes can be
modified directly in the subroutine {\it set\_perturb.f90}

\item LFORCING  \index{LFORCING!\innam{NAM\_CONF\_PRE}}: Flag to 
specify forcing sources.
When .TRUE., the precise definition of the forcing is set in the free-format 
part of PRE\_IDEA1.nam (see \ref{ss:forced}). LFORCING must be then set to .TRUE. in EXSEG1.nam (NAM\_CONF).

\item LSHIFT  \index{LSHIFT!\innam{NAM\_CONF\_PRE}}: flag to shift altitudes in boundary layer. If LGEOSBAL=TRUE, LSHIFT will be set to FALSE.

\item L2D\_ADV\_FRC  \index{L2D\_ADV\_FRC!\innam{NAM\_CONF\_PRE}}: flag to activate advecting forcing (2D simulation only). When .TRUE., the precise definition of the advecting forcing is set in the free-format 
part of PRE\_IDEA1.nam (see \ref{ss:adv_forcing}).

\item L2D\_REL\_FRC  \index{L2D\_REL\_FRC!\innam{NAM\_CONF\_PRE}}: flag to activate relaxation forcing (2D simulation only). When .TRUE., the precise definition of the relaxation forcing is set in the free-format 
part of PRE\_IDEA1.nam (see \ref{ss:rel_forcing}).

 \item
NHALO: Size of the halo for parallel distribution.
This variable is related to computer performance but has no
impact on simulation results.\\
\index{NHALO!\innam{NAM\_CONF\_PRE}}

\item
JPHEXT:  Horizontal External points number\\
JPHEXT must be equal to 3 for cyclic cases with WENO5.
\index{JPHEXT!\innam{NAM\_CONF\_PRE}}

\item
LOCEAN: flag to activate the Ocean version of Meso-NH. Pronostic variables are: Current (U \& V), Vertical velocity (W), Temperature (TH), Subgrid Turbulent Kinetic Energy (TKE). Salinity (RV) can be activated with LUSERV=T. The Z-axis is directed upward (as in the atmosphere version), i.e. top of model domain corresponds to the sea surface. The initial profile must be defined in the free-format part (see Section \ref{ss:ocean_iniprofile}).
\index{LOCEAN!\innam{NAM\_CONF\_PRE}}

\end{itemize}


\subsection{Namelist NAM\_CONFn (configuration variables for
modeln)}
\index{NAM\_CONFn!namelist description}

\begin{center}
\begin{tabular} {|l|l|l|}
\hline
Fortran name & Fortran type & default value \\
\hline
LUSERV    & logical & TRUE  \\
LUSERC    & logical & FALSE  \\
LUSERI    & logical & FALSE  \\
NSV\_USER  & integer & 0   \\  
\hline
\end{tabular}
\end{center}

(see \thechapter.\thesection \ for more details for these cases)
\begin{itemize}

\item  LUSERV  \index{LUSERV!\innam{NAM\_CONFn}} : Flag to write
$r_{v}$ (vapor mixing ratio) in initial file. It is reset to .TRUE.
when CIDEAL ='RSOU' or 'CSTN'. This has been done in order to avoid
to treat the dry case as a particular case but as a moist case with
humidity equal to 0.

\item  LUSERC  \index{LUSERC!\innam{NAM\_CONFn}} : Flag to write
$r_{c}$ (cloud mixing ratio) in initial file. This case is only
allowed when CIDEAL ='RSOU' (radiosounding case) and KIND='PUVTHDMR'
or KIND='ZUVTHLMR'

\item  LUSERI  \index{LUSERI!\innam{NAM\_CONFn}} : Flag to write
$r_{i}$ (ice mixing ratio) in initial file. This case is only
allowed when CIDEAL ='RSOU' (radiosounding case) and KIND='PUVTHDMR'

\item NSV\_USER \index{NSV\_USER!\innam{NAM\_CONFn}}: number of
scalar variables Note that if NSV\_USER is different from 0, the
Scalar Variables are initialized to 0  by the program

\end{itemize}

\subsection{Namelist NAM\_CONFZ (configuration variables for
splitting along z)\label{s:namconfz}}
\index{NAM\_CONFZ!namelist description}

\begin{center}
\begin{tabular} {|l|l|l|}
\hline
Fortran name & Fortran type & default value \\
\hline
NZ\_VERB & integer & 0 \\
NZ\_PROC & integer & 0 \\
NB\_PROCIO\_R & integer & 1 \\
NB\_PROCIO\_W & integer & 1 \\
MPI\_BUFFER\_SIZE & integer & 40 \\
LMNH\_MPI\_BSEND  & logical & TRUE \\
LMNH\_MPI\_ALLTOALLV\_REMAP & logical & FALSE \\
NZ\_SPLITTING & integer & 10 \\
\hline
\end{tabular}
\end{center}

\begin{itemize} 
\index{NZ\_VERB!\innam{NAM\_CONFZ}}
\item NZ\_VERB: level of message for NZ solver and I/O  
              
\index{NZ\_PROC!\innam{NAM\_CONFZ}}
\item NZ\_PROC: number of processors to use in the Z splitting. The default value (0) yields an automatic calculation of the number.
             
\index{NB\_PROCIO\_R!\innam{NAM\_CONFZ}}
\item NB\_PROCIO\_R: number of processors to use for parallel I/O when reading file. The default value (1) yields a reading from 1 file only. If more than 1 file, the 3D field are written as several 2D slides.

\index{NB\_PROCIO\_W!\innam{NAM\_CONFZ}}
\item NB\_PROCIO\_W: Number of processors to use for parallel I/O when writing file. The default value (1) yields a writing into 1 file only. If more than 1 file, the 3D field are written as several 2D slides.
              
\index{MPI\_BUFFER\_SIZE!\innam{NAM\_CONFZ}}
\item MPI\_BUFFER\_SIZE: default size for MPI\_BSEND buffer in $10^6$ bytes. MPI\_BUFFER\_SIZE corresponds approximately to the size of the domain, that is, NX*NY*NZ for I/O in 1 file, and NX*NY for I/O in N 2D-slide files.

\index{LMNH\_MPI\_BSEND!\innam{NAM\_CONFZ}}
\item LMNH\_MPI\_BSEND:
during HALO exchange and FFT transposition, switch to use bufferized either MPI\_BSEND routine or asynchrone MPI\_ISEND routine. Depending on the computer and size of the problem, one or the other option could run faster. MPI\_ISEND also uses less memory so MPI BUFFER SIZE should be decreased. 

\index{LMNH\_MPI\_ALLTOALLV\_REMAP!\innam{NAM\_CONFZ}}
\item LMNH\_MPI\_ALLTOALLV\_REMAP: 
\begin{itemize} 
\item FALSE: FFT remap with send/recv $<=>$ NZ\_SPLITTING=10
\item TRUE: FFT remap with mpi\_alltoallv $<=>$ NZ\_SPLITTING=14 (BG/MPICH optimization) 
\end{itemize}

\index{NZ\_SPLITTING!\innam{NAM\_CONFZ}}
\item NZ\_SPLITTING: setting by namelist for debugging by expert user only. 
The non-expert user will use LMNH\_MPI\_ALLTOALLV\_REMAP=T/F only:
IZ=1=flat\_inv; IZ=2=flat\_invz; IZ=1+2=the two; +8=P1/P2.

\end{itemize}


\subsection{Namelist NAM\_DIMn\_PRE (contains  dimensions) }
\index{NAM\_DIMn\_PRE!namelist description}

\begin{center}
\begin{tabular} {|l|l|l|}
\hline
Fortran name & Fortran type & default value \\
\hline
NIMAX & integer & 10 \\
NJMAX & integer & 10 \\
\hline
\end{tabular}
\end{center}

\begin{itemize} 
\index{NIMAX!\innam{NAM\_DIMn\_PRE}}
\item NIMAX : number of mass points in x-direction of the 
initial file is $NIMAX +2 JPHEXT$ ( $JPHEXT$ corresponds to the number of
marginal points in the horizontal directions and is fixed to 1 for the present 
Meso-NH version ). NIMAX must be equal to $2^m \times 3^n \times 5^p$ with $(m,n,p) \ge 0$

\index{NJMAX!\innam{NAM\_DIMn\_PRE}}
\item NJMAX : number of mass points in y-direction of the 
physical domain. The total size of the array written in the initial
file is $NJMAX +2 JPHEXT$. NJMAX must be equal to $2^m \times 3^n \times 5^p$ with $(m,n,p) \ge 0$


\end{itemize}


\subsection{Namelist NAM\_DYNn\_PRE (pressure solver)} 
\index{NAM\_DYNn\_PRE!namelist description}

\begin{center}
\begin{tabular} {|l|l|l|}
\hline
Fortran name & Fortran type & default value \\
\hline
CPRESOPT    &  5 characters  & 'CRESI'  \\
NITR        &   integer      &  4      \\
XRELAX      &    real        &  1.   \\
LRES        & logical        & .FALSE. \\
XRES        & real           & 1.E-07  \\
\hline
\end{tabular}
\end{center}


\begin{itemize}

\item
CPRESOPT \index{CPRESOPT!\innam{NAM\_DYNn\_PRE}}:  gives the type of pressure solver used for
the elliptic equation ('RICHA', 'CGRAD', 'CRESI', 'ZRESI').
 This equation is solved in order to ensure the anelastic 
constraint for the initial wind field. Note that the solver is applied even for
the FLAT case when the Earth spericity is taken into account.

\item
NITR  \index{NITR!\innam{NAM\_DYNn\_PRE}}: number of  iterations used for the resolution of the elliptic equation (solver = "CPRESOPT").

\item
XRELAX \index{XRELAX!\innam{NAM\_DYNn\_PRE}} : relaxation factor used by the Richardson method (CPRESOPT = "RICHA").

\item LRES : flag to change the residual divergence limit
\index{LRES\innam{NAM\_DYNn\_PRE}}

\item XRES : Value of the residual divergence limit
\index{XRES\innam{NAM\_DYNn\_PRE}}

\end{itemize}  


\subsection{Namelist NAM\_GRID\_PRE (grid definition)}
\index{NAM\_GRID\_PRE!namelist description}


\begin{center}
\begin{tabular} {|l|l|l|}
\hline
Fortran name & Fortran type & default value \\
\hline
XLON0   & real & 0. \\
XLAT0   & real & 60. \\
XBETA   & real & 0. \\
XRPK    & real & 1. \\
XLONORI & real & 350.    \\
XLATORI & real & 37.    \\
\hline
\end{tabular}
\end{center}

\noindent Namelist not used if a PGD is used.

\begin{itemize}          
\item  XLON0 \index{XLON0!\innam{NAM\_GRID\_PRE}}: reference
longitude for conformal projection and cartesian plane 
(if LCARTESIAN =.TRUE. this
value can be usefull to compute local solar time)

\item XLAT0 \index{XLAT0!\innam{NAM\_GRID\_PRE}}: reference
latitude for conformal projection and cartesian plane

\item XBETA \index{XBETA!\innam{NAM\_GRID\_PRE}}: rotation angle
for conformal projection and cartesian plane

\item XRPK \index{XRPK!\innam{NAM\_GRID\_PRE}}: 
 cone factor for the projection (only if LCARTESIAN =.FALSE.):
\begin{itemize}
\item XRPK=1: polar stereographic projection from south pole
\item 1$>$XRPK$>$0: Lambert projection from south pole
\item XRPK=0: Mercator projection from earth center
\item -1$<$XRPK$<$0: Lambert projection from north pole
\item XRPK=-1: polar stereographic projection from north pole
\end{itemize}

\item XLONORI \index{XLONORI!\innam{NAM\_GRID\_PRE}}: Longitude (in degrees) 
of the origine point (not used if LCARTESIAN =.TRUE.). This point is the mass
point of conformal coordinates (x=0,y=0) of the Meso-NH grids (See annexe \ref{a:grid} for more details on the Meso-NH grids).

\item XLATORI \index{XLATORI!\innam{NAM\_GRID\_PRE}} : Latitude (in degrees)
 of the origine point (not used if LCARTESIAN =.TRUE.)

\end{itemize}


  
\subsection{Namelist NAM\_GRIDH\_PRE (horizontal grid definition)}
\index{NAM\_GRIDH\_PRE!namelist description}


\begin{center}
\begin{tabular} {|l|l|l|}
\hline
Fortran name & Fortran type & default value \\
\hline
XLATCEN   & real & XUNDEF    \\
XLONCEN   & real & XUNDEF     \\
XDELTAX   & real & 5000.  \\
XDELTAY   & real & 5000.  \\
XHMAX     & real &  300. / 0.  \\
NEXPX     & integer&  3      \\
NEXPY     & integer&  1      \\
XAX       & real &  10000.   \\
XAY       & real &  10000.   \\
NIZS      & integer & 5    \\
NJZS      & integer & 5  \\
\hline
\end{tabular}
\end{center}


\begin{itemize}          
\item XLATCEN \index{XLATCEN!\innam{NAM\_GRIDH\_PRE}} : latitude  of the center of the domain for initialization. This  point is vertical vorticity point (See annexe \ref{a:grid} for more details on the Meso-NH grids)

\item XLONCEN \index{XLONCEN!\innam{NAM\_GRIDH\_PRE}} :  longitude of the center of the domain for initialization. This  point is vertical vorticity point (See annexe \ref{a:grid} for more details on the Meso-NH grids)

\item XDELTAX \index{XDELTAX!\innam{NAM\_GRIDH\_PRE}} : mesh length (in meters)
 in x-direction on the conformal or cartesian plane. It is  not
used if you read information in a Meso-NH constant file (PGD\_FILE). 

\item XDELTAY \index{XDELTAY!\innam{NAM\_GRIDH\_PRE}} : mesh length (in meters)
 in y-direction on the conformal or cartesian plane. It is  not
used if you read information in a Meso-NH constant file (PGD\_FILE).

\item  XHMAX\footnote{default is 300. for mountain and 0 for flat orography}  
\index{XHMAX!\innam{NAM\_GRIDH\_PRE}}: Maximum height (in meters)
 $h_{max}$ for orography (case CZS $\neq$ 'FLAT') or ground level for
flat orography

\item   NEXPX  \index{NEXPX!\innam{NAM\_GRIDH\_PRE}}: Exponent $exp_{x}$ for 
 orography in case of CZS='SINE'

\item   NEXPY  \index{NEXPX!\innam{NAM\_GRIDH\_PRE}} : Exponent $exp_{y}$ 
for  orography in case of CZS='SINE'

\item   XAX  \index{XAX!\innam{NAM\_GRIDH\_PRE}}:  Widths (in meters)
 $a_{x}$ along x 
 for orography in case CZS='BELL'
$$ z_s \left( \hat{x} , \hat{y} \right) = { h_{max} \over  \left(
  1 +
 \left(  { \hat{x} - NIZS * XDELTAX \over XAX } \right) ^2 + 
 \left(  { \hat{y} - NJZS * XDELTAX \over XAY } \right) ^2 
                                                      \right) ^{1.5 } } $$ 
in the three-dimensional case.
$$ z_s \left( \hat{x}  \right) = {  h_{max} \over  
1 + \left(  { \hat{x} - NIZS * XDELTAX \over XAX } \right) ^2 } $$
in the two-dimensional case.

\item   XAY \index{XAY!\innam{NAM\_GRIDH\_PRE}} :  Width  (in meters)
$a_{y}$ along y  for orography in case CZS='BELL'

\item  NIZS \index{NIZS!\innam{NAM\_GRIDH\_PRE}}:  Localization in x-direction in the physical domain of the mountain center in 
the  case CZS ='BELL'. ($x_{s} = NIZS * XDELTAX$) It refers to a vertical
velocity point at the ground ( NIZS, NJZS )(See annexe \ref{a:grid} for more details on the Meso-NH grids)

\item   NJZS \index{NJZS!\innam{NAM\_GRIDH\_PRE}}: Localization in y-direction in the physical domain of the mountain center in 
 the     case CZS ='BELL'. ($y_{s} = NJZS * XDELTAY$)

\end{itemize}


\subsection{Namelist NAM\_GRn\_PRE (surface scheme choice)}
\index{NAM\_GRn\_PRE!namelist description}

\begin{center}
\begin{tabular} {|l|l|l|l|}
\hline
Fortran name & Fortran type & default value \\
\hline
\hline
CSURF    & 4 characters   &  "NONE"  \\
\hline
\end{tabular}
\end{center}

\begin{itemize}

\item
CSURF\index{CSURF!\innam{NAM\_GRn\_PRE}} : ground selector.

\begin{itemize}
\item 'NONE' no surface scheme will be activated during the future MesoNH
simulation, we therefore do not need any surface parameters. All the namelists of the externalized surface will be ignored.

\item 'EXTE' the externalized surface is used. See the SURFEX documentation for more details.
\end{itemize}  

\end{itemize}  

\subsection{Namelist NAM\_IBM\_LSF (LevelSet Funct. for Immersed Boundary Method)}
\index{NAM\_IBM\_LSF!namelist description}
\label{s:namibmlsf}
\begin{center}
\begin{tabular} {|l|l|l|}
\hline
Fortran name  & Fortran type  & default value \\
\hline
LIBM\_LSF     & logical       & .FALSE.       \\
CIBM\_TYPE    & 4 characters  & 'NONE'        \\
NIBM\_SMOOTH  & integer       & 1             \\
XIBM\_SMOOTH  & real          & 0.0001        \\
\hline
\end{tabular}
\end{center}

\begin{itemize}

\item LIBM\_LSF \index{LIBM\_LSF!\innam{NAM\_IBM\_LSF}} : Flag to
  calculate LevelSet Function (the minimum distance to the obstacles) or not.
\begin{itemize}
\item .TRUE.: The LevelSet Function is calculated.
\item .FALSE.: The LevelSet Function is not calculated.
\end{itemize}

\item CIBM\_TYPE \index{CIBM\_TYPE!\innam{NAM\_IBM\_LSF}} : The way the
  obstacles are described.
\begin{itemize}
\item 'NONE': No obstacles.
\item 'IDEA': Idealised ellipsoide or parallelepiped obstacles that
  are described in the additional namelist ibm\_idea.nam (see section \ref{s:namibmidea}).
\item 'GENE': Generic user defined obstacles in .obj format (see the description of the ibm\_gene.obj file below).
\end{itemize}

\item NIBM\_SMOOTH \index{NIBM\_SMOOTH!\innam{NAM\_IBM\_LSF}} : The
  number of iterations for smoothing the LevelSet Function. In the
  case a considerable smoothing shall be done, it is recommended to use NIBM\_SMOOTH=10.

\item XIBM\_SMOOTH \index{XIBM\_SMOOTH!\innam{NAM\_IBM\_LSF}} : The
  characteristic length scale used for smoothing the LevelSet
  Function. It is recommended to use a value of XIBM\_SMOOTH close to the grid size.

\end{itemize}

\underbar{The ibm\_gene.obj file}
\\
The ibm\_gene.obj file contains the information of the triangles constituting the faces of the obstacles. 
The .obj file must have a particular organization:
\begin{enumerate}
 \item A line with 'usemtl' indicates the two materials of each side of the interface. Only the faces with  their external face in contact with the outside air are read (mat2=air).
 \item A line starting with 'v' indicates the location (x,y,z coordinates) of a triangle vertex.
 \item A line starting with 'f' indicates the vertex constituting one triangle.
\end{enumerate}
\begin{verbatim}
usemtl mat1:mat2
v      xv1      yv1      zv1
v      xv2      yv2      zv2
v      xv3      yv3      zv3
v      xv4      yv4      zv4
f         1         2         3
f         1         3         4
usemtl mat3:mat2
v      xv5      yv5      zv5
v      xv6      yv6      zv6
v      xv7      yv7      zv7
v      xv8      yv8      zv8
f         5         6         7
f         5         7         8
\end{verbatim}
In the above example, the first triangle is formed by vertex numbers
1, 2, 3 and the second by vertex numbers 1, 3, 4. The two triangles form the interface mat1:mat2.
The interface mat3:mat2 is defined by two other triangles. The vertex number increment starts at the first vertex defined in the file.
\\There are numerous other ways to write an .obj file. For instance,
the normal or the texture of the faces can be defined. For the sake of
simplicity only material ('usemtl'), vertex coordinates ('v') and face
definition ('f') are considered in Meso-NH. The face normal is
computed directly in the code and the face texture is irrelevant
in the present version of the code.

\subsection{Namelist ibm\_idea.nam (Idealised obstacles definition)}
\index{ibm\_idea.nam!namelist description}
\label{s:namibmidea}
\begin{center}
\begin{tabular} {|l|l|l|}
\hline
Fortran name  & Fortran type  & default value \\
\hline
NOBJ          & integer       & none       \\
NOBJ\_TYPE    & integer       & none       \\
NTYPE         & integer       & none       \\
NOBJ\_ETYPE   & integer       & none       \\
XX1           & real          & none       \\
XX2           & real          & none       \\
XY1           & real          & none       \\
XY2           & real          & none       \\
XZ1           & real          & none       \\
XZ2           & real          & none       \\
\hline
\end{tabular}
\end{center}

\begin{itemize}

\item NOBJ  \index{NOBJ!\innam{ibm\_idea.nam}} : Total number of obstacles (parallelepiped and ellipsoide).

\item NOBJ\_TYPE  \index{NOBJ!\innam{ibm\_idea.nam}} : Number of obstacles type (parallelepiped and/or ellipsoide).

\item NTYPE  \index{NTYPE!\innam{ibm\_idea.nam}} : Obstacle type:
\begin{itemize}
\item 1: Parallelepiped.
\item 2: Ellipsoide.
\end{itemize}

\item NOBJ\_ETYPE  \index{NOBJ\_ETYPE!\innam{ibm\_idea.nam}} : Number of obstacles for each type.

\item XX1, XX2, XY1, XY2, XZ1, XZ2  \index{XX1!\innam{ibm\_idea.nam}} : Locations of the obstacles.
\begin{itemize}
 \item if type 1 (parallelepiped): I1/I2 = min/max in direction (X,Y,Z)
 \item if type 2 (ellipsoide): X1/Y1/Z1 locations of object center, X2/Y2/Z2 axe lenght in each direction
\end{itemize}
\end{itemize}

\underbar{Example (case of 4 parallelepipeds and 1 ellipsoide)}
\begin{verbatim}
5    2                                         # NOBJ  / NOBJ_TYPE
1    4                                         # NTYPE / NOBJ_ETYPE
+51.00 +53.42  +51.45  +63.65  -1.00  +2.54    # XX1/XX2/XY1/XY2/XZ1/XZ2
+51.00 +53.42  +71.55  +83.75  -1.00  +2.54    #
+51.00 +53.42  +91.65  +103.85  -1.00  +2.54   #
+51.00 +53.42  +111.75  +123.95  -1.00  +2.54  #
2    1                                         # NTYPE / NOBJ_ETYPE
+100. +10. +100. +10. -1. 10.                  # XX1/XX2/XY1/XY2/XZ1/XZ2
\end{verbatim}
In the case of obstacles in contact with the ground it is necessary to have negative value of XZ1.

\subsection{Namelist NAM\_LBCn\_PRE (lateral boundary conditions)} 
\index{NAM\_LBCn\_PRE!namelist description}

\begin{center}
\begin{tabular} {|l|l|l|}
\hline
Fortran name & Fortran type & default value \\
\hline
CLBCX    & array(2 characters)  &  2*"CYCL"  \\
CLBCY    & array(2 characters)  &  2*"CYCL"  \\
\hline
\end{tabular}
\end{center}

\begin{itemize}

\item
CLBCX  \index{CLBCX!\innam{NAM\_LBCn\_PRE}}: represent the type of lateral boundary
condition at the left and right boundaries along x (CLBCX(1) and CLBCX(2)
respectively). Possible values are "CYCL", "OPEN", "WALL" for cyclic, open and
rigid wall boundary conditions respectively. It should be note that CLBCX(1) or CLBCY(1)
refers to the lowest index values ( IIB , IJB for X and Y directions) and
CLBCX(2) or CLBCY(2) to the highest index values ( IIE  and IJE). Please note
that :  
$$CLBCo(1) ="CYCL" \Rightarrow CLBCo(2) ="CYCL" $$
  {\bf  The same boundary conditions must be used for the MESO-NH run itself} (see EXSEG1.nam namelist)
Note also that CYCLIC conditions are not possible  
with a PGD file (CPGD\_FILE different to '  ' in NAM\_REAL\_PGD).
\item
CLBCY  \index{CLBCY!\innam{NAM\_LBCn\_PRE}}: same as CLBCX but for the left and right boundaries along y (CLBCY(1) and CLBCY(2) respectively). They are strings of 4 characters.
\end{itemize}  


\subsection{Namelist NAM\_LUNITn (logical unit names) }
\index{NAM\_LUNITn!namelist description}

\begin{center}
\begin{tabular} {|l|l|l|}
\hline
Fortran name & Fortran type & default value \\
\hline
CINIFILE     & 28 characters & 'INIFILE'\\
CINIFILEPGD  & 28 characters & ' '\\
\hline
\end{tabular}
\end{center}

\begin{itemize}
\item CINIFILE \index{CINIFILE!\innam{NAM\_LUNITn}} : name of the initial FM-file  produced by PREP\_IDEAL\_CASE, it will then 
be used as initial file in a MESONH numerical simulation.
\item CINIFILEPGD \index{CINIFILEPGD!\innam{NAM\_LUNITn}} : name of the PGD file if CSURF$\neq$'NONE'. If you use an input PGD file for the step PREP\_IDEAL\_CASE (CPGD\_FILE in NAM\_REAL\_PGD), you must have CINIFILEPGD=CPGD\_FILE. If there is no input PGD, CINIFILEPGD is the name of the PGD file produced by PREP\_IDEAL\_CASE.
\end{itemize} 

\subsection{Namelist NAM\_PERT\_PRE (set analytical perturbations) }
\index{NAM\_PERT\_PRE!namelist description}

\begin{center}
\begin{tabular} {|l|l|l|}
\hline
Fortran name & Fortran type & default value \\
\hline
CPERT\_KIND         & characters & 'TH'  \\
XAMPLITH            & real       & 1.5   \\
XAMPLIRV            & real       & 0.0   \\
XAMPLIUV            & real       & 1.0834   \\
XAMPLIWH            & real       & 0.1   \\
NKWH                & integer    & 2    \\
LSET\_RHU           & logical    & TRUE \\
XCENTERZ            & real       & 2000. \\
XRADX               & real       & 10000. \\
XRADY               & real       & 10000. \\
XRADZ               & real       & 2000. \\
LWH\_LBXU           & logical    & FALSE \\
LWH\_LBYV           & logical    & FALSE \\
\hline
\end{tabular}
\end{center}

\begin{itemize}

\index{CPERT\_KIND !\innam{NAM\_PERT\_PRE}}
\item CPERT\_KIND: Defines the type of the perturbation
\begin{itemize}
\item 'TH' : thermodynamical fields perturbation ($\theta$ and $r_v$)
\item 'UV' : horizontal wind fields perturbation ($U$ and $V$)
\item 'WH' :  white noise applied to $\theta$
\item 'WW' :  white noise applied to wind components
\end{itemize}

\index{XAMPLITH!\innam{NAM\_PERT\_PRE}}
\item XAMPLITH: maximum perturbation for $\theta$

\index{XAMPLIRV!\innam{NAM\_PERT\_PRE}}
\item XAMPLIRV: maximum perturbation  for $r_v$

\index{XAMPLIUV!\innam{NAM\_PERT\_PRE}}
\item XAMPLIUV: maximum perturbation for $U$ and $V$

\index{XAMPLIWH!\innam{NAM\_PERT\_PRE}}
\item XAMPLIWH: maximum perturbation for the normalized white noise (temperature or wind)

\index{NKWH!\innam{NAM\_PERT\_PRE}}
\item NKWH: Upper level of the layer starting from the ground where the white noise is applied

\index{LSET\_RHU !\innam{NAM\_PERT\_PRE}}
\item LSET\_RHU: Conservation of the relative humidity
\begin{itemize}
\item TRUE the relative humidity is conserved in the $\theta$ perturbation
\item FALSE the $r_v$ perturbation is computed with the XAMPLIRV amplitude
\end{itemize}

\index{XCENTERZ!\innam{NAM\_PERT\_PRE}}
\item XCENTERZ: Height of the maximum of the $\theta$ perturbation (m)

\index{XRADX!\innam{NAM\_PERT\_PRE}}
\item XRADX: radius of the perturbation along X (m)

\index{XRADY!\innam{NAM\_PERT\_PRE}}
\item XRADY: radius of the perturbation along Y (m)

\index{XRADZ!\innam{NAM\_PERT\_PRE}}
\item XRADZ: radius of the perturbation along Z (m)

\index{LWH\_LBXU!\innam{NAM\_PERT\_PRE}}
\item LWH\_LBXU : White noise in inflow and outflow LBC of U

\index{LWH\_LBXV!\innam{NAM\_PERT\_PRE}}
\item LWH\_LBXV : White noise in inflow and outflow LBC of V	

\end{itemize}


\subsection{Namelist NAM\_REAL\_PGD (PGD file flags) }
\index{NAM\_REAL\_PGD!namelist description}

\begin{center}
\begin{tabular} {|l|l|l|}
\hline
Fortran name & Fortran type & default value \\
\hline
CPGD\_FILE          & characters & '   ' \\
LREAD\_ZS           & logical    & FALSE \\ 
LREAD\_GROUND\_PARAM & logical    & FALSE \\
\hline
\end{tabular}
\end{center}


\begin{itemize}
\index{CPGD\_FILE !\innam{NAM\_REAL\_PGD}}
\item CPGD\_FILE : name of the physiographic data file containing the ground data
                   fields. The file must be generated by the PRE\_PGD program.
{\bf For a purely ideal case, the CPGD\_FILE variable may be deleted from the
namelist or set to its default value '   '.}
\index{CPGD\_FILE!\innam{NAM\_REAL\_PGD}}
{\bf The horizontal grid will be read in the PGD file and therefore, the mesh
increments XDELTAX and XDELTAY are no more used}.

\item LREAD\_GROUND\_PARAM : Flag to use or not the surface cover types (COVERnnn)
                             and all other physiographic fields (except orographic ones)
                             read in the PGD file.
\begin{itemize}
\item .TRUE. to read the data in the PGD file 
\item .FALSE. to use XUNIF\_COVER idealized homogeneous values given in the
namelist NAM\_COVER (from the externalized surface) and scratch the PGD\_FILE  data
\end{itemize}
\index{LREAD\_GROUND\_PARAM!\innam{NAM\_REAL\_PGD}}


\item LREAD\_ZS : Flag to use or not the orography parameters 
                  read in the PGD file.
\begin{itemize}
\item .TRUE. to use the data read in the PGD\_FILE 
\item .FALSE. to use an idealized orography given in the
namelist NAM\_GRIDH\_PRE and scratch the PGD\_FILE  data
\end{itemize}
\index{LREAD\_ZS!\innam{NAM\_REAL\_PGD}}

\end{itemize}



\subsection{Namelist NAM\_SLEVE (smoothed orography for Sleve coordinate) }
\index{NAM\_SLEVE!namelist description}

\begin{center}
\begin{tabular} {|l|l|l|}
\hline
Fortran name & Fortran type & default value \\
\hline
NSLEVE  & integer & 12   \\  
XSMOOTH\_ZS & real & XUNDEF  \\
\hline
\end{tabular}
\end{center}

\begin{itemize}
\item NSLEVE \index{NSLEVE!\innam{NAM\_SLEVE}}:
 number of iteration for computation of smooth orography.
\item XSMOOTH\_ZS \index{XSMOOTH\_ZS!\innam{NAM\_SLEVE}}:
 optional uniform smooth orography.
\end{itemize}


  


            
\subsection{Namelist NAM\_VER\_GRID (contains vertical grid definition)}
\index{NAM\_VER\_GRID!namelist description}

There are three ways to compute the vertical grid, as in PREP\_REAL\_CASE:
\begin{enumerate}
\item
constant grid mesh: only the number of levels NKMAX and the grid mesh sizes 
ZDZGRD and ZDZTOP are used. ZDZGRD and ZDZTOP must have the same value. The type of grid 
YZGRID\_TYPE is set to 'FUNCTN'.
\item
two layers are defined, with constant stretching in each layer. The grid
mesh size is given near the ground and at top of the model. It is possible
that the top grid size is never reached, if the number of points is not enough
for the prescribed stretchings. The type of grid 
YZGRID\_TYPE is also set to 'FUNCTN'.
\item
the levels are given by the user. The type of grid YZGRID\_TYPE is set to
'MANUAL' in the namelist, and only the number of levels NKMAX is also used in it. 
\end{enumerate}

The variables of this namelist are:

\begin{center}
\begin{tabular} {|l|l|l|}
\hline
Fortran name & Fortran type & default value\\
\hline
\hline
LTHINSHELL     & logical    & .FALSE.    \\
NKMAX          & integer    & 10         \\
YZGRID\_TYPE   & 6 characters & 'FUNCTN' \\
ZDZGRD         & real & 300.  \\
ZDZTOP         & real & 300.  \\
ZZMAX\_STRGRD  & real & 0.    \\
ZSTRGRD        & real & 0.    \\
ZSTRTOP        & real & 0.    \\
LSLEVE         & logical & FALSE   \\
XLEN1          & real & 7500.   \\
XLEN2          & real & 2500.   \\
\hline
\end{tabular}
\end{center}

\begin{itemize}
\item LTHINSHELL : Flag for the thinshell approximation (logical)
\index{LTHINSHELL!\innam{NAM\_VER\_GRID}}
\item NKMAX : number of points in z-direction of the required 
              physical domain. The total size of the array written in initial
file will be $NKMAX +2 JPVEXT$ ($JPVEXT$ is fixed to 1 for the present version
 of Meso-NH) 
\index{NKMAX!\innam{NAM\_VER\_GRID}}

\item  YZGRID\_TYPE : type of vertical grid definition:
\index{YZGRID\_TYPE!\innam{NAM\_VER\_GRID}}
\begin{itemize}
\item 'FUNCTN': the vertical grid is given by a regular logarithmic function, 
whose variation is determined by the values of free parameters ZDZGRD, ZDZTOP,
ZSTRGRD, ZSTRTOP, ZZMAX\_STRGRD described below.
\item 'MANUAL': the levels are explicitly given in the free-formatted
part with the keyword \texttt{ZHAT} by entering the heights of the different levels from  K=2 to K= KMAX + 2 (see \ref{ZHAT}).
\end{itemize}  
 
\item  ZDZGRD : mesh length in z-direction near the ground
\index{ZDZGRD!\innam{NAM\_VER\_GRID}}

\item  ZDZTOP : mesh length in z-direction near the top of the model
\index{ZDZTOP!\innam{NAM\_VER\_GRID}}

\item  ZZMAX\_STRGRD : Altitude separating the two constant stretching layers
\index{ZZMAX\_STRGRD!\innam{NAM\_VER\_GRID}}

\item  ZSTRGRD : Constant imposed stretching (in \%) in the lower layer
(below ZZMAX\_STRGRD)
\index{ZSTRGRD!\innam{NAM\_VER\_GRID}}

\item  ZSTRTOP : Constant imposed stretching (in \%) in the upper layer
(above ZZMAX\_STRGRD)
\index{ZSTRTOP!\innam{NAM\_VER\_GRID}}

\item  LSLEVE \index{LSLEVE!\innam{NAM\_VER\_GRID}} : flag for Sleve 
vertical coordinate.
\item  XLEN1 \index{XLEN1!\innam{NAM\_VER\_GRID}} : decay scale for
smooth topography (in meters)
\item  XLEN2 \index{XLEN2!\innam{NAM\_VER\_GRID}} : decay scale for
smale-scale topography deviation (in meters)

\end{itemize}

            
\subsection{Namelist NAM\_VPROF\_PRE (variables for CIDEAL ='CSTN' or 'RSOU')}
\index{NAM\_VPROF\_PRE!namelist description}

\begin{center}
\begin{tabular} {|l|l|l|}
\hline
Fortran name & Fortran type & default value \\
\hline
LGEOSBAL    &  logical   & FALSE   \\
CFUNU       & 3 characters  & ZZZ  \\
CFUNV       & 3 characters  & ZZZ  \\
CTYPELOC     &  6 characters & IJGRID \\
XLATLOC  &  real   & 45.  \\
XLONLOC  &  real   &  0.  \\
XXHATLOC  &  real   &  20000.  \\
XYHATLOC  &  real   &  20000.  \\
NILOC     &  integer & 4    \\
NJLOC     &  integer & 4    \\
\hline
\end{tabular}
\end{center}


\begin{itemize}

\item LGEOSBAL \index{LGEOSBAL!\innam{NAM\_VPROFn\_PRE}} : Flag to fulfill the geostrophic balance or not
\begin{itemize}
\item .TRUE. the geostrophic balance is satisfied by the initial fields
\item .FALSE. the geostrophic balance is not satisfied by the initial fields
\end{itemize}


\item CFUNU  \index{CFUNU!\innam{NAM\_VPROFn\_PRE}}: String of 3 characters, describing the type of function, which 
gives the  x component of the wind. Possible configurations are listed below

\begin{itemize}

\item 
'ZZZ'    :  U = U(z). 
The  U(z) values are taken from the
Radio-Sounding or analitycal profile given in the free-formatted part of the 
PRE\_IDEA1.nam file.

\item 
'Y*Z'    : U= F(Y)*U(Z).  
The U(z) values are build in the same way as the
'ZZZ' case and the function F(Y) is a simple function of Y, which must be
adapted by modifying its definition directly in the routine FUNUY. The default
 function is :
$$  FUNUY(\hat{y}) = { 1 \over \cosh \left( 
   { \hat{y} - \hat{y_0} \over zwidth } \right) } $$

\item
 'Y,Z'    : U= G(Y,Z).  
The function  G must also be adapted  by modifying its definition directly in the
 routine FUNUYZ. The default function is :
$$  FUNUYZ(\hat{y},z) = { 1 \over \cosh \left( 
  \left( { \hat{y} - \hat{y_0} \over zwidthy } \right) ^2 +
  \left( { z - z_0 \over zwidthz } \right) ^2 
 \right) } $$
Notice that in this case the U(z) values given by the profile are not used. 
\end{itemize}

\item CFUNV  \index{CFUNV!\innam{NAM\_VPROFn\_PRE}}: String of 3 characters, describing the type of function, which 
gives the  y component of the wind. Possible configurations are listed below

\begin{itemize}

\item 
'ZZZ'    :  V = V(z). 
The  V(z) values are taken from the
Radio-Sounding or analitycal profile given in the free-formatted part of the 
PRE\_IDEA1.nam file.

\item 
'X*Z'    : V= F(X)*V(Z).  
The V(z) values are build in the same way as the
'ZZZ' case and the function F(X) is a simple function of X, which must be
adapted by modifying its definition directly in the routine FUNVX. The default
 function is :
$$  FUNVX(\hat{x}) = { 1 \over \cosh \left( 
   { \hat{x} - \hat{x_0} \over zwidth } \right) } $$

\item
 'X,Z'    : V= G(X,Z).  
The function  G must also be adapted  by modifying its definition directly in the
 routine FUNVXZ. The default function is :
$$  FUNVXZ(\hat{x},z) = { 1 \over \cosh \left( 
  \left( { \hat{x} - \hat{x_0} \over zwidthx } \right) ^2 +
  \left( { z - z_0 \over zwidthz } \right) ^2 
 \right) } $$
Notice that in this case the V(z) values given by the profile are not used.
\end{itemize}


\item 
CTYPELOC  \index{CTYPELOC!\innam{NAM\_VPROFn\_PRE}}: Type of information used to give the
                    localization of vertical profile (string of 6 characters)
\begin{itemize}
\item
  'IJGRID'  for (i,j) point  on index space
\item
 'XYHATM' for (x,y) coordinates on
conformal plane or cartesian plane
\item
 'LATLON' for (latitude,longitude) on   spherical earth  
\end{itemize}


\item XLATLOC \index{XLATLOC!\innam{NAM\_VPROFn\_PRE}}: Latitude (in degrees) of the vertical
                profile localization  (used in case  CTYPELOC='LATLON') 


\item
 XLONLOC\index{XLONLOC!\innam{NAM\_VPROFn\_PRE}} : Longitude (in degrees) of the vertical
                profile localization  (used in case  CTYPELOC='LATLON') 

\item
   XXHATLOC \index{XXHATLOC!\innam{NAM\_VPROFn\_PRE}}: position (in meters) x of the vertical profile
                    localization  (used in cases 
                                         CTYPELOC='XYHATM') 


\item
 XYHATLOC \index{XYHATLOC!\innam{NAM\_VPROFn\_PRE}}:  position (in meters) y of the vertical profile
                    localization  (used in cases 
                                         CTYPELOC='XYHATM') 

\item
 NILOC \index{NILOC!\innam{NAM\_VPROFn\_PRE}}: position i  in the physical domain  of the vertical profile localization   (used in cases 
                                         CTYPELOC='IJGRID') 
If you use a 1D model, then NILOC is reset to 1 by the program.
 
                                         
\item
 NJLOC \index{NJLOC!\innam{NAM\_VPROFn\_PRE}}:  position    j  in the physical domain of the vertical profile localization   (used in cases 
                                         CTYPELOC='IJGRID') 
If you use a 1D or a 2D  model, then NJLOC is reset to 1 by the program.

\end{itemize}

\section{Namelists for the externalized surface}
\subsection{Principles}

Further definition of surface parameters is not done by MESONH itself, but by the externalized surface included in it.
Three cases are encountered:
\begin{enumerate}

\item You do not have any PGD input file or externalized surface. You have a fixed surface state (e.g., the surface temperature does not evolve) so you just need to set
CSURF="NONE" in namelist NAM\_GRn\_PRE (default value).
\item You want to use the externalized surface (CSURF="EXTE" in namelist NAM\_GRn\_PRE) but you do not have any input PGD file or you do not want to use the surface fields included in it (LREAD\_GROUND\_PARAM = .FALSE.). Then, you must  define both the physiographic and prognostic fields, and you must fill the following namelists (see SURFEX documentation for details):

\begin{itemize}
\item NAM\_PGD\_SCHEMES
\item NAM\_COVER
\item NAM\_ISBA (if you chose to use the ISBA scheme).
\item NAM\_CH\_EMIS\_PGD
\item NAM\_DUMMY\_PGD
\item NAM\_PREP\_SURF\_ATM
\item NAM\_PREP\_SEAFLUX (if you chose to use the SEAFLX scheme)
\item NAM\_PREP\_WATFLUX (if you chose to use the WATFLX scheme)
\item NAM\_PREP\_TEB (if you chose to use the TEB urban scheme)
\item NAM\_PREP\_ISBA (if you chose to use the ISBA scheme)
\end{itemize}

You can choose to :
\begin{enumerate}
\item Use one or all the surface schemes. So you need to fill NAM\_PGD\_SCHEMES\index{NAM\_PGD\_SCHEMES!surfex namelist} CSEA='SEAFLUX' or/and
CNATURE='ISBA' or/and CWATER='WATFLUX' or/and CTOWN='TEB' and to set
NAM\_COVER\index{NAM\_COVER!surfex namelist} XUNIF\_COVER(i) where index "i" corresponds to the cover type, among those defined in
routine mode\_cover.f90 in SURFEX. Notice that no coherence
check is performed between CSEA, CNATURE, CWATER, CTOWN on one side and
the XUNIF\_COVER type you choose on the other side.
An example of namelist is given in the following part (Example 1).

\item Prescribe your own surface fluxes 
and surface state. Before MASDEV49 version, you had to fill them in the dedicated routine init\_ideal\_flux.f90
in SURFEX and to recompile the routine. From MASDEV49, you just have to fill them in the SURFEX namelist of 
EXSEG1.nam (only for the run) : NAM\_IDEAL\_FLUX\index{NAM\_IDEAL\_FLUX!surfex namelist} (see SURFEX user's guide).
You need to fill also NAM\_PGD\_SCHEMES CSEA='FLUX' or/and
CNATURE='FLUX' or/and CWATER='FLUX' or/and CTOWN='FLUX' /
according to the surface type you consider and to set
NAM\_COVER XUNIF\_COVER(i).\\
NAM\_PGD\_SCHEMES CSEA='FLUX' and NAM\_COVER XUNIF\_COVER(1)=1. is often met.
An example of namelist is given in the following part (Example 2).
\end{enumerate}
 
\item You want to use all the information contained in a PGD file. Only the prognostic variables must be defined, and the following namelists must be filled:

\begin{itemize}
\item NAM\_PREP\_SURF\_ATM\index{NAM\_PREP\_SURF\_ATM!surfex namelist}
\item NAM\_PREP\_SEAFLUX\index{NAM\_PREP\_SEAFLUX!surfex namelist} (if you chose to use the SEAFLX scheme)
\item NAM\_PREP\_WATFLUX\index{NAM\_PREP\_WATFLUX!surfex namelist} (if you chose to use the WATFLX scheme)
\item NAM\_PREP\_TEB\index{NAM\_PREP\_TEB!surfex namelist} (if you chose to use the TEB urban scheme)
\item NAM\_PREP\_ISBA\index{NAM\_PREP\_ISBA!surfex namelist} (if you chose to use the ISBA scheme)
\end{itemize}
An example of namelist is given in the following (Example 3).

\end{enumerate}

Note that orography either comes from :
\begin{itemize}
\item the input PGD file (if any and if LREAD\_ZS =.TRUE.). In this case, the atmospheric orography is also set equal to the one in this input PGD file.
\item or from the orography you have defined from the MESONH namelists (in this case, the surface orography is forced to be equal to the atmosphere orography).
\end{itemize}

\subsection{Examples  :}
\begin{verbatim}

Example 1 : You do not want to use a PGD file but you want to use a surface scheme, 
without prescribed fluxes :
&NAM_REAL_PGD /
&NAM_DIMn_PRE NIMAX=20, NJMAX=20 /
&NAM_VER_GRID NKMAX=36,YZGRID_TYPE='MANUAL'  /
&NAM_CONF_PRE LCARTESIAN=.TRUE., NVERB=10,
              CIDEAL='RSOU',  CZS='FLAT', LFORCING=.FALSE., LPACK=.FALSE.,
              LBOUSS=.FALSE., CEQNSYS='DUR',
              LPERTURB=.FALSE.  /
&NAM_PERT_PRE /
&NAM_CONFn LUSERV=.TRUE. /
&NAM_GRID_PRE XLAT0=35.762 / 
&NAM_GRIDH_PRE XDELTAX=500.,XDELTAY=500.  /
&NAM_LUNITn CINIFILE='IDEA_ISBA' , CINIFILEPGD='IDEA_ISBA_PGD' /
&NAM_PREP_ISBA XHUG_SURF=0., XHUG_ROOT=0.2, XHUG_DEEP=0.2,
               XTG_SURF=293., XTG_ROOT=293., XTG_DEEP=293. / 
&NAM_POST_PRE /
&NAM_DYNn_PRE  /
&NAM_LBCn_PRE / 
&NAM_VPROF_PRE LGEOSBAL = .FALSE., CTYPELOC='IJGRID', NILOC=2, NJLOC=2 /
&NAM_GRn_PRE CSURF='EXTE'/
&NAM_CH_MNHCn_PRE /
&NAM_BLANKn /
&NAM_PGD_SCHEMES CNATURE='ISBA' /
&NAM_ISBA XUNIF_CLAY = 0.3, XUNIF_SAND = 0.3 /
&NAM_COVER XUNIF_COVER(208)=1. /   


Example 2 : You do not want to use a PGD file and you want to prescribed your own fluxes 
(case ARM) :
&NAM_REAL_PGD /
&NAM_DIMn_PRE NIMAX=1, NJMAX=1 /
&NAM_CONF_PRE LCARTESIAN=.TRUE., NVERB=10,
              CIDEAL='RSOU',  CZS='FLAT', LFORCING=.TRUE., LPACK=.FALSE.,
              LBOUSS=.FALSE., CEQNSYS='DUR', LPERTURB=.FALSE.  /
&NAM_PERT_PRE /
&NAM_CONFn LUSERV=.TRUE. /
&NAM_GRID_PRE XLAT0=35.762 / 
&NAM_GRIDH_PRE XDELTAX=40000., XDELTAY=40000. /
&NAM_VER_GRID  LTHINSHELL=.TRUE., NKMAX=100, ZDZGRD=40., ZDZTOP=40.,
               ZZMAX_STRGRD=1000. , ZSTRGRD=0., ZSTRTOP=0. /
&NAM_LUNITn CINIFILE='eurocs',CINIFILEPGD='eurocs_PGD' /
&NAM_POST_PRE /
&NAM_DYNn_PRE  /
&NAM_LBCn_PRE CLBCX=2*"CYCL", CLBCY=2*"CYCL" / 
&NAM_VPROF_PRE /
&NAM_GRn_PRE CSURF='EXTE'/
&NAM_CH_MNHCn_PRE /
&NAM_BLANKn /
&NAM_PGD_SCHEMES CSEA='FLUX  ' /
&NAM_COVER XUNIF_COVER(1)=1. /

Example 3 : You want to use a PGD file and surface schemes :
&NAM_DIMn_PRE NIMAX=40, NJMAX=40 /
&NAM_VER_GRID NKMAX=36, YZGRID_TYPE='MANUAL' /
&NAM_CONF_PRE LCARTESIAN=.FALSE., CIDEAL='RSOU',LBOUSS=.FALSE., LPERTURB=.FALSE. ,
	      CEQNSYS='DUR',NVERB=10 /
&NAM_GRn_PRE CSURF='EXTE' /
&NAM_REAL_PGD CPGD_FILE='PGD_CORSE', LREAD_ZS=.TRUE., 
              LREAD_GROUND_PARAM=.TRUE. /
&NAM_PGD_SCHEMES CNATURE='ISBA', CSEA='NONE',CWATER='WATFLX',CTOWN='TEB' /
&NAM_PREP_SURF_ATM NYEAR=2007, NMONTH=07, NDAY=26, XTIME=54000. /
&NAM_PREP_WATFLUX XTS_WATER_UNIF = 293. /
&NAM_PREP_ISBA XHUG_SURF=0.2, XHUG_ROOT=0.2, XHUG_DEEP=0.2, XTG_SURF=293.,
               XTG_ROOT=293., XTG_DEEP=293. / 
&NAM_PREP_TEB XWS_ROAD=0., XWS_ROOF=0., XTS_ROAD=309.,
              XTS_ROOF= 298., XTS_WALL=298.,XTI_BLD=298., XTI_ROAD=298. /
&NAM_CONFn /              
&NAM_GRIDH_PRE XDELTAX=250.,XDELTAY=250.  / 
&NAM_LUNITn CINIFILE='IDEA_CORSE',CINIFILEPGD='PGD_CORSE' /   
&NAM_DYNn_PRE CPRESOPT= 'RICHA' NITR=10 XRELAX=1. /
&NAM_LBCn_PRE CLBCX= 2*'OPEN' CLBCY= 2*'OPEN' /
&NAM_VPROF_PRE CTYPELOC='IJGRID', NILOC=20, NJLOC=20,LGEOSBAL=.FALSE. /
\end{verbatim}
            


\section{ Free-format part }

{\bf Each section of the free format part must be introduced by its corresponding
keyword (writen on a separated line)}


There is always a moist variable written in PRE\_IDEA1.nam file, even in idealized dry
cases, for which  the moist variable should be equal to zero in the PRE\_IDEA1.nam file.
The produced initial file will always contain a moist variable  in 'CSTN'
and 'RSOU' cases.
 

\subsection{Optional Vertical grid : }
\label{ZHAT}
keyword: {\bf ZHAT}

If the vertical grid generation selector CZGRID\_TYPE is equal to 'MANUAL',  
you must enter at the end of your namelist file, the heights of the vertical
velocity levels. You must start from the ground level (K=2) to the model top
(K=KMAX +2), thus you only have to enter  KMAX + 1 values, because the level below the
ground (i.e. K=1) is at the same distance from the ground ( K=2 ) as the first
level above the ground ( K=3 ). Note also that the K= KMAX + 2 level represents
the model top. In this case the free parameters (ZDZGRD, ZDZTOP, ZSTRGRD, ...)
are not used


\subsection{Radiosounding case :}

keyword: {\bf RSOU}

The radiosounding data are written in 
the free-format part of PRE\_IDEA1.nam file, where 
the  altitude variable is  :


\begin{itemize}
\item the pressure in case KIND='STANDARD' or '{\bf P}UVTHVMR' or 
'{\bf P}UVTHVHU' or '{\bf P}UVTHDHU' or '{\bf P}UVTHDMR' (real, in Pascal)
\item the height in case '{\bf Z}UVTHVMR' or 
'{\bf Z}UVTHVHU'or '{\bf Z}UVTHDMR' or '{\bf Z}UVTHLMR' (real, in meters)
\end{itemize}


The first wind  variable is  :


\begin{itemize}
\item the wind direction in case KIND='STANDARD' (real,in degrees)
\item the zonal wind in cases KIND='P{\bf UV}THVMR' or 'P{\bf UV}THDMR' or 
'Z{\bf UV}THDMR' or 'Z{\bf UV}THLMR'or 'Z{\bf UV}THVHU' or 'P{\bf UV}THDHU' or 'Z{\bf UV}THVMR' or
'P{\bf UV}THVHU' (real, in m/s)
\end{itemize}


The second wind  variable is  :


\begin{itemize}
\item the wind force in case KIND='STANDARD' (real, in m/s)
\item the meridian wind in cases KIND='P{\bf UV}THVMR' or 'P{\bf UV}THDMR' or 
'Z{\bf UV}THDMR' or 'Z{\bf UV}THLMR' or 'Z{\bf UV}THVHU' or 'P{\bf UV}THDHU' or 'Z{\bf UV}THVMR' or
'P{\bf UV}THVHU' (real, in m/s)
\end{itemize}


The temperature variable is  :


\begin{itemize}
\item the temperature in case KIND='STANDARD' (real, in Kelvin)
\item the virtual potential
temperature in cases KIND='PUV{\bf THV}MR' or 'PUV{\bf THV}HU' or
'ZUV{\bf THV}MR' or 'ZUV{\bf THV}HU' (real,in Kelvin)
\item  the dry potential temperature in cases
KIND='PUV{\bf THD}MR' or 'PUV{\bf THD}HU' or \\
'ZUV{\bf THD}MR' (real, in Kelvin)
\item the liquid potential temperature in case KIND='ZUV{\bf THL}MR'(real, in Kelvin)
\end{itemize}


The moist variable is  :

\begin{itemize}
\item the dew point temperature in case KIND='STANDARD' (real, in Kelvin) 
\item the vapor mixing ratio
 in cases KIND='PUVTHV{\bf MR}' or 'ZUVTHD{\bf MR}' or 
'ZUVTHV{\bf MR}' or 'PUVTHD{\bf MR}' (real, in Kg/Kg) 
\item the total water  mixing ratio  in case KIND= 'ZUVTHLMR (real, in Kg/Kg)
\item  the relative humidity in cases KIND= 'ZUVTHV{\bf HU}', or 'PUVTHD{\bf HU}' or 'PUVTHV{\bf HU}' (real, in
percents)
\end{itemize}

Additional cloud variables

For the moment, this configuration works only for  KIND='PUVTHDMR' or 'ZUVTHDMR'
and  L1D=.TRUE.. It is planned to compute radiation
diagnostics with the {\bf DIAG} program (see chapter \ref{ch:diag}).

\begin{itemize}
\item
cloud mixing ratio if LUSERC=T or LUSERI=T (real, in Kg/Kg) 
\item
ice  mixing ratio if  LUSERI=T (real, in Kg/Kg) 
\end{itemize}

\vspace{0.5cm}

{\bf You should make sure that the levels are dense enough so that the Laplace
relation, which gives the thickness between successive levels, can be applied.}
The radiosounding information is written in the file  in the following order :


\begin{itemize}
\item YEAR (integer, exemple : 1994), MONTH (integer, exemple : 4), DAY
(integer, exemple : 22), TIME (real, in seconds, exemple : 36000 for 10 h)

\item KIND of data used for the radiosounding (string of 8 charcaters)
Nine kind of data are possible : 'STANDARD', 'PUVTHVMR', 'PUVTHVHU',
'ZUVTHVMR', 'ZUVTHVHU', 'PUVTHDMR', 'PUVTHDHU', 'ZUVTHDMR', 'ZUVTHLMR'.

Except for the STANDARD kind :  \begin{itemize}
\item  the first letter of KIND represents the kind of altitude
variable (P for pressure and Z for height), 
\item the second and third letters represent  the kind of wind
variables (U for zonal wind, V for meridian  wind), 
\item the fourth, fifth and sixth letters represent  the kind of
temperature variable (THV for virtual potential temperature, THD
for dry potential temperature and THL for liquid potential
temperature), 
\item the seventh and eighth letters represent  the kind of moist
variable (HU for relative humidity and MR for vapor mixing ratio). 
\end{itemize}

(In case of STANDARD kind, the altitude
variable is the pressure, the wind variables are direction and wind force,
the temperature variable is the temperature and the moist variable is the due
point temperature. )

\item HEIGHT of GROUND LEVEL (real, in meters)
\item PRESSURE at GROUND LEVEL (real,in Pascal)
\item a TEMPERATURE variable  at GROUND LEVEL (real, in Kelvin) 
\item a MOIST variable  at GROUND LEVEL

\item NUMBER of WIND data LEVELS (integer)

\item level 1 :  ALTITUDE variable ,  first WIND variable,  second WIND variable
at wind level 1 (the lowest wind-level).

\item level 2 :  ALTITUDE variable, first WIND variable,  second WIND variable.

\hspace{5cm}  \vdots

\hspace{5cm}  \vdots


\item  uppermost wind level : ALTITUDE variable,  first WIND variable, 
second WIND variable.

\item NUMBER of mass data LEVELS (integer) {\bf Note  that this number includes
the ground level (i.e. the first level).} That is why the following list starts
at level 2. 

\item level 2 :  ALTITUDE variable, TEMPERATURE variable,  MOIST
variable, additional cloud variable(s)  (the mass level 1 is at
ground).

\item level 3 : ALTITUDE variable,  TEMPERATURE variable,  MOIST
variable, additional cloud variable(s) .

\hspace{5cm}  \vdots

\hspace{5cm}  \vdots

\item uppermost mass level:  ALTITUDE variable, TEMPERATURE
variable, MOIST variable, additional cloud variable(s) 
\end{itemize}

{\bf You should make sure that the highest level of the radiosounding is located above the highest vertical level of the model.}


\vspace{0.5cm}

\underbar{Example of free part of PRE\_IDEA1.nam}
\begin{verbatim}
RSOU
1990 10 3 72000.
'STANDARD'
200.
100240.
287.5
276.
2
85000. 20. 10.
70000. 30. 10.
3
90000. 280. 275.
60000. 271. 269.
\end{verbatim}


\subsection{Constant moist Brunt-Vaisala case :}

keyword: {\bf CSTN}

Data of the vertical profile are written in 
the free-format part of PRE\_IDEA1.nam file in the following order :

\begin{itemize}
\item YEAR (integer, example : 1994), MONTH (integer, example : 4), DAY
(integer, example : 22), TIME (real, in seconds, example : 36000. for 10 h)

\item NUMBER of LEVELS (integer)
\item VIRTUAL POTENTIAL TEMPERATURE  at GROUND LEVEL 
({\it i.e} at the first level) (real,in Kelvin)
\item PRESSURE  at GROUND LEVEL ({\it i.e} at the first level) (real, in Pascal)
\item HEIGHT  at all levels. {\bf the first level is the  ground level}
\item ZONAL WIND COMPONENT at all  levels (the first level is the  ground level)
\item MERIDIAN WIND COMPONENT at all  levels (the first level is the  ground level)
\item RELATIVE HUMIDITY at all  levels (the first level is the  ground level)
\item MOIST BRUNT VAISALA FREQUENCY at all layers (the number of layers is the
number of levels - 1)
\end{itemize}

In this case, the level number can even be equal to 1, because the profile
information is linearly interpolated on the model grid without orography
(wind components, $\theta _v$ and humidity) before the application of the Laplace
relation to deduce the pressure and the vapor mixing ratio. Thus, the
layers' thicknesses are never too large to invalidate the Laplace relation.

\vspace{0.5cm}

\underbar{Example of free part of PRE\_IDEA1.nam}
\begin{verbatim}
CSTN
2006 06 06 21600.
5
287.5
100240.
200. 1000. 1500. 3000. 4000. 
10. 20. 25. 30. 35.
2. 10. 12.5 11.5 15.
80. 84. 85. 79. 87.
0.01 0.014 0.015 0.016
\end{verbatim}


\subsection{The forced version} \label{ss:forced}

keyword: {\bf ZFRC} or {\bf PFRC}

For idealized simulations a forced mode can be useful to impose the effects of
a simplified large scale environment to the model solution. This functionality
works (LFORCING=.TRUE. in module MODD\_CONF) when CIDEAL='RSOU' or 'CSTN'
(see 5.2.10 and 5.3) and only in the case 
LGEOSBAL =.FALSE. for inclusion of a geostrophic wind forcing.
All forcing fields are  issued from spatial interpolation of chronological series of 1D data (provided by the user onto the model grid). They are prepared during the
{\bf prep\_ideal\_case} sequence and are stored in the LFI files for further use
in case of RESTART model run.

The forcing fields can be time dependent. Application of the forcing begins as
soon as the date and time of the first set of forcing field given by the
user, is lower or equal to the current date and time of the model run. The
forcing action of the last forcing field is remanant, this is a way to impose
a stationnary forcing. When the current date and time of the model run is
bounded by two successive forcing fields, a simple linear interpolation in time
is made.

Note that an available Newtonian relaxation forcing type on $[u, v]$ and/or
$[\theta$, $r_v]$ is exclusive from the other physical forcings.

The forcing information and soundings have to be added at the end of the
free-format part already written for CIDEAL='CSTN' or 'RSOU'. First, the
type of forcing and the number of time dependent forcing are given:

\begin{itemize}
\item keyword forcing type (character*4)
\begin{itemize}
\item ZFRC means that the altitude of the forcing data are in height scale
(meters).
\item PFRC means that the altitude of the forcing data are in pressure scale
(Pascal).
\end{itemize}

\item number of time dependent forcing (integer)
\end{itemize}

The 1D forcing data are different from the one used to initialize the model
because specific data have to be entered. The data used to define each forcing
are given sequentially in the following order (one item per line):

\begin{itemize}
\item date and time of the forcing in the format:

year (integer),

month (integer),

day (integer) and

time of the day (real, s).
\item ground height (real, m)
\item ground pressure (real, Pa) 
\item $\theta_d$ (real, K) at ground level (Nota: it is used later in
the code to compute - if asked - a time varying sea surface temperature).
\item $r_v$ (real, kg/kg) at ground level
\item number of level (integer)
\item height of level1 (real, m) if ZFRC or pressure at level1 (real, Pa) if
PFRC,

$u_{frc}$ component at level1 (real, m/s),

$v_{frc}$ component at level1 (real, m/s),

$\theta_{frc}$ at level1 (real, K),

$r_{v\ frc}$ at level1 (real, kg/kg),

$w_{frc}$ at level1 (real, m/s),

$(\partial\theta / \partial t)_{frc}$ at level1 (real, K/s) and

$(\partial r_v/ \partial t)_{frc}$ at level1 (real, 1/s).

$(\partial u/ \partial t)_{frc}$ at level1 (real, m/$s^2$).

$(\partial v/ \partial t)_{frc}$ at level1 (real, m/$s^2$).

\item idem at level2
\item   ...
\item idem at levelN
\end{itemize}

If PFRC is the forcing type, an additional sounding is given in order to
convert the pressure levels into height levels with enough accuracy. Data
are organized as follows:

\begin{itemize}
\item number of level (integer)
\item pressure at level1 (real, Pa),

$\theta$ at level1 (real, K) and

$r_{v}$ at level1 (real, kg/kg).
\end{itemize}

This operation is repeated until the previous number of sounding is reached.

\underbar{Example of free part of PRE\_IDEA1.nam}
\begin{verbatim}
ZFRC
   1
1983 07 01 0.
0
1000000
284.5
 .008
 6
   5.   -7.0   0.0  281.10   0.00540     -0.00000   0. 0. 0. 0.
  15.   -7.0   0.0  281.10   0.00540     -0.00000   0. 0. 0. 0.
1095.   -7.0   0.0  280.75   0.00540     -0.00300   0. 0. 0. 0.
1145.   -7.0   0.0  290.60   0.00190     -0.00300   0. 0. 0. 0.
3000.   -7.0   0.0  304.15   0.00190     -0.00300   0. 0. 0. 0.
9000.   -7.0   0.0  346.15   0.00190     -0.00300   0. 0. 0. 0.
\end{verbatim}

\subsection{The advective forcing} \label{ss:adv_forcing}

keyword: {\bf ZFRC\_ADV} 

For 2D idealized simulation, an advective forcing can be used to impose effects to the model solution. This functionality works (L2D\_ADV\_FRC=.TRUE. in MODD\_CONF) only in 2D cases. The advecting forcings mimic the latidudinal humidity and temperature advection not taken into account in a 2D model.
         
The forcing information and soundings have to be added at the end of the
free-format part already written for CIDEAL='CSTN' or 'RSOU'. They are set in the following order : 
\begin{itemize}
\item ZFRC\_ADV : keyword for advective forcing
\item number of forcing files
\item type of forcing : ZADV2D for Z levels or PADV2D for  pressure levels
\item number vertical levels for the file
\item Date of first forcing : YYYY MM DD T (secondes)
\item name of the file with horizontal mean profile of theta, rv
\item name of the advective forcing file
\end{itemize}

                                   
\underbar{Example of free part of PRE\_IDEA1.nam}
\begin{verbatim}
ZFRC_ADV
1
ZADV2D
52
1997 07 15 00000.
"mean_atm_07.dat"
"frc_ideal_7_70km.dat"
\end{verbatim}

\subsection{The relaxation forcing} \label{ss:rel_forcing}

keyword: {\bf ZFRC\_REL} 

For 2D idealized simulation, a relaxation forcing can be used to impose effects to the model solution. This functionality works (L2D\_REL\_FRC=.TRUE. in MODD\_CONF) only in 2D cases.  The relaxation forcing allows the relax the model fields towards a 2D climatology for temperature and humidity.
         
The forcing information and soundings have to be added at the end of the
free-format part already written for CIDEAL='CSTN' or 'RSOU'. They are set in the following order : 
\begin{itemize}
\item ZFRC\_REL : keyword for advective forcing
\item number of forcing files
\item type of forcing : ZREL2D for Z levels or PREL2D for  pressure levels
\item number vertical levels for the file
\item Date of first forcing : YYYY MM DD T (secondes)
\item name of the file with horizontal mean profile of theta, rv
\item name of the advective forcing file
\end{itemize}

                                   
\underbar{Example of free part of PRE\_IDEA1.nam}
\begin{verbatim}
ZFRC_REL
1
ZREL2D
52
1997 07 15 00000.
"mean_atm_07.dat"
"frc_ideal_7_70km.dat"
\end{verbatim}

\subsection{Discretized orography}

keyword: {\bf ZSDATA}

Only the orography corresponding to the computational domain must be
provided in the free format part. For 3D orography, data are read like if it
was a map (the first line is the Northern border and the first data
is the North-West corner) with one line per Y-axis increment.

\underbar{Example of free part of PRE\_IDEA1.nam}
\begin{verbatim}
ZSDATA
30.    30.    35.     50.     30.     30.
30.    59.5  133.3   100.2   136.7   100.
35.    89.5  183.3   200.2   299.7   170.5
50.    112.5 193.0   210.2   206.7   120.
40.    82.5  153.0   180.5   156.7   100.3
\end{verbatim}

\subsection{The ocean version} \label{ss:ocean_iniprofile}

keyworkd: {\bf RSOU}

For oceanic version of Meso-NH, the initial and forcing profiles of the ocean are written in the free-format part of PRE\_IDEA1.nam file, where
the altitude variable is the depth. To follow usual convention for ocean data, the 1D profiles are given starting from the surface (positive value).
The profile and forcing information are written in the following order :
\begin{itemize}
\item YEAR (integer, exemple : 1994), MONTH (integer, exemple : 4), DAY (integer, exemple
: 22), TIME (real, in seconds, exemple : 36000 for 10 h)
\item KIND of data used for the profile. Two kind are possible : KIND='IDEALOCE' (data written in PRE\_IDEA1.nam) or KIND='STANDOCE' (data written in a NetCDF file). The following format is valid for KIND='IDEALOCE'.
\item ATMOSPHERIC PRESSURE at the surface which is the top domain of the oceanic model (real, in Pascal)
\item SEA SURFACE TEMPERATURE at the surface (real, in Kelvin)
\item SEA SURFACE SALINITY at the surface (real, in g/kg)
\item NUMBER of SEA CURRENT levels (integer)
\item level 1 : DEPTH HEIGHT variable (real, meters) , U-CURRENT (real, m/s), V-CURRENT at CURRENT level (real, m/s)
\item level 2 : DEPTH HEIGHT variable (real, meters) , U-CURRENT (real, m/s), V-CURRENT at CURRENT level (real, m/s)\\
  \vdots

\item NUMBER of mass data LEVELS (integer).
\item level 2 : DEPTH HEIGHT variable (real, meters), WATER TEMPERATURE (real, Kelvin), SALINITY at mass level (real, g/kg)
\item level 3 : DEPTH HEIGHT variable (real, meters), WATER TEMPERATURE (real, Kelvin), SALINITY at mass level (real, g/kg)\\
 \vdots

\item NUMBER of time-varying FORCING (integer). The data used to define each forcing are given sequentially in the
following order (one item per line):
\item YEAR (integer, exemple : 1994), MONTH (integer, exemple : 4), DAY (integer, exemple
: 22), TIME (real, in seconds, exemple : 36000 for 10 h)
\item U-STRESS  (real, $m^2/s^2$)
\item V-STRESS  (real, $m^2/s^2$)
\item HEAT TURBULENT FLUX (real, $W/m^2$)
\item RADIATIVE FLUX (real, $W/m^2$)
\end{itemize}
Surface fluxes are positive when going upward (from the ocean to the atmosphere)  \\

In the KIND='STANDOCE': initial 1D profiles and surface fluxes(t) are read from 2 netcdf files (See set\_rsou.f90 for details)
 

\section{Example of PRE\_IDEA1.nam : }
The selected case is the following:
\begin{itemize}
\item 2D mountain 
\item  one moist  layer atmosphere
\end{itemize}

\underline{FILE PRE\_IDEA1.nam}

\begin{verbatim}
&NAM_DIMn_PRE  NIMAX=128, NJMAX=1 /
&NAM_VER_GRID  NKMAX=32, YZGRID_TYPE = 'FUNCTN', ZDZGRD=500., ZDZTOP=500., 
               ZZMAX_STRGRD=1000.    , ZSTRGRD=0., ZSTRTOP= 0., 
&NAM_CONFn     LUSERV=.TRUE., NSV_USER = 0 /
&NAM_GRID_PRE  XLAT0 = 48.25 , XLON0 = 0.,
               XRPK  = 0.    , XBETA = 0.,
               XLONORI = 48.25, XLATORI = 0. /
&NAM_CONF_PRE  LCARTESIAN=.TRUE., LBOUSS=.FALSE., 
               CIDEAL='CSTN', CZS='BELL', 
               LPERTURB= .FALSE., NVERB=1 /
&NAM_GRIDH_PRE XDELTAX=5.E2 , XDELTAY=5.E2, 
               XHMAX=500., XAX=10.E3, XAY=10.E3, NIZS=64, NJZS=2, 
               NEXPX = 1, NEXPY=1 /
&NAM_LUNITn    CINIFILE='HYD2D',CINIFILEPGD='HYD2D_PGD' /
&NAM_DYNn_PRE  CPRESOPT ='RICHA', NITR=4, XRELAX=1.0 /
&NAM_LBCn_PRE  CLBCX(1)='OPEN', CLBCX(2)='OPEN', 
               CLBCY(1)='OPEN', CLBCY(2)='OPEN' /
&NAM_VPROF_PRE CTYPELOC='IJGRID', NILOC=10, NJLOC=2,  
               CFUNU='ZZZ', CFUNV='ZZZ',
               LGEOSBAL=.FALSE. /
&NAM_GRn_PRE   CSURF='EXTE' /
&NAM_CH_MNHCn_PRE LUSECHEM = F /
CSTN
2
285.
100000.
0. 20000.
10. 10.
0. 0.
40. 40.
0.01
\end{verbatim}

This file contains the  information necessary to generate the initial 
conditions for a quasi-hydrostatic flow, in the weakly
non-linear regime, with a regular vertical grid.

