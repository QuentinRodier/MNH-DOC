\chapter{Compute spectra after a MESO-NH simulation}\label{ch:spectre}
\section{Presentation}
After running the model, you can compute spectra with the program SPECTRE,
that gives the kinetic energy density according to the wavelength or the wave number (see Ricard et al., 2011).
The calculation uses a discrete cosinus transform 
to convert grid-point fields into spectral space ones.


\subsection{Input file}
The "SPECTRE" program computes spectra on MESONH files issued from a simulation or on AROME file. In this second case (CTYPEFILE='AROME'), the program does not read the true AROME file but an ASCII file obtained with the tools "edf". You must create one ASCII file per variable you want : the files will be named YINIFILE\_U, YINIFILE\_V...

\subsection{Output files}
You will obtain one ascii file per variable you have asked. A script exists to trace easily the results. (on MESONH web site in Wiki Team's)


\section{The namelist file SPEC1.nam}

\subsection{Namelist NAM\_SPECTRE\_FILE}
\begin{center}
\begin{tabular} {|l|l|l|}
\hline
Fortran name & Fortran type & default value \\
\hline
YINIFILE    & array of character (len=28)  & ' '\\
CTYPEFILE   & character (LEN=6)  & 'MESONH'      \\
YOUTFILE    & array of character (len=28)  & ' '\\
LSTAT       & logical  & .FALSE. \\        
\hline
\end{tabular}
\end{center}
\begin{itemize}
\item YINIFILE : name of the input FM file.
\item CTYPEFILE : type of the input  file ('AROME ', 'MESONH')
\item YOUTFILE  : prefix of the output file.\\
If the user does specify this name, the output file will be named YOUTFILE\_U, YOUTFILE\_V ....\\
If the user does NOT specify this name, the output file will be named spectra\_YINIFILE\_U, spectra\_YINIFILE\_V ....
\item LSTAT : flag to have some statistiques on fields if .TRUE.
\end{itemize}

\subsection{Namelist NAM\_SPECTRE}
\begin{center}
\begin{tabular} {|l|l|l|}
\hline
Fortran name & Fortran type & default value \\
\hline
LSPECTRE\_U &  logical &.FALSE. \\
LSPECTRE\_V &  logical &.FALSE. \\
LSPECTRE\_W &  logical &.FALSE. \\
LSPECTRE\_TH &  logical &.FALSE. \\
LSPECTRE\_RV &  logical &.FALSE. \\
LSPECTRE\_LSU &  logical &.FALSE. \\
LSPECTRE\_LSV &  logical &.FALSE. \\
LSPECTRE\_LSW &  logical &.FALSE. \\
LSPECTRE\_LSTH &  logical &.FALSE. \\
LSPECTRE\_LSRV &  logical &.FALSE. \\
LSMOOTH     &  logical & .FALSE.\\
\hline
\end{tabular}
\end{center}
\begin{itemize}
\item LSPECTRE\_U : flag to compute spectrum of U component
\item LSPECTRE\_V : flag to compute spectrum of V component
\item LSPECTRE\_W : flag to compute spectrum of W
\item LSPECTRE\_TH : flag to compute spectrum of Theta
\item LSPECTRE\_RV : flag to compute spectrum of vapor mixing ratio
\item LSPECTRE\_LSU : flag to compute spectrum of large scale U component
\item LSPECTRE\_LSV : flag to compute spectrum of large scale V component
\item LSPECTRE\_LSW : flag to compute spectrum of large scale W
\item LSPECTRE\_LSTH : flag to compute spectrum of large scale Theta
\item LSPECTRE\_LSRV : flag to compute spectrum of large scale vapor mixing ratio
\item LSMOOTH : flag to smooth the results
\end{itemize}


\subsection{Namelist NAM\_ZOOM\_SPECTRE}

\begin{center}
\begin{tabular} {|l|l|l|}
\hline
Fortran name & Fortran type & default value \\
\hline
LZOOM &  logical &.FALSE. \\
NXDEB & integer & \\
NYDEB& integer & \\
NITOT& integer & \\
NJTOT& integer & \\
\hline
\end{tabular}
\end{center}

\begin{itemize}
\item LZOOM : flag to make a zoom on the file domain
\item NXDEB : first point I index, left to and out of the wanted domain
\item NYDEB : first point J index, under and out of the wanted domain
\item NITOT : number of grid points in I direction. It must be equal to $2^m \times 3^n \times 5^p$ with $(m,n,p) \ge 0 $
\item NJTOT : number of grid points in J direction. It must be equal to $2^m \times 3^n \times 5^p$ with $(m,n,p) \ge 0$
\end{itemize}

\subsection{Namelist NAM\_DOMAIN\_AROME}
This namelist is only available for CTYPEFILE='AROME'. It contains the caracteristics of the domain arome in the input file.

\begin{center}
\begin{tabular} {|l|l|l|}
\hline
Fortran name & Fortran type & default value \\
\hline
NI & integer & 750\\
NJ& integer & 720\\
NK& integer & 60\\
XDELTAX& integer & 2500\\
XDELTAY& integer & 2500\\
\hline
\end{tabular}
\end{center}

\begin{itemize}
\item NI : number of points to read in I direction 
\item NJ : number of points to read in J direction 
\item NK : : number of vertical levels to read 
\item XDELTAX : gridsize of arome file in I direction (m)
\item XDELTAY : gridsize of arome file in J direction (m)
\end{itemize}


