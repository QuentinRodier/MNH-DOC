%%%%%%%%%%%%%%%%%%%%%%%%%%%%%%%%%%%%%%%%%%%%%%%%%%%%%%%%%%%%%%%%%%%%%%%%%%%%%%%
% CONTRIBUTION TO THE MESONH BOOK1: "Surface Processes Scheme"
% Author        : J. S. Belair,  A. Boone and V. Masson
% Original      : October 15, 1997
% Last Update   : April 7, 2000
%%%%%%%%%%%%%%%%%%%%%%%%%%%%%%%%%%%%%%%%%%%%%%%%%%%%%%%%%%%%%%%%%%%%%%%%%%%%%%%

\chapter{Surface Processes Scheme}
\minitoc

%{by S. Belair, A. Boone and V. Masson}

Since the {\bf Meso-NH} Masdev4-6 version, the surface schemes were externalized
in the so-called {\bf SURFEX} scheme. In absence of scientific documentation
associated with SURFEX, the documentation of the surface schemes, as it was in the Meso-NH Masdev4-3 version is given below, as the principles of
most of the surface schemes remain the same.

\section{Introduction}


By providing more realistic
lower boundary conditions,
land surface schemes
can help to improve the numerical simulation
of the atmosphere.
The surface and the atmosphere
interact via energy fluxes applied at the base of the
atmospheric numerical model.
The fluxes provided to the atmospheric model
are the momentum flux, turbulent sensible and latent heat fluxes,
the upward radiative fluxes, and, as an option, the CO2 flux.

The MESO-NH grid box is partitioned in 4 main surface types:
sea, inland water, towns (artificial areas only),
natural and cultivated areas (including gardens and town parks).
In order to keep the subgrid surface type information,
an appropriate surface scheme is used to compute the fluxes
between each of the surfaces and the atmosphere
(assumed identical for all the sub-components).
Then, the averaging of the 4 fluxes gives
the flux received by the atmospheric part of MESO-NH.\\

\begin{figure}[h]
\hspace*{2.cm}
\psfig{figure=\EPSDIR/4flux.eps,width=12cm}
\caption{Partitioning of the MESO-NH grid box, and corresponding turbulent fluxes.
F stands either for M (momentum flux), H (sensible heat flux), LE (latent heat flux),
$S^\uparrow$ (the reflected solar radiation) or $L^\uparrow$ (the
upward longwave radiation).
\label{surf1}}
\end{figure}

\begin{displaymath}
\begin{array}{lclclclcl}
M &= & M_{sea} & + & M_{water} & + & M_{town} & + & M_{nature} \\
H &= & H_{sea} & + & H_{water} & + & H_{town} & + & H_{nature} \\
LE &= & LE_{sea} & + & LE_{water} & + & LE_{town} & + & LE_{nature} \\
S^\uparrow &= & S^\uparrow_{sea} & + & S^\uparrow_{water} & + & S^\uparrow_{town} & + & S^\uparrow_{nature} \\
L^\uparrow &= & L^\uparrow_{sea} & + & L^\uparrow_{water} & + & L^\uparrow_{town} & + & L^\uparrow_{nature} \\
\end{array}
\end{displaymath}

\bigskip

For the natural surfaces,
an 'aggregated' vegetation is used, viz
a vegetation representative of all the species present
in the grid mesh.

As an option, it will be possible to use the soil and
vegetation scheme on the different landscape types present in the grid mesh.
Then averaging the fluxes coming from each landscape type will give
the flux coming from the natural part of the grid mesh.
Only a limited number of landscape types will be available:
forests, grassland, C3 type cropland, C4 type cropland, irrigated crops,
bare soil, rocks and permanent snow. For each lanscape,
there will be a temporal
evolution of the surface variables independently of the other landscape types.\\

The fluxes coming from the urban area are themselves average
of fluxes from the roofs, roads and walls (see TEB scheme hereafter).\\




The surface schemes used for the 4 main surface types are:

\begin{itemize}
\item \underline{over seas}: a  Charnock's formulation, with stationnary SST
\item \underline{over inland waters}: a Charnock's formulation, with stationnary
temperature.
\item \underline{over natural areas}:
an improved version of the
Interactions Soil-Biosphere-Atmosphere (ISBA) scheme
by Noilhan and Planton (1989), Noilhan and Mahfouf (1996),
Calvet et al. (1998) and Boone et al. (1999),
\item \underline{over artificial areas}:
the 'Town Energy Budget' scheme by Masson (2000).
\end{itemize}

%%%%%%%%%%%%%%%%%%%%%%%%%%%%%%%%%%%%%%%%%%%%%%%%%%%%%%%%%%%%%%%%%%%%%%%%%%%%%%
\clearpage
\section{Fluxes over water surfaces}

\subsection{Free water surfaces}

For ocean surfaces and over inland waters,
all the prognostic variables are kept constant.

The surface fluxes are calculated using Eqs. \ref{eqnRN}, \ref{eqnH},
\ref{eqnLEG} and
Eqs. \ref{eqn_H}, \ref{eqn_LE}, \ref{eqn_FM},
taking the relative humidity of the ocean $hu=1$, and
$veg=p_{sn}=0$.
The roughness length is given by Charnock's relation:
\begin{eqnarray}
z_{0sea} = 0.015 {u^2_* \over g}
\end{eqnarray}

\subsection{Sea ice}

Sea ice is detected in the model when sea surface temperature (SST) is
two degrees below 0$^\circ$C (i.e. 271.15K). In this case, in order
to avoid an overestimation of the evaporation flux, the calculations
are performed with the roughness length of flat snow surfaces:
\begin{eqnarray}
z_{0ice} = 2.4 \; \times \; 10^{-4} m
\end{eqnarray}

In the same manner, the sea ice albedo is set equal to the fresh snow
albedo instead of the free water albedo. This leads to a much brighter
surface. This has no effect on the sea ice cover (since there is no evolution
of the sea surface parameters), but modifies the lower boundary shortwave flux
input for the atmospheric radiative scheme.

%%%%%%%%%%%%%%%%%%%%%%%%%%%%%%%%%%%%%%%%%%%%%%%%%%%%%%%%%%%%%%%%%%%%%%%%%%%%%%
\clearpage
\section{Urban and artificial areas: the TEB surface cheme}

The Town Energy Budget (TEB) scheme is aimed
to simulate the turbulent fluxes into the atmosphere
at the surface of a mesoscale atmospheric model which is covered
by buildings, roads, or any artificial material
(see Masson 2000 for all details and equations).
It should parameterize both the urban surface and the roughness sublayer,
so that the atmospheric model only 'sees' a constant flux layer
as its lower boundary.
At mesoscale (with a grid mesh larger than a few hundred meters), spatial
averaging of the town characteristics, as well as its effect on the atmosphere,
are necessary.
The individual shapes
of each building are no longer taken into account.
The TEB geometry is based on the canyon hypothesis. However, a
single canyon would be too restrictive at the
considered horizontal scale.
We therefore use the following original city representation:
\begin{enumerate}
\item the buildings have the same height and width (in the model mesh).
The roof level is at the surface level of the atmospheric model.
\item buildings are located along identical roads, the length
of which is considered
far greater than their width. The space contained between two facing buildings
is defined as a canyon.
\item any road orientation is possible, and they all exist
with the same probability. This hypothesis allows the computation
of averaged forcing
for road and wall surfaces. In other words, when the canyon orientation
appears in a formula (with respect to the sun or the wind direction),
it is averaged over 360$^\circ$. In this way, no discretization is performed
on the orientation.
\end{enumerate}

\begin{figure}[h]
\hspace*{2.cm}
\psfig{figure=\EPSDIR/geom.eps,width=12cm}
\caption{Canyon geometry in the TEB scheme ,and its prognostic variables.
\label{TEB2}}
\end{figure}

These hypotheses, as well as the
formulations chosen for the physics (see Masson 2000), allow
the development of a relatively
simple scheme. The parameters describing the city
are displayed in Table \ref{paramTEB} in chapter \ref{PGD}.
The prognostic variables of the scheme are in Table \ref{TEB1}.
{\bf The TEB model
does not use one urban surface temperature} (representative of the
entire urban cover), but {\bf three} surface temperatures
representative of roofs, roads and walls (see figure \ref{TEB2}).
There are two reasons for that:
\begin{itemize}
\item this allows to keep the maximum of results gathered by the
urban climatologists about the urban surface energy budget,
\item the use of only one temperature should be debatable, because
it is observed that the Monin-Obukhov similarity theory does not
apply for temperature in the urban roughness sublayer.
\end{itemize}


\begin{figure}[t]
\hspace*{0.cm}
\psfig{figure=\EPSDIR/flux.eps,width=16cm}
\caption{Energy fluxes between the artificial surfaces and the atmosphere.
\label{TEB3}}
\end{figure}

As a consequence, {\bf three} surface energy budgets are computed, one
for each surface type. The resulting fluxes at town scale are obtained
by averaging (see figure \ref{TEB3} for the links between surfaces and
atmosphere).

\clearpage
The physics of the scheme are relatively complete, thanks to
the use of this non-flat canyon geometry:
\begin{itemize}
\item interception of snow and water.
\item longwave computations (with one reemission), infra-red trapping effect.
\item shortwave computations (with an infinity of reflections),
solar trapping effect. The solar zenith angle influences
each surface energy input (see figure \ref{TEB4}) and thus the trapping effect.
\item momentum fluxes (with an urban roughness).
\item sensible and latent heat fluxes, dew.
\item storage flux (by explicit conduction equation),
domestic heating (buildings internal temperature).
\item anthropogenic fluxes (car traffic, factories).
\end{itemize}


\begin{figure}[t]
\hspace*{0.cm}
\psfig{figure=\EPSDIR/solar1.eps,width=16cm}
\caption{Solar radiation input for a road perpendicular to the sun azimuth.
In the TEB scheme, the contribution of all the other road directions
are averaged with this one.
\label{TEB4}}
\end{figure}

\clearpage
\begin{table}[h]
{\footnotesize{
\begin{tabular}{||c |l |c||}
\hline
\hline
symbol                             & designation of symbol & unit    \\
\hline
\hline
prognostic variables && \\
\hline
$T_{R_k}$, $T_{r_k}$, $T_{w_k}$  & temperature of the $k^{th}$ roof, road or wall layer & K \\
$W_R$, $W_r$              & roof and road water interception reservoir& kg m$^{-2}$ \\
${W_{snow}}_R$, ${W_{snow}}_r$ & roof and road snow interception reservoir& kg m$^{-2}$ \\
${T_{snow}}_R$, ${T_{snow}}_r$ & roof and road snow temperature& K \\
${\rho_{snow}}_R$, ${\rho_{snow}}_r$ & roof and road snow density& kg m$^{-3}$ \\
${\alpha_{snow}}_R$, ${\alpha_{snow}}_r$ & roof and road snow albedo& -  \\
\hline
diagnostic variables && \\
\hline
$T_{can}$ & canyon air temperature & K \\
$q_{can}$ & canyon air specific humidity & kg kg$^{-1}$ \\
$U_{can}$ & along canyon horizontal wind & m s$^{-1}$ \\
$\alpha_{town}$ & town effective albedo& - \\
$T_{s_{town}}$ & town area averaged radiative surface temperature & K \\
\hline
input energy fluxes && \\
\hline
$L^\downarrow$ & downward infra-red radiation on an horizontal surface  & W m$^{-2}$ \\
$S^\downarrow$ & downward {\bf diffuse} solar radiation on an horizontal surface  & W m$^{-2}$ \\
$S^\Downarrow$ & downward {\bf direct} solar radiation on an horizontal surface  & W m$^{-2}$ \\
$H_{traffic} $ & anthropogenic sensible heat flux released in the canyon & W m$^{-2}$ \\
$LE_{traffic} $ & anthropogenic latent heat flux released in the canyon & W m$^{-2}$ \\
$H_{industry} $ & anthropogenic  sensible heat flux released by industries & W m$^{-2}$ \\
$LE_{industry} $&  anthropogenic latent heat flux released by industries & W m$^{-2}$ \\
\hline
other energy input && \\
\hline
$T_{i_{bld}}$  & building interior temperature & K \\
\hline
output energy fluxes && \\
\hline
$S^*_R$, $S^*_r$, $S^*_w$ & net solar radiation budget for roofs, roads and walls & W m$^{-2}$ \\
$L^*_R$, $L^*_r$, $L^*_w$ & net infra-red radiation budget for roofs, roads and walls & W m$^{-2}$ \\
$H_R$, $H_r$, $H_w$ & turbulent sensible heat flux for roofs, roads and walls & W m$^{-2}$ \\
$LE_R$, $LE_r$, $LE_w$ & turbulent latent heat flux for roofs, roads and walls & W m$^{-2}$ \\
$G_{R_{k,k+1}}$, $G_{r_{k,k+1}}$, $G_{w_{k,k+1}}$ & conduction heat flux between $k^{th}$ and $k+1^{th}$ roof,
 road or wall layers & W m$^{-2}$ \\
$H_{town}$ & town averaged turbulent sensible heat flux & W m$^{-2}$ \\
$LE_{town}$ & town averaged turbulent latent heat flux & W m$^{-2}$ \\
\hline
\hline
\end{tabular}
}}
\caption{Energy fluxes and variables in the TEB scheme}
\label{TEB1}
\end{table}
%%%%%%%%%%%%%%%%%%%%%%%%%%%%%%%%%%%%%%%%%%%%%%%%%%%%%%%%%%%%%%%%
\clearpage
\section{Soil and Vegetation}

\subsection{Treatment of the soil heat content}

The prognostic equations for the surface temperature
$T_s$ and its mean value $T_2$ over one day $\tau$,
are obtained from the force-restore method proposed by
Bhumralkar (1975) and Blackadar (1976):

\begin{eqnarray} \label{eqTs}
{\partial T_s \over \partial t}&=&C_T (R_n-H-LE) -
{2 \pi \over \tau} (T_s-T_2),   \\
{\partial T_2 \over \partial t}&=& {1 \over \tau}
(T_s-T_2),
\end{eqnarray}
where $H$ and
$LE$ are the sensible and latent heat fluxes, and
$R_n$ is the net radiation at the surface.
The surface temperature $T_s$ evolves due to both the
diurnal forcing by the heat flux $G = R_n-H-LE$ and
a restoring term towards its mean value $T_2$.
In contrast, the mean temperature $T_2$ only varies
according to a slower relaxation towards $T_s$.

The coefficient $C_T$ is expressed by

\begin{equation}
C_T=1 / \left(  {(1-veg)(1-p_{sng}) \over C_g}
    +       { veg(1-p_{snv}) \over C_v }
    +       { p_{sn} \over C_s } \right)
\end{equation}
where $veg$ is the fraction of vegetation, $C_g$ is the ground heat capacity,
$C_s$ is the snow heat capacity, $C_v$ is the vegetation heat capacity,
and
\begin{eqnarray}
p_{sng} = {W_s \over W_s + W_{crn} }; \ \ \ \ \ \
p_{snv} = { h_s \over h_s + 5000 z_0 }; \ \ \ \ \ \
p_{sn} = (1-veg) p_{sng} + veg p_{snv}
\end{eqnarray}
are respectively the fractions of the bare soil and vegetation
covered by snow, and the fraction of the grid covered by snow.
Here, $W_{crn} = 10 \ mm$, and
$h_s = W_s / \rho_s$ is the thickness of the snow canopy ($\rho_s$ is the
snow density).
The partitioning of the grid into bare soil, vegetation, and snow regions,
is indicated in Fig.(\ref{isba1}) .

\begin{figure}[h]
\psfig{figure=\EPSDIR/isba_fig1.eps,width=15cm}
\caption{Partitioning of the grid \label{isba1}}
\end{figure}

The heat capacities of the ground and snow canopies are
respectively given by

\begin{equation}
C_g=C_{gsat} \left( {w_{sat} \over w_2} \right)^
{b / 2log10} ; C_g \leq 1.5 \times 10^{-5} \ K m^2 J^{-1}
\end{equation}
where $G_{gsat}$ is the heat capacity at saturation, and $w_{sat}$ the
volumetric moisture content of the soil at saturation; and
\begin{equation}
C_s=2 \times \left( {\pi \over \lambda_s c_s \tau}
\right)^{1/2}
\end{equation}
where $\lambda_s = \lambda_i \times {\rho_s}^{1.88}$;
$c_s = c_i (\rho_s / \rho_i)$:
$\lambda_i$ is the ice conductivity;
$c_i$ is the heat capacity
of ice; and $\rho_i$ is the relative density
of ice (see Douville 1994, Douville et al. 1995).
\\

After an intermediate surface
temperature ${T_s}^*$ is evaluated from Eq. (\ref{eqTs}), the cooling
due to the melting of snow is considered following
\begin{eqnarray}
{T_s}^+ = {T_s}^* - C_T L_f (melt) \Delta t
\end{eqnarray}
where $L_f$ is the latent of fusion, $\Delta t$ is the timestep,
and the melting rate of snow is
\begin{eqnarray}
melt = p_{sn} \left( { T_n - T_0 \over C_s L_f \Delta t } \right); \ \ \ \ \
melt \geq 0
\end{eqnarray}
Here,
\par $T_0=273.16 \ K$;
\par $T_n = (1-veg) {T_s}^* + veg T_2$

Similarly, the intermediate mean temperature ${T_2}^*$ obtained
from Eq. (2) is also modified due to the melting/freezing of
water in the soil layer occurring for temperatures
between $-5^\circ C$ and $0^\circ C$.
The resulting mean temperature is
\begin{eqnarray}
{T_2}^+ = {T_2}^* + (\Delta w_2)_{frozen} L_f C_g d
\end{eqnarray}
with
\begin{eqnarray}
(\Delta w_2)_{frozen} = \left[ 1 - \left( { {T_2}^*-268.16
\over 5. } \right) \right] \left( w_2(t) - w_2(t-\Delta t) \right) \\
(\Delta w_2)_{frozen} = 0  \ \ \ \ if \ T_2 \leq -5^\circ C \
or \ if \ T_2 \geq 0^\circ C
\end{eqnarray}
where $d=15 \ cm$ is an estimated average of the penetration of the
diurnal wave into the soil.
Only the mean temperature $T_2$ is modified
by this factor.
The surface temperature $T_s$, however, indirectly
feels this effect through the relaxation term in Eq. (1).

\subsection{Treatment of the soil water}

Equations for $w_g$ and $w_2$ are derived from the force-restore
method applied by Deardorff (1977) to the ground soil moisture:
\begin{eqnarray}
{\partial w_g \over \partial t}&=&{ C_1 \over \rho_w d_1}
(P_g - E_g) - {C_2 \over \tau} ( w_g - w_{geq} ) \ ;
0 \leq w_g \leq w_{sat} \\
{\partial w_2 \over \partial t}&=&{1 \over \rho_w d_2}
(P_g - E_g - E_{tr}) - { C_3 \over d_2 \tau} \ max
\left[ 0., ( w_2 - w_{fc} ) \right] ; 0 \leq w_2 \leq w_{sat}
\end{eqnarray}
where $P_g$ is the flux of liquid water reaching the soil surface
(including the melting),
$E_g$ is the evaporation at the soil surface,
$E_{tr}$ is the transpiration rate,
$\rho_w$ is the density of liquid water,
and $d_1$ is an arbitrary normalization depth of 10 centimeters.
In the present formulation, all the liquid water from
the flux $P_g$ goes into the
reservoirs $w_g$ and $w_2$, even when snow covers fractions of the
ground and vegetation.
The first term on the right hand side of Eq. (12) represents
the influence of surface atmospheric fluxes when the
contribution of the water extraction by the roots is neglected.
The coefficients $C_1$ and $C_2$, and the equilibrium surface volumetric
moisture $w_{geq}$,
have been calibrated for different soil textures and
moistures.

The expression for $C_1$ differs depending on the moisture content
of the soil.  For wet soils
(i.e., $w_g \geq w_{wilt}$), this coefficient is expressed as
\begin{eqnarray}
C_1 = C_{1sat} \left( {w_{sat} \over w_g} \right)^{b/2+1}
\end{eqnarray}
For dry soils (i.e., $w_g < w_{wilt}$),
the vapor phase transfer needs to be considered in order to reproduce the
physics of water exchange.  These transfers are parameterized
as a function of the wilting point $w_{wilt}$, the soil water
content $w_g$,
and the surface temperature $T_s$, using the Gaussian expression
(see Braud et al. 1993, Giordani 1993)
\begin{eqnarray}
C_1 = C_{1max} exp \left[ - { (w_g-w_{max})^2 \over 2 \sigma^2 } \right]
\end{eqnarray}
where $w_{max}$, $C_{1max}$, and $\sigma$ are respectively the abscissa
of the maximum, the mode, and the standard deviation of the
Gaussian functions (see Appendix B).
The other coefficient, $C_2$, and the equilibrium water
content, $w_{geq}$, are given by
\begin{eqnarray}
C_2&=&C_{2ref} \left( {w_2 \over w_{sat}-w_2+0.01} \right) \\
w_{geq}&=&w_2 - a w_{sat} \left( { w_2 \over w_{sat}} \right)^p
\left[ 1 - \left({ w_2 \over w_{sat}} \right)^{8p} \right]
\end{eqnarray}

For the $w_g$ evolution,
Eq. (13) represents the water budget over the soil
layer of depth $d_2$.  The drainage, which is proportional to the water
amount exceeding the field capacity (i.e., $w_2-w_{fc}$),
is considered in the second term of the equation (see
Mahfouf et al. 1994).  The coefficient $C_3$ does not depend
on $w_2$ but simply on the soil texture (see Appendix A).  Similarly,
run-off occurs when $w_g$ or $w_2$ exceeds the saturation
value $w_{sat}$.

\subsubsection{Root zone soil layer option}

In the standard two-soil layer version of ISBA, it is not possible
to distinguish the root zone and sub-root zone soil water reservoirs.
With the three-layer version,
the deep soil layer may provide water
to the system through capillary rises only, and the available
water content (for transpiration) is clearly defined.

The bulk soil layer (referred to as $w_2$ in the previous sections)
is divided into a root-zone layer (with a depth $d_2$) and base-flow layer
(with a thickness defined as $d_3-d_2$).
The governing equations for the time evolution of
soil moisture for the two sub-surface soil layers are written
following (Boone et al. 1999) as
%
\begin{eqnarray}
%
{\partial w_2\over\partial t} &=&
{1\over \rho_w d_2}\left(P_g-E_g-E_{tr}\right)
\,-\, {C_3\over d_2\tau} {\rm max}\left[0,\,\left(w_2-w_{fc}\right)\right]
\,-\, {C_4\over\tau} \left(w_2-w_3\right) \\
%
{\partial w_3\over\partial t} &=&
{d_2\over \left(d_3-d_2\right)}
\Biggr\lbrace
{C_3\over d_2\tau} {\rm max}\left[0,\,\left(w_2-w_{fc}\right)\right]
\,+\, {C_4\over\tau} \left(w_2-w_3\right)
\Biggr\rbrace \\
& & \,-\,
{C_3\over \left(d_3-d_2\right)\tau}
{\rm max}\left[0,\,\left(w_3-w_{fc}\right)\right] \,;
\hskip.5in 0 \leq w_3 \leq w_{sat}
%
\end{eqnarray}
%
where $C_4$ represents the vertical diffusion coefficient.
It is defined as
%
\begin{eqnarray}
%
C_4 & =& {C_{4\, ref}} \,{{\overline w}_{2,3}}^{C_{4b}} \\
%
\end{eqnarray}
%
where ${\overline w_{2,3}}$ represents the interpolated volumetric water
content representative of the values at the layer interface ($d_2$).
The $C_{4\, ref}$ and $C_{4b}$ coefficients are defined using the soil
sand and clay contents, consistent with the
other model parameters (see the section on model coefficients). In addition,
the $C_{4\, ref}$ coefficient is scaled as a function of grid geometry.
The equations are integrated in time using a fully implicit method.

\subsection{Treatment of the intercepted water}

Rainfall and dew intercepted by the foliage feed a
reservoir of water content $W_r$.  This amount of water
evaporates in the air at a potential rate from the fraction
$\delta$ of the foliage covered with a film of water, as the
remaining part $(1-\delta)$ of the leaves transpires.
Following Deardorff (1978), we set
\begin{eqnarray}
{\partial W_r \over \partial t} = veg P - (E_v-E_{tr}) - R_r \ ; \
0 \leq W_r \leq W_{rmax}
\end{eqnarray}
where $P$ is the precipitation rate at the top of the vegetation,
$E_v$ is the evaporation from the vegetation including the
transpiration $E_{tr}$ and the direct evaporation $E_r$ when
positive, and the dew flux when negative (in this case $E_{tr} = 0$),
and $R_r$ is the runoff of the interception reservoir.
This runoff occurs when $W_r$ exceeds a maximum value $W_{rmax}$
depending upon the density of the canopy, i.e., roughly proportional
to $veg LAI$.
According to Dickinson (1984), we use the simple equation:
\begin{eqnarray}
W_{rmax} = 0.2 veg LAI \ \ \ \ \ [mm]
\end{eqnarray}

\subsection{Treatment of soil ice}

The inclusion of soil freezing necessitates the addition
of so-called phase change to the thermal and hydrologic
transfer equations. In addition,
a freezing/drying wetting/thawing analogy is
used to model changes in the force-restore coefficients
so that they must also be modified accordingly. Terms which have been
added to the baseline ISBA scheme are underlined in this
section, while terms which are modified are denoted using
an $\ast$ superscript.
Additional details related to soil freezing
scheme can be found in Boone et al. (2000).

The basic prognostic equations
including soil ice are expressed as
%
\begin{eqnarray}
{\partial T_s\over\partial t} &=&
{C_T}^\ast \left[ R_n \,-\, H \,-\, {LE}^\ast \,-\,
L_f(M_s-{\underline{F_{g\,w}}})\right] \,-\,
{2\pi\over\tau}(T_s-T_2)
\,\,\,, \\
%
{\partial T_2\over\partial t} &=&
{1\over\tau}(T_s-T_2)
\,+\,{C_G}^\ast L_f {\underline{F_{2\,w}}}
\,\,\,, \\
%
{\partial w_g\over\partial t} &=&
{1\over d_1\rho_w} \left[ {C_1}^\ast\left(P_g-E_{g\,l}+M_s \right)
\,-\, {\underline{F_{g\,w}}} \right]
\,-\, {{C_2}^\ast\over\tau}(w_g-{w_{g\,{\rm eq}}}^\ast) \\
& & \hskip2.6in
(w_{{\rm min}} \leq w_g \leq w_{\rm sat}-w_{g\,f})
\,\,\,, \\
%
{\partial w_2\over\partial t} &=&
{1\over d_p\rho_w}
\left(P_g-{E_{g\,l}}-{E_{tr}}^\ast+M_s-
{\underline{F_{2\,w}}} \right) \,-\,{C_3\over\tau}{\rm max}
(0\,,\,\,w_2-{w_{\rm fc}}^\ast) \\
& & \hskip2.6in
(w_{{\rm min}}\leq w_2 \leq w_{\rm sat}-w_{2\,f})
\,\,\,, \\
%
{\partial w_{g\,f}\over\partial t} &=&
{1\over d_1 \rho_w} \left(
F_{g\,w} \,-\, E_{g\,f} \right)
\hskip1.35in
(0 \leq w_{g\,f}\leq w_{\rm sat}-w_{{\rm min}} )
\,\,\,, \\
%
{\partial w_{2\,f}\over\partial t} &=&
{1\over \left(d_2-d_1\right)\rho_w} F_{2\,w}
\hskip1.6in
(0 \leq w_{2\,f}\leq w_{\rm sat}-w_{{\rm min}} )
\,\,\,.
\end{eqnarray}
%
where $w_{g\,f}$ and $w_{2\,f}$ represent the volumetric soil ice content
(m$^3$ m$^{-3}$) in the surface and deep-soil reservoirs, respectively.
The phase change mass and heat sink (source) terms ($F$; kg m$^{-2}$ s$^{-1}$)
are expressed as
%
\begin{eqnarray}
F_{g\,w} &=& \left(1-p_{sng}\right)\,\left( F_{g\,f}
\,-\, F_{g\,m}\right)\,\,\,, \\
%
F_{2\,w} &=& \left(1-p_{sng}\right)\,\left( F_{2\,f}
\,-\, F_{2\,m}\right)\,\,\,,
\end{eqnarray}
%
where the $m$ and $f$ subscripts represent melting and
freezing, respectively.
The freezing and melting phase change terms are formulated using simple
relationships based on the potential
energy available for phase change. They are expressed
for the surface soil layer as
%
\begin{eqnarray}
F_{g\,f} &=& \left(1/\tau_i \right)\,{\rm min} \left[
K_s \,\epsilon_{s\,f} \,{\rm max}(0,\,T_0-T_s)/C_I \, L_f ,\,
\rho_w\,d_1\,\left( w_g-w_{\rm min} \right)
\right]
\,\,\, , \\
%
F_{g\,m} &=& \left(1/\tau_i \right)\,{\rm min} \left[
K_s \,\epsilon_{s\,m} \,{\rm max}(0,\,T_s-T_0)/C_I \, L_f ,\,
\rho_w\,d_1\,w_{g\,f}
\right]
\,\,\, ,
\end{eqnarray}
%
and for the deep soil layer as
%
\begin{eqnarray}
F_{2\,f} &=& \left(\delta_{2\,f}/\tau_i \right)\,{\rm min} \left[
\epsilon_{2\,f} \,{\rm max}(0,\,T_0-T_2)/C_I \, L_f , \,
\rho_w\,\left(d_2-d_1\right)\left( w_2-w_{\rm min}\right)\right]
\,\,\, ,\\
%
F_{2\,m} &=& \left(1/\tau_i \right)\,{\rm min} \left[
\epsilon_{2\,m} \,{\rm max}(0,\,T_2-T_0)/C_I \, L_f , \,
\rho_w\,\left(d_2-d_1\right)\,w_{2\,f}\right]
\,\,\,.
\end{eqnarray}
%
The characteristic time scale for freezing is represented by $\tau_i$ (s).
The phase change efficiency coefficients, $\epsilon$, introduce
a dependence on the water mass available for phase changes
which are expressed as
the ratio of the liquid volumetric water content
to the total soil porosity for freezing,
and the ratio of ice content to the porosity for melting.
The ice thermal inertia coefficient is defined as
$C_I = 2 {\left( {\pi/\lambda_i\,C_i\rho_i\tau} \right)}^{1/2}$
(J m$^{-2}$ K$^{-1}$).
The insulating effect of vegetation is modeled using a coefficient
defined as
%
\begin{eqnarray}
K_s =
\, \left({1-{veg\over K_2}}\right)
\left({1-{LAI\over K_3}}\right) \,\,\,,
\end{eqnarray}
%
where the dimensionless coefficients have
the values $K_2\,=\,5.0$ and $K_3\,=\,30.0$ (Giard and Bazile 1999).
The most direct
effect of vegetation cover
is to slow the rate of phase changes for more dense vegetation cover
as energy not used for phase change is assumed to cool/warm the
vegetative portion of the lumped soil-vegetation layer.

The deep-soil phase change (freezing) term
is multiplied by a factor ($\delta_{2\,f}$)
which essentially limits ice production during prolonged
cold periods.  It is defined as 0 if $z_f \geq z_{f\,{\rm max}}$
where
%
\begin{eqnarray}
z_{f\,{\rm max}} &=& 4/\left( {C_G}^\ast\,c_g\right)
\,\,\,
\end{eqnarray}
and the actual depth of ice in the soil is defined as
\begin{eqnarray}
z_f &=& d_2 \, \left({w_{2\,f}\over w_{2\,f} + w_2}\right)
\hskip1.in \left(0 \leq z_f < d_2\right)
\end{eqnarray}
%
Ice is assumed to become
part of the solid soil matrix. This is accomplished by
defining the modified porosity
(eg. Johnsson and Lundin 1991) as
%
\begin{eqnarray}
{w_{sat}}^\ast = w_{sat} - w_{j\,f}
\end{eqnarray}
%
where $j$ corresponds to the surface ($g$) or sub-surface ($2$)
soil water reservoirs.
This, in turn, is used to modify the force-restore coefficients
(see Boone at al. 2000 for more details).

\subsection{Treatment of the snow}

\subsubsection{One-layer snow scheme option}

The evolution of the equivalent water content of the snow reservoir
is given by
\begin{eqnarray}
{\partial W_s \over \partial t} = P_s - E_s - melt
\end{eqnarray}
where $P_s$ is the precipitation of snow, and $E_s$ is the sublimation
from the snow surface.

The presence of snow covering the ground and
vegetation can greatly influence the energy and mass
transfers between the land surface and the atmosphere.
Notably, a snow layer modifies the radiative
balance at the surface by increasing the albedo.
To consider this effect, the albedo of snow $\alpha_s$ is treated
as a new prognostic variable.
Depending if the snow is melting or not,
$\alpha_s$ decreases
exponentially or linearly with time.

If there is no melting (i.e., $melt=0$):
\begin{eqnarray}
\alpha_s (t) = \alpha_s (t-\Delta t) - \tau_a {\Delta t \over \tau}
+ { P_s \Delta t \over W_{crn} } ( \alpha_{smax} - \alpha_{smin} ) \\
\alpha_{smin} \leq \alpha_s \leq \alpha_{smax}
\end{eqnarray}
where $\tau_a=0.008$ is the linear rate of decrease per day,
$\alpha_{smin}=0.50$
and $\alpha_{smax}=0.85$ are the minimum and maximum values of the snow
albedo.

If there is melting (i.e., $melt \ > \ 0$):
\begin{eqnarray}
\alpha_s (t) = \left[ \alpha_s(t - \Delta t) - \alpha_{smin} \right]
exp \left[ -\tau_f {\Delta t \over \tau } \right]
+ \alpha_{smin}
+ { P_s \Delta t \over W_{crn} } ( \alpha_{smax} - \alpha_{smin} ) \\
\alpha_{smin} \leq \alpha_s \leq \alpha_{smax}
\end{eqnarray}
where $\tau_f=0.24$ is the exponential decrease rate per day.
Of course, the snow albedo increases as snowfalls occur, as shown
by the second terms of Eqs. (21) and (23).

The average albedo of a model grid-area is expressed as
\begin{eqnarray}
\alpha_t = (1-p_{sn}) \alpha + p_{sn} \alpha_s
\end{eqnarray}
Similarly, the average emissivity $\epsilon_t$ is also
influenced by the snow coverage:
\begin{eqnarray}
\epsilon_t = (1-p_{sn}) \epsilon + p_{sn} \epsilon_s
\end{eqnarray}
where $\epsilon_s = 1.0$ is the emissivity of the snow.
Thus, the overall albedo and emissivity of the ground for infrared radiation
is enhanced by snow.

Because of the significant variability of thermal properties related
with the snow compactness,
the relative density of snow $\rho_s$ is also considered
 as a prognostic variable.
Based on Verseghy (1991), $\rho_s$ decreases exponentially at a rate
of $\tau_f$ per day:
\begin{eqnarray}
\rho_s(t)=\left[ \rho_s(t-\Delta t) - \rho_{smax} \right]
exp \left[ -\tau_f {\Delta t \over \tau } \right] + \rho_{smax}
+ {P_s \Delta t \over W_s} \rho_{smin}  \\
\rho_{smin} \leq \rho_s \leq  \rho_{smax}
\end{eqnarray}
where
$\rho_{smin} = 0.1$ and $\rho_{smax} = 0.3$
are the minimum and maximum relative density of snow.

Finally, the average roughness length $z_{0t}$ is
\begin{eqnarray}
z_{0t} = ( 1 - p_{snz0} ) z_0 + p_{snz0} z_{0s}
\end{eqnarray}
where
\begin{eqnarray}
p_{snz0} = { W_s \over W_s + W_{crn} + \beta_s g z_0 }
\end{eqnarray}
Here, $\beta_s = 0.408 \ s^2 m^{-1}$ and $g=9.80665 \ m s^{-2}$ are physical constants,
whereas $z_{0s}$ is the roughness length of the snow.

\subsubsection{Multi-layer snow scheme option}

An additional snow scheme option has been added to ISBA.
It is a so-called intermediate complexity scheme which
is representative of a class of snow models which
use several layers and have simplified physical parameterization schemes
based on those of the highly detailed internal-process models
while having computational requirements more closely resembling
the relatively simple
composite/force-restore or single layer schemes
(eg.s Loth et al. 1993; Lynch-Steiglitz 1994; Sun et al. 1999).
%
Compared to the baseline ISBA snow scheme,
the explicit multi-layered approach
resolves the large thermal and the density gradients
which can exist in the snow cover, distinguishes the surface
energy budgets of the snow and non-snow covered portions
of the surface, includes the effects of liquid water storage
in the snow cover, models the
absorption of incident radiation within the pack,
and calculates explicit heat conduction between the
snow and the soil.

The conservation equation for the total snow cover mass
is expressed as
%
\begin{eqnarray}
{\partial W_s \over \partial t} =
P_s + p_{sn}\,\left(P-P_s\right) - E_{s} - E_{sl} - Q_n
\,\,\,,
\end{eqnarray}
%
where $E_{sl}$ represents evaporation of liquid water
from the snow surface, and the product $p_{sn}\,\left(P-P_s\right)$
represents the portion of the total rainfall that is
intercepted by the snow surface while the remaining
rainfall is assumed to be intercepted by the snow-free soil and vegetation
canopy. The snow-runoff rate, $Q_n$,
is the rate at which
liquid water leaves the base of the snow cover.

The snow state variables are the heat content ($H_s$),
the layer thickness ($D$), and the layer average density ($\rho_s$).
The temperature ($T_{sn}$) and liquid water content ($w_{sl}$) are defined
using the heat content.
The total snow depth, $D_s$ (m) is defined as
%
\begin{eqnarray}
D_s = \sum_{i=1}^{N_s} D_i
\end{eqnarray}
%
where a three-layer configuration is currently used by default (i.e. $N_s=3$).
%The layering scheme is described in Boone and Etchevers (2000).
The surface snow layer is always less than or equal to 0.05 m, and this
temperature is used to calculate the fluxes between the atmosphere
and the snow surface.
The snow density is compacted using standard empirical relationships
(Anderson 1976). Additional changes arise to snowfall which
generally reduces the snow density, and densification resulting from ripening.
The snow heat content (J m$^{-2}$) is defined as
%
\begin{eqnarray}
H_{s\,i} = c_{s\,i}\,D_i\,\left(T_{sn\,i}-T_0\right)
\,-\, L_f\,\rho_w \left(w_{s\,i}- w_{sl\,i}\right)\,\,\,,
\end{eqnarray}
%
where $w_s$ is the total snow layer water equivalent depth (m),
$w_{sl}$ is the snow layer liquid water content (m), and $c_s$
is the snow heat capacity (J m$^{-3}$ K$^{-1}$) (using the same
definition as the baseline ISBA snow scheme).
The snow heat content is used in order to allow
the presence of either
cold (dry) snow which has a temperature less
than or equal to the freezing point
%and contains no liquid water,
or warm (wet)
snow which is characterized by a temperature at the freezing point
and contains water in liquid form.
The snow temperature
and liquid water content can then be defined as
%
\begin{eqnarray}
T_{sn\,i} &=& T_f \,+\, \left(H_{s\,i}  + L_f\,\rho_w\,w_{s\,i}\right)
/\left(c_{s\,i}\,D_i \right) \,;
\hskip.3in
w_{l\,i} = 0 \\
%
w_{sl\,i} &=& w_{s\,i} \,+\, \left(H_{s\,i}/L_f\,\rho_w\right) \,;
\hskip1.3in
T_{sn\,i} = T_f \,\,\,{\rm and} \,\,\, w_{sl\,i} \leq w_{sl\,{\rm max}\,i}
\end{eqnarray}
%
where $w_{sl\,{\rm max}\,i}$ is the maximum liquid water
holding capacity of a snow layer,
which is based on empirical relations. All
water exceeding this flows into the layer below where
it can do one or all of the following;
add to the liquid water content, refreeze, or continue
flowing downward.

Snow heat flow is along the thermal gradient
as any snow melt or percolated water within the snow cover
is assumed to have zero heat content.
The layer-averaged snow temperature equation
($T_{s\,i}$) is expressed as
%
\begin{eqnarray}
c_{s\,i} D_i {\partial T_{sn\,i}\over\partial t}
= G_{s\,i-1} - G_{s\,i} + R_{s\,i-1} - R_{s\,i}  - S_{s\,i}
\,\,\,,
\end{eqnarray}
%
where $S_s$ represents an energy sink/source term associated with
phase changes between the liquid and solid phases of water.
Incoming short wave radiation ($R_s$)
transmission within the snowpack decreases exponentially
with increasing snow depth. At the surface, it is expressed as
%
\begin{eqnarray}
R_{s\,0} = R_g \,\left(1-\alpha_s\right)
\end{eqnarray}
%
where the snow albedo is defined using the same relationships
as in the baseline version of ISBA (Douville et al. 1995).
The sub-surface heat ($G_s$) flux terms are evaluated using
simple diffusion. At the surface, this flux is expressed as
%
\begin{eqnarray}
G_{s\,0}
= \epsilon_s \left( R_A - \sigma_{SB} {T_{sn\,1}}^4 \right)
\,-\, H\left(T_{sn\,1}\right) \,-\, LE\left(T_{sn\,1}\right) \,-\,
c_w \,p_{sn} \left(P-P_s\right)\left(T_f-T_r\right)\,\,\,,
\end{eqnarray}
%
The last term on the right hand side of the above equation
represents a latent heat source when rain
with a temperature ($T_r$) greater than $T_0$ falls on the snow cover,
where $c_w$ represents the heat
capacity of water (4187 J kg$^{-1}$ K$^{-1}$).
Rainfall is simply assumed to have a temperature which is the larger of
the air temperature ($T_a$) and the freezing point.
The latent heat flux from the snow
includes the liquid fraction weighted
contributions from the
evaporation of liquid water and sublimation.

The ISBA surface soil/vegetation layer temperature is then
coupled to the snow scheme using
%
\begin{eqnarray}
{1\over C_T}{\partial T_s\over\partial t} &=&
\left(1-p_n\right) \left[R_g\left(1-\alpha\right) +
\epsilon_t\left(R_A-\sigma {T_s}^4\right)- H - LE
- {2\pi \over C_T \tau} \left(T_s-T_2\right)
\right] \\
%
& & \,+\, p_n \, \left[G_{s\,N}+R_{s\,N}
+ c_w\,Q_n\,\left(T_f-T_s\right)
\right] \,\,\,.
\end{eqnarray}
%
The term on the right hand side of the above equation
involving the snow runoff ($Q_n$) represents an advective term.
The net surface fluxes to/from the atmosphere
are then calculated as the snow-cover
fraction weighted sums over the snow and non-snow covered
surfaces. When the 3-layer option is in use, the default ISBA scheme
is used when the snow cover is relatively thin (arbitrarily
defined as 0.05 m depth). When the snow depth exceeds this
threshold, the snow mass and heat is transferred to the multi-layer
scheme. This prevents numerical difficulties
%and a more complex computer code
for vanishingly thin snow packs.

\subsection{The surface fluxes}

Only one energy balance is considered for the whole system
ground-vegetation-snow (when the 3-layer snow scheme option is not in use).
As a result, heat and mass transfers between the surface and
the atmosphere are related to the mean values $T_s$ and $w_g$.

The net radiation at the surface is the sum of the absorbed
fractions of the incoming solar radiation $R_G$ and of the
atmospheric infrared radiation $R_A$, reduced by the emitted
infrared radiation:
\begin{eqnarray} \label{eqnRN}
R_n = R_G (1-\alpha_t) + \epsilon_t \left( R_A-\sigma_{SB}{T_s}^4 \right)
\end{eqnarray}
where $\sigma_{SB}$ is the Stefan-Boltzmann constant.

The turbulent fluxes are calculated by means of the classical
aerodynamic formulas.  For the sensible heat flux:
\begin{eqnarray}
H = \rho_a c_p C_H V_a (T_s - T_a) \label{eqnH}
\end{eqnarray}
where $c_p$ is the specific heat; $\rho_a$, $V_a$, and $T_a$
are respectively the air density, the wind speed, and the
temperature at the lowest atmospheric level; and $C_H$,
as discussed below, is the
drag coefficient depending upon the thermal stability of the
atmosphere. The explicit snow scheme sensible heat flux
is calculated using the same formulation (but with $T_{sn}$).
The water vapor flux $E$ is the sum of the evaporation
of liquid water
from the soil surface (i.e., $E_{g\,l}$), from the vegetation (i.e., $E_v$),
and sublimation from the snow and soil ice (i.e, $E_s$ and $E_{g\,f}$):
%
\begin{eqnarray}
LE &=& LE_{g\,l} + LE_v + L_i \left(E_s + E_{g\,f}\right) \\
E_{g\,l} &=& (1-veg)(1-p_{sng})\left(1-\delta_i\right)\, \rho_a C_H V_a
        \left( h_u q_{sat}(T_s) - q_a \right) \label{eqnLEG} \\
E_v &=& veg (1-p_{snv}) \rho_a C_H V_a h_v
        \left( q_{sat}(T_s) - q_a \right) \\
E_s &=& p_{sn} \rho_a C_H V_a
        \left( q_{sat}(T_s) - q_a \right) \\
E_{g\,f} &=&
\,\left(1-veg\right)\left(1-p_{sng}\right)\,\delta_i\, \rho_a C_H V_a
\left( h_{ui} \,q_{\rm sat}\left(T_s\right) \,-\, q_a \right)
\end{eqnarray}
where $L$ and $L_i$ are the specific heat of evaporation
and sublimation, $q_{sat}(T_s)$ is the saturated
specific humidity at the
temperature $T_s$, and $q_a$ is the atmospheric specific humidity
at the lowest atmospheric level.
The water vapor flux $E$
from the explicit snow surface is expressed as
%
\begin{eqnarray}
LE\left(T_{sn\,1}\right) &=& L E_{sl} + L_i E_s \\
E_{sl} &=& \delta_{sn} \,\rho_a C_{Hs} V_a
        \left( q_{sat}\left(T_{sn\,1}\right) - q_a \right) \\
E_s &=& \left(1-\delta_{sn}\right) \,
        \rho_a C_{Hs} V_a \left( q_{sat}\left(T_{sn\,1}\right) - q_a \right) \\
\delta_{sn} &=& w_{sl\,1}/w_{sl\,{\rm max}\,1}\,;
\hskip2.2in
0 \leq \delta_{sn} \leq 1
\end{eqnarray}
%
where evaporation of liquid water is zero when $T_{sn\,1}<T_0$.
The transfer coefficient ($C_{Hs}$) is calculated over the snow
covered surface using the same formulation as $C_H$.

The surface ice fraction is
is used to partition the bare soil latent heat flux
between evaporation and sublimation, and it is defined as
%
\begin{eqnarray}
\delta_i = w_{g\,f}/\left(w_{g\,f}+w_g\right) \,;
\hskip.5in
0 \leq \delta_i < 1   \,\,\,.
\end{eqnarray}

The relative humidity $h_u$ at the ground surface is related to the
superficial soil moisture $w_g$ following
\begin{eqnarray}
h_u &=& {1 \over 2} \left[ 1-cos \left( {w_g \over {w_{fc}}^\ast} \pi \right)
      \right] \ , \ {\rm if} \ w_g < {w_{fc}}^\ast \\
h_u &=& 1 \ \ \ \ \ \ \ \ \ \ \ \ \ \ \ \ \ \
          \ \ \ \ \ \ \ \ \ , \ {\rm if} \ w_g \geq {w_{fc}}^\ast
\end{eqnarray}
%
where the field capacity with respect to the liquid water
is defined using the modified soil porosity so
that ${w_{fc}}^\ast = w_{fc}\,w_{sat}^\ast/w_{sat}$.
The humidity for the ice covered portion of the grid box
is calculated in a similar fashion as
%
\begin{eqnarray}
h_{ui} &=& {1 \over 2} \left[ 1-cos \left( {w_{g\,f} \over {w_{fc}}^{\ast\ast}} \pi \right)
      \right] \ , \ {\rm if} \ w_{g\,f} < {w_{fc}}^{\ast\ast} \\
h_{ui} &=& 1 \ \ \ \ \ \ \ \ \ \ \ \ \ \ \ \ \ \
          \ \ \ \ \ \ \ \ \ , \ {\rm if} \ w_{g\,f} \geq {w_{fc}}^{\ast\ast}
\end{eqnarray}
%
where ${w_{fc}}^{\ast\ast} = w_{fc}(w_{sat}-w_g)/w_{sat}$.
In case of dew flux when $q_{sat}(T_s) < q_a$, $h_u$ is also set
to 1 (see Mahfouf and Noilhan 1991 for details).
When the flux $E_v$ is positive, the Halstead coefficient $h_v$
takes into account the direct evaporation $E_r$ from the fraction
$\delta$ of the foliage covered by intercepted water, as well as
the transpiration $E_{tr}$ of the remaining part of the leaves:
\begin{eqnarray}
h_v &=& (1-\delta) R_a / (R_a+R_s) + \delta \\
E_r &=& veg(1-p_{snv}) {\delta \over R_a}
        \left( q_{sat} (T_s) - q_a \right) \\
E_{tr} &=& veg(1-p_{snv}) {1-\delta \over R_a + R_s}
           \left( q_{sat}(T_s) - q_a \right)
\end{eqnarray}
When $E_v$ is negative, the dew flux is supposed to occur
at the potential rate, and $h_v$ is taken equal to 1.

Following Deardorff (1978), $\delta$ is a power function of the
moisture content of the interception reservoir:
\begin{eqnarray}
\delta = (W_r / W_{rmax})^{2/3}
\end{eqnarray}
The aerodynamic resistance is $R_a = ( C_H V_a )^{-1}$.
The surface resistance, $R_s$, depends upon both atmospheric
factors and available water in the soil; it is given by:
\begin{eqnarray}
R_s = {R_{smin} \over F_1 F_2 F_3 F_4 LAI}
\end{eqnarray}
with the limiting factors $F_1$, $F_2$, $F_3$, and $F_4$:
\begin{eqnarray}
F_1 &=& {f+R_{smin}/R_{smax} \over 1+f} \\
F_2 &=& {w_2-w_{wilt} \over w_{fc} - w_{wilt} } \ \ \ \
and \ 0 \leq F_2 \leq 1 \\
F_3 &=& 1 - \gamma \left( q_{sat}(T_s) - q_a \right) \\
F_4 &=& 1-1.6\times 10^{-3} (T_a - 298.15)^2
\end{eqnarray}
where the dimensionless term $f$ represents the incoming
photosynthetically active radiation on the foliage,
normalized by a species-dependent threshold value:
\begin{eqnarray}
f = 0.55 {R_G \over R_{Gl}} {2 \over LAI}
\end{eqnarray}
Moreover,
$\gamma$ is a species-dependent parameter (see Jacquemin and
Noilhan 1990) and $R_{smax}$ is arbitrarily set to $5000 \ s m^{-1}$.

The surface fluxes of heat, moisture, and momentum can
be expressed as
\begin{eqnarray}
(\overline{w'\theta'})_s &=& {H \over \rho_a c_p T_a / \theta_a} \label{eqn_H} \\
(\overline{w'r'_v})_s &=& {E \over \rho_a (1-q_a)} \label{eqn_LE} \\
|\overline{w'V'}|_s &=& C_D |V_a|^2  = u^2_* \label{eqn_FM}
\end{eqnarray}
where $r_v$ is the water vapor mixing ratio,
$w$ is the vertical motion, $\theta_a$ is the potential
temperature at the lowest atmospheric level.  The primes and
overbars denote perturbation and average quantities.

For the drag coefficients $C_H$ and $C_D$, the formulation of
Louis (1979) was modified in order to consider different
roughness length values for heat $z_0$ and momentum $z_{0h}$
(Mascart et al. 1995):
\begin{eqnarray}
C_D = C_{DN} F_m \ ; \
C_H = C_{DN} F_h
\end{eqnarray}
with
\begin{eqnarray}
C_{DN} = {k^2 \over [ln ( z / z_0) ]^2} \\
\end{eqnarray}
where $k$ is the Von Karmann constant.  Also
\begin{eqnarray}
F_m &=& 1 - {10 Ri \over 1 + C_m
        \sqrt{|Ri|} }  \ \ \ \ \ if \ Ri \leq 0 \\
F_m &=& {1 \over 1 + {10Ri \over \sqrt{1+5Ri}}} \
        \ \ \ \ \ \ \ \ \ \ \ \ \ if \ Ri > 0
\end{eqnarray}
and
\begin{eqnarray}
F_h = \left[ 1-{15Ri \over 1 + C_h \sqrt{|Ri|}} \right]
      \times \left[ {ln(z/z_{0}) \over ln(z/z_{0h})} \right] \
      \ \ \ \ \ \ \ \ if \ Ri \leq 0 \\
F_h = {1 \over 1+15Ri \sqrt{1+5Ri}}
      \times \left[ {ln(z/z_{0}) \over ln(z/z_{0h})} \right] \
      \ \ \ \ \ \ \ \ if \ Ri > 0
\end{eqnarray}
where $Ri$ is the gradient Richardson number.
The coefficients $C_m$ and $C_h$ of the unstable case are given by
\begin{eqnarray}
C_m &=& 10 {C_m}^* C_{DN} (z/z_{0})^{p_m} \\
C_h &=& 15 {C_h}^* C_{DN} (z/z_{0h})^{p_h} \times
        \left[ {ln(z/z_{0}) \over ln(z/z_{0h})} \right]
\end{eqnarray}
where $C^*_m$, $C^*_h$, $p_m$, and $p_h$ are functions of the ratio
$\mu = ln(z_{0}/z_{0h})$ only:
\begin{eqnarray}
C^*_h &=& 3.2165 + 4.3431 \times \mu + 0.5360 \times \mu^2
        - 0.0781 \times \mu^3 \\
C^*_m &=& 6.8741 + 2.6933 \times \mu - 0.3601 \times \mu^2
        + 0.0154 \times \mu^3 \\
p_h &=& 0.5802 - 0.1571 \times \mu + 0.0327 \times \mu^2
        - 0.0026 \times \mu^3 \\
p_m &=& 0.5233 - 0.0815 \times \mu + 0.0135 \times \mu^2
        - 0.0010 \times \mu^3
\end{eqnarray}


\subsection{Summary of Useful Parameters}

The parameters have been chosen in order to characterize the
main physical processes, while attempting to reduce the number
of independant variables.  They can be divided into two
categories:  primary parameters needing to be specified by
spatial distribution, and secondary parameters which values
can be associated with those of the primary parameters.
\\

In the present state of the method,
the primary parameters describe the nature of the land surface
and its vegetation coverage by means of only four numerical
indices:  the percentage of
sand and clay in the soil, the dominant vegetation type,
and the land-sea mask.\\

The secondary parameters associated with the soil type are
evaluated from the sand and clay composition of the soil, according
to the continuous formulation discussed in Giordani (1993) and
Noilhan and Lacarr\`ere (1995) (see Appendix).
These parameters are:
\begin{itemize}
\item the saturated volumetric moisture content $w_{sat}$;
\item the wilting point volumetric water content $w_{wilt}$;
\item the field capacity volumetric water content $w_{fc}$;
\item the slope $b$ of the retention curve;
\item the soil thermal coefficient at saturation $C_{Gsat}$;
\item the value of $C_1$ at saturation (i.e., $C_{1sat}$);
\item the reference value of $C_2$ for $w_2=0.5 w_{sat}$ (i.e., $C_{2ref}$);
\item the drainage coefficient $C_3$  ;
\item the diffusion coefficients $C_{4\,ref}$ and $C_{4b}$ ;
\item and the coefficients $a,p$ for the $w_{geq}$ formulation.
\end{itemize}
On the other hand,
the parameters associated with the vegetation can either
be derived from the dominant vegetation type, or
be specified from existing classification or observations.
They are
\begin{itemize}
\item the fraction of vegetation $veg$;
\item the depth of the soil column $d_2$ (or the root zone depth);
\item the depth of the soil column $d_3$ (if third soil layer option in use);
\item the minimum surface resistance $R_{smin}$;
\item the leaf area index $LAI$;
\item the heat capacity $C_v$ of the vegetation;
\item the $R_{Gl}$ and $\gamma$ coefficients found in the formulation
of the surface resistance $R_s$;
\item and the roughness length for momentum $z_0$ and for heat $z_{0h}$.
\end{itemize}
Other necessary parameters are
\begin{itemize}
\item the albedo $\alpha$
\item the emissivity $\epsilon$.
\item and characteristic time scale for phase changes (currently constant) $\tau_i$.
\end{itemize}


\subsection{Apendix A: Continuous formulation of the soil secondary parameters}
Following Giordani (1993), Noilhan and Lacarr\`ere (1995),
the sand and clay composition (i.e., $SAND$ and $CLAY$) are
expressed in percentage.

\bigskip

The saturated volumetric water content ($m^3 m^{-3}$):
\begin{eqnarray}
w_{sat} =  ( -1.08 SAND + 494.305 ) \times 10^{-3}
\end{eqnarray}

The wilting point volumetric water content ($m^3 m^{-3}$):
\begin{eqnarray}
w_{wilt} = 37.1342 \times 10^{-3} (CLAY)^{0.5}
\end{eqnarray}

The field capacity volumetric water content ($m^3 m^{-3}$):
\begin{eqnarray}
w_{fc} = 89.0467 \times 10^{-3} (CLAY)^{0.3496}
\end{eqnarray}

The slope of the retention curve:
\begin{eqnarray}
b = 0.137 CLAY + 3.501
\end{eqnarray}

The soil thermal coefficient at saturation ($K m^2 J^{-1}$):
\begin{eqnarray}
C_{Gsat} = -1.557 \times 10^{-2} SAND -1.441 \times 10^{-2} CLAY
+ 4.7021
\end{eqnarray}

The value of $C_1$ at saturation:
\begin{eqnarray}
C_{1sat} = (5.58 CLAY + 84.88) \times 10^{-2}
\end{eqnarray}

The value of $C_2$ for $w_2=0.5 w_{sat}$:
\begin{eqnarray}
C_{2ref} = 13.815 CLAY^{-0.954}
\end{eqnarray}

The coefficient $C_3$:
\begin{eqnarray}
C_3 = 5.327 CLAY^{-1.043}
\end{eqnarray}

The coefficient $C_{4b}$:
\begin{eqnarray}
C_{4b} = 5.14 \,+\,0.115 \, CLAY
\end{eqnarray}

The coefficient $C_{4\,ref}$:
\begin{eqnarray}
C_{4\, ref} &=& {2(d_3-d_2)\over (d_2\,{d_3}^2)}
{\rm {log}_{10}}^{-1}
\Bigg[
\beta_0 +
\sum_{j=1}^3 \left( \beta_j \,{SAND}^{\,j} \,+\,
\alpha_j \,{CLAY}^{\,j} \right)
\Bigg]
\end{eqnarray}
%
where the $\beta_j \,(j=0,3)$ coefficients are
$4.42 \times {10}^{-0},\, 4.88 \times {10}^{-3},\,
5.93 \times {10}^{-4}$ and $-6.09 \times {10}^{-6}$.
The $\alpha_j \,(j=1,3)$ coefficients are defined as
$-2.57 \times {10}^{-1} ,\,
8.86 \times {10}^{-3}$ and
$-8.13 \times {10}^{-5}$.

The coefficients for the $w_{geq}$ formulation:
\begin{eqnarray}
a&=&732.42 \times 10^{-3} CLAY^{-0.539} \\
p&=&0.134 CLAY + 3.4
\end{eqnarray}

\subsection{Appendix B: Gaussian formulation for the $C_1$ coefficient}

Following Giordani (1993) and Braud et al. (1993),
for dry soils (i.e., $w_g < W_{wilt}$), the $C_1$ coefficient
in Eq. (13) is approximated by the Gaussian distribution:
\begin{eqnarray}
  C_1(w) = C_{1max} exp \left[ - {(w_g-w_{max})^2 \over 2 \sigma^2} \right]
\end{eqnarray}
In this expression,
\begin{eqnarray}
  C_{1max} &=& (1.19w_{wilt}-5.09) \times 10^{-2} T_s
           + (-1.464w_{wilt} + 17.86) \\
  w_{max}  &=& \eta w_{wilt}
\end{eqnarray}
with
\begin{eqnarray}
  \eta = (-1.815 \times 10^{-2} T_s + 6.41) w_{wilt}
       + (6.5 \times 10^{-3} T_s -1.4)
\end{eqnarray}
and
\begin{eqnarray}
  \sigma^2 = - {W^2_{max} \over 2 ln \left( {0.01 \over C_{1max} } \right) }
\end{eqnarray}


\section{References}

\begin{description}

\item
Anderson, E. A., 1976:
A point energy and mass balance model of a snow cover.
{\it NOAA Tech. Rep. NWS 19}, 150 pp. U.S. Dept. of
Commer., Washington, D.C.

\item
Bhumralkar, C.M., 1975:
Numerical experiment on the computation of ground surface
temperature in an atmospheric general circulation model.
{\em J. Appl. Meteor.}, {\bf 14}, 1246-1258.

\item
Blackadar, A.K., 1976:
Modeling the nocturnal boundary layer.
{\em Proc. Third Symp. on Atmospheric Turbulence,
Diffusion and Air Quality }, Boston, Amer. Meteor. Soc., 46-49.

%\item
%Boone, A.,
%and P. Etchevers, 2001:
%An inter-comparison of three snow schemes of varying complexity coupled
%to the same land-surface model: Local scale evaluation at an Alpine site.
%{\em J. Hydrometeor.}, {\bf 2}, 374-394.

\item
Boone, A.,
J.-C. Calvet and J. Noilhan, 1999:
Inclusion of a third soil layer in a
land-surface scheme using the force-restore method,
{\em J. Appl. Meteor.},
{\bf 38},
1611-1630.

\item
Boone, A.,
V. Masson, T. Meyers, and J. Noilhan, 2000:
The influence of the inclusion of soil freezing
on simulations by a soil-atmosphere-transfer scheme.
{\em J. Appl. Meteor.}, {\bf 39}, 1544-1569.

\item
Braud, I., J. Noilhan, P. Bessemoulin, P. Mascart, R. Haverkamp,
and M. Vauclin, 1993:
Bare-ground surface heat and water exchanges under dry conditions:
Observations and parameterization.
{\em Bound.-Layer Meteror.},
{\bf 66},
173-200.

\item
Deardorff, J.W., 1978:
Efficient prediction of ground surface temperature and moisture
with inclusion of a layer of vegetation.
{\em J. Geophys. Res.},
{\bf 20},
1889-1903.

\item
Deardorff, J.W., 1977:
A parameterization of ground surface moisture content for
use in atmospheric prediction models.
{\em J. Appl. Meteor.},
{\bf 16},
1182-1185.

\item
Dickinson, R.E., 1984:
Modeling evapotranspiration for three dimensional global
climate models.
{\em Climate Processes and Climate Sensitivity.
Geophys. Monogr.},
{\bf 29},
58-72.

\item
Douville, H., 1994:
D\'eveloppement et validation locale d'une nouvelle
param\'etrisation du manteau neigeux.
Note 36 GMME/M\'et\'eo-France.

\item
Douville, H., J.-F. Royer, and J.-F. Mahfouf, 1995:
A new snow parameterization for the French community climate
model.  Part I:  Validation in stand-alone experiments.
{\em Climate Dyn.}, {\bf 12}, 21-35.

\item
Giard, D., and E. Bazile, 1999:
Implementation of a new assimilation scheme for
soil and surface variables in a global NWP model.
{\em Mon. Wea. Rev.}, {\bf 128}, 997-1015.

\item
Giordani, H., 1993:
Exp\'eriences de validation unidimensionnelles du sch\'ema
de surface NP89 aux normes Arp\`ege sur trois sites de la
campagne EFEDA 91.
Note de travail 24 GMME/M\'et\'eo-France.

\item
Jacquemin, B., and J. Noilhan, 1990:
Validation of a land surface parameterization using the
HAPEX-MOBILHY data set.
{\em Bound.-Layer Meteor.},
{\bf 52},
93-134.

\item
Johnsson, H., and L.-C. Lundin, 1991:
Surface runoff and soil water percolation as affected by snow
and soil frost.
{\em J. Hydro.},
{\bf 122},
141-158.

\item
Loth, B.,
H.-F. Graf, and J. M. Oberhuber, 1993:
Snow cover model for global climate simulations.
{\em J. Geophys. Res.},
{\bf 98},
10451-10464.

\item
Louis, J.F., 1979:
A parametric model of vertical eddy fluxes in the atmosphere.
{\em Bound.-Layer Meteor.},
{\bf 17},
187-202.

\item
Lynch-Stieglitz, M., 1994: The development and validation
of a simple snow model for the GISS GCM.
{\em J. Clim.},
{\bf 7},
1842-1855.

\item
Mahfouf, J.-F., J. Noilhan, and P. P\'eris, 1994:
Simulations du bilan hydrique avec ISBA:  Application
au cycle annuel dans le cadre de PILPS.
Atelier de mod\'elisation de l'atmosph\`ere,
CNRM/M\'et\'eo-France,
December 1994, Toulouse, France, 83-92.

\item
Mahfouf, J.-F., and J. Noilhan, 1991:
Comparative study of various formulations of evaporation
from bare soil using in situ data.
{\em J. Appl. Meteor.}, {\bf 9}, 1354-1365.


\item
Mascart, P., J. Noilhan, and H. Giordani, 1995:
A modified parameterization of flux-profile relationships
in the surface layer using different roughness length
values for heat and momentum.
{\em Bound.-Layer Meteor.}, {\bf 72}, 331-344.

\item
Masson, 2000:
A physically-based scheme for the urban energy budget in atmospheric models.
{\it Bound. Layer. Meteor.},  {\bf 94}, 357-397.

\item
Noilhan, J., and P. Lacarr\`ere, 1995:
GCM grid-scale evaporation from mesoscale modeling.
{\em J. Climate}, {\bf 8}, 206-223. 

\item
Noilhan, J., and S. Planton, 1989:
A simple parameterization of land surface processes for
meteorological models.
{\em Mon. Wea. Rev.}, {\bf 117}, 536-549.

\item
Sun, S.,
J. Jin, and Y. Xue, 1999:
A simple snow-atmosphere-soil transfer (SAST) model.
{\em J. Geophys. Res.},
{\bf 104},
19587-19579.

\item
Verseghy, D., 1991:
CLASS - A Canadian land surface scheme for GCMs.
I:  Soil model.
{\em Int. J. Climatol.},
{\bf 11},
111-133.

\end{description}
%%%%%%%%%%%%%%%%%%%%%%%%%%%%%%%%%%%%%%%%%%%%%%%%%%%%%%%%%%%%%%%%%%%%%%%%%%%%%%%
%%%%%%%%%%%%%%%%  END OF THE "Surface Processes Scheme" CHAPTER     %%%%%%%%%%%
%%%%%%%%%%%%%%%%%%%%%%%%%%%%%%%%%%%%%%%%%%%%%%%%%%%%%%%%%%%%%%%%%%%%%%%%%%%%%%%

