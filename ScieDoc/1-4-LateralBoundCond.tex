%%%%%%%%%%%%%%%%%%%%%%%%%%%%%%%%%%%%%%%%%%%%%%%%%%%%%%%%%%%%%%%%%%%%%%%%%%%%%%
% CONTRIBUTION TO THE MESONH BOOK1: "LATERAL BOUNDARY CONDITIONS"
% Author : P. Lafore
% Original : June, 1995
% Last Update   : March, 2000
%%%%%%%%%%%%%%%%%%%%%%%%%%%%%%%%%%%%%%%%%%%%%%%%%%%%%%%%%%%%%%%%%%%%%%%%%%%%%%

\chapter{Lateral Boundary Conditions}
\minitoc

%{\em by J.P. Lafore}

The lateral boundary conditions (l.b.c. hereafter) of Meso-NH
have been designed to offer various possibilities:

\begin{itemize}
\item different types of l.b.c. formulations (cyclic, rigid wall or open)
\item coupling with large-scale fields provided by the ARPEGE or ECMWF models
 (either analyses or forecasts)
\item two-way interactive gridnesting (detailed in the following chapter)
\end{itemize}

\section{Principles and equations}

\subsection{Cyclic boundary condition (CLBCX or CLBCY = 'CYCL')}

This widely used boundary condition is most simple to prescribe. For instance,
giving the periodicity length $L_x$ (corresponding to the
model domain width) for the x-direction, the conditions imposed to all
model variables $\alpha $ are
\begin{equation}
\alpha (x,y,z) = \alpha (x+L_x,y,z) \label{perlbc}.
\end{equation}
It should be noted that this option implies some specific treatments not only
in the routine taking care of l.b.c., but at various places in the model.
First, the normal velocity component at the physical boundaries is explicitly
predicted by the equations.
Second, the pressure solver accounts for this hypothesis by using complex
FFTs, whereas for other types of l.b.c. the solver uses cosine FFTs.
 In fact the domain is physically unbounded in that case, as cyclic
boundary condition assumes that the physical domain is infinitely large.

\subsection{Rigid wall lateral boundary condition ('WALL')}

 For a free-slip rigid wall, a mirror type boundary condition is assumed.
At lateral boundaries, the conditions imposed on the model become:
\begin{eqnarray}
\vec{U}\cdot\vec{n}&=&0
\end{eqnarray}
where $\vec{n}$ is a horizontal unit vector normal to the lateral boundary.
 It results that normal velocity components are zero at the boundary,
whereas all others variables are symmetric relative to the boundary,
\begin{equation}
\dfrac{\partial \alpha }{\partial n }=0.
\end{equation}

\subsection{Wave-radiation open boundary ('OPEN')}

There exists several variants of radiative, or open boundary conditions
in the literature.  For instance, some authors apply the wave-radiation
boundary condition to all prognostic variables
(Xue and Thorpe, 1991), whereas others apply it only to the normal
velocity component (Klemp and Wilhelmson, 1978). For complex cases
with ambient wind shear or with intense convection, the second approach
works better.

For the present version of Meso-NH, we choose the simplest
formulation, that gives from our experience, satisfactory results.
The prognostic variables are separated in two groups: on the one hand, the
scalars, including the tangential velocity components; on the other hand
the normal wind.

\begin{itemize}
\item {\bf Case of scalars}

To compute the advection, the basic need is the prescription of the flux
at the boundary for each scalar. This is done depending
on the sign of the normal velocity component.

\subitem - {\it For outflow boundaries:} fluxes normal to the boundary are
extrapolated from the interior using one-sided upstream differencing.
\subitem - {\it For inflow boundaries:}  each scalar is taken as its
large-scale value $LB$ at that boundary location.

\item {\bf Case of normal wind}

For open l.b.c., normal velocity components $u_n$ are computed by using a
general Sommerfeld equation proposed by Carpenter (1982), already used by
Redelsperger and Lafore (1988):


\begin{equation}
\label{carpenter}
{\dr u_n \over \dr t} =             {\l{{\dr u_n \over \dr t}}\r}_{LB}
- C^*{\l  {\dr u_n \over \dr x}  -  {\l{{\dr u_n \over \dr x}}\r}_{LB}  \r}
- K \, \, {\l u_n - {u_n}_{LB}   \r},
\end{equation}

\noindent
 where the subscript $LB$ stands for large-scale value of the field,
$C^*$ denotes the phase speed of the perturbation field
$u_n - {\l{u_n}\r}_{LB}$, and $K$ is the inverse of a damping time.
The large scale gradient
${\l {\dr u_n} / {\dr x} \r}_{LB}$ and the time evolution
${\l {\dr u_n} / {\dr t} \r}_{LB}$ are specified by the coupling model.
For idealized simulations including no larger-scale effects, they are of
course set to zero.

This formulation allows waves coming from the interior of the model domain
to pass out freely through the boundary with minimal reflection. It also
allows the large scale flow to force the evolution of the inner domain at
large time scales. The last term is a relaxation that has been added to
avoid a slow drift of the solution away from the LB field.

Sophisticated methods may be used to evaluate the phase speed $C^*$,
such as proposed by Orlanski (1976). As a first step however,
we use the simple method proposed by Klemp and Wilhelmson (1978), reading
\begin{equation}
C^*= u_n + C,
\end{equation}
\noindent
where C is an adjustable, constant phase speed (20 to 50 ms$^{-1}$ typically).


\item {\bf Pressure function}

Prescribing the normal velocity components $u_n$ at the boundaries, owing to
the use of a Sommerfeld equation or another assumption (rigid wall one),
infers the knowledge of the normal component of the gradient of the
pressure function. The result is an elliptic equation for the pressure
equation, with Neuman type boundary conditions when 'wall' or 'open'
l.b.c. are used. This is discussed in Chapter 7.

\item {\bf Adjustment for mass conservation}

Since the anelastic continuity equation

\begin{equation}
\label{continuity}
  \dfrac{\partial }{\partial \overline{x} } (\tilde{\rho} U^{c} \; )
+ \dfrac{\partial }{\partial \overline{y} } (\tilde{\rho} V^{c} \; )
+ \dfrac{\partial }{\partial \overline{z} } (\tilde{\rho} W^{c} \; ) =0.
\end{equation}

\noindent
is enforced throughout the domain,
it is necessary to guarantee that it is still verified at the model domain
scale. As the upper and lower boundaries are
assumed to be rigid walls, the total mass flux through the
lateral boundaries should be zero. Due to the use of the general Sommerfeld
equation (Eq. \ref{carpenter}), this requirement is not insured by open lateral boundary
conditions. An adjustment is thus necessary for this l.b.c. type (see section
2.4).


\end{itemize}

\subsection{Combinations of different types of l.b.c.}

 The three types of l.b.c. can be specified independently for each lateral
boundary, with the only obvious limitation that
if the cyclic boundary condition is chosen for one lateral boundary,
the opposing side b.c. must also be cyclic. Note in particular that it is
possible to mix a wall condition on one side and an open condition of the
other side in the same direction.
 The combination of these different types of l.b.c. offers
a wide range of geometrical configurations, such as a close tank, a
channel, etc...

\subsection{The coupling}

One of the most difficult problems inherent to limited area models is the
specification of the boundary conditions, that should achieve
two partially contradictory goals:

\begin{enumerate}
\item evacuate through boundaries the physical (gravity...) and numerical
waves generated by the LAM,
\item prescribe the large-scale evolution provided by the {\bf coupling}
models or analysis.
\end{enumerate}

 The wave-radiation open l.b.c are well designed to solve the first problem,
but also to partly treat the coupling, owing to the use of modification proposed by
Carpenter (1982). Another strategy of coupling used in most hydrostatic models,
is the flow relaxation scheme (Davies, 1976). Prognostic variables $\alpha$ in
a marginal zone near the boundaries are forced to relax towards the large-scale
fields $LB$ on a time scale $1/K$, zero at the boundaries and inwards decreasing,
using:

\begin{equation}
  \dfrac{\partial \alpha}{\partial t} =
 \epsilon (\alpha) - K (\alpha - \alpha_{LB})
\end{equation}

\noindent
where $\epsilon (\alpha)$ represents the sources terms of the $\alpha$ prognostic
equation.

 In practice, to treat the coupling the user has 2 possibilities:
\begin{enumerate}
\item either using the wave-radiation open l.b.c ('OPEN' version),
\item or activating the lateral sponge zones corresponding to the flow
relaxation scheme (see section 6.3 of this book). To get this treatment
the user must also force the phase speed to zero ($C^* = 0$).
 In that case the damping
coefficient at the boundaries $F_{SZ}^{max}$ must be maximum (relative to the
time step). As tested by I. Mallet the shape of the variation of the damping
coefficient in the rim zone may be important to reduce reflections.
\end{enumerate}

 In practice, the coupling can be performed by a combination of the
wave-radiation open l.b.c with the flow relaxation scheme.

 Concerning the large-scale fields $LB$, they are available on MNH files provided by
the user at the segment simulation beginning. The Meso-NH code reads each CouPLing
FILE [CPLFILE(n)] which must span the whole time duration of the segment.
The current large-scale fields are linearly interpolated at each
time step between the closest large-scale states given by the coupling files
[CPLFILE(n) and CPLFILE(n+1)].


\section{Discretization and implementation}

\subsection{Grid structure}

\begin{figure}[pbh]
\psfig{figure=\EPSDIR/horzgrid1.eps}
%\psfig{figure=horzgrid1.eps}
\caption{Horizontal grid structure for HEXT=1  outside the computation domain
\label{horzgrdlbc} }
\end{figure}

Figure (\ref{horzgrdlbc}) shows the horizontal grid structure, including the
outer points. By convention, the physical boundary of the simulation domain
is shown by the
thick line and corresponds to localization of the velocity components normal to
the boundary. To implement the boundary conditions, it is convenient to
define EXTra grid points outside the physical domain, both Horizontally and
Vertically (variables HEXT and VEXT respectively). The number of extra points
in the direction normal to the boundary is parameterized. At present time,
only the "one extra point" case (HEXT=VEXT=1) is implemented, although
provision has been made in many places in the code to facilitate further
implementation of the more general option.

The strategy for coding the l.b.c. has been guided by the idea that
the advection and diffusion terms should be calculated in the corresponding
routines, without making any particular cases at the boundaries.
The adequate values in the boundary zone, are therefore prepared by
specialized routines (BOUNDARIES and RAD\_BOUND). We take advantage of the
fact that the computation of advection uses values at $t$, and the
computation of diffusion uses values at $t-1$: the routine BOUNDARIES and
RAD\_BOUND are called only once per time step, and they set the boundary
points to the values needed by the advection scheme for variables at $t$,
and the values needed by the diffusion scheme for variables at $t-1$. Note
that these values are, in general, different.



\subsection{Cyclic l.b.c. ('CYCL')}

In case of cyclic l.b.c., the periodic lengths $L_x$ and $L_y$ involved in
Eq. (\ref{perlbc}) correspond to the horizontal sizes of the simulation domain
outlined by the thick line on Fig. 1.
For instance, at the $\overline x$ boundaries, the cyclic l.b.c. are written
for the three velocity components and the temperature:
\begin{verbatim}
    DO JEXT=1,JPHEXT
       PRUT  (IIB-JEXT,:,:)   = PRUT  (IIE+1-JEXT,:,:)    !  Left side
       PRVT  (IIB-JEXT,:,:)   = PRVT  (IIE+1-JEXT,:,:)    !  =========
       PRWT  (IIB-JEXT,:,:)   = PRWT  (IIE+1-JEXT,:,:)
       PRTHT (IIB-JEXT,:,:)   = PRTHT (IIE+1-JEXT,:,:)
!
       PRUT  (IIE+JEXT,:,:)   = PRUT  (IIB-1+JEXT,:,:)    !  Right side
       PRVT  (IIE+JEXT,:,:)   = PRVT  (IIB-1+JEXT,:,:)    !  ==========
       PRWT  (IIE+JEXT,:,:)   = PRWT  (IIB-1+JEXT,:,:)
       PRTHT (IIE+JEXT,:,:)   = PRTHT (IIB-1+JEXT,:,:)
    END DO
\end{verbatim}

Note that the I prefix is used above in all indexes to conform to the DOCTOR
norm.

\subsection{Rigid wall l.b.c. ('WALL')}

For instance, at the left $\overline x$ boundary, the rigid wall l.b.c.
reads for the three velocity components and the temperature:
\begin{verbatim}
       PRUT  (IIB     ,:,:)   = 0.
    DO JEXT=1,JPHEXT
       PRUT  (IIB-JEXT,:,:)   =-999                      !  not used
       PRVT  (IIB-JEXT,:,:)   = PRVT  (IIB-1+JEXT,:,:)
       PRWT  (IIB-JEXT,:,:)   = PRWT  (IIB-1+JEXT,:,:)
       PRTHT (IIB-JEXT,:,:)   = PRTHT (IIB-1+JEXT,:,:)
    END DO
\end{verbatim}

\subsection{Wave-radiation open boundary ('OPEN')}

\subsubsection{a. Case of scalars (in routine BOUNDARIES)}

For prognostic variables, except the wind component normal to the lateral
boundary (hereafter subscript b), all source terms can be computed as for
points inside the domain,
except for advection and diffusion in the direction normal to the
boundary, where the computations involve a point outside
of the physical domain.

\begin{itemize}
\item {\bf For outflow boundary at time $t$ and $t - \Delta t$:} gradients
normal to the boundary are
extrapolated from the interior using one-sided upstream differencing. In
fact it is equivalent to extrapolating each scalar at the location of the
first extra point (hereafter subscript $b+1$).

\begin{equation}
\label{inflow}
\alpha^{n}_{b+1} = 2 \alpha^{n}_b -  \alpha^{n}_{b-1}
\end{equation}

 For instance for the left $\overline x$ boundary and for variables at time $t$,
it gives for the two
tangential wind components and the temperature:
\begin{verbatim}
    WHERE ( PRUT(IIB,:,:) <= 0. )               !  OUTFLOW condition
!
      PRVT  (IIB-1,:,:)   = 2.*PRVT  (IIB,:,:)   -PRVT  (IIB+1,:,:)
      PRWT  (IIB-1,:,:)   = 2.*PRWT  (IIB,:,:)   -PRWT  (IIB+1,:,:)
      PRTHT (IIB-1,:,:)   = 2.*PRTHT (IIB,:,:)   -PRTHT (IIB+1,:,:)
\end{verbatim}

\item {\bf For inflow boundary at time $t$ and $t - \Delta t$:}
each scalar $\alpha^{n}$ ($\alpha^{n-1}$) is taken as its large-scale value
$\alpha^{n}_{LB}$ ($\alpha^{n-1}_{LB}$) at the boundary location.

\begin{equation}
\label{inflow}
\alpha^{n}_{b+1} =  {\alpha^{n}_{LB}}_{b+1}
\end{equation}

 For instance for the left $\overline x$ boundary at time t, it gives for the
two tangential wind components and the temperature:

\begin{verbatim}
!
    ELSEWHERE                                   !  INFLOW  condition
      PVT  (IIB-1,:,:) = ZLBXVT   (1,:,:)
      PWT  (IIB-1,:,:) = ZLBXWT   (1,:,:)
      PTHT (IIB-1,:,:) = ZLBXTHT  (1,:,:)
\end{verbatim}

\end{itemize}

\subsubsection{b. Case of normal wind (in routine RAD\_BOUND)}

 A semi-implicit numerical scheme is used to discretize the
radiative equation (\ref{carpenter}), as recommended by Orlanski (1976).
This reads

\begin{equation}
u^{n+1}_{n_b} =  {1-r_L-K \Delta t \over 1+r_L+K \Delta t} u^{n-1}_{n_b}
               + {2 r_L            \over 1+r_L+K \Delta t} u^{n  }_{n_{b-1}}
               + {2 \Delta t       \over 1+r_L+K \Delta t}
                             {\l F^{n}_{LB} +K \, u^{n}_{{LB}_b} \r},
\end{equation}


\noindent
where $n$ is the time level, and
$r_L = C^* \; \Delta  t / \Delta x $ is the normalized phase
velocity, and $F^{n}_{LB}$ the LB forcing term of Eq. \ref{carpenter}. It should
be noted that $r_L$ must be positive to be outwards and results in a stable
numerical scheme.

\subsubsection{c. Adjustment for mass conservation (in routine MASS\_LEAK)}

The adjustment procedure simply consists in the following.
First, compute the total mass flux through lateral boudaries by integrating
the continuity equation (Eq. \ref{continuity}) over the whole domain of simulation:

\begin{equation}
LEAK =
 \left[{ \sum_{j=JB}^{JE} \; \sum_{k=KB}^{KE}
 \dfrac{1}{d_{xx} } \overline{  {\tilde{\rho} } }^{x} u \;
d{\overline{y}} d{\overline{z}} }\right]_{i=IB}^{IE+1}
+\left[{ \sum_{i=IB}^{IE} \; \sum_{k=KB}^{KE}
 \dfrac{1}{d_{yy} } \overline{  {\tilde{\rho} } }^{y} v \;
d{\overline{x}} d{\overline{z}} }\right]_{j=JB}^{JE+1}
\end{equation}

Second, as it is not strictly equal to zero, all normal wind components are
corrected of the same value $u_{stop}$ to eliminate this leak of mass
(positive or negative at left or right lateral boundaries respectively).

\begin{equation}
u_{stop} = {LEAK \over LINMASS}
\end{equation}

\noindent where LINMASS is the mass per unit length
in the direction normal to the open
lateral boundaries (i.e. $\int \int_{S_{open}}\rho ds$).

\begin{eqnarray}
LINMASS  = &
 \left[{ \sum_{j=JB}^{JE} \; \sum_{k=KB}^{KE}
 \dfrac{1}{d_{xx} } \overline{  {\tilde{\rho} } }^{x} \;
d{\overline{y}} d{\overline{z}} }\right]_{i=IB}
+\left[{ \sum_{j=JB}^{JE} \; \sum_{k=KB}^{KE}
 \dfrac{1}{d_{xx} } \overline{  {\tilde{\rho} } }^{x} \;
d{\overline{y}} d{\overline{z}} }\right]_{i=IE+1}
\nonumber \\
+&\left[{ \sum_{i=IB}^{IE} \; \sum_{k=KB}^{KE}
 \dfrac{1}{d_{yy} } \overline{  {\tilde{\rho} } }^{y} \;
d{\overline{x}} d{\overline{z}} }\right]_{j=JB}
+\left[{ \sum_{i=IB}^{IE} \; \sum_{k=KB}^{KE}
 \dfrac{1}{d_{yy} } \overline{  {\tilde{\rho} } }^{y} \;
d{\overline{x}} d{\overline{z}} }\right]_{j=JB+1}
\end{eqnarray}

 The horizontal wind correction $u_{stop}$ is
applied only to open l.b., and LINMASS only accounts for these boundaries.


\section*{References}
\parindent 0cm

\por
Carpenter, K. M., 1982: Note on the paper "Radiation conditions for lateral
boundaries of limited area numerical models".
{\it Quart. J. Roy. Meteor. Soc. }, {\bf 110}, 717-719.

\por
Clark, T. L., 1984: Severe downslope windstorm calculations in two and three
spatial dimensions using anelastic interactive grid nesting: a possible
mechanism for gustiness. {\it J. Atmos. Sci.},  {\bf 41}, 329-350.

\por
Davies, H. C., 1976: A lateral boundary formulation for multi-level prediction
models.
{\it Quart. J. Roy. Meteor. Soc. }, {\bf 102}, 405-418.

\por
Klemp, J. B. and R. B. Wilhelmson, 1978: The simulation of three-dimensional
convective storm dynamics. {\it J. Atmos. Sci.},  {\bf 35}, 1070-1096.

\por
Orlanski, I., 1976: A simple boundary condition for unbounded hyperbolic flows.
{\it J. Comput. Phys.}, {\bf 21}, 251-269.

\por
Redelsperger, J. L. and J. P. Lafore, 1988: A three-dimensional simulation of a
tropical squall-line: Convective organization and thermodynamic vertical
transport. {\it J. Atmos. Sci.},  {\bf 45}, 1334-1356.

\por
Xue, M. and A. J. Thorpe, 1991: A mesoscale numerical model using nonhydrostatic
pressure-based sigma coordinate equations: Model experiments with dry mountain
flows.
{\it Quart. J. Roy. Meteor. Soc. }, {\bf 119}, 1168-1185.

