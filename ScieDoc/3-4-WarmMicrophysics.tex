%%%%%%%%%%%%%%%%%%%%%%%%%%%%%%%%%%%%%%%%%%%%%%%%%%%%%%%%%%%%%%%%%%%%%%%%%%%%%%
% CONTRIBUTION TO THE MESONH BOOK1: "Microphysical Scheme for Warm Clouds"
% Author : Evelyne Richard
% Original : October 15, 1997
% Update   : October 15, 1997
%%%%%%%%%%%%%%%%%%%%%%%%%%%%%%%%%%%%%%%%%%%%%%%%%%%%%%%%%%%%%%%%%%%%%%%%%%%%%%%

\chapter{Microphysical Scheme for Warm Clouds}
\minitoc

%{\em by E. Richard}

\section{Equations}
We note $r_v$, $r_c$, and $r_r$ the water vapor, cloud water and rainwater
mixing ratios, as defined in MESO-NH. For any constituent,
$r$ is the mass of the
constituent divided by the reference mass of dry air.\footnote{Notice that
$\rho_{d\, ref}r = \rho_d \tilde r$ where  $\tilde r$ is the usual mixing ratio
i.e. the mass of the constituant divided by the mass of dry air.}
The consevation equations for these quantities are written:
\begin{eqnarray}
 &\dfrac{d}{dt}(  \rho_{d\,ref} r_v)&= \rho_{d\,ref} (P_{RE}-  P_{CON})\\
 &\dfrac{d}{dt}(  \rho_{d\,ref} r_c)&= \rho_{d\,ref}(- P_{RA} -  P_{RC}+  P_{CON})\\
 &\dfrac{d}{dt}(  \rho_{d\,ref} r_r)&= \rho_{d\,ref} (P_{RA} +  P_{RC} -  P_{RE}
+  P_{RS})
\end{eqnarray}
where $P$ designates the sources and where the subscritps
 $CON$, $RE$, $RA$, $RC$ and $RS$  respectively refer to the following processes:
evaporation/condensation, rain evaporation, accretion of cloud droplets by
raindrops, conversion of cloud droplets into raindrops (autoconversion),
and rain sedimentation.
Condensation/evaporation is a very fast process, it cannot be computed
explicitly and is obtained through an implicit saturation adjustment
procedure, taking subgrid-scale processes into account (see Chapter 13).
All others terms are computed explicitly.


\section{Raindrops characteristics}
\subsection{Distribution}
Raindrops are assumed to follow a Marshall-Palmer distribution. The number of
raindrops whose diameter lies in the interval
$D$ and $D+dD$ is given by:
\begin{equation}
N(D)dD=N_0 \exp (-\lambda D) dD
\end{equation}
Observations for different types of rain show a range of values for
$N_0$ from
$0.4\ 10^7$ to $3.5\ 10^7\ m^{-4}$. $N_0= 10^7\ m^{-4}$
is a value frequently used.


The $\lambda$ parameter is obtained from
\begin{equation}
 \rho_{d\,ref} r_r=\int_0^\infty {\pi \over 6} \rho_{lw} D^3 N_0 \exp(-\lambda D)dD=
{\pi \rho_{lw} N_0 \over \lambda^4},
\end{equation}
which leads to
\begin{equation}
\lambda= \Big({\pi \rho_{lw} N_0 \over \rho_{d\,ref} r_r}\Big)^{1 \over 4}.
\end{equation}
\subsection{Fall velocity}
The terminal fall velocity for a raindrop of diameter $D$ is expressed as
\begin{equation}
V(D)=  \big( \dfrac{ \rho_{00}}{ \rho_{d\,ref}} \big)^{\alpha} aD ^{b},
\label{eqvterm}
\end{equation}
with $\alpha=0.4$, $a=842 m^{0.2}/s$ and $b=0.8$, $\rho_{00}$ being the air
density at the reference pressure level $P_{00}$.
This parameterization follows Liu and Orville (1969) and includes the
effect of mean density variation as suggested by Foote and Du Toit (1969).

The rain terminal fall velocity is then given by
\begin{equation}
V_T \rho_{d\,ref} r_r= \int_0^\infty {\pi \over 6} \rho_{lw} D^3 V(D) N(D)dD,
\end{equation}
which leads to
\begin{equation}
V_T = {a \over 6} \big( {\rho_{00}\over \rho_{d\,ref}}\big) ^\alpha
{\Gamma(b+4)\over \lambda ^b},
\end{equation}
or
\begin{equation}
V_T= {a \over 6} \Gamma(b+4)\big( {\rho_{00}\over \rho_{d\,ref}}\big) ^\alpha
\big( {\rho_{d\,ref} r_r \over \pi \rho_{lw} N_0}\big)
^{b \over 4}.
\end{equation}
\section{Explicit sources}

\subsection{Autoconversion}
The sole rainwater initiation mechanism is the autoconversion process which is
parameterized according to Kessler (1969).
The Kessler rate relies on intuitive considerations:
the autoconversion rate increases linearly with the cloud water content
$(\rho_{d\,ref} r_c)$ but
cloud conversion does not occur below a threshold value $ q_{crit}$.
\begin{equation}
P_{RC}= k\  Max( r_c - \dfrac{q_{crit}}{\rho_{d\,ref}}, 0)
\end{equation}
The parameters $k$ and $q_{crit}$ are
usually set to $10^{-3}\ s^{-1}$ et $0.5g/m^3$.

In the code $P_{RC}$ is written as:
\begin{equation}
P_{RC}= C1_{RC} Max ( r_c - \dfrac{C2_{RC}}{\rho_{d\,ref}}, 0)
\end{equation}
with $C1_{RC}= k$ and $C2_{RC}=q_{crit}$.


\subsection{Accretion}
Once embryonic precipitation particles are formed, rainwater mixing ratio
growth occurs
primarily by accretion of cloud water in the form
\begin{equation}
P_{RA}=\int_0^\infty {\pi \over 4}D^2 V(D) E  r_c N(D)dD,
\end{equation}
where E is the collision efficiency (here taken equal to 1).
After integration,
\begin{equation}
P_{RA}={\pi \over 4}a N_0( \dfrac{ \rho_{00}}{ \rho_{d\,ref}})^{\alpha}
 r_c{\Gamma(b+3) \over \lambda ^{b+3}}.
\end{equation}
After replacing $\lambda$,
\begin{equation}
P_{RA}={\pi \over 4}a N_0( \dfrac{ \rho_{00}}{ \rho_{d\,ref}})^{\alpha}
 r_c  \Gamma(b+3) \big( {\rho_{d\,ref} r_r \over \pi \rho_{lw} N_0}\big)
^{b+3 \over 4}.
\end{equation}
In the code $P_{RA}$ is written as:
\begin{equation}
P_{RA}= C_{RA}( \rho_{d\,ref})^{C'_{RA}-\alpha}  r_c ( r_r )^{C'_{RA}}
\end{equation}
with
$$ C_{RA}={\pi \over 4}a N_0( \rho_{00})^{\alpha} \Gamma(b+3)
\big( {1 \over \pi \rho_{lw} N_0}\big)
^{b+3 \over 4}$$
and $$C'_{RA}={b+3 \over 4}$$

The accretion and autoconversion sources are limited by the amount of available
cloud water. In the case where both processes are simultaneously operating,
accretion is computed first.

\subsection{Rain evaporation}
According to Pruppacher and Klett (1978, p420), the evaporation rate of a drop
of diameter D is given by
\begin{equation}
D {dD \over dt} = {4S \bar f \over \rho_{lw} A}, \label{eq17}
\end{equation}
where $\bar f$ is a ventilation factor and
\begin{eqnarray}
S&=& {r_{vs}-r_v \over r_{vs}}, \\
A&=& {R_v T \over e_s(T) D_v} + {L_v(T) \over k_a T}({L_v(T) \over R_v T} -1)\\
 &\simeq &{R_v T \over e_s(T) D_v} + {L_v(T)^2 \over k_a R_v T^2}  .
\end{eqnarray}

$r_{vs}$ is the saturated vapor mixing ratio, $D_v$ is the diffusivity of water
vapor in air and $k_a$ is the heat conductivity of air.
For simplicity, $D_v$ and  $k_a$ are taken constants:
$D_v=2.26\ 10^{-5} m^2/s$ and $k_a= 24.3 \ 10 ^{-3} J/(msK)$.

$e_s$ is the saturation vapor pressure and is computed according to
\begin{equation}
\label{eq21}
e_s(T)= exp \big( \alpha_w - {\beta_w \over T} - \gamma_w \ln (T) \big),
\end{equation}
with
\begin{eqnarray}
&\alpha_w   &= \ln (e_s(T_t))+{\beta_w \over T_t} - \gamma_w \ln (T_t), \\
&\beta_w   &= {L_v(T_t) \over R_v}\gamma_w T_t, \\
&\gamma_w  &= {C_l -C_{pv} \over R_v}.
\end{eqnarray}

$L_v$ is the latent heat of vaporization and is computed according to
\begin{equation}
L_v(T) = L_v(T_t) + (C_{pv} - C_l)(T-T_t).
\end{equation}
The ventilation factor $\bar f$ is given by
\begin{equation}
\bar f = 1 + F (Re) ^{0.5},\label{eq26}
\end{equation}
where $Re$ is the Reynolds number which can be expressed as
\begin{equation}
Re= {V(D) D \over \nu}, \label{eq27}
\end{equation}
$\nu$  being  the air kinematic viscosity which is here assumed to be constant:
$\nu=0.15 \ 10^{-4} kg/(ms)$.

$F$ is a ventilation coefficient taken equal to 0.22.

After replacing (\ref{eqvterm}) and (\ref{eq27}) in (\ref{eq26}), one gets
\begin{equation}
\bar f =  1 + F \big[ \big( {\rho_{00} \over \rho_{d\,ref}} \big) ^\alpha
{aD^{b+1}  \over \nu} \big] ^{0.5}.
\end{equation}
The integration of (\ref{eq17}) over the rain drop spectrum
leads to  the expression of the evaporation source
\begin{equation}
P_{RE} = {1 \over \rho_{d\,ref}}\int_0^{+\infty} {2 \pi S \bar f \over A} D N(D) dD.
\end{equation}
After replacing $\bar f$,
\begin{equation}
P_{RE} = {2 \pi S N_o \over A}{1 \over \rho_{d\,ref}} \big[ {1 \over \lambda ^2}
+ F ( {\rho_{00} \over \rho_{d\,ref}} )^{\alpha /2} ({a  \over \nu})^{1/2}
{\Gamma ({b+5 \over 2}) \over \lambda ^{{b+5 \over 2}}} \big],
\end{equation}
or
\begin{equation}
P_{RE} = {2 \pi S N_o \over A} {1 \over \rho_{d\,ref}} \big[
\big( {\rho_{d\,ref} r_r \over \pi \rho_{lw} N_o}\big) ^{1 \over 2}
+ F ( {\rho_{00} \over \rho_{d\,ref}} )^{\alpha /2} ({a  \over \nu})^{1/2}
\Gamma ({b+5 \over 2})
\big( {\rho_{d\,ref} r_r \over \pi \rho_{lw} N_o}\big) ^{b+5 \over 8} \big].
\end{equation}
In the code $P_{RE}$ is written as
\begin{equation}
P_{RE}= { S \over A}\big [ C1_{RE} (\rho_{d\,ref}) ^{-{1 \over 2}} (r_r) ^{1 \over 2} +
 C2_{RE} (\rho_{d\,ref} ) ^{C'_{RE}-1- \alpha/2} (r_r ) ^{C'_{RE}} \big],
\end{equation}
with
$$C1_{RE}=2 \pi  N_o ( {1 \over \pi \rho_{lw} N_o}) ^{1 \over 2}$$
$$C2_{RE}=2 \pi  N_o  F \big( \rho_{00} )^{\alpha /2} ({a \over \nu})^{1/2}
\Gamma ({b+5 \over 2})
\big( { 1 \over \pi \rho_{lw} N_o}\big) ^{b+5 \over 8} \big)$$
$$C'_{RE}={b+5 \over 8}$$

The rain evaporation source is limited by the amount of available rainwater.

\subsection{Rain sedimentation}
The sedimentation rate is given by
\begin{eqnarray}
&P_{RS} &= {1 \over \rho_{d\,ref}}{\partial \over \partial z} \int_0^\infty N(D)V(D) {\pi \over 6}
\rho_{lw} D^3 V(D)N(D) dD \\
&&= {1 \over \rho_{d\,ref}}{\partial \over \partial z} (V_T \rho_{d\,ref} r_r).
\end{eqnarray}

In the code, $P_{RS}$ is written as
\begin{equation}
P_{RS} = {1 \over \rho_{d\,ref}}{\partial \over \partial z} \big[ C_{RS}
(\rho_{d\,ref})^{C'_{RS}-\alpha} (r_r )^{C'_{RS}}\big],
\end{equation}
with
$$C_{RS}= {a \over 6} \Gamma(b+4)( \rho_{00}) ^\alpha
\big( {1\over \pi \rho_{lw} N_0} \big)^{b \over 4},$$
$$C'_{RS}={4+b \over 4}.$$

In order to maintain stability, the rain sedimentation source is computed with
a time splitting technique and with  an upstream differencing scheme. The small
time step used for this computation is determined from the CFL stability
criterium based on a maximum raindrop fall velocity $V_{TRmax}$ of 7m/s.

\section{Implicit sources}
Once the explicit sources are computed, the condensation/evaporation rate is
obtained through a saturation adjustment procedure following Langlois (1973).
If $T^*$ and $r_v^*$ are the temperature and vapor mixing ratio obtained after
adding the explicit sources, we seek the zero-crossing of $F(T)$, defined as
\begin{equation}
F(T)= (T-T^*) + {L_v(T) \over C_{ph}} (r_{vs}(T)-r_v^*).
\end{equation}
To obtain a rapidly convergent algorithm, Langlois suggests to use a
generalized
Newton-Raphson procedure which employs the first and second derivatives of $F$:
\begin{equation}
T \simeq T^*- {F(T^*) \over F'(T^*)} \big[ 1+ {F(T^*)F''(T^*) \over 2 F'^2(T^*)}\big].
\end{equation}
The saturated vapor mixing ratio is given by
\begin{equation}
r_{vs}(T)= {\epsilon e_s(T) \over p-e_s(T)},
\end{equation}
where $\epsilon = M_v/M_d$.

According to (\ref{eq21}),
\begin{equation}
e_s'(T)= ( {\beta_w \over T^2} - {\gamma_w \over T}) e_s(T) = A(T) e_s(T).
\end{equation}
$r'_{vs}$ is then given by
\begin{equation}
r'_{vs} = A(T) r_{vs}(T)(1 + {r_{vs}(T) \over \epsilon}).
\end{equation}
It follows:
\begin{equation}
T= T^* - \Delta_1 (1 +{1\over 2} \Delta_1 \Delta_2),
\end{equation}
with
\begin{eqnarray}
\Delta_1&= \dfrac {F(T^*)}{ F'(T^*)}&= {L_v(T) \over C_{ph} + L_v(T)r'_{vs}(T^*)}
         \big[r_{vs}(T^*)-r_v^*\big], \\
\Delta_2&= \dfrac {F''(T^*)}{F'(T^*)}&= {L_v(T) r'_{vs}(T^*) \over C_{ph} + L_v(T) r'_{vs}(T^*)}
         \big[ {A'(T^*) \over A(T^*)} + A(T^*) +
{2 r_{vs}(T^*) \over \epsilon}\big] ,
\end{eqnarray}
and
\begin{eqnarray}
A(T)&=\dfrac{\beta_w}{T^2} - \dfrac{\gamma_w}{T}, \\
A'(T)&=-\dfrac{2\beta_w}{T^3} + \dfrac{\gamma_w}{T^2}. \\
\end{eqnarray}
In the above derivation, the variations of $L_v$ with respect to T are ignored,
being considered much smaller than the variations of $r_{vs}$. Langlois shows
that with this procedure, iteration is unnecessary.

The condensation/evaporation rate is then computed as:
\begin{equation}
P_{CON}= - \Delta_1 (1 +{1\over 2} \Delta_1 \Delta_2){C_{ph}\over L_v(T)}
{1 \over 2\Delta t}
\end{equation}
In the case of evaporation (condensation), $P_{CON}$ is limited by the amount
of available cloud water (water vapor).

\section{Global correction for negative values}
The microphysical sink/sources are computed in such a way they never return
negative values for $r_v$, $r_c$, or $r_r$.
However, in the present version of the code, the advection
scheme is not positive definite. It is therefore necessary to remove all the
negative mixing ratio values before applying the microphysical calculations.
This is currently done inside the microphysical scheme,
by a global filling algorithm based on a multiplicative method
(Rood, 1987). The
negative values of the mixing ratio source distribution are found and
corrected (i.e set equal to zero). The total mass of the corrected distribution
is calculated. Then the corrected distribution is mutiplied grid point by
grid point by the ratio of the mass of the original distribution to the mass of
the corrected distribution.

\section{Practical implementation}

The microphysical constants ($N_o$, $a$, $b$, $\alpha$, $C1_{RC}$, $C2_{RC}$,
$C_{RA}$, $C'_{RA}$, $D_v$, $k_a$, $C1_{RE}$, $C2_{RE}$, $C'_{RE}$, $C_{RS}$,
$C'_{RS}$ and $V_{TRmax}$) are set up in  routine INI\_CLOUD called during the
initialization process.

During the model run,  the computations related to the resolved cloud and rain
parameterization are  monitored by the routine RESOLVED\_CLOUD. When entering
RESOLVED\_CLOUD, the source array $\psi S$ of a variable $\psi$ contains
$$ {\hat \rho \psi^{t-1} \over {2 \Delta t}} + \sum_i S_i(\hat \rho \psi^t) $$
where  $S_i$ designate the previously computed tendencies (i.e. advection,
numerical diffusion, turbulence, ...). $\psi S$ can be interpreted as a guess
of $\hat \rho \psi^{t+1} / 2 \Delta t$.  RESOLVED\_CLOUD computes the
microphysical tendencies and returns updated
values of the source arrays affected by the explicit cloud and rain
parameterization i.e.  $\theta S$, $r_v S$, $r_c S$, and $r_r S$. The main steps
of the scheme are the following

\begin{itemize}
\item
The negative mixing ratios sources ( $r_vS$, $r_cS$, and  $r_rS$)
are corrected according to the global filling algorithm described in 11.5.

\item
The  $\theta$, $r_v$, $r_c$, and $r_r$ sources are divided by $\hat \rho$
to minimize computations in this section.


\item
Routine SLOW\_TERMS is called and proceeds to the computation of the
explicit sources:

\begin{itemize}
\item
Computes the rain sedimentation source $P_{RS}$ and updates the rain
source [$r_r S = r_rS + P_{RS}$].

\item
Computes the  accretion source $P_{RA}$, limits the accretion source
by the amount of cloud water available at this stage
[$P_{RA} = Min (P_{RA}, r_cS)$] and updates the cloud water and rainwater
sources
[$r_c S = r_c S - P_{RA}$ and  $r_r S = r_r S + P_{RA}$].

\item
Computes the  autoconversion source $P_{RC}$, limits the autoconversion
source by the amount of cloud water available at this stage
[$P_{RC} = Min (P_{RC}, r_cS)$] and updates the cloud water and rainwater
sources
[$r_c S = r_c S - P_{RC}$ and  $r_r S = r_r S + P_{RC}$].

\item
Computes the  rain evaporation source $P_{RE}$, limits the rain
evaporation  source by the amount of rainwater available at this stage
[$P_{RE} = Min (P_{RE}, r_rS)$] and updates the water vapor, rainwater, and
potential temperature sources
[$r_v S = r_v S + P_{RE}$,  $r_r S = r_r S - P_{RE}$ and $\theta S = \theta S
- P_{RE} L_v / (\pi_{ref} C_{ph})$].
\end{itemize}

\item
Routine FAST\_TERMS is called and performs the implicit saturation
adjustment:

\begin{itemize}
\item
Computes the condensation/evaporation source $P_{CON}$, limits this
source by  the amount of cloud water (water vapor) available at this stage in
the case of evaporation (condensation) [$P_{CON} = Min (P_{CON}, r_vS)$
or $P_{CON} = Min (P_{CON}, r_cS)$], and updates  the water vapor,
cloud water, and potential temperature sources
[$r_v S = r_v S - P_{CON}$,  $r_c S = r_c S + P_{CON}$ and $\theta S = \theta S
+ P_{CON} L_v / (\pi_{ref} C_{ph})$].
\end{itemize}

\item
The  $\theta$, $r_v$, $r_c$, and $r_r$ sources are multiplied by $\hat \rho$
to go back to the original tendencies.
\end{itemize}

\section{Available options and summary of the ajustable constants}
According to the value of the CLOUD parameter given in  namelist (see The
Meso\_NH user's guide), the
microphysical scheme can be used in three different ways:
\begin{itemize}
\item
CLOUD = 'NONE' :  no microphysics, the water vapor (if present) is computed
as a passive tracer,
\item
CLOUD = 'REVE' : only reversible processes are considered, no rain is generated
(i.e the call to SLOW\_TERM is by-passed),
\item
CLOUD = 'KESS' : the full scheme is operating.
\end{itemize}

Some others parameters might be reasonably modified  by the user in
routine INI\_CLOUD. These are:

$N_o$ the Marshall-Palmer distribution parameter,

$a$, $b$, and $\alpha$ the parameters used in the raindrop fall velocity
expression,

$C1_{RC}$ and $C2_{RC}$, the autoconversion time constant and threshold,

$V_{TRmax}$, the maximum raindrop fall velocity used to ensure stability of
the sedimentation computation.

\section{References}
\parindent 0truecm
\por
Foote, G.B., and P.S. Du Toit, 1969: Terminal velocity of raindrops aloft.
{\it J. Apl. Meteor.}, {\bf 8}, 249-253.
\por
Kessler, E., 1969: On the distribution and continuity of water sustance in
atmospheric circulations. {\it Meteor. Monog.}, {\bf 10}, N$^0$ 32, 84pp.
\por
Langlois, W.E., 1973: A rapidly convergent procedure for computing large-scale
condensation in a dynamical weather model. {\it Tellus}, {\bf 25}, 86-87.
\por
Liu, J.Y, and H.D. Orville, 1969: Numerical modeling of precipitation and cloud
shadow effects on mountain-induced cumuli. {\it J. Atmos. Sci.},{\bf 26},
1283-1298.
\por
Pruppacher, H.R and J.D. Klett, 1978: Microphysics of clouds and precipitation.
Reidel, 714pp
\por
Rood, R.B., 1987: Numerical advection algorithms and their role in atmospheric
transport and chemistry models. {\it Review of Geoph.}, {25}, 71-100.
%%%%%%%%%%%%%%%%%%%%%%%%%%%%%%%%%%%%%%%%%%%%%%%%%%%%%%%%%%%%%%%%%%%%%%%%%%%%%%
%%%%%%%%%%%%%%%%%%%%%%%%%%%%%%%%%%%%%%%%%%%%%%%%%%%%%%%%%%%%%%%%%%%%%%%%%%%%%%
%%%%%%%%%%%%%%%%%%%%%%%%  END OF WARM MICROPHYSICS %%%%%%%%%%%%%%%%%%%%%%%%%%%
%%%%%%%%%%%%%%%%%%%%%%%%%%%%%%%%%%%%%%%%%%%%%%%%%%%%%%%%%%%%%%%%%%%%%%%%%%%%%%
%%%%%%%%%%%%%%%%%%%%%%%%%%%%%%%%%%%%%%%%%%%%%%%%%%%%%%%%%%%%%%%%%%%%%%%%%%%%%%
