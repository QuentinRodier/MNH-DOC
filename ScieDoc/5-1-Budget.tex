%%%%%%%%%%%%%%%%%%%%%%%%%%%%%%%%%%%%%%%%%%%%%%%%%%%%%%%%%%%%%%%%%%%%%%%%%%%%%%%
%%%%%%%%%%%%%%%%%%%%%%%%%%%%%%%%%%%%%%%%%%%%%%%%%%%%%%%%%%%%%%%%%%%%%%%%%%%%%%%
% CONTRIBUTION TO THE MESONH BOOK1: "Budget Analysis"
% Author : J. Nicoleau and J.P. Lafore
% Original : October 15, 1997
% Update   : October 15, 1997
%%%%%%%%%%%%%%%%%%%%%%%%%%%%%%%%%%%%%%%%%%%%%%%%%%%%%%%%%%%%%%%%%%%%%%%%%%%%%%%
%%%%%%%%%%%%%%%%%%%%%%%%%%%%%%%%%%%%%%%%%%%%%%%%%%%%%%%%%%%%%%%%%%%%%%%%%%%%%%%

\chapter{Budget Analysis}
\minitoc

%{\em by J. Nicoleau and J.P. Lafore}

\section{Introduction}

The budget analysis is a major component of the {\bf comprehensive
physical package} offered by the Meso-NH Atmospheric Simulation System. This
analysis is performed in three steps.

\begin{enumerate}
\item In-line computation of the budget for all prognostic variables of the
model. The results are written on a diachronic MNH file.
%\item Post-processing programs read these files, in order
%to derive new budgets of non-prognostic variables (such as kinetic energy,
%potential vorticity,...), or diagnostics (such as apparent source of heat, or
%moisture sink,...). The results are also written on a diachronic MNH file.
\item The results are written on a diachronic MNH file.
\item Finally the graphical package reads these files, and visualizes
the results.
\end{enumerate}

 The budget analysis has been designed to be easily performed. The user has to
prescribe variables for which he wants a budget, processes
to store, and spatio-temporal domain type and characteristics,
on which the budgets are computed. For simulations with several nested models,
the budget analysis has been limited to only one model.

\section{Budget definition}

\subsection{Equations}

 The budget analysis is based on the model equations written in flux form,
discretized on the computational grid 
(see chapter on discretization in Part I).
 For instance for a prognostic variable
$\alpha$, we have the following generic equation;

\begin{equation}
\dfrac{\partial}{\partial t}(\tilde{\rho}\alpha) \, = \, - \,
 \dfrac{\partial }{\partial \overline{x}} (\tilde{\rho} U^{c} \;  \alpha)
\, - \, \dfrac{\partial }{\partial \overline{y}} (\tilde{\rho} V^{c} \;  \alpha)
\, - \, \dfrac{\partial }{\partial \overline{z}} (\tilde{\rho} W^{c} \;  \alpha)
\, + \, \sum_{p=4}^{p_{max}} \; {\cal S}\alpha_p
\end{equation}

\noindent where ${\cal S}\alpha_p$ represents the $\alpha$ source
due to the $p^{th}$ process. The first 3 terms of the r.h.s. of this equation
($\it i.e.$ $p= 1$, $2$ and $3$) correspond to the advection contributions in the
$\overline{x}$, $\overline{y}$ and $\overline{z}$-directions respectively.
The budget analysis simply consists in storing all the sources terms,
necessary to explain the temporal evolution of the $\alpha$ variable.
To check the budget closure, and to physically understand it, it is necessary to
add the storage of variable $\alpha$ at the initial and final instant
of the budget, and of its mean value for this time period.

 At this point, it is important to note that the budget is not performed for the
physical variable $\alpha$, but for $\tilde{\rho}\alpha$
(with $\tilde{\rho}  = \rho_{d\,ref} J$), i.e. the quantity multiplied
by the mass of dry
air contained in the grid box volume. The flux form of the advection equation,
allows the conservation properties of the transport process to be exactly
accounted for.

\subsection{Average operators}

 Such budget analysis quickly generates a huge amount of information, difficult
to handle and to physically understand. Thus in practice, the budgets terms
can be conveniently compressed in different ways, by applying the following
operator:

\begin{equation}
\dfrac{1}{t_2 - t_1} \,  \sum_{t=t_1}^{t_2}
 \,  \sum_{i} \,  \sum_{j} \,  \sum_{k} \,
 \, {\cal P}_{i,j,k}^{t}
 \, \Delta \overline{x} \, \Delta \overline{y} \, \Delta \overline{z}
 \, \Delta t
\end{equation}

\noindent where ${\cal P}$ is for a given process.

\begin{enumerate}
\item Several {\bf processes} can be added to form a single process. For example
the summation of the 3 first processes of Eq. 1, corresponds to the total
advection.

\item The {\bf temporal average}
$\dfrac{1}{t_2 - t_1} \,  \sum_{t=t_1}^{t_2} \, \Delta t$
of the budgets on a time interval ($t_1$ to
$t_2$) longer than a time step, allows to obtain more significant results
for system presenting high temporal fluctuations.

\item The {\bf vertical integration} of the budgets allows understandng of the
behaviour of a column of atmosphere for a given layer ($k_1$ to $k_2$).
 It corresponds to the operator $\sum_{k=k_1}^{k_2} \, \Delta \overline{z}$.

\item The {\bf "horizontal" integration} (performed on the computational levels)
gives significant information for given horizontal subdomain. It can be
performed for
2 types of subdomain; either a cartesian zone, or a complex area defined by
a mask.

\end{enumerate}

\begin{itemize}
\item {\bf Cartesian domain}

 In that case the user defines a cartesian sub-domain, where the budget will be
performed, by setting the horizontal coordinates of the edges of this zone
($i_1$ to $i_2$ and $j_1$ to $j_2$). The horizontal average operator is:

\begin{equation}
\sum_{i=i_1}^{i_2} \, \sum_{j=j_1}^{j_2}
\, \Delta \overline{x} \, \Delta \overline{y}
\end{equation}


\noindent
The budget can be compressed or not in the 2 horizontal directions.
In case of no compression, the resulting budget fields are 3D over the selected
cartesian horizontal zone, whereas 1D budget for this zone is returned in case of total
compression. For some systems characterized by a strong slab symmetric structure
(fronts, convection line,...), a compression in the along-system direction
results in 2D budget fields, well adapted to analyze them.

\item {\bf Defined by a mask}

 For a wide range of systems, it is convenient to analyze budgets over
complex zones, which can temporally evolve. For example a cloud population
simulation can be analyzed, by considering zones with precipitation, and/or
clear air regions... In that case, the user must define the differents zones,
owing to a set of horizontal masks ($mask(i,j)$), refreshed every time
step if necessary. The budgets are horizontally averaged in each zone, so that
1D vertical budget profiles are provided. Thus, the horizontal integration operator is:

\begin{equation}
\sum_{mask \, (i,j)} \, \Delta \overline{x} \, \Delta \overline{y}
\end{equation}

\end{itemize}

\noindent {\bf NB:} In the case of non-flat topography, the budget compression
by the above horizontal operators, results in budgets over volumes following
the non-horizontal computational levels. Thus to obtain budget over horizontal
boxes having a physical meaning, spatial interpolations must be performed
before compression. As a consequence, in case of flow over topography, the
best is to avoid the horizontal compression. The interpolation and the compression
will be therefore performed by the post-processing programs.

%%%%%%%%%%%%%%%%%%%%%%%%%%%%%%%%%%%%%%%%%%%%%%%%%%%%%%%%%%%%%%%%%%%%%%%%%%%%%%%
% END OF CONTRIBUTION "Budget Analysis"
%%%%%%%%%%%%%%%%%%%%%%%%%%%%%%%%%%%%%%%%%%%%%%%%%%%%%%%%%%%%%%%%%%%%%%%%%%%%%%%
