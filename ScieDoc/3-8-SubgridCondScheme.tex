\chapter{Sub-Grid Condensation Schemes}
\minitoc

\section{Sub-Grid Condensation Scheme for warm-phase clouds}

%{\em by J.M. Carriere and P. Bougeault}

\subsection{Introduction}

Sommeria and Deardorff (1977) have first suggested that taking into account
sub-grid condensation in the turbulence parameterization
would improve the physical behaviour of a model:
The numerical shocks between cloudy
and non cloudy grid cells are attenuated, and the vertical turbulent flux
of virtual potential temperature can then be positive in partially
cloudy layers.

The turbulence scheme of Meso-NH has been described in the previous chapter
in the hypothesis of a cloudless atmosphere.  Here we explain how the sub-grid
clouds are represented. This presentation assumes that the reader knows
the gross characteristics of the turbulence scheme. The principle of the
sub-grid cloud scheme is to use {\em conservative forms} of the prognostic
variables describing the entropy and the total non-precipitating water
content. This was first proposed by Betts (1973). As a consequence, there
are only minor modifications to the general turbulence computations.

\subsection{Definition of conservative variables}

The difficulty of treating the source terms for the mixing
ratios $r_*$ due to such processes as evaporation/condensation, accretion
of cloud droplets by raindrops, etc.  , can be considerably reduced by using
quasi-conservative variables.  Let us consider
the equations for $r_v$, $r_c$, and $r_r$:
\begin{eqnarray}
\dfrac{d(\rho_{d\,ref} r_v)}{dt}&=&\rho_{d\,ref}(S_{RE}-S_{CON}), \\
\dfrac{d(\rho_{d\,ref} r_c)}{dt}&=&\rho_{d\,ref}(-S_{RA}-S_{RC}+S_{CON}),\\
\dfrac{d(\rho_{d\,ref} r_r)}{dt}&=&\rho_{d\,ref}
 (S_{RA}+S_{RC}-S_{RE}+S_{RS}),
\end{eqnarray}
where the $S_*$ represent the source terms, with subscripts CON, RE, RA, RC, RS
corresponding respectively to evaporation/condensation, rain evaporation,
accretion of cloud droplets by raindrops, conversion of cloud droplets into
raindrops, and rain sedimentation.
If one defines a new variable
\begin{equation} \label{defrnp}
r_{np} = r_v+r_c,
\end{equation}
the equation of evolution of $r_{np}$ will read
\begin{equation}\label{eqn:rnpev}
\dfrac{d(\rho_{d\,ref} r_{np})}{dt} = \rho_{d\,ref} (S_{RE}-S_{RA}-S_{RC}).
\end{equation}
Note that the standard Meso-NH notations compell to use the subscript
np (for {\it non precipitating}) instead of w, since $r_w$ is already defined
as the sum of all species $r_*$.
$r_{np}$ is the sum of the mixing ratios of water species which {\it accompany
air movements}, as opposed to $r_r$, whose evolution equation involves rain
sedimentation, thus an independent fall velocity, assumed to be much larger
than the air velocity. {\it For the time beeing, we
consider that the turbulent fluctuations of rain are
negligible, and the turbulence scheme will only take into account
the vapour vapor and the cloud liquid water}.
The right-hand side of Eq.~(\ref{eqn:rnpev}) involving only source terms that
have slow characteristic time scales, $r_{np}$ can be considered as
quasi-conservative.  In the future,
the inclusion of ice ($r_i$), snow ($r_s$), or graupels ($r_g$) in Meso-NH
may require careful
attention to the definition of a suitable, quasi-conservative mixing ratio.

Similarly, we may define
\begin{equation} \label{defthetal}
\theta_l = \theta - {\dfrac{L_v}{C_{ph}}
{\Pi_{ref}}^{-1}r_c},
\end{equation}
where $L_v$ is the
latent heat of vaporization of water, $C_{ph}$ is the specific heat at
constant pressure for moist air ($C_{ph} = C_{pd} + r_v C_{pv} +
(r_c+r_r) C_l + (r_i+r_s+r_g+r_h) C_i$),
and $\Pi_{ref}^{-1} = \left(\dfrac{\theta}{T}\right)_{ref}$.  $\theta_l$ is a
quasi-conservative potential temperature, except for the same source terms
that remain in the equation of evolution of $r_{np}$, as can be seen if
we come back to the equation of evolution of $\theta$ in the absence
of moist correction, solid species of water, and diabatic effects due to
radiation and diffusion:  starting from the usual expressions
\begin{eqnarray}
\dfrac{d(\rho_{d\,ref} \theta)}{dt}&=&
{\dfrac{L_v}{C_{ph}}{\Pi_{ref}^{-1}}
\underbrace{\rho_{d\,ref} (-S_{RE}+S_{CON})}_{ =
-\dfrac{d(\rho_{d\,ref} r_{v})}{dt}}}, \\
\dfrac{d(\rho_{d\,ref} r_{c})}{dt}&=&\rho_{d\,ref}
(-S_{RA}-S_{RC}+S_{CON}),
\end{eqnarray}
we deduce
\begin{equation}\label{eqn:thetalev}
\dfrac{d(\rho_{d\,ref} \theta_l)}{dt} = \dfrac{L_v}{C_{ph}} {\Pi_{ref}^{-1}}
\rho_{d\,ref} (S_{RA}+S_{RC}-S_{RE}).
\end{equation}

This shows that except for the slow microphysical terms, the new variables
are approximately as well conserved as $\theta$ and $r_v$ in the absence of
condensation. We take advantage of this to apply the turbulence scheme
described in the previous chapter to $\theta_l$ and $r_{np}$, with a
single modification  linked to the expression of the buoyancy.\\

The course of computations is therefore the following:
\begin{itemize}
\item enter the turbulence scheme with the usual prognostic variables
\item compute conservative variables
\item compute turbulent fluxes of conservative variable formally
as in the cloudless case (except for the
expression of the buoyancy flux $\overline{u_i' \theta'_v}$)
\item diagnose turbulent fluxes in non-conservative variables from
their conservative counterparts
\item come back to non-conservative variables
\item come back to the main model
\end{itemize}

\subsection{Retrieval of cloud water mixing ratio}

As shown by Sommeria and Deardorff (1977), the mean value of $r_c$ can
be diagnosed from the grid scale values of $\theta_{l}$ and $r_{np}$,
and their variances, which are supplied by the general turbulence scheme.
This diagnosis relies
on a simple statistical theory based on the shape of the sub-grid scale
fluctuations histogram.

We first note that whenever cloud is present, the mixing ratio is equal to
the saturation mixing ratio $r_{vs}(\theta,p)$. This can be expressed using
$\theta_l$ and a first-order Taylor expansion as
\begin{equation}
r_v = r_{vs} \left( \theta,p \right) \approx r_{vs}(\theta_l) +
(\dfrac {\partial r_{vs}} {\partial \theta})_{\theta_l}
(\theta-\theta_l).
\end{equation}
Here, the variations of $r_{vs}$ with pressure have been neglected, because
$|\dfrac {\partial r_{vs}} {\partial \theta}\Delta \theta| \gg
|\dfrac {\partial r_{vs}} {\partial p}\Delta p|$.
In the following, we will use the notation
$J= \left( \dfrac {\partial  r_{vs}} {\partial \theta} \right)_{\theta_l} $.
Using the Clausius-Clapeyron relation, we have
$J = \dfrac {r_{vs}(T_l) L_v} {R_v T_l \theta_l} $.
Thus, we get
\begin{equation}
r_c = r_{np} - r_v = r_{np} - r_{vs}(\theta_l)- J (\theta-\theta_l).
\end{equation}
Since by definition, $\theta-\theta_l=\dfrac{L_v}{C_{ph}}\Pi_{ref}^{-1}r_c$,
this gives
\begin{eqnarray}
r_c &=& r_{np}-r_{vs} (\theta_l)- J \dfrac{L_v}{C_{ph}}\Pi_{ref}^{-1}r_c, \\
r_c &=& \dfrac {r_{np}-r_{vs}(\theta_l)} {1+M},
\end{eqnarray}
where $ M= J \dfrac{L_v}{C_{ph}}\Pi_{ref}^{-1} $. \\

On the other hand, in the unsaturated case, $r_c = 0$, thus a
general expression is
\begin{equation} \label{rcdef}
r_c = \mbox{Max} \left(0,\dfrac {r_{np}-r_{vs}(\theta_l)} {1+M} \right).
\end{equation}

The mean cloud water mixing ratio in the grid cell is then obtained by
averaging Eq.~(\ref{rcdef}). For historical reasons, we use the notation
\begin{equation}\label{eqn:s}
s = \dfrac {r_{np}-r_{vs}(\theta_l)} {2(1+M)}.
\end{equation}
$s$ may be seen as a turbulent quantity that controls local saturation within
the grid. Whenever $s \ge 0$, there is saturation. Then,
$s=\overline{s} + s'$, and saturation may be defined as $s'\ge -\overline{s}$.

Note that the standard deviation $\sigma_s$ of $s$ may be easily computed,
since to first order approximation,
\begin{equation}\label{eqn:s'}
s' = \dfrac {r'_{np} - J \theta_l'} {2(1+M)},
\end{equation}
leading to
\begin{equation}\label{eqn:sigma2}
\sigma_s = \dfrac {\left( \overline{r_{np}^{'2}} + J^2\overline{\theta_l^{'2}}-
2J\overline{r'_{np}\theta_l'} \right)^{1\over 2}} {2(1+M)}.
\end{equation}

To compute the statistical average of $r_c$ within the grid, it is useful
to introduce the centered, normalized variable $t= s'/\sigma_s$, and its
probability distribution $G(t)dt$. In terms of $t$, saturation is present
whenever $t \ge -\overline{s} /\sigma_s$. We will therefore define the
quantity
\begin{equation} \label{eqq1}
Q_1=  { \overline{r_{np}} - r_{vs}(\overline{\theta_l})
\over 2(1+M)\sigma_s }.
\end{equation}

The fraction of the grid cell occupied by clouds is then
\begin{equation}\label{eqn:N}
N = \int_{-Q_{1}}^{+\infty} G(t) dt,
\end{equation}
and the mean cloud water content is
\begin{equation}\label{eqn:rc}
\dfrac{\overline{r_c}}{2\sigma_s} = \int_{-Q_1}^{+\infty} (Q_1+t)G(t) dt.
\end{equation}
The problem is therefore solved.

Note that in the case of elongated grids with $\Delta x,\Delta y \gg \Delta z$,
it will be useful to assimilate the cloud fraction $N$ to the cloud {\em
cover}, in the usual meteorological sense. For isotropic grid, the cloud
fraction is more of a three dimensional nature.

\subsection{Flux of liquid water}

We also need to compute the  fluxes $\overline{{u'_{i}}{r'_{c}}}$, where
$u_{i}$ can be any of the three components of the velocity $u_{1}$, $u_{2}$, or
$u_{3}$.  Indeed, these fluxes will be used to compute the buoyancy fluxes
$\overline{{u'_{i}}{{\theta}'_{v}}}$ (appearing in the TKE prognostic
equation), and to retrieve the fluxes of non-conservative variables, once
the fluxes of conservative variables are known.

With the same notations as in the previous section, according to
Bougeault (1981, 1982), we have
\begin{equation}\label{eqn:s'rc'}
\dfrac{\overline{{s'}{r'_{c}}}}{2{{\sigma}_{s}}^2} =
\int_{-Q_{1}}^{+\infty} t(Q_{1}+t)G(t) dt~.
\end{equation}
We assume that this second-order correlation can be used as a generic model
for the $\overline{u_i' r_c'}$ correlations. However, Bechtold et al. (1993)
have shown this to underestimate the real correlation in cases of low cloud
fraction. We therefore introduce an empirical coefficient $\lambda_i$ to
obtain more realistic results:
\begin{equation}\label{eqn:key}
\dfrac{\overline{u'_i r'_c}}{\overline{u'_i s'}} =
\lambda_i \dfrac{\overline{s' r'_c}}{\sigma_s^2}.
\end{equation}

\smallskip
Once the turbulent fluxes of cloud liquid water are known,
 going back to turbulent
fluxes in non-conservative variables is straightforward, because they are
linear combinations of turbulent fluxes in conservative variables.  From
Eqs. (\ref{defrnp}) and (\ref{defthetal}), we obtain readily
\begin{eqnarray}
\overline{{u'_{i}}{{\theta}'}} &=&\overline{{u'_{i}}{{\theta}'_{l}}}  +
\dfrac{L_{v}}{{C_{p}}_{h}}{{\Pi}_{ref}}^{-1} \overline{{u'_{i}}{r'_{c}}}, \\
\overline{{u'_{i}}{r'_{v}}}&=&\overline{{u'_{i}}{r'_{np}}}  -
\overline{{u'_{i}}{r'_{c}}}.
\end{eqnarray}

\subsection{Closure of the scheme}

The closure relies on adequate choices for $G$ and $\lambda_i$.
In their original paper, Sommeria and Deardorff proposed to use by simplicity
a gaussian distribution for $G$. However, Bougeault (1981, 1982) showed
that this would underestimate the cloud fraction in most cases, because
actual distributions are skewed. He proposed several solutions, the
most general being the Gamma probability density. The importance of using
skewed distributions has been confirmed by later work (Cuijpers and
Bechtold 1995).


Using this theory, N, $\dfrac{\overline{r_{c}}}{2{\sigma}_{s}}$, and
$\dfrac{\overline{{s'}{r'_{c}}}}{2{{\sigma}_{s}}^2}$ are then computed from
Eqs. (\ref{eqn:N}), (\ref{eqn:rc}), and (\ref{eqn:s'rc'}) and expressed as:
\begin{eqnarray}
\label{eqn:NF0} N = F_0(Q_1,A_s), \\
\label{eqn:rcF1} \dfrac{\overline{r_c}}{2\sigma_s}=F_1(Q_1,A_s),\\
\label{eqn:s'rc'F2} \dfrac{\overline{s'r'_c}}{2\sigma_s^2}=F_2(Q_1,A_s).
\end{eqnarray}
$Q_1$ has been defined above, and $A_s$ is the skewness of the distribution
of $t$, that requires parametrization.
The detailed expressions of $F_0$, $F_1$, and $F_2$ may be found in Bougeault
(1982).

As far as the coefficients ${\lambda}_{i}$ are concerned, they are unfortunately
not well known when $i = 1$ or $i = 2$.  By simplicity, we will assume
${\lambda}_{1} = {\lambda}_{2} = 1$.  For $i = 3$, the study of
Cuijpers and Bechtold (1995) shows that its value can range from 1 when G is
a gaussian function, to 3 when $Q_{1}$ is of the order of -2, and up to 50 when
$Q_{1}$ is as low as -4.  However, in the later case, these values of $Q_{1}$
are sparse and the dispersion of the numerical value of
${\lambda}_{3}$ obtained after LES 3D simulations for different conditions of
saturation is high, which makes difficult
the fitting of any curve for ${\lambda}_{3}$. \\

The current choice of parameters in Meso-NH is the following:
\begin{enumerate}
\item For 1D or quasi-1D mode,
\begin{itemize}
\item $Q_{1} \geq 0$: $A_{s} = 0$ and ${\lambda}_{3} = 1$ ,
\item $Q_{1} < -2$: $A_{s} = 2$ and ${\lambda}_{3} = 3$ ,
\item $-2 \leq Q_{1} < 0$: $A_{s} = -Q_{1}$ and ${\lambda}_{3} = 1-Q_{1}$ .
\end{itemize}
The values of $A_{s}$ for the first two cases correspond respectively to
a gaussian function and a skewed exponential function.
The latter case is only a linear interpolation of the two previous ones.  In any
of the three previous cases, $\lambda_1=\lambda_2 \equiv 1$.
\item For the LES mode,
\begin{itemize}
\item $A_{s} \equiv 0$ and
$\lambda_1=\lambda_2=\lambda_3 \equiv 1$ in all cases.
\end{itemize}
\end{enumerate}

The expression of the fluxes of cloud water is finally obtained from
Eqs. (\ref{eqn:s'}), (\ref{eqn:key}), and (\ref{eqn:s'rc'F2}), as
\begin{equation}\label{equirc}
\overline{{u'_{i}}{r'_{c}}} = A_{moist} \overline{{u'_{i}}{r'_{np}}} +
A_{\theta} \overline{{u'_{i}}{{\theta}'_{l}}},
\end{equation}
with
\begin{eqnarray}\label{ArAtheta}
A_{moist} = \dfrac {\lambda_i F_2(Q_1,A_s)} {1+M}, \\
A_{\theta} = -J \dfrac {\lambda_i F_2(Q_1,A_s)} {1+M}.
\end{eqnarray}

\subsection{Buoyancy flux}

As mentionned above, the only modification in the turbulence scheme
in presence of saturation is the expression of the virtual potential
temperature. Cloud water increase the density of the parcel, but the
latent heat produced by condensation has an opposite effect. In general
cloudy parcels are therefore more buoyant than clear air parcels. Using
the conservative variables, this is expressed as
\begin{eqnarray}\label{eqn:thetav}
\theta_{v} &=& \theta (1+\delta\dfrac{r_{v}}{1+r_{w}}-
\dfrac{r_{c}}{1+r_{w}}-\dfrac{r_{w}-r_{v}-r_{c}}{1+r_{w}}) \nonumber\\
&=& \left( \theta_{l}+\dfrac{L_{v}}{{C_{p}}_{h}}{{\Pi}_{ref}}^{-1}r_{c} \right)
\left( 1+ \delta \dfrac {r_{np}}{1+r_{w}}-
(1+\delta)\dfrac{r_{c}}{1+r_{w}}-\dfrac{r_{w}-r_{np}}{1+r_{w}} \right).
\end{eqnarray}
The fluctuations of $\theta$ are then given by
\begin{eqnarray}\label{eqn:theta'v}
{\theta}'_{v} &\approx&
\delta \dfrac {\overline{\theta}}{1+\overline{r_{w}}} {r'_{np}}\nonumber\\
&+& \left(\dfrac{L_{v}}{{C_{p}}_{h}}{{\Pi}_{ref}}^{-1}
\left( 1+\delta \dfrac {\overline{r_{np}}}{1+\overline{r_{w}}}-
(1+\delta)\dfrac {\overline{r_{c}}}{1+\overline{r_{w}}}
-\dfrac{\overline{r_{w}}-\overline{r_{np}}}{1+\overline{r_{w}}}\right)-(1+\delta)
\dfrac {\overline{\theta}}{1+\overline{r_{w}}}\right) {r_{c}}'\nonumber\\
&+& \left(1+\delta \dfrac {\overline{r_{np}}}{1+\overline{r_{w}}} -
(1+\delta)\dfrac{\overline{r_{c}}}{1+\overline{r_{w}}}
-\dfrac{\overline{r_{w}}-\overline{r_{np}}}{1+\overline{r_{w}}}\right)
{\theta}'_{l},
\end{eqnarray}
where $\delta = \dfrac{R_{v}}{R_{d}}-1$.  This shows that the expression of
$\overline{{u'_{i}}{{\theta}'_{v}}}$ will include a
$\overline{{u'_{i}}{{r_{c}}'}}$ term.\\

It is convenient to define the following coefficients:
\begin{eqnarray}
A &=& 1+\delta \dfrac {\overline{r_{np}}}{1+\overline{r_{w}}} -
(1+\delta)\dfrac {\overline{r_{c}}} {1+\overline{r_{w}}}-
\dfrac{\overline{r_{w}}-\overline{r_{np}}}{1+\overline{r_{w}}}, \\
B&=&\delta \dfrac {\overline{\theta}} {1+\overline{r_{w}}}, \\
C&=&\dfrac{L_{v}}{{C_{p}}_{h}}{{\Pi}_{ref}}^{-1}
\left(1+\delta \dfrac {\overline{r_{np}}} {1+\overline{r_{w}}} -
(1+\delta)\dfrac {\overline{r_{c}}} {1+\overline{r_{w}}}-
\dfrac{\overline{r_{w}}-\overline{r_{np}}}{1+\overline{r_{w}}} \right)
 - (1+\delta) \dfrac {\overline{\theta}} {1+\overline{r_{w}}} \nonumber \\
&=&\dfrac {L_{v}} {{C_{p}}_{h}} {{\Pi}_{ref}}^{-1} A -
\dfrac{1+\delta}{\delta} B.
\end{eqnarray}

With these notations, the buoyancy flux
$\overline{{u'_{i}}{{\theta}'_{v}}}$ can be written
\begin{equation}\label{eqn:u'itheta'v}
\overline{{u'_{i}}{{\theta}'_{v}}} = A \overline{{u'_{i}}{{\theta}'_{l}}} +
B \overline{{u'_{i}}{r'_{np}}} + C \overline{{u'_{i}}{r'_{c}}}~.
\end{equation}

A more compact expression may be obtained by use of eq.~(\ref{equirc}):
%PM/02/07/2001 Modif P. Marquet
%\begin{equation} \label{ErEtheta}
%\overline{{w'}{{\theta}'_{v}}} = E_{\theta} \overline{{w'}{{\theta}'_{l}}} +
%E_{moist} \overline{{w'}{r'_{np}}}~,
%\end{equation}
\begin{equation} \label{ErEtheta}
\overline{{u'_{i}}{{\theta}'_{v}}} = E_{\theta} \overline{{u'_{i}}{{\theta}'_{l}}} +
E_{moist} \overline{{u'_{i}}{r'_{np}}}~,
\end{equation}
where $E_{\theta} = A + C A_{\theta}$ and $E_{moist} = B + C A_{moist}$.


\subsection{Modification to the mixing length}


%>>>>>>>>>>>>>>>>>>>>>>>>>>>>>>>>>>>>>>>>>>>>>>>>>>>>>>>>>>>>>>>
%>>>>>>>>>>>>>>>>>>>>>>>>>>>>>>>>>>>>>>>>>>>>>>>>>>>>>>>>>>>>>>>
%>>>>>>>>>>>>>>>>>>>>>>>>>>>>>>>>>>>>>>>>>>>>>>>>>>>>>>>>>>>>>>>
%>>>>>>>>>>>>                                     >>>>>>>>>>>>>>
%>>>>>>>>>>>>    CETTE SECTION EST A REVOIR       >>>>>>>>>>>>>>
%>>>>>>>>>>>>                                     >>>>>>>>>>>>>>
%>>>>>>>>>>>>>>>>>>>>>>>>>>>>>>>>>>>>>>>>>>>>>>>>>>>>>>>>>>>>>>>
%>>>>>>>>>>>>>>>>>>>>>>>>>>>>>>>>>>>>>>>>>>>>>>>>>>>>>>>>>>>>>>>
%>>>>>>>>>>>>>>>>>>>>>>>>>>>>>>>>>>>>>>>>>>>>>>>>>>>>>>>>>>>>>>>
%En principe c'est fait! PM 27 juin 2001

There are three different ways of computing the mixing length in Meso-NH.
\begin{itemize}
  \item  For the LES mode, the mixing length is usually set
equal to $\left(d_{xx}d_{yy}d_{zz}\right)^{1/3}$.  In this case, no change is
necessary to account for SGC.
  \item  In the case of a K-$\epsilon$ turbulence scheme, condensation
processes will be taken into account implicitely in the length scale
when working in conservative variables.
  \item  Finally, when the Bougeault-Lacarr\`ere
scheme is adopted, one has to use the
potential temperature
$\theta_{v} = \theta \left(\dfrac{1+\delta r_{v}}{1+r_{w}}\right)$ to obtain
the level, either upward or
downward, where all the TKE has been transformed into potential energy due
to buoyancy, yielding:
%PM 27 juin 2001 Correction signe e(z) ci-dessous
$$- \: \overline{e}(z) =
\int_{z}^{z+l_{up}} \beta
({\overline{\theta_{v}}}_{p}(z')-{\overline{\theta_{v}}}_{e}(z')) d z' =
\int_{z-l_{down}}^{z} \beta
({\overline{\theta_{v}}}_{e}(z')-{\overline{\theta_{v}}}_{p}(z')) d z'~,$$
where the subscripts $e$ and $p$ refer respectively to the environment and
to the particle which has left its initial height.
\end{itemize}
If the transformation is adiabatic, when a particle leaves its initial height,
then:
\begin{equation}
{\overline{\theta_{v}}}_{p}(z') = {\overline{\theta}}_{p}(z')
\left(\dfrac{1+\delta{\overline{r_{v}}}_{p}(z')}
{1+{\overline{r_{w}}}_{p}(z')}\right)~,
\end{equation}
\begin{equation}
{\overline{\theta_{v}}}_{e}(z') = {\overline{\theta}}_{e}(z')
\left(\dfrac{1+\delta{\overline{r_{np}}}_{e}(z')}
{1+{\overline{r_{w}}}_{e}(z')}\right)~,
\end{equation}
where ${\overline{\theta}}_{p}(z')$, ${\overline{r_{v}}}_{p}(z')$, and
${\overline{r_{w}}}_{p}(z')$ account for the change of phase of water when the
particle is lifted from height $z$.


\subsection{Spatial Discretization}

The discretization in space follows from the general turbulence scheme.
The cloud fraction, the cloud mixing ratio, the conservative variables and
the variances are all located at mass points. The application of the
theory is therefore straightforward.

On the other hand, the fluxes are located at $u_i$ points. Thus,
application of Eq.~(\ref{equirc}) requires an average of
$A_{r}$ and $A_{\theta}$ in the direction $i$.
This reads:
\begin{equation}
\overline{{u'_{i}}{r'_{c}}} = {\overline{A_{r}}}^{x_{i}} \overline{{u'_{i}}{r'_{np}}} +
{\overline{A_{\theta}}}^{x_{i}} \overline{{u'_{i}}{{\theta}'_{l}}}.
\end{equation}

%The buoyancy flux ....


\subsection{Practical implementation}


As this method is providing a diagnostic estimate of
the cloud water mixing ratio,
it substitutes to the condensation adjustment procedure.  This could be done in
the turbulence scheme very conveniently.  However, the general code organization
would become very complicated, and different wether the SGC is used or not.
Therefore, we decided not to mix the turbulence scheme and the condensation
adjustment in one routine and to adhere to the usual order
of call of the routines inside one model time step,
where the condensation adjustment
is the last routine called before the end of the time step.

The code organization follows from these considerations. It is understood
that the objective of the time step is to compute values at $t+1$.

\begin{enumerate}

\item The input variables to the turbulence scheme are the usual prognostic
variables at $t$ and $t-1$, their sources, and the value of $\lambda_3
\overline{s'r_c'} / 2 \sigma_s^2 \equiv \lambda_3 F_2(Q_1,A_s)$,
saved from the previous time step.

\item
In routine TURB, the conservative variables and their sources are computed
from the input variables, through Eqs. (\ref{defrnp}) and (\ref{defthetal}).

\item
The computation of turbulent fluxes follows as described in the previous
chapter. However, when the buoyancy flux is needed, the formulation of the
$E_{\theta}$ and $E_{moist}$ coefficients uses eq.~(\ref{ErEtheta})
instead of the formulation given in the previous chapter. Note that the
two formulations coincides in the total absence of cloud. This computation
is based on the value of $\lambda_3 \overline{s'r_c'}/2\sigma_s^2$.

\item
The fluxes of cloud water $\overline{u_i'r_c'}$ are diagnosed
through Eq.~(\ref{equirc}).
This uses also Eq.~(\ref{ArAtheta}). Note that only $\lambda_3 F_2$ is transmitted
to the routine to save ressources. This trick is possible because in the
present version of the scheme, $\lambda_1$ and $\lambda_2$ are not used
in the 1DIM case, and equal to $\lambda_3$ in the 3DIM case.

\item
The divergence of $\overline{u_i'r_c'}$ is used to evaluate the source
of $r_c$ due to the
turbulent exchanges, at the exclusion of the condensation/evaporation process.

\item The value of $\sigma_s^t$ is computed, and is an
output of the turbulence scheme.

\item
At the end of the TURB routine, we get back to the usual form of the
temperature, the water mixing ratio, and their sources. Note that the
cloud water mixing ratio $r_c$ is not modified during the turbulence
computation, but its source now includes the effect of turbulent transport,
as do the sources of the potential temperature and water vapor.


\item At the end of the model time step, routine FAST\_TERMS receives
as input the non-adjusted value of the prognostic variables at time $t+1$.
We shall call
these $\theta^*$, $r_v^*$, and $r_c^*$. It also receives $\sigma_s^t$.

\item FAST\_TERMS computes the final value of the conservative variables
by applying Eqs. (\ref{defrnp}) and (\ref{defthetal}) to $\theta^*$, $r_v^*$,
and $r_c^*$.
This does not involve any approximation, since the conservative variables
are conserved by definition in the saturation adjustment.

\item It computes $Q_1$ by applying Eq.~(\ref{eqq1}) with the final values of
the conservative variables, and $\sigma_s^t$. This step thus assimilates
$\sigma_s^t$ to $\sigma_s^{t+1}$. This is done in order to save computer
ressources. The alternative would be to carry a large number of additionnal
variables, or to redo part of the turbulence computation. This is a price
to pay for clarity and efficiency of the code.

\item The adjusted value $r_c^{**}$ is obtained from Eq.~(\ref{eqn:rcF1}), again
assimilating $\sigma_s^t$ to $\sigma_s^{t+1}$. At the same time,
$\lambda_3 \overline{s'r_c'} / 2 \sigma_s^2$ is
computed and saved for the next time step.

\item Finally, the adjusted value of the non-conservative, prognostic
variables at time $t+1$ are obtained through

\begin{equation}\label{eqn:rct+1}
{\overline{r_{c}}}^{t+\Delta t} = {\overline{r_{c}}}^{**}
\end{equation}
\begin{equation}\label{eqn:thetat+1}
{\overline{\theta}}^{t+\Delta t} - {\overline{\theta}}^{*} =
\dfrac{L_{v}}{{C_{p}}_{h}}{{\Pi}_{ref}}^{-1}
\left( {\overline{r_{c}}}^{**} - {\overline{r_{c}}}^{*} \right)
\end{equation}
\begin{equation}\label{eqn:rvt+1}
{\overline{r_{v}}}^{t+\Delta t} - {\overline{r_{v}}}^{*} =
-\left( {\overline{r_{c}}}^{**} - {\overline{r_{c}}}^{*} \right)
\end{equation}

\end{enumerate}

%%%%%%%%%%%%%%%%%%%%%%%%%%%%%%%%%%%%%%%%%%%%%%%%%%%%%%%%%%%%%%%%%%%%%%%%%%%%%%
\section{Sub-Grid condensation scheme for ice-phase and convective clouds}

% Authors : Peter Bechtold and Jean-Pierre Chaboureau
% Original : Mai 21, 2002 
%%%%%%%%%%%%%%%%%%%%%%%%%%%%%%%%%%%%%%%%%%%%%%%%%%%%%%%%%%%%%%%%%%%%%%%%%%%%%%

\subsection{Introduction}
The application of a statistical sub-grid condensation scheme to deep convective
tropospheric clouds and to upper tropospheric stratiform clouds requires a
couple of extensions to the warm-phase scheme presented in the previous chapter.
These extensions include i) the representation of ice phase and mixed-phase clouds
ii) a definition of statistical cloud relations that approximately hold for all types of
tropospheric clouds, and iii)
 a proper definition of sub-grid variance including a convective contribution.
These specific developments are described below and closely follow the
M\'eso-nh Cloud Resolving Model (CRM) study by Chaboureau and Bechtold (2002).
However, all issues concerning the coupling of the sub-grid condensation scheme
with the M\'eso-nh prognostic thermodynamic framework (section "practical implementation")
as described in the previous chapter keep unchanged.

\subsection{Thermodynamic framework}
The properties of a moist adiabatically ascending air parcel are conveniently
expressed assuming conservation (in the absence of precipitation) of enthalpy
or "liquid water static energy" $h_{l}$  and total
water mixing ratio $r_w$ 
\begin{eqnarray}
h_{l}&=& C_{pm} T - L_v r_c - L_s r_i + ( 1 + r_w ) g z\label{eqh}\\
r_w&=&r_v+r_c+r_i\label{eqr}, 
\end{eqnarray}
where the specific heat of moist air is defined as
$C_{pm}= C_{pd} + r_w C_{pv}$, $L_v$ and $L_s$ are the specific latent heats
of vaporization and sublimation, $g$ denotes the gravitational acceleration,
$z$ is height, and $r_v, r_c$ and $r_i$ denote the mixing ratios of water
vapor and non-precipitating cloud water/ice, respectively.
We further define a "liquid" temperature as
\begin{eqnarray}
T_l & = & T - L_v/C_{pm}\, r_c - L_s/C_{pm}\, r_i \nonumber\\
 & = & (h_{l}- ( 1 + r_w ) g z)/C_{pm}\label{Tl},
\end{eqnarray}
and combine the moisture and temperature effects to one single variable
$s=a r_w - b T_l$ (see e.g. Mellor 1977) with
\begin{eqnarray}
\lefteqn{
a= (1+L r_{sl}/C_{pm})^{-1},\quad\quad\quad b=a\,\, r_{sl},}
\nonumber \\ &
r_{sl}= \partial r_{sat}/\partial T(T=T_l)=L r_{sat}(T_l)/(R_v T_l^2).
\end{eqnarray}
Here $L$ and $r_{sat}$ are the latent heat and water vapor saturation mixing
ratio that inside a given glaciation interval $T_0>T>T_1$ are linearly
interpolated as a function of temperature between their respective values  for
liquid water and ice, i.e. $L=(1-\chi)L_v+\chi L_s$,  $r_{sat}=(1-\chi)r_{satw}+\chi r_{sati}$,
with $\chi=(T_0-T)/(T_0-T_1)$, 
$T_0$ =273.16 K, and $T_1$=253 K. $r_{satw}$ and $r_{sati}$ are the saturation mixing ratios
over water and ice, respectively.

Finally, with the above definitions $Q_1$ is expressed as the saturation 
deficit of the ensemble or grid average (denoted by overbars) normalized by
$\sigma_s$, the variance of $s$,
with primes denoting deviations from the ensemble (grid) mean
\begin{eqnarray}
\lefteqn{ Q_1=\overline{a}\, (\overline{r}_w-r_{sat}(\overline{T}_l))/\sigma_s,
}\nonumber \\ &
\sigma_s=(\overline{a}^2\,\overline{r_w^{\prime 2}}-
2\overline{a}\overline{b}\,\,\overline{r_w^\prime T_l^\prime}+\overline{b}^2 \,
\overline{T_l^{\prime 2}})^{1/2}.
\label{eqq1}
\end{eqnarray}

\subsection{Fractional cloudiness and cloud condensate}
As shown in Chaboureau and Bechtold (2002) using CRM data,
the cloud fraction $N$ and the
grid-mean condensate mixing ratio $r_l=r_c+r_i$ of tropospheric clouds
can be represented by the
following relations that have been earlier established for boundary-layer clouds
\begin{equation}
N =\max \{0, \min [1, 0.5 + 0.36 \arctan (1.55 Q_1)]\}
\label{eqN}
\end{equation}
\begin{equation}
\begin{array}{lll}
{\overline{r}_l\over\sigma_s}  = & e^{( 1.2Q_1-1 )}, & Q_1< 0. \\
& & \\
{\overline{r}_l\over\sigma_s}  = & e^{-1}+0.66 Q_1 +0.086Q_1^2 &0\le Q_1\le 2\\
& & \\
{\overline{r}_l\over\sigma_s}  = & Q_1 &Q_1>2.\\
\end{array}
\label{eqrl}
\end{equation}

The respective mixing ratios for cloud water and cloud ice  are then retrieved
using $r_c=(1-\chi)\,r_l$, and $r_i=\chi\,r_l$.

\subsection{Parameterization of $\sigma_s$}

The practical application of the cloud relations [Eqs. (\ref{eqN}) and (\ref{eqrl})]
in meteorological models requires the knowledge of the second-order moment
$\sigma_s$. This quantity is provided by the M\'eso-nh turbulence scheme and used
as that in the warm-phase sub-grid condensation scheme.

However, it turned out that this formulation is not yet well adapted
to deep and upper tropsopheric clouds as it generally has zero value above
the boundary-layer.
Here, a simple parameterization is suggested representing the total variance of $s$
as a sum of a turbulent and  a convective contribution
\begin{equation}
\sigma_s^2=\sigma_{sturb}^2+\sigma_{sconv}^2
\end{equation}

\subsubsection{Turbulent contribution}

The turbulent contribution is either the value computed by the turbulence scheme
(possible option in the computer code) or (present default option) expressed as

\begin{equation}
\sigma_s  = c_\sigma\,l\, \left[
\overline{a}^2\,\left({\partial\overline{r}_w\over\partial z}\right)^2
-  2\overline{a}\overline{b}\, C_{pm}^{-1}
{\partial\overline{h}_l\over\partial z}
{\partial\overline{r}_w\over\partial z}
 + \overline{b}^2 \, C_{pm}^{-2}
\left({\partial\overline{h}_l\over\partial z}\right)^2\right]^{1/2},
\label{eqsp}
\end{equation}
\noindent
where $C_{pm}^{-1}\,\partial h_l/\partial z=\partial T_l/\partial z\,\,+g/C_{pm} (1+r_w)$, 
$l$ is a constant length-scale of value $l_0=600$ m  (for $z>l_0$; $l=z$ for $z<=l_0$), and
$c_\sigma$ is a constant of value 0.2 (Cuxart et al. 2000).

\subsubsection{Convective contribution}

Using the top-hat approximation, the variance and the flux of a quantity $s$ can be
expressed as (see e.g. Lappen and Randall  2001)
\begin{equation}
\overline{s^{\prime 2}}=\,N\,(1-N) (s^c -s^e)^2\label{var1},
\end{equation}
\noindent where $N$ is the cloud fraction,  and
where $c$ and $e$ denote cloud and environmental values, respectively. 
However, as noted by the authors and Siebesma (personal communication) the top-hat
approximation reasonably represents  the convective fluxes but not the variances, and is also sensible
to the decomposition chosen, i.e. cloud/environment or updraft/downdraft.
Therefore, instead of Eq.~(\ref{var1}) we seek a simple expression as a function of
the convective mass flux (a quantity that is readily available from the mass flux convection
parameterization):
\begin{equation}
\sigma_s^{conv}=\overline{s^{\prime 2}}^{1/2}\approx {\overline{w^\prime s^\prime }\over w^*}
\approx M\,\,{(s^c -s^e)\over w^* \rho^*}\label{var2}
\end{equation}

\noindent where $M=\overline{\rho}\,\, N\,w$ is the convective
mass flux (kg s$^{-1}$ m$^{-2}$), $w^*$ a convective scale velocity and 
$\rho^*$ a tropospheric density scale. Finally, Eq.~(\ref{var2}) is further simplified to
\begin{equation}
\sigma_s^{conv}\approx  M\,\,{(s^c -s^e)\over w^* \rho^*} \approx \alpha\,\,M\,f(z/z^*)\label{var3}
\end{equation}
where  a vertical scaling function $f$ has been introduced.
It turns out that $\sigma_s$ is mainly determined (especially in the tropics and the upper troposphere)
by the moisture variance. For simplicity
  the scaling function is set to $f=a^{-1}$ so that it is proportional to the
saturation mixing ratio.


Finally, the  value of the proportionality coefficient $\alpha$ can be obtained by minimizing the function
$(\sigma_s^2-\sigma_{s  turb}^2-\sigma_{s  conv}^2)^2$, where $\sigma_s^2$ are the  values
derived from convective data sets
and where $\sigma_{s turb}$ and $\sigma_{s conv}$ are replaced by
Eqs. (\ref{eqsp}) and (\ref{var3}), respectively.  A value $\alpha=3\times 10^{-3}$ is obtained
using the cloud mass flux from the CRM.


\section{Treatment of mixed-phase clouds}

\subsection{Definition of new conservative variables}

The standard Meso-NH sub-grid condensation scheme computes the turbulent fluxes
in a moist atmosphere assuming a reversible condensation/evaporation process 
for non-precipitating cloud droplets. The inclusion of the ice phase leads to
a generalization of the scheme to mixed-phase clouds. This is done using new
quasi-conservative variables. First, the total {\it non precipitating} mixing 
ratio is redefined by
\begin{equation} \label{newdefrnp}
r_{np} = r_v+r_c+r_i,
\end{equation}
where the additional term $r_i$ is the non-precipitating pristine ice mixing 
ratio. Second, the ice-liquid potential temperature, $\theta_{il}$, is 
substituted for
$\theta_l$ which is only applicable to warm clouds 
\begin{equation} \label{defthetail}
\theta_{il} = \theta - {\dfrac{L_v}{C_{ph}}{\Pi_{ref}}^{-1}r_c} 
                   - {\dfrac{L_s}{C_{ph}}{\Pi_{ref}}^{-1}r_i},
\end{equation}
where $L_s$ is the latent heat of sublimation of ice and with the usual
definition of $C_{ph}$. Note that $r_{np}$ and $\theta_{il}$ still revert to 
$r_v$ and $\theta$ in a cloudless atmosphere so the moist turbulence scheme 
can be employed with only slight modifications to account for the presence of 
the ice phase. 

\subsection{Some weighted thermodynamical functions}

In most of the cases, a simple weighting function 
$f_{ice}=\dfrac {r_i}{r_c+r_i}$ is sufficient to update some ancillary 
thermodynamical functions from their "warm cloud" definition. For instance the 
mixed-phase saturation mixing ratio is defined by the barycentric formula
$r_{vs}({T})=(1-f_{ice})r^l_{vs}({T}) +f_{ice} r^i_{vs}({T})$
with ${r^l_{vs}}(T)$ and ${r^i_{vs}}(T)$ being the saturation mixing ratios over
water and ice, respectively. Consequently, the function 
$J=\left( \dfrac {\partial  r^l_{vs}} {\partial \theta} \right)_{\theta_{il}}$
which is introduced in the warm cloud case ($J \equiv J_l$) becomes
\begin{equation} \label{newdefJ}
J \equiv J_{il}=(1-f_{ice}) J_l + f_{ice} J_i,
\end{equation}
where $J_i=\dfrac {r^i_{vs}(T_{il}) L_s} {R_v T_{il} \theta_{il}} $. 
The same approximation is made for the function $M$, for the 
stability functions $\phi_i$ and $\psi_i$, and hence for the resulting
$A_{moist}$ and $A_{\theta}$ functions. We recall that the assumption of 
saturation mixing ratio is always made when a cloud is present (although this 
may not be accurate in the case of cold ice clouds where a significant degree
of supersaturation over ice can be found), so:
\begin{equation}
r_c + r_i = \dfrac {r_{np}-r_{vs}(\theta_l)} {1+M},
\end{equation}
with a new definition of $M$ which can be derived analytically as
$ M \equiv M_{il} = J_{il} \dfrac{L_v (1-f_{ice})+L_s f_{ice})}{C_{ph}}\Pi_{ref}^{-1} $. \\


\subsection{Ice and liquid fluxes}

Then we need to compute the divergence of the  turbulent fluxes 
$\overline{{u'_{i}}{r'_{c}}}$ and
$\overline{{u'_{i}}{r'_{i}}}$, where $u_{i}$ can be any of the three components
of the velocity. This is done using the same theory developed in the pure warm 
cloud case but with $r'_{(c+i)}=r'_{c}+r'_{i}$
\begin{equation}\label{eqn:s'rc+i'}
\dfrac{\overline{{s'}{r'_{(c+i)}}}}{2{{\sigma}_{s}}^2} =
\int_{-Q_{1}}^{+\infty} t(Q_{1}+t)G(t) dt~,
\end{equation}
and with an actualized definition of 
$s' = \dfrac {r'_{np} - J \theta_{il}'} {2(1+M)}$ which now incorporates both 
$r'_c$ and $r'_i$ fluctuations. $Q_1$ still has its usual meaning. Similar 
characteristics of the probability distribution $G(t)dt$ are implicitly assumed
for liquid and mixed-phase clouds. The simplest way to partition the liquid and
ice turbulent fluxes of condensed material is to consider the following closure
\begin{equation}
\overline{{u'_{i}}{r'_{(c+i)}}}=\lambda_i \dfrac{\overline{u'_i s'}\;.\;\overline{s' r'_{(c+i)}}}{\sigma_s^2} = \overline{{u'_{i}}{r'_{c}}}+\overline{{u'_{i}}{r'_{i}}}
\end{equation}
with $\overline{{u'_{i}}{r'_{c}}} = (1-f_{ice})\overline{{u'_{i}} r'_{(c+i)}}$ 
and
with $\overline{{u'_{i}}{r'_{i}}} =    f_{ice} \overline{{u'_{i}} r'_{(c+i)}}$.


\subsection{Buoyancy flux}

The last point of concern is the computation of the buoyancy term in presence of
ice in the TKE equation. The expression of $\theta_{v}$ in Eq.~(\ref{eqn:thetav}) is
slightly modified when switching from $\theta_l$ to $\theta_{il}$ and when
using the new expression for $r_{np}$
\begin{eqnarray}\label{eqn:newthetav}
\theta_{v} &=& \theta (1+\delta\dfrac{r_{v}}{1+r_{w}}-
\dfrac{r_{c}+r_{i}}{1+r_{w}}-\dfrac{r_{w}-r_{v}-r_{c}-r_{i}}{1+r_{w}}) 
\nonumber\\
&=& \left( \theta_{il}+\dfrac{L_{v}}{{C_{p}}_{h}}{{\Pi}_{ref}}^{-1}r_{c}
                      +\dfrac{L_{s}}{{C_{p}}_{h}}{{\Pi}_{ref}}^{-1}r_{i} \right)
\nonumber\\
&\times& \left( 1+ \delta \dfrac{r_{np}}{1+r_{w}}-
(1+\delta)\dfrac{r_{c}+r_{i}}{1+r_{w}}-\dfrac{r_{w}-r_{np}}{1+r_{w}} \right),
\end{eqnarray}
where again $\delta = \dfrac{R_{v}}{R_{d}}-1$.
The fluctuations of $\theta_v$ now depend on both ${r_{c}}$ and ${r_{i}}$
\begin{eqnarray}\label{eqn:newtheta'v}
{\theta}'_{v} &\approx&
\delta \dfrac {\overline{\theta}}{1+\overline{r_{w}}} {r'_{np}}\nonumber\\
&+& \left(\dfrac{L_{v}}{{C_{p}}_{h}}{{\Pi}_{ref}}^{-1}
\left( 1+\delta \dfrac{\overline{r_{np}}}{1+\overline{r_{w}}}-
(1+\delta)\dfrac{\overline{r_{c}}+\overline{r_{i}}}{1+\overline{r_{w}}}-
\dfrac{\overline{r_{w}}-\overline{r_{np}}}{1+\overline{r_{w}}}\right)-(1+\delta)
\dfrac{\overline{\theta}}{1+\overline{r_{w}}}\right) {r'_{c}}\nonumber\\
&+& \left(\dfrac{L_{s}}{{C_{p}}_{h}}{{\Pi}_{ref}}^{-1}
\left( 1+\delta \dfrac{\overline{r_{np}}}{1+\overline{r_{w}}}-
(1+\delta)\dfrac{\overline{r_{c}}+\overline{r_{i}}}{1+\overline{r_{w}}}-
\dfrac{\overline{r_{w}}-\overline{r_{np}}}{1+\overline{r_{w}}}\right)-(1+\delta)
\dfrac{\overline{\theta}}{1+\overline{r_{w}}}\right) {r'_{i}}\nonumber\\
&+& \left(1+\delta \dfrac{\overline{r_{np}}}{1+\overline{r_{w}}} -
(1+\delta)\dfrac{\overline{r_{c}}+\overline{r_{i}}}{1+\overline{r_{w}}}
-\dfrac{\overline{r_{w}}-\overline{r_{np}}}{1+\overline{r_{w}}}\right)
{\theta}'_{il},
\end{eqnarray}
After recasting the above expression as
${\theta}'_{v}=A^{il}{\theta}'_{il}+B{r'_{np}}+C^l{r'_{c}}+C^i{r'_{i}}$
(see Eq.(\ref{ErEtheta})), a good approximation of $\overline{u'_{i}\theta'_{v}}$ 
is deduced
\begin{equation} \label{newErEtheta}
\overline{{u'_{i}}{{\theta}'_{v}}} \approx 
E_{\theta} \overline{u'_{i}\theta'_{il}} +
E_{moist}  \overline{u'_{i}r'_{np}},
\end{equation}
where the updated factors $E_{\theta} = A + C A_{\theta}$ and 
$E_{moist} = B + C A_{moist}$ are a function of $A \equiv A^{il}$ (including
$\overline{r_{i}}$) and of $C=(1-f_{ice}) C^l + f_{ice} C^i$.

\section{Mixing length}

No modification is brought for the moment.

\section{Fractional nebulosity of mixed-phase clouds}

Although all ingredients of the "warm cloud" scheme are available and are
directly applicable to mixed-phase clouds, the partial cloudiness $N$ has not 
yet been tested in this case. Note that $N$ is already coupled to the radiative 
transfer scheme.




\section{References}

\por 
Bechtold, P., J.-P. Pinty and P. Mascart, 1993:
The use of partial cloudiness in a warm-rain parameterization:
            a subgrid-scale precipitation scheme,
           {\it Mon. Wea. Rev.}, {\bf 121}, 3301-3311.
\por Betts, A. K., 1973:
on precipitating cumulus convection and its parameterization,
           {\it Quart. J. Roy. Meteorol. Soc.}, {\bf 99}, 178-196.
\por 
Bougeault, P. 1981:
Modeling the trade-wind cumulus boundary layer.  Part I:
                testing the ensemble cloud relations against numerical data,
           {\it J. Atmos. Sci.}, {\bf 38}, 2414-2428, 1981.
\por 
Bougeault, P., 1982:
Cloud-ensemble relations based on the gamma probability
distribution for the higher-order models of the planetary boundary layer,
{\it J. Atmos. Sci.}, {\bf 39}, 2691-2700.
\por
Bougeault, P., and P. Lacarr\`ere, 1989:
Parameterization of orography-induced turbulence in a meso-beta scale model,
{\it Mon. Wea. Rev.}, {\bf 117}, 1872-1890.
\por
Chaboureau J.-P., and P. Bechtold, 2002: 
A simple cloud parameterization derived from
cloud resolving model data: Diagnostic and prognostic applications.
{\it J. Atmos. Sci.}, {\bf 59}, 2362-2372.
\por
Chaboureau J.-P., and P. Bechtold, 2005: Statistical representation of clouds
in a regional model and the impact on the diurnal cycle of convection
during Tropical Convection, Cirrus and Nitrogen Oxides (TROCCINOX). 
{\it J. Geophys. Res.}, {\bf 110}, D17103, doi:10.1029/2004JD005645.
\por
Cuxart, J., P. Bougeault, and J.-L. Redelsperger, 2000:
A turbulence scheme allowing for mesoscale and large-eddy simulations.
{\it Quart. J. Roy. Meteor. Soc.}, {\bf 126,} 1--30.
\por 
Cuijpers, J. W. M. and P. Bechtold, 1995:
A Simple Parameterization of Cloud Water Related Variables 
for Use in Boundary Layer Models,
{\it J. Atmos. Sci.}, {\bf 52}, 2486-2490.
\por
Lappen, C.-L., and D. A. Randall, 2001: 
Toward a unified parameterization of the boundary layer and moist convection.
Part I: A new type of mass-flux model.
{\it J. Atmos. Sci.,} {\bf 58,} 2021-2036.

\por
Mellor, G. L., 1977: The Gaussian cloud model relations.
{\it J. Atmos. Sci.,} {\bf 34,} 356-358.

\por 
Sommeria, G. and J. W. Deardorff, 1977:
Subgrid scale condensation in models for non precipitating clouds,
{\it J. Atmos. Sci.}, {\bf 34}, 344-355.

%%%%%%%%%%%%%%%%%%%%%%%%%%%%%%%%%%%%%%%%%%%%%%%%%%%%%%%%%%%%%%%%%%%%%%%%%%%%%%%
%%%%%%%%%%%%%%% END OF "Sub-Grid Condensation Scheme" CHAPTER  %%%%%%%%%%%%%%%%
%%%%%%%%%%%%%%%%%%%%%%%%%%%%%%%%%%%%%%%%%%%%%%%%%%%%%%%%%%%%%%%%%%%%%%%%%%%%%%%
