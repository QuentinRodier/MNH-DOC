%%%%%%%%%%%%%%%%%%%%%%%%%%%%%%%%%%%%%%%%%%%%%%%%%%%%%%%%%%%%%%%%%%%%%%%%%%%
%%%%%%%%%%%%%%%%%%     NUMERICAL DIFFUSION SECTION    %%%%%%%%%%%%%%%%%%%%%
%%%%%%%%%%%%%%%%%%%%%%%%%%%%%%%%%%%%%%%%%%%%%%%%%%%%%%%%%%%%%%%%%%%%%%%%%%%

\chapter{Numerical diffusion terms}
\minitoc

%{\em by J.P. Pinty}

As in most numerical models, it is necessary to prevent the occurence of
numerical waves due to the inaccurate representation of the dynamical
processes and reflection at the top or lateral boundaries. This is done
in a fairly classical way, through (i) a weak background diffusion, (ii) a
top absorbing layer, and (iii) a lateral sponge zone. Note that in these
three regions, the flow is relaxed towards the "large-scale" values, which
may be non uniform in space, and time dependent. For idealized runs, the
user may of course choose uniform and steady large-scale values.

\section{Background diffusion}
\subsection{Diffusion operator}
A diffusion operator is applied to the {\em fluctuations} of the prognostic
variables $\phi$. The fluctuations are defined here as the departure
from the large scale value $\phi_{LS}$.
The diffusion operator is a fourth-order operator
($\delta_{x^4}$) used everywhere except at the first interior grid point where
a second-operator ($\delta_{x^2}$) is substituted in the case of non-periodic
boundary conditions.
The background diffusion source for any prognostic variable noted $\phi$
is
$$
S_{BD}=-K4 \big[ \delta_{x^4}[\phi(t-\delta t)- \phi_{LS}]+
\delta_{y^4}[\phi(t-\delta t)- \phi_{LS}]\big]
$$
or
$$
S_{BD}=+K2 \big[ \delta_{x^2}[\phi(t-\delta t)- \phi_{LS}]+
\delta_{y^2}[\phi(t-\delta t)- \phi_{LS}]\big]
$$
where $K2$ and $K4$ are positive coefficients, and $\phi_{LS}$ represents the
Large Scale value of the considered variable.


\subsection{Choice of the diffusion coefficient}
Let us consider a single harmonic wave defined by:
$$ \phi(x,t)=\Phi(t)e^{ikx}$$
where $\Phi(t)$ is the wave amplitude and $k$ the wavenumber.

The application of a second-order diffusion operator during $N$ time steps leads
to:
$$\phi(x,t+N \Delta t)=\Phi(t)[1- 2 {K_2 \Delta t \over {d_{xx}}^2}
(1- cos k d_{xx} )]^N$$
where $\Delta t$ is the time step, ${d_{xx}}^2$ the grid interval, and
$K2=K_2/{d_{xx}}^2$ the diffusion coefficient.
The time $T_2$ at which the initial wave is damped by $e^{-1}$ is then:
$$T_2=N \Delta t= {-\Delta t \over \ln [1- 2{K_2 \Delta t \over {d_{xx}}^2}
(1-cos k d_{xx} )]}$$
which can be approximated by:
$$ T_2 \simeq {{d_{xx}}^2 \over 2 K_2 (1-cos k d_{xx})}$$
The corresponding time in the case of a fourth-order diffusion operator is given
by:
$$T_4=N \Delta t= {-\Delta t \over \ln [1 - 4{K_4 \Delta t \over {d_{xx}}^4}
(1-cos k d_{xx})^2]} $$
which can be approximated by:
$$ T_4 \simeq {{d_{xx}}^4 \over {4 K_4 (1-cos k d_{xx})^2}} $$.

If $k$ is the wavenumber associated to the $n d_{xx}$ wavelength, $T_2$ and
$T_4$ are given by
\begin{equation}
T_2 (n) \simeq {{d_{xx}}^2 \over {2 K_2 (1-cos (2 \pi /n))}}
\end{equation}
\begin{equation}
T_4 (n) \simeq {{d_{xx}}^4 \over {4 K_4 (1-cos (2 \pi /n))^2}}
\end{equation}

To set up  the diffusion coefficients, it might be more convenient
to specify $T_2$ or $T_4$ rather than $K_2$ or $K_4$. $T_2$ and
$T_4$  can be more easily related to the physical processes being studied.
From previous experience, $T_4(2)$
was set to 10-15 mn in the case of PBL convective
rolls, 20-30 mn for moist convection, 1-2 hours for orographic flows.

For a specified wavelength $n_0 d_{xx}$, an equivalent damping timescale with
a second-order or a fourth-order diffusion scheme requires
\begin{equation}
 K_4= - {K_2 \over 2} {{d_{xx}}^2 \over 1-cos(2 \pi /{n_0}) }.
\end{equation}
The same arguments hold for the diffusion in the $y$ direction.


In the code $n_0$ is set equal to 2 to select the highest wavenumber and so
the user specifies $T_4(2)$.
Noting that  $K2= K_{2x}/{d_{xx}}^2= K_{2y}/{d_{yy}}^2$,
 $K4= K_{4x}/{d_{xx}}^4 = K_{4y}/{d_{yy}}^4$, and according to (1), (2)
and (3)
$K4$ and $K2$ are then given by:
$$K4={1 \over 16T_{4}(2)}$$
and
$$K2=K4  ( 1-cos(2 \pi /n_0))= 4K4$$
\vskip 2truecm

\section{Top absorbing layer}
To prevent spurious reflection from the model top boundary, an absorbing layer
in which damping increases with height occupies the top fraction of the
 domain. A Rayleigh damping has been chosen, it is applied on the three
components of the wind and on the thermodynamical variable. Only the
perturbations of a variable from its local large scale values are damped
on $\bar z$ surfaces. In the absorbing layer, the implicit damping source
for any variable $\phi$ is written as:
$$
S_{AL}=-K_{AL} (\bar z)\big[\phi(t+\Delta t)- \phi_{LS} \big]
      =-KAL(\bar z) \big[\phi(t-\Delta t)- \phi_{LS} \big]$$
where
$$ KAL= {K_{AL}(\bar z) \over 1+2 \Delta t K_{AL}(\bar z)}$$
and where $K_{AL}(\bar z)$ is given by
$$K_{AL}(\bar z)= K_{AL} (H) \sin ^2
\Big({\pi \over 2}
     { \bar z - \bar z_{alb} \over H- \bar z_{alb}} \Big)
\qquad for  \quad  \bar z \geq  \bar z_{alb}$$
with $ \bar z_{alb}$  the Gal-Chen and Sommerville
 height of the absorbing layer base and
$ \phi_{LS} $ the relaxation value of $\phi$. In the first version of the model
the relaxation fields are the initial fields and the maximum damping rate
$K_{AL} (H)$ must be provided for each model run.
\vskip 2truecm

\section{Lateral sponge zone}
An additional sponge zone is inserted close to the lateral boundaries to either
damp outward propagating waves or slowly incorporate inward propagating
larger scale waves. A first order damping rate has been retained and its
application to any prognostic variable $\phi$ leads to a source term of the
form:
$$
S_{SZ}=-K_{SZ} (\bar x, \bar y)\big[\phi(t+\Delta t)- \phi_{LS} \big]
      =-KSZ(\bar x, \bar y) \big[\phi(t-\Delta t)- \phi_{LS} \big]$$
where
$$ KSZ= {K_{SZ}(\bar x, \bar y) \over 1+2 \Delta t K_{SZ}(\bar x, \bar y)}.$$
The damping coefficient $KSZ(\bar x, \bar y)$ is non-zero in a rim zone of
width ${rim}_{\bar x}$ and ${rim}_{\bar y}$ (in each $x$ and $y$ direction,
respectively) following immediately the lateral boundaries. For example, near
the left ($x$) lateral boundary, the damping coefficient has the generic form:
%\begin{equation}
$$
KSZ(\bar x) = \left\{ \begin{array}{ll}
K_{SZ}^{max} \sin^2 \Big(\displaystyle{{\pi \over 2}{\bar x-{rim}_{\bar x} \over {rim}_{\bar x}}} \Big)
            & \mbox{if $0 < \bar x < {rim}_{\bar x}$,} \\
 \\
 0          & \mbox{otherwise,}
                   \end{array}
        \right.
$$
%\end{equation}
In the four corners of the domain of simulation, there is a smooth transition
between the pure $\bar x$ and pure $\bar y$ dependencies of the damping
coefficient $KSZ$ that is obtained in the following manner:
%\begin{equation}
$$
KSZ(\bar x, \bar y) = \left\{ \begin{array}{ll}
K_{SZ}^{max}& \mbox{if $1 \le d_{rim}^2$,} \\
 \\
K_{SZ}^{max} \sin^2 \Big({\pi \over 2} d_{rim} \Big)
            & \mbox{if $d_{rim}^2 \le 1$,} \\
 \\
 0          & \mbox{if $\bar x \ge {rim}_{\bar x}$ and $\bar y \ge {rim}_{\bar y}$} \\
                         \end{array}
        \right.
$$
%\end{equation}
where
$d_{rim}= \sqrt{\Big( \displaystyle{{\bar x-{rim}_{\bar x} \over {rim}_{\bar x}}\Big)^2 +
                \Big( {\bar y-{rim}_{\bar y} \over {rim}_{\bar y}}\Big)^2 }}$.
The maximum value of the relaxation coefficient $K_{SZ}^{max}$ and the rim zone
depths ${rim}_{\bar x}$ and ${rim}_{\bar y}$ are prescribed externally by the
user.


\vspace{2.5cm}

%>>>>>>>>>>>>>>>>>>>>>>>>>>>>>>>>>>>>>>>>>>>>>>>>>>>>>>>>>>>>>>>
%>>>>>>>>>>>>>>>>>>>>>>>>>>>>>>>>>>>>>>>>>>>>>>>>>>>>>>>>>>>>>>>
%>>>>>>>>>>>>>>>>>>>>>>>>>>>>>>>>>>>>>>>>>>>>>>>>>>>>>>>>>>>>>>>
%>>>>>>>>>>>>                                     >>>>>>>>>>>>>>
%>>>>>>>>>>>>>    CETTE SECTION EST A REVOIR       >>>>>>>>>>>>>>
%>>>>>>>>>>>>                                     >>>>>>>>>>>>>>
%>>>>>>>>>>>>>>>>>>>>>>>>>>>>>>>>>>>>>>>>>>>>>>>>>>>>>>>>>>>>>>>
%>>>>>>>>>>>>>>>>>>>>>>>>>>>>>>>>>>>>>>>>>>>>>>>>>>>>>>>>>>>>>>>
%>>>>>>>>>>>>>>>>>>>>>>>>>>>>>>>>>>>>>>>>>>>>>>>>>>>>>>>>>>>>>>>
% Il manque une figure pour illustrer la structure en "hippodrome" de la
%"RIM zone" (J. Phi.)
%Par ailleurs il faut revoir les notations des variables de RELAXATION
%
%%%%%%%%%%%%%%%%%%%%%%%%%%%%%%%%%%%%%%%%%%%%%%%%%%%%%%%%%%%%%%%%%%%%%%%%%%%%
\chapter{The pressure problem}

%{\em by P. Hereil}

\section{Continuous pressure equation}
\subsection{Full elliptic problem}
\par In the  Meso-NH model formulation, the equations of motion are
expressed as:
\begin{eqnarray}
\label{motion1}
\frac {\partial{\tilde{\rho} { u }} } {\partial{t} } = S_x - \tilde{\rho}
\frac{\partial{\Phi} }{\partial{x} } \\
\label{motion2}
\frac {\partial{\tilde{\rho} { v }} } {\partial{t} } = S_y - \tilde{\rho}
\frac{\partial{\Phi} }{\partial y} \\
\label{motion3}
\frac {\partial{\tilde{\rho} {w}} } {\partial{t} } = S_z - \tilde{\rho}
\frac{\partial{\Phi} }{\partial z}
\end{eqnarray}
where $S_x$, $S_y$ and $S_z$ are the source terms apart from the
pressure terms; $\tilde{\rho} = \rho_{d\,ref} J$; $\Phi = C_p \,
\theta_{v\,ref}\,\Pi '$ is called here the "pressure" for short.
The anelastic constraint reads\\
\begin{equation}
\label{conti}
     \dfrac {\partial{\tilde{\rho} U^{c} } } {\partial{\overline{x} } }
   + \dfrac {\partial{\tilde{\rho} V^{c} } } {\partial{\overline{y} } }
   + \dfrac {\partial{\tilde{\rho} W^{c} } } {\partial{\overline{z} } } = 0
\end{equation}
If we combine the equations (\ref{motion1}-\ref{motion3}) in order to build and
nullify the divergence of the wind at the future instant in (\ref{conti}), we
obtain a diagnostic equation for the pressure:
\begin{equation}
\label{conteq}
GDIV(\tilde{\rho}  \overrightarrow{\nabla} \Phi) = GDIV \left(
\overrightarrow {S} -
\dfrac{\tilde{\rho} \vec{U}_{(t-\Delta t)} }
{2 \Delta t}
\right)
\end{equation}
where
\begin{displaymath}
GDIV(\vec{B}) = \dfrac {\partial{ B^{c1} } } {\partial{\overline{x} } }
               + \dfrac {\partial{ B^{c2} } } {\partial{\overline{y} } }
               + \dfrac {\partial{ B^{c3} } } {\partial{\overline{z} } }
\end{displaymath}
$B^{c1}, B^{c2}, B^{c3}$ are the contravariant components of
$\overrightarrow{B}$ and can be written as:
\begin{eqnarray}
 B^{c1} \;  &  = &
 \dfrac{{B_1}}{ d_{xx}} \\
 B^{c2} \;  &  = &
 \dfrac{{B_2}}{ d_{yy}} \\
 B^{c3} \;  & = &
\dfrac{1}{d_{zz}}
\left[  B_3 -
   \dfrac  {B_1} {d_{xx}}  d_{zx}
-  \dfrac  {B_2} {d_{yy}}  d_{zy}  \right]
\end{eqnarray}
where $(B_1, B_2, B_3)$ are the components of $\vec{B}$ along
$\left( \vec{i}, \vec{j}, \vec{k} \right)$.\\
$\overrightarrow{\nabla}$ is defined by:
$
\overrightarrow{\nabla}=
\vec{i} \dfrac {\partial{  } } {\partial{x} }  +
\vec{j} \dfrac {\partial{  } } {\partial{y} }  +
\vec{k} \dfrac {\partial{  } } {\partial{z} }
$
where
\begin{eqnarray*}
\dfrac {\partial{  } } {\partial{x} } &=&
\dfrac{1}{d_{xx}}
  \dfrac{\partial  }{\partial \overline{x}}- \dfrac{d_{zx}}{d_{xx}d_{zz}}
\dfrac{\partial  }{\partial \overline{z}}\\
\dfrac {\partial{  } } {\partial{y} } &=&
\dfrac{1}{d_{yy}}
  \dfrac{\partial  }{\partial \overline{y}}- \dfrac{d_{zy}}{d_{yy}d_{zz}}
   \dfrac{\partial  }{\partial \overline{z}}\\
\dfrac {\partial{  } } {\partial{z} } &=&
\dfrac{1}{d_{zz}}
  \dfrac{\partial  }{\partial \overline{z}}
\end{eqnarray*}
For convenience, we define the $QLAP$ operator by: $QLAP =
GDIV(\tilde{\rho} \overrightarrow{\nabla})$.

Note that because an iterative method is used to solve the pressure equation,
the residual term $GDIV(\tilde{\rho} \vec{U}_{(t-\Delta t)})$
must be kept and injected in the right hand side
of 6.5 to get the most accurate solution for $\Phi$.
%%%%%%%%%%%%%%%%%%%%%%%%%%%%%%%%%%%%%%%%%%%%%%%%%%%%%%%%%%%%%%%%%%%%%%%%%%%%%%%
\subsubsection{Top and Bottom Boundaries}
For the highest and the lowest vertical level, a Neumann condition is
imposed:\\
\begin{equation}
\label{vb1}
\tilde{\rho} \vec{n} \cdot \overrightarrow{\nabla}\Phi= \vec{n} \cdot
\overrightarrow{S}
\end{equation}
%%
%%%%%%%%%%%%%%%%%%%%%%%%%%%%%%%%%%%%%%%%%%%%%%%%%%%%%%%%%%%%%%%%%%%%%%%%%%%%%%%
%%
\subsubsection{Lateral Boundaries}
We allow for several possibilities:
\begin{itemize}

\item For a periodic simulation domain, the lateral boundary conditions are
a periodisation of the pressure field.
\item For an open area, the ${ u }$ and ${ v }$ evolution on the boundaries
are performed by the Sommerfeld's equations:
\begin{eqnarray}
\label{som1}
\frac {\partial{\tilde{\rho} { u }} } {\partial{t} } = -  c_x \frac
{\partial{\tilde{\rho} { u }} } {\partial{x} } \\
\label{som2}
\frac {\partial{\tilde{\rho} { v }} } {\partial{t} } = -  c_y \frac
{\partial{\tilde{\rho} { v }} } {\partial{y} }
\end{eqnarray}
where $c_x$ and $c_y$ are the phase velocities
of the fastest waves in  respectively x and y directions and determined by the
Orlanski's method (1976\nocite {orl76}) or fixed (Klemp and Wilhelmson, 1978
\nocite{kle78}). \\
\item If a Davies' scheme is used, the tendency on the boundary is given by:\\
\begin{eqnarray}
\frac {\partial{\tilde{\rho} { u }} } {\partial{t} } =
\left( \frac {\partial{\tilde{\rho} { u }} } {\partial{t} } \right)_{Large
\; Scale} \\
\frac {\partial{\tilde{\rho} { v }} } {\partial{t} } =
\left( \frac {\partial{\tilde{\rho} { v }} } {\partial{t} } \right)_{Large
\; Scale}
\end{eqnarray}
\item If we use a Carpenter's scheme (Carpenter, 1982\nocite{car82}),
the tendency on the boundary is expressed as:
\begin{eqnarray}
\frac {\partial{\tilde{\rho} { u }} } {\partial{t} } =
\left( \frac {\partial{\tilde{\rho} { u }} } {\partial{t} } \right)_{Large
\; Scale}-  c_x  \left[
 \frac {\partial{\tilde{\rho} { u }} } {\partial{x} }
- \left( \frac {\partial{\tilde{\rho} { u }} } {\partial{x} }\right)_{Large
\; Scale}
\right]\\
%
\frac {\partial{\tilde{\rho} { v }} } {\partial{t} } =
\left( \frac {\partial{\tilde{\rho} { v }} } {\partial{t} } \right)_{Large
\; Scale} -  c_y \left[
 \frac {\partial{\tilde{\rho} { v }} } {\partial{y} }
- \left( \frac {\partial{\tilde{\rho} { v }} } {\partial{y} }\right)_{Large
\; Scale}
\right]
\end{eqnarray}
%%
\end{itemize}
Therefore, in every case, the temporal evolution of the horizontal momentum is
known at the boundary by these equations and from (\ref{motion1}-
\ref{motion2}), we can infer the normal component of the pressure gradient
when the sources components $\vec{S} \cdot \vec{n}$ is known.
The result is a non-homogeneous Neumann condition for the full elliptic
problem. We can make this condition homogeneous if we replace in
(\ref{motion1}-\ref{motion2}) the source terms $S_x$ and $S_y$ at the
boundaries by the prescribed values of $\frac {\partial{\tilde{\rho} { u }}
} {\partial{t} }$ and $\frac {\partial{\tilde{\rho} { v }} }
{\partial{t} }$ respectively. This leads to an algebraic simplification
in the pressure equation (\ref{conteq}).The homogeneous boundary condition
for the pressure field is:
\begin{equation}
\label{hb1}
\dfrac{\partial{\Phi } } {\partial{n} } = 0
\end{equation}
%%%%%%%%%%%%%%%%%%%%%%%%%%%%%%%%%%%%%%%%%%%%%
%%%
\par We want to solve efficiently and accurately the linear system
(\ref{conteq}) where the source term $\overrightarrow {S}$ is known and with
the previous boundary conditions. As it is
difficult and time-consuming to invert such a large system, we have
implemented two iterative methods in order to offer a good compromise between
versatility and efficiency. Both methods are described
in Golub and Meurant (1983\nocite{gol83}): the first one consists in
a conjugate gradient algorithm and the second one is the Richardson's method.
The convergence of such algorithms is considerably speeded-up with the use
of preconditioning techniques. Here, we precondition with the "flat"
problem (Bernardet\nocite{ber94}, 1994), i.e. without orography, which
compresses the eigenvalue spectrum of the original problem, leaving only a few
eigenvalues significantly different from 1, resulting in much faster
convergence. For both methods, the first guess is also set equal to the
solution of the flat problem.
%%
%%%%%%%%%%%%%%%%%%%%%%%%%%%%%%%%%%%%%%%%%%%%%%%%%%%%%%%%%%%%%%%%%%%%%%%%%%%%%%%
%%
\subsection{Conjugate gradient algorithm}
\par The Golub and Meurant's version of the conjugate gradient (CG) is devoted
to symmetric matrices. As our problem is non symmetric, we use a
generalization of the CG called ORTHOMIN, detailled in Young and Jea
(1981\nocite{you81}). Considering the linear system:
\begin{equation}
Qx = y
\end{equation}
where $x$ is unknown. In our context, $Q$ is the $QLAP$ operator defined before.
Preconditionning the previous system with the "flat" problem gives:
\begin{equation}
F^{-1} Q x = F^{-1} y
\end{equation}
where $F^{-1}$ is the preconditioner with $F$ defined as the $QLAP$ operator
where the topography is neglected (flat problem).\\
The ORTHOMIN algorithm can be put in the following form (Kapitza and Eppel,
1992\nocite{kap92}):
\begin{eqnarray*}
%%
x^{(n+1)} &=& x^{(n)} + \lambda _n p^{(n)}\\
%%
\lambda _n &=& \frac {(\delta^{(n)}, F^{-1} Q p^{(n)} )}
  { \left( F^{-1} Q p^{(n)}, F^{-1} Q p^{(n)} \right) } \\
%%
\delta^{(n)} &=& F^{-1} ( y - Q x^{(n)}) \\
%%
p^{(0)} &=& \delta^{(0)}\\
%%
p^{(n)} &=& \delta^{(n)} + \alpha _n p^{(n-1)} \\
%%
\alpha _n &=& - \frac {\left( F^{-1} Q \delta^{(n)},  F^{-1} Q p^{(n-1)} )
\right)} {\left( F^{-1} Q p^{(n-1)}, F^{-1} Q p^{(n-1)} \right)} \\
%%
\end{eqnarray*}
Here $x^{(n)}$ is the n-th iteration of $x$ and (.,.) is the dot product of two
vectors. Concerning the computing time, at each iteration, we have to
invert three times the $F$ matrix.
\\
%%
%%%%%%%%%%%%%%%%%%%%%%%%%%%%%%%%%%%%%%%%%%%%%%%%%%%%%%%%%%%%%%%%%%%%%%%%%%%%%%%
%%
\subsection{Richardson's method}
\par This technique allows to divide by 3 the number of matrix inversion,
compared to the CG algorithm . Furthermore, this method, including a
relaxation factor $\alpha$, allows an under-relaxation  which is useful to
improve the solver convergence in the context of simulations with large slopes.
For example, for slopes greater than 1, preliminary tests have shown that the
optimal value of $\alpha$ is equal to 1. for such simulations. If we assume the
following linear system preconditioned with the inverse of the "flat" problem:
\begin{equation}
F^{-1} Q x = F^{-1} y
\end{equation}
the Richardson's method reads:
\begin{eqnarray*}
x^{(n+1)} &=& x^{(n)} + \alpha (F^{-1} y - F^{-1} Q x^{(n)} )
\end{eqnarray*}
where $x^{(n)}$ is the n-th iteration of $x$.\\
%%
%%%%%%%%%%%%%%%%%%%%%%%%%%%%%%%%%%%%%%%%%%%%%%%%%%%%%%%%%%%%%%%%%%%%%%%%%%%%%%%
%%
\subsection{Flat operator}
\par Both iterative methods require the inversion of the operator $QLAP$ for
the "flat" problem. "Flat" means the topography is neglected, which induces
that $x=\overline{x}$, $y=\overline{y}$, $z=\overline{z}$.
This involves that $d_{zx}$, $d_{zy}$ vanish. Here, we consider a regular grid
and the stretching functions ${\cal{D}}_x (\widehat{x})$,
${\cal{D}}_y (\widehat{y})$, ${\cal{D}}_z (\widehat{z})$ are neglected
in order to use a FFT decomposition and will be implemented later. Thus, we
obtain for the flat operator $F$:\\
\begin{equation}
\label{flatop1}
F \Phi =  \frac { \partial{} } { \partial{\overline{x}} } \left(
          \dfrac{1}{d_{xx}} \tilde{\rho} \dfrac{1}{d_{xx}}
          \frac { \partial{\Phi} } { \partial{\overline{x}} } \right)
+          \frac { \partial{} } { \partial{\overline{y}} } \left(
          \dfrac{1}{d_{yy}} \tilde{\rho} \dfrac{1}{d_{yy}}
          \frac { \partial{\Phi} } { \partial{\overline{y}} } \right)
+          \frac { \partial{} } { \partial{\overline{z}} } \left(
          \dfrac{1}{d_{zz}} \tilde{\rho} \dfrac{1}{d_{zz}}
          \frac { \partial{\Phi} } { \partial{\overline{z}} } \right)
\end{equation}
In the flat problem, $\tilde{\rho}$ only depends on $z$, (\ref{flatop1})
can be rewritten as:\\
\begin{equation}
\label{flatop}
F \Phi =  < \tilde{\rho}>_{x,y}
  \left[ \dfrac{1}{<{d_{xx}}^2>_{x,y}}
         \frac { \partial{^2\Phi} } { \partial{\overline{x}^2} }
+        \dfrac{1}{<{d_{yy}}^2>_{x,y}}
         \frac { \partial{^2\Phi} } { \partial{\overline{y}^2} }
  \right]
+          \frac { \partial{} } { \partial{Z} } \left(
           <\tilde{\rho}>_{x,y} \dfrac{1} {<{d_{zz}}^2>_{x,y}}
          \frac { \partial{\Phi} } { \partial{Z} } \right)
\end{equation}
where $Z = <\overline{z}>_{xy}$.\\
We have applied a spatial averaging on a whole $xy$ plan because
$\tilde{\rho}$ and $\overline{z}$ are functions of $z$. When
the orography is neglected, we must recover a function of only $\overline{z}$ (this
is also necesary to lead to a separable problem). A Fast Fourier Transform
(Schumann and Sweet, 1988\nocite{sch88}) is used for the inversion of the
horizontal part of $F$. The form of this FFT depends on the lateral boundary
conditions. The vertical part of $F$ leads to a tridiagonal matrix and thus a
classical inversion algorithm is applied (the double sweep method, see
"Numerical Recipes").
%%%%%%%%%%%%%%%%%%%%%%%%%%%%%%%%%%%%%%%%%%%%%%%%%%%%%%%%%%%%%%%%%%%%%%%%%%%%%%%
%%
%%%%%%%%%%%%%%%%%%%%%%%%%%%%%%%%%%%%%%%%%%%%%%%%%%%%%%%%%%%%%%%%%%%%%%%%%%%%%%%
%%
\section{Discrete pressure equation}
%%
%%%%%%%%%%%%%%%%%%%%%%%%%%%%%%%%%%%%%%%%%%%%%%%%%%%%%%%%%%%%%%%%%%%%%%%%%%%%%%%
%%
\subsection{Full elliptic problem}
\par The continuous equation (\ref{conteq}) is:\\
\begin{equation}
\label{pressure}
GDIV(\rho  \overrightarrow{\nabla} \Phi) = GDIV \left( \overrightarrow {S} -
\dfrac{\tilde{\rho} \vec{U}_{(t-\Delta t)} }
{2 \Delta t}
\right)
\end{equation}
where
$
\overrightarrow{\nabla}\Phi =
\dfrac {\partial{\Phi  } } {\partial{x} }  \vec{i} +
\dfrac {\partial{\Phi } } {\partial{y} }  \vec{j} +
\dfrac {\partial{\Phi  } } {\partial{z} }  \vec{k}
$
is discretized by:
\begin{eqnarray}
\label{gradx}
\dfrac { \partial {\Phi}} {\partial{x}} &\rightarrow&  \frac {1}
{d_{xx} } \left[ \delta_x \Phi -
\overline { d_{zx} \overline { \frac { \delta_z \Phi} {d_{zz} } }^x }
^z \right] \\
\label{grady}
\dfrac {\partial {\Phi}} {\partial{y}} &\rightarrow&  \frac {1}
{d_{yy} } \left[ \delta_y \Phi -
\overline { d_{zy} \overline { \frac { \delta_z \Phi} {d_{zz} } }^y } ^z
\right] \\
\label{gradz}
\dfrac {\partial {\Phi}} {\partial{z}} &\rightarrow& \frac {\delta_z \Phi}
{ d_{zz}}
\end{eqnarray}
where $\delta _x$, $\delta _y$, $\delta _z $ are finite difference operators.\\
The discrete form of the contravariant components of a vector $\vec{B}
(B_1, B_2, B_3)$ is:
\begin{eqnarray}
 B^{c1} \;  &  = &
 \dfrac{{B_1}}{ d_{xx}} \\
 B^{c2} \;  &  = &
 \dfrac{{B_2}}{ d_{yy}} \\
 B^{c3} \;  & = &
\dfrac{1}{d_{zz}}
\left[  B_3 -
\overline{\left( \overline{\left(
 \dfrac{ {B_1}}{d_{xx}} \right)}^{z}
d_{zx}\right)}^{x}
-  \overline{\left( \overline{\left(\dfrac{
{B_2}}{d_{yy}}\right)}^{z}d_{zy}\right)}^{y}\right]
\end{eqnarray}
$(B_1, B_2, B_3)$ are the components of $\vec{B}$ along
$\left( \vec{i}, \vec{j}, \vec{k} \right)$
and thus we have:
\begin{equation}
GDIV(\vec{B}) = \delta _x B^{c1} + \delta _y B^{c2} + \delta _z B^{c3}
\end{equation}
We note $QLAP$ the numerical discretization of $QLAP$ and we obtain:
\begin{eqnarray*}
QLAP (\Phi) = GDIV(\tilde{\rho} \vec{\nabla}  \Phi ) &  = &  \delta _{x}
\left[ \overline {\tilde{\rho} }^{x} \dfrac{\delta _{x} \Phi }{{d_{xx}}^2}
 \right]
  - \delta _{x} \left[
 \overline{\tilde{\rho} }^{x}
  \dfrac{1 }{{d_{xx}}^2} \overline{\left( d_{zx} \overline{\left(\dfrac
{\delta  _{z} \Phi }{d_{zz} }\right) }^{x}\right) }^{z}   \right]
  \nonumber \\ & & \nonumber \\
 & + &   \delta _{y} \left[ \overline{\tilde{\rho} }^{y}
 \dfrac{\delta _{y} \Phi }{{d_{yy}}^2}
 \right] - \delta _{y} \left[ \overline{\tilde\rho  }^{y}
  \dfrac{1 }{{d_{yy}}^2} \overline{\left( d_{zy} \overline{\left( \dfrac
{\delta  _{z} \Phi}{d_{zz} } \right) }^{y}\right) }^{z}   \right]
  +    \delta _{z} \left[ \overline{\tilde{\rho} }^{z}
 \dfrac{\delta _{z} \Phi }{{d_{zz}}^2} \right]  \nonumber \\ & & \nonumber \\
 & -  & \delta _{z} \left[
 \dfrac{1}{d_{zz}}
 \overline{
  \left\{ d_{zx}
 \overline{ \left\{ \overline{\tilde{\rho} }^{x}\dfrac{\delta _{x} \Phi }
{{d_{xx}}^2}
  -\overline{\tilde{\rho} }^{x}\dfrac{1 }{d_{xx}}
 \overline{ \left(  d_{zx} \overline{  \left( \dfrac{\delta
  _{z} \Phi}{d_{zz} }\right) }^{x}\right) }^{z}
\right\}  }^{z}
\right\}  }^{x}
 \right] \nonumber \\ & & \nonumber \\
& -  & \delta _{z} \left[
 \dfrac{1}{d_{zz}}
 \overline{
  \left\{ d_{zy}
\overline{\left\{\overline{\tilde{\rho}}^{y}\dfrac{\delta _{y}\Phi}{{d_{yy}}^2}
  -\overline{\tilde{\rho}}^{y}\dfrac{1 }{d_{yy}}
 \overline{ \left(  d_{zy} \overline{  \left( \dfrac{\delta
  _{z} \Phi}{d_{zz} }\right) }^{y}\right) }^{z}
\right\}  }^{z}
\right\}  }^{y}
 \right]
\end{eqnarray*}
%%%%
For the first inner points, the calculation of the pseudo-Laplacian
operator $QLAP$ requires the evaluation of pressure outside the physical
domain. \\
%%%%%%%%%%%%%%%%%%%%%%%%%%%%%%%%%%%%%%%%%%%%%%%%%%%%%%%%%%%%%%%%%
\subsubsection {Vertical boundaries }
The discrete form of the boundary equation (\ref{vb1}) is expressed as:
\begin{equation}
\left(\vec{S}\right)^{c3} = \left( \tilde{\rho} \overrightarrow{\nabla} \Phi
 \right) ^{c3}
\end{equation}
where
\begin{equation}
\label{vb3}
\left( \tilde{\rho}  \overrightarrow{\nabla} \Phi \right) ^{c3} \;   =
\dfrac{1}{d_{zz}}
\left[ \left( \tilde{\rho} \dfrac {\partial \Phi}{\partial z} \right)  -
\overline{\left( \overline{\left(
 \dfrac{ 1}{d_{xx}}\left(\tilde{\rho}  \dfrac {\partial \Phi}{\partial x}
\right) \right)}^{z} d_{zx}\right)}^{x}
-  \overline{\left( \overline{\left(\dfrac{ 1
}{d_{yy}}\right)\left(\tilde{\rho} \dfrac {\partial
\Phi}{\partial y} \right)}^{z}d_{zy}\right)}^{y}\right]
\end{equation}
and the pressure gradient expressions are given by (\ref{gradx}-\ref{gradz}).
In (\ref{vb3}), the vertical spatial average requires the points located at
$z = - \Delta z/2$. In this case, we copy the value of the first
pressure gradient over the surface:
\begin{eqnarray*}
\left( \dfrac {\partial \Phi} {\partial x} \right)_{-\frac{\Delta z}{2} } =
 \left( \dfrac {\partial \Phi} {\partial x} \right)_{\frac{\Delta z}{2} } \\
\left( \dfrac {\partial \Phi} {\partial y} \right)_{-\frac{\Delta z}{2} } =
 \left( \dfrac {\partial \Phi} {\partial y} \right)_{\frac{\Delta z}{2} } \\
\end{eqnarray*}
%%%%%%%%%%%%%%%%%%%%%%%%%%%%%%%%%%%%%%%%%%%%%%%%%
%%
\subsubsection {Horizontal boundaries }
Here, we have to consider how to evaluate the pressure gradient terms at the
boundary (wind point), i.e. $i = 1+jphext$, $i= imax+1+jphext$, $j = 1+jphext$,
$j = jmax+1+jphext$ and this for $kmax+1+jpvext > k > jpvext$. $jphext$ and
$jpvext$ are the variables used to extend the number of horizontal and vertical
outer points respectively ($jphext \geq 1$, $jpvext \geq 1$). We have to
distinguish three cases:\\
%%
%%%%%%%%%%%%%%%%%%%%%%%%%%%%%%%%%%%%%%%%%%%%%%%%%
%%
$\bullet$\underline{Cyclic case:}\\
The periodic condition reads:\\
\begin{eqnarray}
\label{cyclic1}
 \Phi (jphext, j, k) &=& \Phi (imax+jphext, j, k) \\
\nonumber
\Phi (imax+1+jphext, j, k) &=& \Phi (1+jphext, j, k)
\end{eqnarray}
 in the x direction and
\begin{eqnarray}
\label{cyclic2}
 \Phi (i, jphext, k) &=& \Phi (i, jmax+jphext, k) \\
\nonumber
\Phi (i, jmax+1+jphext, k) &=& \Phi (i, 1+jphext, k)
\end{eqnarray}
in the y direction.\\
%%
%%%%%%%%%%%%%%%%%%%%%%%%%%%%%%%%%%%%%%%%%%%%%%%%%
%%
$\bullet$\underline{Open and Davies' cases:}\\
The x lateral boundary condition $\dfrac{\partial{\Phi } } {\partial{x} } = 0
$ for $i = 1+jphext, i= imax+1+jphext$ ($jmax+1+jphext>j>jphext$) becomes:
\begin{equation}
\label{open1}
\dfrac{\delta_{x} \Phi}{d_{xx}}  \, +  \,  \dfrac{1}{d_{xx}}
\overline{ \left(\overline{ \left(\dfrac{\delta_{z} \Phi}{d_{zz}}
\right)}^{x}d_{zx}\right)}^{z} = 0
\nonumber \\
\end{equation}
The y lateral boundary condition $\dfrac{\partial{\Phi } } {\partial{y} } = 0 $
for $j = 1+jphext, j= jmax+1+jphext$ ($imax+1+jphext>i>jphext$)
becomes:
\begin{equation}
\label{open2}
 \dfrac{\delta_{y} \Phi}{d_{yy}}  \, + \,   \dfrac{1}{d_{yy}}
\overline{ \left(\overline{ \left(\dfrac{\delta_{z} \Phi}{d_{zz}}
\right)}^{y}d_{zy}\right)}^{z} = 0
\end{equation}
%%
%%%%%%%%%%%%%%%%%%%%%%%%%%%%%%%%%%%%%%%%%%%%%%%%%
$\bullet$\underline{Mixed boundary conditions:}\\
In this case, we have to combine the previous set of equations (\ref{cyclic1})-
(\ref{open2}) to be consistent with the specification of the horizontal
boundaries. For example, if we assume a periodic condition along $y$ and a
Davies' condition along $x$ we use (\ref{cyclic2}) and (\ref{open1}).\\
\\
%%
%%%%%%%%%%%%%%%%%%%%%%%%%%%%%%%%%%%%%%%%%%%%%%%%%%%%%%%%%%%%%%%%%%%%%%%%%%%%%%%
%%
\underline {\bf Edges:}\\
The edges points at $k = jpvext$, (i.e. $\Phi (jphext, j, jpvext)$,
$\Phi (imax+1+jphext, j, jpvext)$, $\Phi (i, jphext,$  $jpvext)$,
$\Phi (i, jmax+1+jphext, 1)$, where $imax+jphext+1> i >  jphext$ and
$jmax+jphext+1 > j > jphext$) and their corresponding points located at the
coordinate $k = kmax+1+jpvext$, are necessary for the calculation of the
pseudo-Laplacian of the first inner grid points. So, we have imposed boundary
conditions consistent with the Neumann staggered condition defined for the
horizontal boundaries. These conditions are expressed as:\\
- for $i=1+jphext$, $imax+1+jphext$ and $z = \Delta z/2, H-\Delta z/2$:
\begin{displaymath}
\dfrac{\partial{\Phi } } {\partial{x} } = 0
\end{displaymath}
- for $i=1+jphext$, $imax+1+jphext$ and $z = 0, H$:
\begin{displaymath}
\delta _z \dfrac{\partial{\Phi } } {\partial{x} } = 0
\end{displaymath}
- for $j=1+jphext$, $jmax+1+jphext$ and $z = \Delta z/2, H - \Delta z/2$:
\begin{displaymath}
\dfrac{\partial{\Phi } } {\partial{y} } = 0 \\
\end{displaymath}
- for $j=1+jphext$, $jmax+1+jphext$ and $z = 0, H$:
\begin{displaymath}
\delta _z \dfrac{\partial{\Phi } } {\partial{y} } = 0
\end{displaymath}

\par The corner points, i.e. $\Phi (jphext, jphext, jpvext)$,
$\Phi (imax+1+jphext, jphext, jpvext)$, $\Phi (jphext,$ $jmax+1+jphext, jpvext)$,
$\Phi (imax+1+jphext, jmax+1+jphext, jpvext)$ and
their corresponding points at $k = kmax+1+jpvext$, are not required for the
resolution of the pressure equation.
%%
%%%%%%%%%%%%%%%%%%%%%%%%%%%%%%%%%%%%%%%%%%%%%%%%%%%%%%%%%%%%%%%%%%%%%%%%%%%%%%%
%%%%%%%%%%%%%%%%%%%%%%%%%%%%%%%%%%%%%%%%%%%%%%%%%%%%%%%%%%%%%%%%%%%%%%%%%%%%%%%
%%
\subsection{Flat operator}
%%%
\par We want to invert the following system (\ref{flat1}) where $Y$ is known
and $F$ is the flat operator.
\begin{equation}
\label{flat1}
F \Phi = Y
\end{equation}
The discretized form of the flat operator $F$ (\ref{flatop1})  is written as:
\begin{equation}
\label{flatop2}
F \Phi = <\tilde{\rho}>_{x, y} \left[ \dfrac {1} {<{d_{xx}}^2>_{xy}} \delta _{x^2}
 \Phi
+ \dfrac{1}{<{d_{yy}}^2>_{xy}}\delta _{y^2} \Phi \right]
+ \delta _{Z}  \overline{<\tilde{\rho} >_{x, y}}^{z} \dfrac {1}
{<{d_{zz}}^2>_{xy} } \delta _{Z}\Phi
\end{equation}
The method consists in treating the horizontal part of F in the Fourier space.
First, we compute the FFT of $Y$ noted $\hat{Y}$. Then, we introduce the
horizontal Fourier decomposition of $F$ completed by its classical vertical
part. This results in $imax \times jmax$ tridiagonal matrices where every
matrix corresponds to a different horizontal mode. A classical tridiagonal
matrix inversion is performed for each horizontal mode ($m, n$), which allows
us to compute the solution of (\ref{flat1}) in the Fourier space and denoted
$\hat{\Phi}$. Finally, we apply an inverse FFT to obtain $\Phi$ in the physical
space. As it is mentionned in the FFT documentation, the numbers of horizontal
modes $imax$ and $jmax$ should be a composite number
factorizable as a product of powers of 2, 3 and 5. Vectorization is achieved by
doing parallel transforms.
%%%%%%%%%%%%%%%%%%%%%%%%%%%%%%%%%%%%%%%%%%%%
\subsubsection{Horizontal discretization}
The FFT form depends on the lateral boundary conditions:\\
%%%%%%%%%%%%%%%%%%%%%%%%%%%%%%%%%%%%%%%%%%%%
$\bullet$ \underline{Cyclic case:}\\
%%%%%%%%%%%%%%%%%%%%%%%%%%%%%%%%%%%%%%%%%%%%
In the aim of no confusion with the complex number $i$ ($i^2 = -1$), we adopt
for a while the convention $(i, j, k) \rightarrow (I, J, K)$. With the boundary
conditions (\ref{cyclic1})-(\ref{cyclic2}), we have:
\begin{equation}
\label{foux1}
\Phi (I, J, K) =   \sum_{m=0}^{imax-1} \sum_{n=0}^{jmax-1}
\widehat{\Phi}_{m\, n}(K) e^ {i\frac {2\pi}{imax} m I \, + \,
i\frac {2\pi}{jmax} n J }
\end{equation}
We denote $\tilde{\Phi} _{m n}(K) = \widehat{\Phi} _{m n}(K)
e^ {i\frac {2\pi}{imax} m I \, + \, i\frac {2\pi}{jmax} n J } $ where
$\tilde{\Phi}_{m\, n}(K)$ is complex. Using the FFT decomposition
(\ref{foux1}), we have:\\
\begin{eqnarray*}
\delta _{x^2} \tilde{\Phi} _{m n} = - 4 \sin ^2 \left( \frac {\pi} {imax}
 m \right)     \tilde{\Phi} _{m n}  \\
\delta _{y^2} \tilde{\Phi}_{m n}  = - 4 \sin ^2 \left( \frac {\pi} {jmax}
 n \right)      \tilde{\Phi} _{m n}
\end{eqnarray*}
The horizontal part of the operator $F$ can now be written as:
\begin{equation}
\label{bmn}
<\tilde{\rho}>_{x, y} \left[ \dfrac {1} {<{d_{xx}}^2>_{xy}} \delta _{x^2}
+ \dfrac{1}{<{d_{yy}}^2>_{xy}}\delta _{y^2} \right]
\tilde{\Phi} _{m n} = b_{mn}  \tilde{\Phi}_{m n}
\end{equation}
where the eigenvalues are defined by:
\begin{displaymath}
b_{mn} = - 4 <\tilde{\rho}>_{x,y} \left[
\dfrac{\sin ^2 \left( \frac {\pi} {imax} m \right)} {<{d_{xx}}^2>_{xy}}   +
\dfrac{\sin ^2 \left( \frac {\pi} {jmax} n \right)} {<{d_{yy}}^2>_{xy}}
\right]
\end{displaymath}
where $m\,=\,0, \, 1...imax-1$ and $n\,=\,0,\, 1,...jmax-1$.\\
%%%%%%%%%%%%%%%%%%%%%%%%%%%%%%%%%%%%%%%%%%%%
$\bullet$ \underline{Open and Davies' cases:}\\
%%%%%%%%%%%%%%%%%%%%%%%%%%%%%%%%%%%%%%%%%%%%
The degeneracy of the full discrete operator with orography gives:
\begin{eqnarray*}
\dfrac{\partial \Phi}{\partial x} \rightarrow \delta _x \Phi \\
\dfrac{\partial \Phi}{\partial y} \rightarrow \delta _y \Phi
\end{eqnarray*}
We obtain an homogeneous problem by changing our variable $\Phi$ at the last
point to have an homogeneous condition.
 Then, we employ a FFT cosine decomposition expressed as:
\begin{equation}
\label{foux2}
\Phi (I, J, K) =    \sum_{m=0}^{imax-1} \sum_{n=0}^{jmax-1}
\widehat{\Phi}_{m \, n}(K)
\cos \left( \dfrac {(2I - 1) \pi}{2 \, imax} m\right) \times
\cos \left( \dfrac {(2J - 1) \pi}{2 \, jmax} n\right)
\end{equation}
In this case, the eigenvalues of $F$ are expressed as:
\begin{displaymath}
b_{mn} = - 4 <\tilde{\rho}>_{x,y} \left[
\dfrac{\sin ^2 \left( \frac { \pi} {2 \, imax}  m\right)} {<{d_{xx}}^2>_{xy}}
+
\dfrac{\sin ^2 \left( \frac { \pi} {2 \, jmax}  n\right)} {<{d_{yy}}^2>_{xy}}
\right]
\end{displaymath}
where $m\,=\,0, \, 1...imax-1$ and $n\,=\,0,\, 1,...jmax-1$.\\
%%%%%%%%%%%%%%%%%%%%%%%%%%%%%%%%%%%%%%%%%%%%
$\bullet$ \underline{Mixed boundary conditions:}\\
%%%%%%%%%%%%%%%%%%%%%%%%%%%%%%%%%%%%%%%%%%%%
Here, the decomposition of $\Phi$ results from a combination of (\ref{foux1})
and (\ref{foux2}). If we have a cyclic condition along $x$ and an open
condition along $y$, we have:
\begin{equation}
\Phi (I, J, K) =   \sum_{m=0}^{imax-1} \sum_{n=0}^{jmax-1}
\widehat{\Phi}_{m\, n}(K)
\left[ e^ {i\frac {2\pi}{imax} m I } \times
\cos \left( \dfrac {(2J - 1) \pi}{2 \, jmax} n\right) \right]
\end{equation}
and the eigenvalues are defined by:
\begin{equation}
b_{mn} = - 4 < \tilde{\rho} >_{x,y} \left[
\dfrac{\sin ^2 \left( \frac {\pi} {imax} m \right)} {<{d_{xx}}^2>_{xy}} +
\dfrac{\sin ^2 \left( \frac {\pi} {2 \, jmax}  n\right)} {<{d_{yy}}^2>_{xy}}
\right]
\end{equation}
In the other case, if we have an open condition along $x$ and a cyclic
condition along $y$, the decomposition of $\Phi$ reads:
\begin{equation}
\Phi (I, J, K) =   \sum_{m=0}^{imax-1} \sum_{n=0}^{jmax-1}
\widehat{\Phi}_{m\, n}(K)
\cos \left( \dfrac {(2I - 1) \pi}{2 \, imax} m\right) \times
e^ { i\frac {2\pi}{jmax} n J }
\end{equation}
and the eigenvalues are expressed by:
\begin{equation}
b_{mn} = - 4 < \tilde{\rho} >_{x,y} \left[
\dfrac{\sin ^2 \left( \frac { \pi} {2 \, imax}  m\right)} {<{d_{xx}}^2>_{xy}}
 +
\dfrac{\sin ^2 \left( \frac { \pi} {jmax}  n\right)} {<{d_{yy}}^2>_{xy}}
 \right]
\end{equation}
%%
%%%%%%%%%%%%%%%%%%%%%%%%%%%%%%%%%%%%%%%%%%%%%%%%%%%%%%%%%%%%%%%%%%%%%%%%%%%%%%%
%%
\subsubsection{Vertical discretization}
%%%%%%%%%%%%%%%%%%%%%%%%%%%%%%%%%%%%%%%%%%%%%%%%%%%%%%%%%%%%%%%%%%%%%%%%%%%%%%
The discretized vertical operator is applied after the horizontal operator with
boundary conditions derived from the full problem with orography:
\begin{equation}
\left( \tilde{\rho}  \overrightarrow{\nabla} \Phi \right) ^{c3}
\rightarrow \dfrac{\overline{< \tilde{\rho} >_{x,y}}^z} {<{d_{zz}}^2>_{xy}}
\delta _z \Phi
\end{equation}
Thus, we have:
\begin{displaymath}
F \tilde{\Phi} _{m n} (k) =
a(k) \tilde{\Phi} _{m n }  (k-1) + b(k) \tilde{ \Phi} _{m n} (k)  +  c(k)
\tilde{\Phi} _{m n} (k+1)
\end{displaymath}
where
\begin{eqnarray*}
a(k) &=& \frac {< \tilde{\rho} >_{x,y}(k-1) + < \tilde{\rho} >_{x,y}(k)}
 {2 \,\Delta z(k)^2} \\
b(k) &=& -  \dfrac {< \tilde{\rho} >_{x,y}(k-1) +  < \tilde{\rho} >_{x,y}(k)}
{2 \, \Delta z(k)^2}
-  \dfrac { < \tilde{\rho} >_{x,y}(k)+< \tilde{\rho} >_{x,y}(k+1)}
{2 \, \Delta z(k+1)^2}  + b_{mn} \\
c(k) &=& \frac {< \tilde{\rho} >_{x,y}(k) + < \tilde{\rho} >_{x,y}(k+1) }
 {2 \, \Delta z(k+1)^2}
\end{eqnarray*}
%%%%%%%%%%%%%%%%%%%%%%%%%%%%%%%%%%%%%%%%%%%%%%%%%%%%%%%%%%%%%%%%%%%%%%%%%%%%
\subsubsection{Complete flat operator}
%%%%%%%%%%%%%%%%%%%%%%%%%%%%%%%%%%%%%%%%%%%%%%%%%%%%%%%%%%%%%%%%%%%%%%%%%%%%%
Thus, the flat operator is expressed under the form of $m n$ tridiagonal
matrices of $kmax+2 \times kmax+2$ elements with the following form:
\begin{displaymath}
{\bf F_{mn}}=
\left(
\begin{array}{cccccc}
b(jpvext) & c(jpvext)         &   &   &   &   \\
a(1+jpvext) & b(1+jpvext) & c(1+jpvext)          &   &   &   \\
     &      &                &   &   &   \\
     &      &             &   &   & \huge{0}  \\
     &      &                 &   &   &   \\
     &      &                 &   &   &   \\
\huge{0}     &      &               &   &   &   \\
     &  a(kmax+jpvext)    &  b(kmax+jpvext)         &  c(kmax+jpvext)&   &   \\
     &      &                      a(kmax+1+jpvext) &  b(kmax+1+jpvext) & &\\
\end{array}
\right)
\end{displaymath}
The $ F_{mn}$ components do not depend on time and are
stored in arrays of:\\
- $imax \times jmax \times (kmax+2)$ elements for the $b$ coefficients;\\
- $kmax+2$ elements for the $a$ and $c$ coefficients.\\
%%
All the matrices $F_{mn}$ are invertible except for
($m, n$) = (0, 0) (the uniform mode) because the vertical Neumann conditions
involve an evaluation of the pressure terms to within a constant.
For this reason, this particular mode is inverted apart. Moreover, as all the
$F_{mn}$ vertical matrices are independent,
this computation can easily be vectorized for the $m \, n$ different modes.

{\bf Algorithm Performance}

Both iterative methods presented above showed a good convergence for the
solution of the elliptic equation. All in all, the Richardson method
requires less computation time than the CG method, because it needs less
matrix inversions.

{\bf Current Limitations}

At the present time, the algorithm is not adapted for large stretching
of the horizontal grid. A practical maximum ratio of the larger to the
smaller grid size is about two, which is sufficient to allow simulations
on small domains in conformal projection.

%%%%%%%%%%%%%%%%%%%%%%%%%%%%%%%%%%%%%%%%%%%%%%%%%%%%%%%%%%%%%%%%%%%%%%%%%%%%%
\chapter{Initial Fields for Idealized Flows}

%{\em by J. Stein}

In many occasions, we want to run simulations starting from highly idealized
conditions, such as an atmosphere in geostrophic equilibrium, with simple
vertical and horizontal structure, and simple orography, or no orography
at all.  This is useful
for theoretical research on geophysical flows, or simply to test if recent
changes in the code have not been detrimental to the quality of previous
results.  Meso-NH possesses a general program to prepare initial data for
such experiments, named PREP\_IDEAL\_CASE.

This program reads data on the
vertical structure of the atmosphere from an input file, that may take a
variety of formats for convenience, and prepares a file containing a 2D or 3D
initial state satisfying the geostrophic and anelastic balances,
according to various specifications for the geographical setting.
It is also possible to prepare non balanced fields following simple mathematical
formulas, in order to test the good functioning of the post-processing
and graphical facility (see last section).

The fields are produced either for a cartesian geometry, or for a conformal
projection.  A stretched vertical grid may be used, but only regular horizontal
grids are supported in this program. The orography can be zero everywhere,
or defined by simple bell-shaped or sine-shaped mountains.

\section{Input Atmospheric Profile}
The input atmospheric profile may be specified in several ways, depending
on the willingness of the user to use observations (or pseudo observations)
in various formats, or to achieve a highly idealized experiment.

{\bf Sounding Formats } (Option: 'RSOU')

If one wishes to use observations or pseudo-observations, the most convenient
way to communicate with the program is to use one of several available
"sounding" formats. The data consist of surface level information, followed
by $nu$ levels of temperature and moisture data, and $nm$ levels of wind data.
The date and time of the observation must also be supplied.

The following  data formats are supported, and may be selected by specification
of the appropriate value for the parameter YKIND:
\begin{itemize}
\item 'STANDARD'  :
\begin{itemize} \item altitude, pressure, temperature and dew point temperature at
ground level,
\item $nu$ levels with pressure, wind direction and wind force,
\item and $nm$
levels with pressure, temperature and dew point temperature.
\end{itemize}
\item 'PUVTHVMR'  : \begin{itemize} \item altitude, pressure, virtual potential temperature
and vapor mixing ratio at ground level,\item   $nu$ levels with pressure, zonal wind
component, meridian wind component,\item  $nm$ levels with pressure, virtual potential temperature
and vapor mixing ratio.\end{itemize}
\item 'PUVTHVHU' : \begin{itemize} \item altitude, pressure, virtual potential temperature
and relative humidity at ground level , \item $nu$ levels with pressure, zonal wind
component, meridian wind component,\item  $nm$ levels with pressure, virtual potential temperature
and relative humidity \end{itemize}
\item 'ZUVTHVHU'  : \begin{itemize} \item altitude, pressure, virtual potential temperature
and relative humidity at ground level,\item    $nu$ levels with altitude, zonal wind
component, meridian wind component, \item  $nm$ levels with altitude, virtual potential temperature
and relative humidity.\end{itemize}
\item 'ZUVTHVMR'  :  \begin{itemize} \item  altitude, pressure, virtual potential temperature
and vapor mixing ratio at ground level, \item  $nu$ levels with altitude, zonal wind
component, meridian wind component, \item  $nm$ levels with   altitude, virtual potential temperature
and vapor mixing ratio. \end{itemize}
\item 'PUVTHDMR'  : \begin{itemize} \item   altitude, pressure, dry potential temperature
and vapor mixing ratio at ground level, \item  $nu$ levels with pressure, zonal wind
component, meridian wind component, \item  $nm$ levels with pressure,
dry potential temperature and vapor mixing ratio.\end{itemize}
\item 'PUVTHDHU'  : \begin{itemize} \item altitude, pressure, dry potential temperature
and relative humidity at ground level, \item  $nu$ levels with pressure, zonal wind
component, meridian wind component, \item  $nm$ levels with pressure,
dry potential temperature and relative humidity.\end{itemize}
\item 'ZUVTHDMR'  : \begin{itemize} \item   altitude, pressure, dry potential temperature
and vapor mixing ratio at ground level, \item  $nu$ levels with  altitude, zonal wind
component, meridian wind component, \item  $nm$ levels with  altitude,
dry potential temperature and vapor mixing ratio.\end{itemize}
\end{itemize}

{\bf More Idealized Format} (Option: 'CSTN')

A more idealized manner to communicate with the program is to assume that
the atmosphere is composed of an arbitrary number $n-1$ of layers defined
by their altitudes $Z(k)$. Within each layer,
the stability is constant, and the wind and relative humidity vary linearly
with height. In that case, the input data consist in the surface level
information, followed by the moist V\"ais\"al\"a frequency of each layer
$N_v$, and the wind components and relative humidity at the layer
interfaces.
The virtual potential temperature on the $n$ input levels is retrieved
from the $N_{v}$ data by
\begin{equation}
\theta_{v}(k)  = \theta_{v}(k-1) \exp\left\{ \dfrac{N_{v}^{2}(k-1)}{g}
\left( Z(k)-Z(k-1) \right) \right\}
\end{equation}
for $k=2,n$. The situation concerning input data is then the same as for
the 'ZUVTHVHU' case in the 'RSOU' option, and subsequent treatment is similar.

{\bf Specifying a jet structure}

For other applications, one may want to have a jet structure $U(y,z)$ instead
of a horizontally uniform wind $U(z)$. Then we directly prescribe through
a Fortran function the following  value of the x-component wind U(y,z):

\begin{equation}
U(\hat{y},\hat{z})= { 1 \over \cosh \left(
  \left( { \hat{y} - \hat{y}_0 \over zwidthy } \right) ^2 +
  \left( { \hat{z} - \hat{z}_0 \over zwidthz } \right) ^2
 \right) }
\end{equation}

where $ zwidthy$ and  $ zwidthz $ are the characteristic sizes of the jet along
the horizontal and vertical directions. $\hat{y}_0$ and $\hat{z}_0$ are the
curvilinear coordinates  for the jet center.

An intermediate case is also available where the function U(y,z) is a separable
function of y and z:

$$ U(y,z) = U(z) * f(y)$$
 The vertical part $ U(z)$  is still obtained from the input atmospheric profile and
the horizontal part $f(y)$ is given by a Fortran function:
\begin{equation}
 f(\hat{y}) = { 1 \over \cosh \left(
   { \hat{y} - \hat{y_0} \over zwidth } \right) }
\end{equation}
with the same notations as for the jet case.

The jet structure or the separable function, can also be taken into account for
the V(x,z). All combinations between U(y,z) and V(x,z) (function of only z,
separable or non-separable function) are possible.


All these options lead to the prescription of the wind components along vertical
planes (I,K) for V(x,z) and (J,K) for U(y,z) and subsequent treatments
to balance the wind and the mass fields are the same.

\section{Vertical Profile on the Model Grid}

For all the above options, the program will first convert the vertical profile
 in a common format, featuring only the surface altitude, pressure, virtual
temperature and vapor mixing ratio, together with the altitudes of the
data points, and the corresponding values of $\theta_v, r_v$, and wind
components.

The correspondance between pressure and altitude is achieved, when necessary,
by integration of the hydrostatic relation, starting from the surface level:
\begin{equation}
d \Pi = - \dfrac{g}{C_{pd} \theta_{v}}  dz
\end{equation}
with $P = P_{00} \Pi^{C_{pd}/R_{d}} $.

The virtual temperature is obtained as
\begin{equation}
T_{v} = \theta_{v} \Pi = \theta_{v} \left( \dfrac{P}{P_{00}} \right) ^{R_{d}/C_{pd}}
\end{equation}



In some cases, this requires the use of an iterative procedure
to derive the absolute temperature and the vapor mixing ratio from
available information (virtual temperature and relative humidity).
This is done independantly for each level, and
comprises the following steps, starting with $T=T_v$ as first guess:
\begin{enumerate}
\item The saturation vapor pressure $e_{s}(T)$ at temperature T is computed  :
$$
e_{s}(T) =  \exp\left( \alpha_{w} - \dfrac{\beta_{w}}{T} - \gamma_{w}
\ln (T)\right)
$$

with
\begin{eqnarray*}
\alpha_{w} & = &  \ln (e_{s}(T_{t}) ) + \dfrac{\beta_{w}}{T_{t}} + \gamma_{w}
\ln (T_{t}) \\
\beta_{w} & = & \dfrac{L_{v}(T_{t})}{R_{v}} + \gamma_{w} T_{t} \\
\gamma_{w} & = & \dfrac{C_{l}-C_{pv}}{R_{v}}
\end{eqnarray*}
where $T_t$ is the triple point temperature.

\item  The partial pressure of the vapor $e= e_{s}(T_d)$
is computed from the relative humidity $Hu$ (in percents) and the saturation
vapor pressure $e_{s}(T)$ :
$$
e= \dfrac{Hu}{100} e_{s}(T)
$$
\item  The vapor mixing ratio is evaluated by :
$$
r_{v} = \dfrac{R_{d}}{R_{v}} \;\; \dfrac{e}{P-e}
$$
\item   The temperature is deduced :
$$
T = T_{v} \dfrac{(1+r_{v})}{(1 + \dfrac{R_{v}}{R_{d}}r_{v})}
$$
\end{enumerate}
The iteration continues until convergence is achieved.

Once the variables are known on data points, they are linearly
interpolated to the vertical grid of the model without orography.
Here $\theta_v$ and $r_v$ are located on the mass points of the grid,
but the wind components are located on $w$ points for more accurate
estimate of the thermal wind balance. They will be called hereafter
$U$ and $V$.  The
next sections explain how the 3D fields are produced, starting from this
single profile.

\section{Wind and Mass Fields in the absence of Orography}

The vertical profile obtained above is supposed to be valid at the point
$ILOC,JLOC$ of the horizontal model grid, and we want to reconstitute the
whole 3D fields.  As a first step in this computation, the program computes
the wind and mass fields on a 3D grid ignoring the orography (this grid is
exactly the model grid if there is no orography at all).

The wind is first projected on to model axes. At this stage,
the desired jet structure is introduced (to be completed by J. Stein).

The virtual potential temperature on the model grid without
orography is then derived using the thermal wind balance, assuming the above
mentionned
$U$ and $V$ components of the wind can be taken as the geostrophic wind.

The continuous equation of the thermal wind balance is :
\begin{equation}
\dfrac{\partial \vec{V_{g}}}{\partial z} =
\dfrac{\vec{k}}{f} \wedge \dfrac{1}{\theta_{v\,ref}} g \vec{\nabla}_{z}
\theta_{v}
\end{equation}
where
\begin{equation}
\dfrac{\partial \Phi}{\partial z} = g \dfrac{ \theta_v - \theta_{v\,ref} }
{\theta_{v\,ref} }
\end{equation}
has been used. It follows
\begin{eqnarray}
\dfrac{\partial \theta_{v} }{\partial \overline{x}} & = &
d_{xx} \dfrac{f}{g}  \theta_{v\,ref}
\dfrac{1}{\widehat{d}_{zz}} \dfrac{\partial v_{g} }{\partial \widehat{z}}
\\
\dfrac{\partial \theta_{v} }{\partial \overline{y}} &  = & - d_{yy}
\dfrac{f}{g} \theta_{v\,ref}
\dfrac{1}{\widehat{d}_{zz}} \dfrac{\partial u_{g} }{\partial \widehat{z}}
\end{eqnarray}

This is discretized as
\begin{eqnarray}
\left( \delta _{x} \theta_{v}\right)_{(i,j,k)} & = & d_{xx}
 \dfrac{\overline{f}^x}{g}
\theta_{v\,ref}(k) \dfrac{1}{d_{zz}} \overline{\delta_{z} V}^x  \\
\left( \delta _{y} \theta_{v}\right)_{(i,j,k)} & = & - d_{yy}
 \dfrac{\overline{f}^y}{g}
\theta_{v\,ref}(k) \dfrac{1}{d_{zz}} \overline{\delta_{z} U }^y
\end{eqnarray}
where $U$ and $V$ are used as geostrophic wind components.

If the Coriolis parameter is equal to zero, $\theta_{v}$ is defined everywhere
as $\theta_{v}(k)$ at point $(ILOC,JLOC)$.

Note that at this stage $\theta_{v\,ref}$ is unknown, and an iterative
procedure must be used to enforce an exact
 thermal wind balance. For the first iteration, we initialize
$\theta_{v\,ref}$ by the value of $\theta_{v}$ at the point
$(ILOC,JLOC)$. At the following steps,
the virtual potential temperature for the anelastic reference state
is defined as the horizontal average of $\theta_{v}$ :

$$
\theta_{v\,ref}(k) = \sum _{i,j} \theta_{v}(i,j,k)
$$

When the iteration is converged,
the remaining properties of the reference state can be computed.
The Exner function for the
reference state may then be obtained by integration of the hydrostatic relation
from $z=0$ ($\Pi_{ref} (k=KB) =1$) :

$$
d \Pi_{ref}(k) = - \dfrac{g}{C_{pd} \theta_{v\,ref}(k)}  dz
$$
and the computation of the reference state characteristics is finalized
by the density of dry air:
$$
\rho_{d\,ref}(k) = \dfrac{P_{00} (\Pi)^{C_{pd}/R_d - 1} }{R_{d}
\theta_{v\,ref}(k) \left(1+r_{v}(k) \right) }
$$
\section{Add a Perturbation}

In order to start a non-trivial experiment, it is often useful to add
a perturbation to the previously defined balanced fields. Two possibilities
are presently available:

\begin{itemize}
\item
add a perturbation to the $\theta $ field ( convective bubble)
\begin{eqnarray}
&&Dist = \sqrt{ \left( \left( { \hat{x} - \hat{x}_0    \over  Rad_x } \right) ^2 +
                       \left( { \hat{y} - \hat{y}_0    \over  Rad_y } \right) ^2 +
                       \left( { \hat{z} - \hat{z}_0    \over  Rad_z } \right) ^2
                \right)       }  \nonumber \\
&&\theta ' =  Ampli \cos ^2  \left( {\pi \over 2} Dist  \right)  \ \ \
for \ Dist < 1  \nonumber \\
&&\theta ' = 0  \ \ \  \ \ \  \ \ \  elsewhere  \nonumber
\end{eqnarray}

\item
add a non-divergent perturbation to the horizontal wind, through the use of a
streamfunction $\Phi '$ :
\begin{equation}
\Phi ' = Ampli \ \  e^{\left( { \hat{y} - \hat{y}_0    \over  Rad_y } \right) ^2 }
\  \cos \left( {2 \pi  \hat{x} \over  Rad_x } \right)   \nonumber
\end{equation}

This streamfunction is located at the vertical vorticity point and the $u'$
and $v'$ perturbations are computed at the right locations by :
\begin{eqnarray}
u' = - { \partial \Phi ' \over \partial \hat{y} } \nonumber \\
v' = + { \partial \Phi ' \over \partial \hat{x} } \nonumber
\end{eqnarray}
The discretizations use the grid-point curvilinear $ \hat{x}, \hat{y}$ because
at this point, the orography has  not yet  being  introduced in  the model.
This guarantees that the perturbation does not change the  divergence of the
previously obtained fields.
\end{itemize}



\section{Wind and Mass Fields with Orography}

An orography following simple form can be specified:

It may be sine shaped:
$$
z_{s}(i,j) = h_{max} \left\{  \sin \left( \dfrac{\pi i}{IU-1}\right)^{2}
\right\}^{exp_{x}}\left\{  \sin \left( \dfrac{\pi j}{JU-1}\right)^{2}
\right\}^{exp_{y}}
$$

or bell-shaped :
$$
z_{s}(i,j) = h_{max} / \left\{ 1 + \left(\dfrac{x(i) - x_{s}}{a_{x}}\right)^{2}
+ \left(\dfrac{y(j) - y_{s}}{a_{y}}\right)^{2}
\right\}^{1.5}
$$
$IU$ and $JU$ are the total number of points along x and y directions (
{\it i.e.}
including the external points). $h_{max}$, $exp_{x}$, $exp_{y}$, $a_{x}$,
$a_{y}$,$x_{s}$ and $y_{s}$ are the  degrees of freedom. $h_{max}$ is the maximum
height, $a_{x}$ and
$a_{y}$ represent widths along x and y directions,$x_{s}$ and $y_{s}$  are
the center
positions of the mountain.

In that case, an additional step is performed
to compute the 3D field on the model grid following the terrain.
The previous fields of $\theta_v$, $r_v$, $u$, and $v$
are linearly interpolated to the new grid. We also set the vertical component
$w$ equal to zero.
Then, $\theta$ is recovered by :
\begin{equation}
\theta = \theta_{v}\dfrac{(1+r_{v})}{(1 + \dfrac{R_{v}}{R_{d}}r_{v})}
\end{equation}
Finally, the  3D anelastic reference state (
{\it i.e.} on the grid with orography) is determined
 by vertical interpolation for  $\theta_{v\,ref}$ and
$\rho_{d\,ref}$ and by integration of the hydrostatic relation for
$\Pi_{ref}$
defined on the model grid without  orography.

\section{Enforcement of the Anelastic Constraint}

Obviously, wind field obtained at this stage does not fulfil the anelastic
constraint (\ref{anelcons}) and the
free-slip boundary conditions. Therefore, a
last adjustement is necessary so that this wind field can be used as initial
field for the MESONH model. Thus, a perturbation deduced from a potential $Pot$
gradient  is added to the wind component $\vec{U}_0$ so that the resulting
 wind field $\vec{U}$  satisfies (\ref{anelcons}).

Let us write this constraint:

\begin{eqnarray}
\vec{U} = \vec{U} _0 - \vec {\nabla } Pot  \nonumber \\
\vec { \nabla } . \left( \rho _{dref}  \vec{U}  \right) = 0 \nonumber
\end{eqnarray}

From these 2 equations, we deduce an equation for the $Pot$ field given by:

\begin{eqnarray}
\vec { \nabla } .  \left( \rho _{dref} \vec {\nabla } Pot \right) = \vec { \nabla } .
 \left( \rho _{dref}  \vec{U} _0 \right) \label{eq:potential}
\end{eqnarray}

This equation is equivalent to the Pressure equation solved in the temporal loop
of the model to determine the pressure function. Therefore, the same iterative
solver is used to solve the equation \ref{eq:potential} and the gradient of this
potential is added to find the final field $\vec{U}$ satifying (\ref{anelcons}).


\section{Analytical 3D fields}

For the purpose of testing the post-processing and graphical facility, it
is also possible to generate a family of simple 3D analytical fields.
Three types of such fields are possible: spherical fields, plane
fields and wave fields.
\begin{itemize}
\item \underbar{spherical fields} :
\begin{eqnarray}
\overline{\rho^{*}}^{x} {\cal U} & = &  \overline{\rho^{*}}^{x}
\left\{ f_{0} + 0.5  (f_{1}- f_{0}) \sqrt{\dfrac{x-x_{0}}{L_{x}}
+ \dfrac{y_{m}-y_{0}}{L_{y}} + \dfrac{z_{m}-z_{0}}{L_{z}}}    \right\}
\\
\overline{\rho^{*}}^{y}{\cal V} & = &    \overline{\rho^{*}}^{y}
\left\{ f_{0} + 0.5  (f_{1}- f_{0}) \sqrt{\dfrac{x_{m}-x_{0}}{L_{x}}
+ \dfrac{y-y_{0}}{L_{y}} + \dfrac{z_{m}-z_{0}}{L_{z}}}    \right\}
\\
\overline{ \rho^{*}}^{z} w & = &  \overline{\rho^{*}}^{z}
\left\{ f_{0} + 0.5  (f_{1}- f_{0}) \sqrt{\dfrac{x_{m}-x_{0}}{L_{x}}
+ \dfrac{y_{m}-y_{0}}{L_{y}} + \dfrac{z-z_{0}}{L_{z}}}    \right\}
 \\
\rho^{*} \theta(k) & = &   \rho^{*} \left\{ f_{0}
+ 0.5  (f_{1}- f_{0}) \sqrt{\dfrac{x_{m}-x_{0}}{L_{x}}
+ \dfrac{y_{m}-y_{0}}{L_{y}} + \dfrac{z_{m}-z_{0}}{L_{z}}}    \right\}
\\
\rho^{*} r_{n} & = & 0.
\end{eqnarray}

\item \underbar{plane fields} :
\begin{eqnarray}
\overline{\rho^{*}}^{x} {\cal U} & = & \overline{\rho^{*}}^{x}
\left\{ f_{0} + (f_{1}- f_{0}) \left( \dfrac{x-x_{0}}{L_{x}} N_{x}
+ \dfrac{y_{m}-y_{0}}{L_{y}} N_{y} + \dfrac{z_{m}-z_{0}}{L_{z}} N_{z} \right)\right\}\\
\overline{\rho^{*}}^{y}{\cal V} & = &    \overline{\rho^{*}}^{y}
\left\{ f_{0} + (f_{1}- f_{0}) \left( \dfrac{x_{m}-x_{0}}{L_{x}} N_{x}
+ \dfrac{y-y_{0}}{L_{y}} N_{y} + \dfrac{z_{m}-z_{0}}{L_{z}} N_{z}\right)\right\} \\
\overline{ \rho^{*}}^{z} w & = &  \overline{\rho^{*}}^{z}
\left\{ f_{0} + (f_{1}- f_{0}) \left( \dfrac{x_{m}-x_{0}}{L_{x}} N_{x}
+ \dfrac{y_{m}-y_{0}}{L_{y}} N_{y} + \dfrac{z-z_{0}}{L_{z}} N_{z}\right)\right\} \\
\rho^{*} \theta(k) & = &   \rho^{*}
\left\{ f_{0} + (f_{1}- f_{0}) \left( \dfrac{x_{m}-x_{0}}{L_{x}} N_{x}
+ \dfrac{y_{m}-y_{0}}{L_{y}} N_{y} + \dfrac{z_{m}-z_{0}}{L_{z}} N_{z} \right)\right\}\\
\\
\rho^{*} r_{n} & = & 0.
\end{eqnarray}
where
\begin{eqnarray}
N_{x} & = & \cos(\alpha_{z})\sin(\alpha_{x}) \\
N_{y} & = & - \cos(\alpha_{z})\cos(\alpha_{x}) \\
N_{y} & = & - \sin(\alpha_{z})
\end{eqnarray}
\item \underbar{wave  fields} :
\begin{eqnarray}
\overline{\rho^{*}}^{x} {\cal U} & = & \overline{\rho^{*}}^{x}
\left\{ f_{0} + (f_{1}- f_{0}) \left( \dfrac{x-x_{0}}{L_{x}} N_{x}
+ \dfrac{y_{m}-y_{0}}{L_{y}} N_{y} + \dfrac{z_{m}-z_{0}}{L_{z}} N_{z}
\right) \right. \\
& + & \left. 0.125 (f_{1}+ f_{0}) \sin \left(2 \pi f_{2} (\dfrac{x-x_{0}}{L_{x}} C_{s}
+ \dfrac{y_{m}-y_{0}}{L_{y}} S_{n}) \right) \right\}
 \\
\overline{\rho^{*}}^{y}{\cal V} & = &    \overline{\rho^{*}}^{y}
\left\{ f_{0} + (f_{1}- f_{0}) \left( \dfrac{x_{m}-x_{0}}{L_{x}} N_{x}
+ \dfrac{y-y_{0}}{L_{y}} N_{y} + \dfrac{z_{m}-z_{0}}{L_{z}} N_{z} \right)\right.
\\ & + & \left.0.125 (f_{1}+ f_{0}) \sin \left( 2 \pi f_{2}( \dfrac{x_{m}-x_{0}}{L_{x}} C_{s}
+ \dfrac{y-y_{0}}{L_{y}} S_{n} ) \right) \right\}\\
\overline{ \rho^{*}}^{z} w & = &  \overline{\rho^{*}}^{z}
\left\{ f_{0} + (f_{1}- f_{0}) \left( \dfrac{x_{m}-x_{0}}{L_{x}} N_{x}
+ \dfrac{y_{m}-y_{0}}{L_{y}} N_{y} + \dfrac{z-z_{0}}{L_{z}} N_{z}\right)\right. \\
& + & \left. 0.125 (f_{1}+ f_{0}) \sin \left( 2 \pi f_{2} (\dfrac{x_{m}-x_{0}}{L_{x}} C_{s}
+ \dfrac{y_{m}-y_{0}}{L_{y}} S_{n} ) \right)\right\} \\
\rho^{*} \theta(k) & = &   \rho^{*}
\left\{ f_{0} + (f_{1}- f_{0}) \left( \dfrac{x_{m}-x_{0}}{L_{x}} N_{x}
+ \dfrac{y_{m}-y_{0}}{L_{y}} N_{y} + \dfrac{z_{m}-z_{0}}{L_{z}} N_{z} \right)\right.
\\ & + & \left. 0.125 (f_{1}+ f_{0}) \sin \left( 2 \pi f_{2} (\dfrac{x_{m}-x_{0}}{L_{x}} C_{s}
+ \dfrac{y_{m}-y_{0}}{L_{y}} S_{n} ) \right) \right\}\\
\\
\rho^{*} r_{n} & = & 0.
\end{eqnarray}
where
\begin{eqnarray*}
C_{s} & = &  \cos(\alpha_{x}) \\
S_{n} & = & \sin(\alpha_{x})
\end{eqnarray*}

$f_{0},f_{1},f_{2},\alpha_{x},\alpha_{z}$ are the degrees of freedom . $L_{x}$,
$L_{y}$,$L_{z}$ are the lengths of the physical part of the domain. $x_{0}$,
$y_{0}$ and $z_{0}$ represent the coordinates of the center point of  the
physical part of the domain.
\end{itemize}

