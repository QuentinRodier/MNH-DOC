%%%%%%%%%%%%%%%%%%%%%%%%%%%%%%%%%%%%%%%%%%%%%%%%%%%%%%%%%%%%%%%%%%%%%%%%%%%
%%%%%%%%%%%%%%%%%%     NUMERICAL DIFFUSION SECTION    %%%%%%%%%%%%%%%%%%%%%
%%%%%%%%%%%%%%%%%%%%%%%%%%%%%%%%%%%%%%%%%%%%%%%%%%%%%%%%%%%%%%%%%%%%%%%%%%%
% March 2008, J.-P. Chaboureau, editorial corrections
% June 2008,  E. Richard, adding the horizontal diffusion over montains
%%%%%%%%%%%%%%%%%%%%%%%%%%%%%%%%%%%%%%%%%%%%%%%%%%%%%%%%%%%%%%%%%%%%%%%%%%%

\chapter{Numerical Diffusion Terms}
\label{NumericalDiffusion}
\minitoc

%{\em by J.P. Pinty}

As in most numerical models, it is necessary to prevent the occurrence of
numerical waves due to the inaccurate representation of the dynamical
processes and reflection at the top or lateral boundaries. This is done
in a fairly classical way, through (i) a weak background diffusion, (ii) a
top absorbing layer, and (iii) a lateral sponge zone. Note that in these
three regions, the flow is relaxed towards the "large-scale" values, which
may be non uniform in space, and time dependent. For idealized runs, the
user may of course choose uniform and steady large-scale values.

\section{Background diffusion}
\subsection{Diffusion operator}
A diffusion operator is applied to the {\em fluctuations} of the prognostic
variables $\phi$. The fluctuations are defined here as the departure
from the large scale value $\phi_{LS}$.
The diffusion operator is a fourth-order operator
($\delta_{x^4}$) used everywhere except at the first interior grid point where
a second-operator ($\delta_{x^2}$) is substituted in the case of non-periodic
boundary conditions.
The background diffusion source for any prognostic variable noted $\phi$
is
$$
S_{BD}=-K4 \big[ \delta_{x^4}[\phi(t-\delta t)- \phi_{LS}]+
\delta_{y^4}[\phi(t-\delta t)- \phi_{LS}]\big]
$$
or
$$
S_{BD}=+K2 \big[ \delta_{x^2}[\phi(t-\delta t)- \phi_{LS}]+
\delta_{y^2}[\phi(t-\delta t)- \phi_{LS}]\big]
$$
where $K2$ and $K4$ are positive coefficients, and $\phi_{LS}$ represents the
Large Scale value of the considered variable.


\subsection{Choice of the diffusion coefficient}
Let us consider a single harmonic wave defined by:
$$ \phi(x,t)=\Phi(t)e^{ikx}$$
where $\Phi(t)$ is the wave amplitude and $k$ the wavenumber.

The application of a second-order diffusion operator during $N$ time steps leads
to:
$$\phi(x,t+N \Delta t)=\Phi(t)[1- 2 {K_2 \Delta t \over {d_{xx}}^2}
(1- cos k d_{xx} )]^N$$
where $\Delta t$ is the time step, ${d_{xx}}^2$ the grid interval, and
$K2=K_2/{d_{xx}}^2$ the diffusion coefficient.
The time $T_2$ at which the initial wave is damped by $e^{-1}$ is then:
$$T_2=N \Delta t= {-\Delta t \over \ln [1- 2{K_2 \Delta t \over {d_{xx}}^2}
(1-cos k d_{xx} )]}$$
which can be approximated by:
$$ T_2 \simeq {{d_{xx}}^2 \over 2 K_2 (1-cos k d_{xx})}$$
The corresponding time in the case of a fourth-order diffusion operator is given
by:
$$T_4=N \Delta t= {-\Delta t \over \ln [1 - 4{K_4 \Delta t \over {d_{xx}}^4}
(1-cos k d_{xx})^2]} $$
which can be approximated by:
$$ T_4 \simeq {{d_{xx}}^4 \over {4 K_4 (1-cos k d_{xx})^2}} $$

If $k$ is the wavenumber associated to the $n d_{xx}$ wavelength, $T_2$ and
$T_4$ are given by
\begin{equation}
T_2 (n) \simeq {{d_{xx}}^2 \over {2 K_2 (1-cos (2 \pi /n))}}
\end{equation}
\begin{equation}
T_4 (n) \simeq {{d_{xx}}^4 \over {4 K_4 (1-cos (2 \pi /n))^2}}
\end{equation}

To set up  the diffusion coefficients, it might be more convenient
to specify $T_2$ or $T_4$ rather than $K_2$ or $K_4$. $T_2$ and
$T_4$  can be more easily related to the physical processes being studied.
From previous experience, $T_4(2)$
was set to 10-15 mn in the case of PBL convective
rolls, 20-30 mn for moist convection, 1-2 hours for orographic flows.

For a specified wavelength $n_0 d_{xx}$, an equivalent damping timescale with
a second-order or a fourth-order diffusion scheme requires
\begin{equation}
 K_4= - {K_2 \over 2} {{d_{xx}}^2 \over 1-cos(2 \pi /{n_0}) }.
\end{equation}
The same arguments hold for the diffusion in the $y$ direction.


In the code $n_0$ is set equal to 2 to select the highest wavenumber and so
the user specifies $T_4(2)$.
Noting that  $K2= K_{2x}/{d_{xx}}^2= K_{2y}/{d_{yy}}^2$,
 $K4= K_{4x}/{d_{xx}}^4 = K_{4y}/{d_{yy}}^4$, and according to (1), (2)
and (3)
$K4$ and $K2$ are then given by:
$$K4={1 \over 16T_{4}(2)}$$
and
$$K2=K4  ( 1-cos(2 \pi /n_0))= 4K4$$
\vskip 2truecm

\subsection{Horizontal diffusion over mountains}
Calculating the horizontal diffusion along terrain-following surfaces may 
introduce serious errors over mountainous terrain, particularly for 
atmospheric properties having a strong vertical gradient. Temperature
diffusion along such surfaces, for example, tends to cool the air in valleys
and to heat that air above mountains and likewise, diffusion of the water mixing
ratio tends to dry the air in valleys and to moisten the atmosphere above 
mountains. To reduce the unphysical impact of horizontal diffusion applied 
on terrain following surfaces, the modifications suggested in Z\"angl (2002)
have been implemented in the code. At model levels sufficiently far away from 
the ground, vertical interpolation is used to compute diffusion truly 
horizontally when the coordinate surfaces are sloping. Close to the ground, 
where truly horizontal computation of diffusion is not possible without 
intersecting the topography, diffusion is treated differently for momentum,
moisture and temperature. A simple transition to diffusion along the
coordinate surface is chosen for momentum. For the moisture variables, 
one-sided truly horizontal diffusion is used in combination with 
orography-adjusted diffusion along the $\hat z$ surfaces. This orography 
adjustment is achieved by strongly reducing the diffusion coefficient 
when the grid points involved in the diffusion computation are located 
at greatly different heights. For temperature, one sided horizontal diffusion 
is not used because it affects the slope wind circulation in an unphysical
way. However, a temperature gradient is applied to the terrain-following
part of diffusion so as to ensure that diffusion behaves neutrally with 
respect to the local vertical temperature gradient.

\section{Top absorbing layer}
To prevent spurious reflection from the model top boundary, an absorbing layer
in which damping increases with height occupies the top fraction of the
 domain. A Rayleigh damping has been chosen, it is applied on the three
components of the wind and on the thermodynamical variable. Only the
perturbations of a variable from its local large scale values are damped
on $\bar z$ surfaces. In the absorbing layer, the implicit damping source
for any variable $\phi$ is written as:
$$
S_{AL}=-K_{AL} (\bar z)\big[\phi(t+\Delta t)- \phi_{LS} \big]
      =-KAL(\bar z) \big[\phi(t-\Delta t)- \phi_{LS} \big]$$
where
$$ KAL= {K_{AL}(\bar z) \over 1+2 \Delta t K_{AL}(\bar z)}$$
and where $K_{AL}(\bar z)$ is given by
$$K_{AL}(\bar z)= K_{AL} (H) \sin ^2
\Big({\pi \over 2}
     { \bar z - \bar z_{alb} \over H- \bar z_{alb}} \Big)
\qquad for  \quad  \bar z \geq  \bar z_{alb}$$
with $ \bar z_{alb}$  the Gal-Chen and Sommerville
 height of the absorbing layer base and
$ \phi_{LS} $ the relaxation value of $\phi$. In the first version of the model
the relaxation fields are the initial fields and the maximum damping rate
$K_{AL} (H)$ must be provided for each model run.
\vskip 2truecm

\section{Lateral sponge zone}
An additional sponge zone is inserted close to the lateral boundaries to either
damp outward propagating waves or slowly incorporate inward propagating
larger scale waves. A first order damping rate has been retained and its
application to any prognostic variable $\phi$ leads to a source term of the
form:
$$
S_{SZ}=-K_{SZ} (\bar x, \bar y)\big[\phi(t+\Delta t)- \phi_{LS} \big]
      =-KSZ(\bar x, \bar y) \big[\phi(t-\Delta t)- \phi_{LS} \big]$$
where
$$ KSZ= {K_{SZ}(\bar x, \bar y) \over 1+2 \Delta t K_{SZ}(\bar x, \bar y)}.$$
The damping coefficient $KSZ(\bar x, \bar y)$ is non-zero in a rim zone of
width ${rim}_{\bar x}$ and ${rim}_{\bar y}$ (in each $x$ and $y$ direction,
respectively) following immediately the lateral boundaries. For example, near
the left ($x$) lateral boundary, the damping coefficient has the generic form:
%\begin{equation}
$$
KSZ(\bar x) = \left\{ \begin{array}{ll}
K_{SZ}^{max} \sin^2 \Big(\displaystyle{{\pi \over 2}{\bar x-{rim}_{\bar x} \over {rim}_{\bar x}}} \Big)
            & \mbox{if $0 < \bar x < {rim}_{\bar x}$,} \\
 \\
 0          & \mbox{otherwise,}
                   \end{array}
        \right.
$$
%\end{equation}
In the four corners of the domain of simulation, there is a smooth transition
between the pure $\bar x$ and pure $\bar y$ dependencies of the damping
coefficient $KSZ$ that is obtained in the following manner:
%\begin{equation}
$$
KSZ(\bar x, \bar y) = \left\{ \begin{array}{ll}
K_{SZ}^{max}& \mbox{if $1 \le d_{rim}^2$,} \\
 \\
K_{SZ}^{max} \sin^2 \Big({\pi \over 2} d_{rim} \Big)
            & \mbox{if $d_{rim}^2 \le 1$,} \\
 \\
 0          & \mbox{if $\bar x \ge {rim}_{\bar x}$ and $\bar y \ge {rim}_{\bar y}$} \\
                         \end{array}
        \right.
$$
%\end{equation}
where
$d_{rim}= \sqrt{\Big( \displaystyle{{\bar x-{rim}_{\bar x} \over {rim}_{\bar x}}\Big)^2 +
                \Big( {\bar y-{rim}_{\bar y} \over {rim}_{\bar y}}\Big)^2 }}$.
The maximum value of the relaxation coefficient $K_{SZ}^{max}$ and the rim zone
depths ${rim}_{\bar x}$ and ${rim}_{\bar y}$ are prescribed externally by the
user.

\section{References}
\decrefname
Z\"angl, G., 2002:  An improved method for computing horizontal diffusion
in a sigma-coordinate model and its application to simulations over
mountainous topography. {\it Mon. Wea. Rev.}, {\bf 130}, 1423-1432 