%%%%%%%%%%%%%%%%%%%%%%%%%%%%%%%%%%%%%%%%%%%%%%%%%%%%%%%%%%%%%%%%%%%%%%%%%%%%%%
% CONTRIBUTION TO THE MESONH BOOK1: "ADVECTION SCHEMES FOR SCALAR VARIABLES"
% Author : J. Vila-Guerau de Arellano and J. P. Lafore
% Original : Juin, 1995
% Update   : Janvier, 1998
%%%%%%%%%%%%%%%%%%%%%%%%%%%%%%%%%%%%%%%%%%%%%%%%%%%%%%%%%%%%%%%%%%%%%%%%%%%%%%
%
%\begin{document}
%\setlength{\baselineskip}{15pt}
%\begin{center}
%\large
%%{\bf PROJET MESO-NH}\\[0.6cm]
%%\normalsize
%{\bf ADVECTION SCHEMES}\\[.3cm]
%%{\bf FOR SCALAR VARIABLES}\\[0.3cm]
%J. Vil\`a-Guerau de Arellano and J. P. Lafore \\[0.3cm]
%M\'et\'eo-France \\[0.3cm]
%{\today}\\[1.7cm]
%\end{center}
%\normalsize

\chapter{Advection Schemes}
\minitoc


\section{Introduction}

This chapter describes the advection schemes available in Meso-NH.
 They are of two kinds, those for momentum and those for scalar variables. 
The two available advection schemes for momentum are flux-form conservative 
schemes centered in time. The three available advection schemes for scalar 
variables are positive definite schemes.  The first one, namely the 
flux-corrected transport scheme (FCT), is a centered scheme (spatially 
and temporally) in which the advective fluxes are corrected (if necessary)
by a limiter factor to insure positiveness.
The second one, namely the Multidimensional Positive Definite Advection 
Transport  Algorithm (MPDATA), is based on the upstream scheme. 
An iterative method  based on the anti diffusive velocities is applied 
to correct the excessive  numerical diffusion of such scheme. 
FCT and MPDATA are second-order accurate on space and time.
The third one, the so-called piecewise parabolic method 
(PPM), is a finite-volume method.

The FCT, MPDATA, and PPM schemes are recommended to be used to simulate
the advective transport of quantities such as microphysical variables
and atmospheric chemistry variables. Both schemes can be applied to describe
the advection of the following quantities: temperature, water substances,
turbulent kinetic energy, and scalars (chemical species). 
In Meso-NH the three schemes can be simultaneously apply to
different variables. However, in order to gain consistency, the above 
mentioned variables have been classified in two groups: meteorological 
variables (temperature, water substances, turbulent kinetic energy) and
scalar variables.

Two 2D-horizontal tests have been carried out to evaluate the FCT and MPDATA
schemes. A scalar with a cone distribution was advected in a rotating 
constant velocity field (Test 1) and in a deformational velocity field 
(Test 2). The evaluation of the tests includes the
performance of the advection schemes as well as the computational costs.

\section{Advection schemes for momentum}

Two flux-form conservative schemes are available in the model. Both of them are centered in time but they differ by their truncature order, 2$^{\rm nd}$ for the CEN2ND scheme and 4$^{\rm th}$ for the CEN4TH scheme.

The schemes integrate the non-linear advection terms of the momentum equation:

\begin{eqnarray}
\label{condm}
\dfrac{\partial}{\partial t}(\tilde{\rho}u) \, &=&
 \, - \, \dfrac{\partial }{\partial \overline{x}} (\,\tilde{\rho} U^{c} \; u)
 \, - \, \dfrac{\partial }{\partial \overline{y}} (\,\tilde{\rho} V^{c} \; u)
 \, - \, \dfrac{\partial }{\partial \overline{z}} (\,\tilde{\rho} W^{c} \; u),\\
\dfrac{\partial}{\partial t}(\tilde{\rho}v) \, &=&
 \, - \, \dfrac{\partial }{\partial \overline{x}} (\,\tilde{\rho} U^{c} \; v)
 \, - \, \dfrac{\partial }{\partial \overline{y}} (\,\tilde{\rho} V^{c} \; v)
 \, - \, \dfrac{\partial }{\partial \overline{z}} (\,\tilde{\rho} W^{c} \; v),\\
\dfrac{\partial}{\partial t}(\tilde{\rho}w) \, &=&
 \, - \, \dfrac{\partial }{\partial \overline{x}} (\,\tilde{\rho} U^{c} \; w)
 \, - \, \dfrac{\partial }{\partial \overline{y}} (\,\tilde{\rho} V^{c} \; w)
 \, - \, \dfrac{\partial }{\partial \overline{z}} (\,\tilde{\rho} W^{c} \; w),
\end{eqnarray}

In (\ref{condm}), all the fluxes are evaluated at time level "t". The different estimates of $u$, $v$ and $w$ in the rhs make the difference between the CEN2ND and CEN4TH schemes. In CEN2ND, the discretization of decentered $u$, $v$ and $w$ components results from a simple average along the direction of differentiation (see Chapter 4). This leads to a second order estimate of the fluxes. In CEN4TH, the decentering of $u$, $v$ and $w$ is given by a fourth-order scheme. For instance, the first term in (\ref{condm}) writes

\begin{eqnarray}
\label{diffm}
- \frac{\displaystyle\partial }{\displaystyle\partial \overline{x}} (\tilde{\rho} U^{c} \;   u )
 & \Longrightarrow & - \delta_{x} \left[
\overline{  \left(
\overline{\tilde{\rho}}^{x} U^{c} \;  \right)}^{x}
\overline{  u  }^{x}
 \right] \qquad \qquad {\rm (CEN2ND \; scheme)}, \\
- \frac{\displaystyle\partial }{\displaystyle\partial \overline{x}} (\tilde{\rho} U^{c} \;   u )
 & \Longrightarrow & - \delta_{x} \left[
\overline{  \left(
\overline{\tilde{\rho}}^{x} U^{c} \;  \right)}^{x}
\overline{\overline{  u  }}^{x} 
 \right] \qquad \qquad {\rm (CEN4TH \; scheme)},
\end{eqnarray}

with the definition of the operators:
$\overline{\alpha}^{x}(i,j,k)=(\alpha(i+1,j,k)+\alpha(i,j,k))/2$
and,
$\overline{\overline{\alpha}}^{x}(i,j,k)=7(\alpha(i+1,j,k)+\alpha(i,j,k))/12-(\alpha(i+2,j,k)+\alpha(i-1,j,k))/12$. Of course the CEN4TH scheme reverts to the CEN2ND one at the edge of the computational domain.

The CEN4TH scheme is restricted to $\mathcal{C}\leq0.73$ ($\mathcal{C}$ is the Courant number of the flow) which is less than $\mathcal{C}\leq1.0$ obtained in the CEN2ND case, in each direction. However, the CEN4TH scheme has better damping properties and less phase or wave speed errors than CEN2ND. So it should be preferred.

\section{Flux-Corrected Transport (FCT)}

\subsection{Description}

The advective part of the governing equation of a scalar quantity $\psi$
(temperature, water substances, TKE, dissipation of TKE and scalar (chemical species))
can be written

\begin{equation}
\label{cond1}
\dfrac{\partial}{\partial t}(\tilde{\rho}\psi) \, =
 \, - \, \dfrac{\partial }{\partial \overline{x}} (F_1)
 \, - \, \dfrac{\partial }{\partial \overline{y}} (F_2)
 \, - \, \dfrac{\partial }{\partial \overline{z}} (F_3),
\end{equation}

\noindent where

\begin{eqnarray}
\label{cond2}
F_1\,=\,\tilde{\rho} U^{c} \;  \psi \\
F_2\,=\,\tilde{\rho} V^{c} \;  \psi \\
F_3\,=\,\tilde{\rho} W^{c} \;  \psi,
\end{eqnarray}

\noindent and $U^{c},V^{c}$ and $W^{c}$ are the respective velocities.

At Meso-NH, the advection of a quantity is solved with a second order centered advection
scheme. The solution of equation (\ref{cond1}) is

\begin{equation}
\dfrac{\tilde{\rho} \psi^{t+\Delta t}} {2 \Delta t} \, = \,
\dfrac{\tilde{\rho} \psi^{t-\Delta t}} {2 \Delta t} \,
 \, - \, \sum_{I=1}^{3} \,  \dfrac{\partial }{\partial \overline{x}_I}
F_{I}^{t}
\end{equation}

\noindent where $2\Delta t$ is the leapfrog time step and $F_I$ are the three fluxes defined
in (\ref{cond2}) and calculated at the instant $t$.

The numerical solution of the advective part might produce oscillations and
negative values. This occurs if
the flux calculated is overestimated with respect the analytical value, i.e. the
monotonicity condition is not fulfilled at the lower limit. In order to obtain a
positive definite advection scheme, a limiter factor for the advective flux is
defined below. The development of the expression of the limiter factor follows
the approach suggested by Smolarkiwecickz and Grabowski (1990). In the previous
study, a non-oscillatory option was developed to correct the oscillations produced by
the MPDATA scheme.

Assuming that the Courant number (CFL) is
smaller than one, the scalar variable computed at $t+\Delta t$ must satisfy the
following condition

\begin{equation}
\label{cond3}
\psi_{MIN}^{t-\Delta t}\,\leq \,
\psi^{t+\Delta t} \, = \,
\dfrac{-2 \Delta t}{\tilde{\rho}\,\Delta \overline{x}_{I}}\, (F^{OUT}-F^{IN}) \,+\,
\psi^{t-\Delta t},
\end{equation}

\noindent where $\psi_{MIN}$ is the local minimum of the quantity and $F^{OUT}$ and $F^{IN}$ are
the total
outgoing flux and incoming flux, respectively. A sufficient condition for (\ref{cond3}) is

\begin{equation}
\label{cond4}
\psi^{t-\Delta t}  \, - \,
\psi_{MIN}^{t-\Delta t}\,\geq \,
\dfrac{2 \Delta t}{\tilde{\rho}\,\Delta \overline{x}_{I}}\, F^{OUT}
\end{equation}

The previous condition (\ref{cond4}) can be written as a flux limiter ratio ($\beta^{OUT}$)

\begin{equation}
\beta^{OUT}=
 \dfrac{ \psi^{t-\Delta t}- \psi_{MIN}^{t-\Delta t} }
{ \dfrac{2 \Delta t}{\tilde{\rho}\,\Delta \overline{x}_{I}} {F}^{OUT}}.
\end{equation}

If $\beta^{OUT} \geq 1$, the flux is calculated correctly.
If $\beta^{OUT}<1$, the outgoing flux is overestimated and therefore the flux limiter
ratio must be applied.
The corrected flux are

\begin{equation}
\label{cond5}
{F}_{I} = min(1,\beta^{OUT}_{I-1})[\,F_{I}]^{+}\,+\,
min(1,\beta^{OUT}_{I})[\,F_{I}]^{-}
\end{equation}

\noindent where $[\,.\,]^{+}\,\equiv\, max(0,.)$ and $[\,.\,]^{-}\,\equiv\,min(0,.)$.

\subsection{Discretization}

For the centered advection scheme (ADVECSCALAR routine in Meso-NH) the advective
flux are discretized according to

\begin{eqnarray}
F_1\,=\,
\overline{\tilde{\rho}}^{x}U^{c} \overline{\psi}^{x} \\
F_2\,=\,
\overline{\tilde{\rho}}^{y}V^{c} \overline{\psi}^{y} \\
F_3\,=\,
\overline{\tilde{\rho}}^{z}W^{c} \overline{\psi}^{z}.
\end{eqnarray}

The outgoing flux for the grid cell ($F^{OUT}$) (located at mass point) is

\begin{eqnarray}
F^{OUT}_{i,j,k}\,=\,[F^1_{i+1,j,k}]^+\,-\,[F^1_{i,j,k}]^-
\nonumber \\
\,+\,[F^2_{i,j+1,k}]^+\,-\,[F^2_{i,j,k}]^-
\nonumber \\
\,+\,[F^3_{i,j,k+1}]^+\,-\,[F^3_{i,j,k}]^-.
\end{eqnarray}

Only on the above expression and in order to clarify the notation, the fluxes
components ($F_1,F_2,F_3$) have been written as ($F^1,F^2,F^3$).

An absolute minimum ($\psi_{MIN}$) must be prescribed to calculate the limiter factor.
This minimum can be, for instance, a
background value or as a default the value equal to zero.
Alternatively, one can prescribed a local minimum ($\psi_{MIN}^{t-\Delta {t}}$)
which is defined

\begin{equation}
\psi_{MIN}^{t-\Delta {t}}=
min[\psi^{t-\Delta {t}}_{i,j,k},
\psi^{t-\Delta {t}}_{i-1,j,k}, \psi^{t-\Delta {t}}_{i+1,j,k},
\psi^{t-\Delta {t}}_{i,j-1,k}, \psi^{t-\Delta {t}}_{i,j+1,k},
\psi^{t-\Delta {t}}_{i,j,k-1}, \psi^{t-\Delta {t}}_{i,j,k+1}].
\end{equation}

If one knows $F^{OUT}$ and $\psi^{MIN}$, the $\beta^{OUT}-$ratio
can be calculated

\begin{equation}
\beta^{OUT}=
 \dfrac{ \psi^{t-\Delta {t}}- \psi_{MIN}^{t-\Delta {t}}   }
{\dfrac{ {2 \Delta t} \, {F}^{OUT}}{\tilde{\rho}} +\varepsilon },
\end{equation}

\noindent where $\varepsilon$ is a small value, for example $10^{-15}$. Finally,
the discretization of the three components of the corrected flux yields

\begin{eqnarray}
{F_1} = min(1,\beta_{i-1,j,k}^{OUT})\,[\,F_1]^{+}\,+\,
min(1,\beta^{OUT}_{i,j,k})\,[\,F_1]^{-}\\
{F_2} = min(1,\beta_{i,j-1,k}^{OUT})\,[\,F_2]^{+}\,+\,
min(1,\beta^{OUT}_{i,j,k})\,[\,F_2]^{-}\\
{F_3} = min(1,\beta^{OUT}_{i,j,k-1})\,[\,F_3]^{+}\,+\,
min(1,\beta^{OUT}_{i,j,k})\,[\,F_3]^{-}.
\end{eqnarray}

Once the advective fluxes are calculated and corrected, the advection of the
scalar can be calculated (r.h.s) of equation (\ref{cond1}). Its discretize form
form is

\begin{equation}
-\,\delta_{x}F_1\,-\,\delta_{y}F_2\,-\,\delta_{z}F_3.
\end{equation}

\subsection{Boundary conditions}

The following boundary conditions are prescribed for the limiter factor $\beta^{OUT}$

- Cyclic boundary conditions

\begin{equation}
\beta^{OUT}_{b-1}=\beta^{OUT}_{e}
\end{equation}

- Open boundary conditions

\begin{equation}
\beta^{OUT}_{b-1}=1.
\end{equation}

\section{Multidimensional Positive Definite Advection Transport Algorithm (MPDATA)}

The MPDATA can be summarized as follows.
First, the advection of a quantity is solved by means of an upstream scheme
(Rood 1987). This is the first iteration in the MPDATA scheme.
Second, the excessive numerical diffusion produced by such scheme is
corrected reapplying the scheme, but now one substitutes the velocity field by
introducing an
anti-diffusive velocity field. The anti-diffusive velocity is derived analytically
based on the truncation error analysis of the upstream scheme.
A non-oscillatory option can be applied to the MPDATA scheme to assure monotonicity.
Such procedure may be repeated an optional number
of times. For two iterations, the  MPDATA is a second-order-accurate in time
and space for any
advective velocity field.
The properties
of this scheme are: stability, consistency and conservation of positive
definitiveness.

For more detailed information, the user is addressed to the following articles:
Smolarkiewicz (1983) (an introduction to the scheme in 1-D form), Smolarkiewicz
(1984) (extension of the scheme to a fully multidimensional form), Smolarkiewicz
and Clark (1986) (application to the scheme to a time-dependent velocity field
and generalized form of the scheme
to the anelastic continuity equation) and Smolarkiewicz and
Grabowski (1990) (non-oscillatory option in order to preserve monotonicity).

The formulation and notation presented in this manual closely follow (when
it is possible) the one
presented in the last two papers above mentioned. However, the MPDATA scheme
has been adapted to the following Meso-NH requirements: leap-frog time stepping and
code optimization.

\subsection{Description}

The advection equation of a quantity $\psi$ written on the computational
grid, takes the following form:

\begin{equation}
\dfrac{\partial}{\partial t}(\tilde{\rho}\psi) \, =
 \, - \, \dfrac{\partial }{\partial \overline{x}} (\tilde{\rho} U^{c} \;  \psi)
 \, - \, \dfrac{\partial }{\partial \overline{y}} (\tilde{\rho} V^{c} \;  \psi)
 \, - \, \dfrac{\partial }{\partial \overline{z}} (\tilde{\rho} W^{c} \;  \psi)
\end{equation}

\noindent For compactness it may be written:

\begin{equation}
\dfrac{\partial}{\partial t}(\tilde{\rho}\psi) \, =
 \, - \, \sum_{I=1}^{3} \, \dfrac{\partial }{\partial \overline{x}_I}
 (u_I \; \psi) \, =
 \, - \, \sum_{I=1}^{3} \, \dfrac{\partial F_I }{\partial \overline{x}_I}
\end{equation}

\noindent
 where $u_I = \tilde{\rho} U_{I}^{c}$ is the Ith component of the non-divergent
contravariant velocity and $F_I = u_I \psi = \tilde{\rho} U_{I}^{c} \psi$
is a flux of $\psi$ in that direction.

 The basic MPDATA
iteratively solves the advection equation (at the $l$th iteration) in the
following way

\begin{equation}
\dfrac{\tilde{\rho} \psi^{(*)^{l}}} {2 \Delta t} \, = \,
\dfrac{\tilde{\rho} \psi^{(*)^{l-1}}} {2 \Delta t} \,
 \, - \, \sum_{I=1}^{3} \,  \dfrac{\partial }{\partial \overline{x}_I}
F_I (\psi^{(*)^{l-1}} , \tilde{u}_I^{l} )
\end{equation}

\noindent
where $2\Delta t$ is the leapfrog time step and $l=1,...,L$.
 L is the total number of iterations.
$F_I$ is the donor-cell advective flux evaluated by means of the
upstream scheme.
% Note that when L=1 the algorithm
%results in the first-order accurate upstream scheme.

\begin{itemize}

\item {\em The initial and final values} are given by

\begin{equation}
\psi^{(*)^{0}}\equiv \psi^{t-\Delta t} \, \, \, and \, \, \, \, \, \, \, \,
\psi^{t+\Delta t}\equiv \psi^{(*)^{L}}
\end{equation}

\item {\em For the first iteration $(l=1)$} the scalar field is advected using
the velocity at the central time $t$, resulting in the the first-order accurate
upstream scheme.

\begin{equation}
\tilde{u}_I^{1}\equiv {u}_{I}^{t}
\end{equation}

\item {\em The next iterations $(l>1)$} increase the accuracy and
reduce the strong diffusive character of the upstream scheme. This
is done by introducing an
{\em anti-diffusive velocities} $\tilde{u}_I^{l}$

If $\psi > 0$
\begin{eqnarray}
\tilde{u}_{I}^{l}  =
0.5 \left[ |{\tilde{u}_I^{l-1}}| \Delta \overline{x}_I
       -{2 \Delta t}\dfrac{(\tilde{u}_I^{l-1})^{2}} {\tilde{\rho}}  \right]
\dfrac{1}{\psi^{(*)^{l-1}}}
\dfrac{\partial \psi^{(*)^{l-1}}}{\partial \overline{x}_I} \,
\nonumber \\
\nonumber \\
- \sum_{J=1,J\neq I}^{3}  0.5({2 \Delta t})
\dfrac{ \tilde{u}_I^{l-1}\tilde{u}_J^{l-1} } {\tilde{\rho} \psi^{(*)^{l-1}}}
\dfrac{\partial \psi^{(*)^{l-1}}}{\partial \overline{x}_J}
\end{eqnarray}
If $\psi = 0$
\begin{equation}
\tilde{u}_I^{l}  = 0.
\end{equation}

 Note that the anti-diffusive velocities
$\tilde{u}_I^{l}$ ($l>1$) are dependent on the $\psi$ variable, contrary to
the first iteration ($l=1$) for which all variables are advected by
the same wind field ${u}_{I}$. This feature has
important impact on the code, as it suggests to perform the iterative cycle
of correction ($l=2,...,$ $ L$) variable by variable, in order to save memory.

\end{itemize}

\subsection{Discretization}

\begin{itemize}

\item {\em Fluxes } are estimated by the following upstream algorithm

\begin{eqnarray}
\label{flux1}
F_1(\psi^{(*)^{l-1}}, \,\tilde{u}^{l}) = [\,\tilde{u}^{l}]^{+}
\psi^{(*)^{l-1}}_{i-1,j,k}\,
\, + \ [\,\tilde{u}^{l}]^{-}
\psi^{(*)^{l-1}}_{i,j,k}\\
\label{flux2}
F_2(\psi^{(*)^{l-1}}, \,\tilde{v}^{l}) = [\,\tilde{v}^{l}]^{+}
\psi^{(*)^{l-1}}_{i,j-1,k}\,
\, + \ [\,\tilde{v}^{l}]^{-}
\psi^{(*)^{l-1}}_{i,j,k}\\
\label{flux3}
F_3(\psi^{(*)^{l-1}}, \tilde{w}^{l}) = [\tilde{w}^{l}]^{+}
\psi^{(*)^{l-1}}_{i,j,k-1}\,
\, + \ [\tilde{w}^{l}]^{-}
\psi^{(*)^{l-1}}_{i,j,k}
\end{eqnarray}

\noindent with triplets $(i,j,k)$ corresponding to the variables position
on the computational grid.

\item {\em The "anti-diffusive" velocities $(l>1)$}. In order to insure
that $\dfrac{1}{\psi} \dfrac{\partial \psi}{\partial \overline{x}_J}$
converges when $\psi$ tends to zero, the anti-diffusive velocities are
discretized as

\begin{eqnarray}
\tilde{u}^{l}  =  0.5(2\Delta t) \left[{
\left[ \overline{\tilde{\rho}}^{x} \dfrac{sign(\,\tilde{u}^{l-1})}{2\Delta t}
                                      - \,\tilde{u}^{l-1}  \right] \, A \,
\, - \, {\,\tilde{u}^{l-1}}\,
\left[\overline{ \overline{B}^y \,\,+ \, \overline{C}^z }^{x} \right]}\right] \\
\tilde{v}^{l}  =  0.5(2\Delta t) \left[{
\left[ \overline{\tilde{\rho}}^{y} \dfrac{sign(\,\tilde{v}^{l-1})}{2\Delta t}
                                      - \,\tilde{v}^{l-1}  \right] \, B \,
\, - \, {\,\tilde{v}^{l-1}}\,
\left[\overline{ \overline{A}^x \,\,+ \, \overline{C}^z }^{y} \right]}\right] \\
\tilde{w}^{l}  =  0.5(2\Delta t) \left[{
\left[ \overline{\tilde{\rho}}^{z} \dfrac{sign(  \tilde{w}^{l-1})}{2\Delta t}
                                      -   \tilde{w}^{l-1}  \right] \, C \,
\, - \, {  \tilde{w}^{l-1}}\,
\left[\overline{ \overline{A}^x \,\,+ \, \overline{B}^y }^{z} \right]}\right]
\end{eqnarray}


\noindent where

\begin{eqnarray}
\label{equA}
A =
 \dfrac{ \, \tilde{u}^{l-1} \delta_{x} \psi^{(*)^{l-1}}   }
{ \overline{\tilde{\rho} \psi^{(*)^{l-1}}}^x +\varepsilon }   \\
\label{equB}
B =
 \dfrac{ \, \tilde{v}^{l-1} \delta_{y} \psi^{(*)^{l-1}}   }
{ \overline{\tilde{\rho} \psi^{(*)^{l-1}}}^y +\varepsilon }   \\
\label{equC}
C =
 \dfrac{    \tilde{w}^{l-1} \delta_{z} \psi^{(*)^{l-1}}   }
{ \overline{\tilde{\rho} \psi^{(*)^{l-1}}}^z +\varepsilon }
\end{eqnarray}

\noindent where $\varepsilon$ is a small value, for example $10^{-15}$, to
ensure $\tilde{u}^{l}=\tilde{v}^{l}=\tilde{w}^{l}=0$ when
$\delta_{x} \psi^{(*)^{l-1}}=\delta_{y} \psi^{(*)^{l-1}}=
\delta_{z} \psi^{(*)^{l-1}}=0$ or $\overline{\tilde{\rho} \psi^{(*)^{l-1}}}^x=
\overline{\tilde{\rho} \psi^{(*)^{l-1}}}^y=
\overline{\tilde{\rho} \psi^{(*)^{l-1}}}^z=0$.

It is important here to note that the discretization proposed for Meso-NH
differs from the discretization suggested by Smolarkiewicz and Clark (1986). As a
result, we have not observed the formation of oscillation on the tests carried
out with the MPDATA scheme. Consequently, the non-oscillatory option, has not been
included in the MPDATA scheme.

\item {\em The $CFL<1$ condition } limits the values of the anti-diffusive velocities
in the following way

\begin{equation}
\mid\tilde{u}^{l}\mid\leq \dfrac{\tilde{\rho}}{2 \Delta t} \, \, \, , \, \, \, \, \, \, \, \,
\mid\tilde{v}^{l}\mid\leq \dfrac{\tilde{\rho}}{2 \Delta t} \, \, \, and \, \, \, \, \, \, \,
\mid\tilde{w}^{l}\mid\leq \dfrac{\tilde{\rho}}{2 \Delta t}
\end{equation}

\end{itemize}

\subsection{Boundary conditions}

\begin{itemize}
\item{\em {First iteration ($l=1$)}}

 The scalar values $\psi$ are taken at the $t-\Delta t$ instant ($n-1$),
whose boundary values are prepared by routine BOUNDARIES depending on the
boundary condition type:

- Cyclic if prescribed.

- At lateral boundaries for the "wall" type and at upper and lower boundaries
\begin{equation}
  \dfrac{\partial \psi }{\partial n } = 0 \;\;\;\;\;\; either \;\;\;\;
\psi^{n-1}_{b+1} = \psi^{n-1}_{b}
\end{equation}

- And for open lateral b.c.
\begin{equation}
  \dfrac{\partial \psi - \psi_{LS} }{\partial n } = 0 \;\;\;\;\;\; either \;\;\;\;
\psi^{n-1}_{b+1} = \psi^{n-1}_{b-1}
 + {\psi_{LS}}^{n-1}_{b+1} -{\psi_{LS}}^{n-1}_{b-1}
\end{equation}

\noindent These boundary conditions tested for the numerical diffusion, are also
well suitable for the upstream scheme. Vertically the contravariant velocities
being zero, the b.c. has no impact.

\item{\em {Next iterations ($l>1$)}}

 There is no reason to apply the anti-diffusive procedure at boundaries, as it
is already the case for the 2nd-order advection scheme. Indeed a diffusive
scheme allows to reduce spurious reflections that may occur at boundaries.
We therefore impose at all boundaries a zero anti-diffusive velocity:

\begin{equation}
\tilde{u}^{l}_{{I }_b} = 0
\end{equation}

\noindent except for cyclic boundaries. In that later case cyclic boundaries
must applies to both $\tilde{u}_I^{l}$ and $\psi^{(*)^{l-1}}$.

\item{\em {Fluxes determination}}

 The fluxes expressions (Eq. \ref{flux1}, \ref{flux2} and \ref{flux3}) are well defined except at the first
point, but which does not correspond to a physical point. We can fill up this
point by the dummy value -999.

\end{itemize}

\section{Evaluation of FCT and MPDATA advection schemes}

Two 2D-horizontal tests were carried out to validate FCT and MPDATA. Test 1 is the
advection of a cone in a rotating field with constant velocity. A full description
of the test can be found in Smolarkiewicz (1984). The initial maximum
concentration $(C_{max})$ was 4. The solution after six full rotations are summarized in Table 1.
The computing time is calculated relative to the computer time used by the centered
second-order
scheme. Note that the leapfrog scheme produces negative values.


\begin{table}[htpb]
  \centering
  \caption{Results of Test 1}
  \begin{tabular}{p{3cm}p{3cm}p{3cm}p{3cm}p{3cm}}
    \hline
    Advection scheme  &  $C_{max}$  &  $C_{min}$  & $C_{max}$(Smol84)  &  Computing time  \\
    \hline
    \hline
    Centered      &   2.65  &  -0.32 &  -    & 1.      \\
    FCT          &   2.64  &   0.   &  -    & 3.4     \\
    MPDATA(L=2)  &   2.015 &   0.   & 2.16  & 7.       \\
    MPDATA(L=3)  &   2.85  &   0.   & 3.17  &11.8      \\
    \hline
  \end{tabular}
\end{table}

Test 2 is the advection of the same cone but now in a deformational flow field. The
description and the analytical solution of the test were done by Staniforth et al. (1987).
FCT and MPDATA performed very well in spite of it being an extremely stressful test.

From the evaluation and inter-comparison of the two tests
the following conclusions can be drawn:

\begin{itemize}

\item{The results obtained with MPDATA are in close agreement with the ones obtained
by Smolarkiewicz (1984). However, and due to the different discretization, MPDATA
at Meso-NH is a little bit more diffusive. On the other hand, we have not found oscillations
on the scalar field distribution.}

\item{FCT is less diffusive than MPDATA. Carrying out more iterations with MPDATA the
numerical diffusion can be corrected but at expense of an increase of the computer time.}

\item{FCT is less computer expensive than MPDATA}

\item{MPDATA conserves better the original distribution of the scalar, i.e. more symmetric}

\end{itemize}



\section{The PPM advection scheme}

\subsection{Introduction}

  The PPM advection scheme is based on the \textbf{P}iecewise
  \textbf{P}arabolic \textbf{M}ethod, a numerical technique developed in
  astrophysics for modeling fluid flows (Colella et al.\ 1984, Carpenter
  et al.\ 1990). The PPM differs substantially from conventional
  grid-point methods like the second and fourth order centered advection
  schemes already implemented in the MesoNH. It is a finite-volume
  method in which some assumption is made about the structure of the
  approximate solution between the grid points.  In the PPM case,
  piecewise continuous parabolas are fitted in each grid-cell. This
  design enables the scheme to handle sharp gradients and
  discontinuities very accurately. Although the constructed piecewise
  continuous functions would allow for the explicit calculation of
  derivatives, in practice the advective forcing is computed from the
  fluxes at the grid-cell edges. Various flux limiters can be used to
  ensure the scheme is monotonic. Monotonic schemes do not amplify
  extrema in the initial values and are, thus, also positive
  definite. Application of the PPM schemes to 3D problems requires
  density-corrected directional splitting and essentially
  forward-in-time (FIT) time-marching, so variables advected by the PPM
  schemes follow a modified path through the main Meso-NH loop.

  Three different versions of the PPM advection scheme have been
  implemented in MesoNH: the unrestricted PPM\_00, the monotonic
  version, PPM\_01, based on the original scheme (Colella at al.\ 1984)
  with monotonicity constraints modified by Lin and Rood (1996). The
  third version, PPM\_02, is another monotonic scheme with a flux
  limiter developed by Skamarock (2006). The PPM\_02 permits extension
  to stable semi-Lagrangian integrations using Courant numbers larger
  than one. The PPM schemes are intended for advection of meteorological
  variables (e.g.\ temperature, water species, TKE) and passive
  scalars. All three versions have excellent mass-conservation properties
  and were found to be an order of magnitude more accurate than the
  existing FCT or centered 4\ts{th} order (CEN4TH) schemes in idealized
  2D tests. However, the numerical diffusion associated with the
  monotonicity preserving methods in PPM\_01 and PPM\_02 can damp
  smooth, well resolved extrema. Therefore the use of numerical diffusion
  applied to scalar fields should be prohibited when using the PPM schemes.
  Also, the flux limiters and the
  specific time marching require considerable computational effort
  making the monotonic PPM schemes significantly more expensive than,
  e.g., FCT. Nevertheless, the PPM schemes in combination with the
  CEN4TH advection of momentum overall produce significantly more
  accurate solutions and the advection is still relatively small
  fraction of the total computational time.

  \subsection{Description}

  The advective part of the scalar conservation equation can be written
  as:  
  \begin{equation}
    \label{advection}
    \frac{\pl}{\pl t}(\rho \phi) =
    - \frac{\pl}{\pl x}(\rho U \phi)
    - \frac{\pl}{\pl y}(\rho V \phi)
    - \frac{\pl}{\pl z}(\rho W \phi)    
  \end{equation}
  where $U$, $V$ and $W$ are Cartesian wind components. To solve this
  equation on a discrete grid using the PPM finite-volume method, an
  approximate solution for the scalar field $\phi$ is constructed
  between the grid points. The grid-point value $\phi_{i}$ of a scalar
  variable at a mass grid point $i$ represents the average of the
  function $\phi(x)$ over the grid-cell, or the interval $x_{i-1/2} \le
  x \le x_{i+1/2}$, where the $x_{i-1/2}$ and $x_{1+1/2}$ are the
  neighboring momentum grid points. In the PPM case, the function
  $\phi(x)$ is a parabola which is constructed uniquely for each
  grid-cell using the cell mean ($\phi_{i}$) and values at the cell
  edges ($\phi_{i-1/2}$ and $\phi_{i+1/2}$).  Once the parabolas are
  known, we can calculate fluxes at the grid-cell edges, and the net
  scalar transport in each cell is equal to the flux divergences.

  A solution (in one dimension) to (\ref{advection}) can be calculated
  from forward-in-time (FIT) discretization:
  \begin{equation}
    \label{1DFIT}
    (\rho \phi)_{i}^{t+\Delta t} = (\rho \phi)_{i}^{t} -
    \mathcal{F}_{x,i}(\phi^{t})    
  \end{equation}
  where the operator $\mathcal{F}_{x,i}$ denotes the discrete flux
  divergence in the grid-cell $i$ and is expressed as:
  \begin{equation}
    \label{FIT}
    \mathcal{F}_{x,i}(\phi^{t}) = \frac{\Delta t}{\Delta x_{i}} \left[
      (\rho U)_{i+1/2} f(\phi^{t})_{i+1/2} - (\rho U)_{i-1/2}
      f(\phi^{t})_{i-1/2} \right] 
  \end{equation}
  where $F_{i\pm1/2}=(\rho U)_{i\pm1/2} f(\phi^{t})_{i\pm1/2}$ are the
  scalar fluxes at the grid-cell edges. These fluxes can be determined
  by integrating over the parabolas in each cell. The parabola in the
  grid-cell $i$ (dropping the superscript $t$) is constructed from the
  cell average ($\phi_{i}$) and left and right edge values ($\phi_{L,i}$
  and $\phi_{R,i}$) in the following way:
  \begin{equation}
    \label{par}
    \phi_{i}(\xi) = \phi_{L,i} + \xi \left[ \Delta(\phi)_{i} +
      \phi_{6,i} \left ( 1- \xi \right ) \right ], \quad \xi = \frac{x -
      x_{i-1/2}}{\Delta x_{i}}
  \end{equation}
  where $0 \le \xi \le 1$ ($x_{i-1/2} \le x \le x_{i+1/2}$) is the
  horizontal coordinate. The parameters of the parabola are:
  \begin{eqnarray}
    \Delta(\phi)_{i} & = & \phi_{R,i} - \phi_{L,i}\\
    \phi_{6,i} & = & 6 \left[\phi_{i} - \frac{1}{2}\left (\phi_{R,i} +
    \phi_{L,i} \right ) \right ].
  \end{eqnarray}
  The edge values are determined using the 4\ts{th} order differencing:
  \begin{eqnarray}
    \label{edgevalues1}
    \phi_{L,i} &=& \phi_{i-1/2} = \frac{1}{2}\left ( \phi_{i} + \phi_{i-1}
    \right ) - \frac{1}{6}\left ( \delta(\phi_{i}) - \delta(\phi_{i-1})
    \right )\\
    \label{edgevalues2}
    \phi_{R,i} &=& \phi_{i+1/2} = \frac{1}{2}\left ( \phi_{i+1} + \phi_{i}
    \right ) - \frac{1}{6}\left ( \delta(\phi_{i+1}) - \delta(\phi_{i})
    \right )
  \end{eqnarray}
  where the operator $\delta(\,)$ is defined as:
  \begin{equation}
    \label{endpar}
    \delta(\phi_{i}) = \frac{1}{2} \left( \phi_{i+1} - \phi_{i-1} \right).
  \end{equation}  
  And finally, the scalar fluxes at the edges of grid-cells
  ($f(\phi)_{i\pm1/2}$) are calculated as the average value of the scalar
  over the advective distance in each grid cell:  
  \begin{equation}
    f(\phi)_{i+1/2} = \frac{1}{\Delta t U_{i+1/2}}
    \int_{x_{i+1/2}-\Delta tU_{i+1/2}}^{x_{i+1/2}} \phi_{i}(x) \dif x,
    \quad \mathrm{for} \quad U_{i+1/2} > 0
  \end{equation}
  Substituting equations (\ref{par}) through (\ref{endpar}) we get:
  \begin{equation}
    \label{flux+}
    f(\phi)_{i+1/2}^{+} = \phi_{R,i} -
     \frac{1}{2} Cr_{i+1/2} \left[\Delta(\phi)_{i} - 
     \left(1- \frac{2}{3} Cr_{i+1/2} \right) \phi_{6,i} \right]
  \end{equation}
  for $Cr_{i+1/2} \geq 0$,
  \begin{equation}
    \label{flux-}
    f(\phi)_{i+1/2}^{-} = \phi_{L,i+1} -
     \frac{1}{2} Cr_{i+1/2} \left[ \Delta(\phi)_{i+1} + 
     \left(1 + \frac{2}{3} Cr_{i+1/2} \right) \phi_{6,i+1} \right]
  \end{equation}
  for $Cr_{i+1/2} < 0$. Here $Cr_{i+1/2} = U_{i+1/2}\Delta t/\Delta x$
  is the Courant number. 

  \subsubsection{The unrestricted scheme, PPM\_00}

  The PPM advective fluxes $f_{i\pm1/2}$ can be expressed in a more
  concise way for more efficient calculation. Combining the equations
  (\ref{par}) through (\ref{endpar}) with the expressions for the fluxes
  (\ref{flux+}) and (\ref{flux-}), we get:
  \begin{eqnarray}
    \label{sflux+}
    f(\phi)_{i+1/2}^{+} &=& \phi_{i+1/2} - Cr_{i+1/2}(\phi_{1+1/2} -
    \phi_{i}) \\ \nonumber
    & & - Cr_{i+1/2}(1-Cr_{i+1/2})(\phi_{i-1/2} - 2\phi_{i} +
    \phi_{i+1/2})    
  \end{eqnarray}
  for $Cr_{i+1/2} = U_{i+1/2}\Delta t/\Delta x \geq 0$,
  \begin{eqnarray}
    \label{sflux-}
    f(\phi)_{i+1/2}^{-} &=& \phi_{i+1/2} + Cr_{i+1/2}(\phi_{1+1/2} -
    \phi_{i+1}) \\ \nonumber
    & & + Cr_{i+1/2}(1+Cr_{i+1/2})(\phi_{i+1/2} - 2\phi_{i+1} +
     \phi_{i+3/2}) 
  \end{eqnarray}
  for $Cr_{i+1/2} = U_{i+1/2}\Delta t/\Delta x < 0$. The values at the
  grid-cell edges (from equations (\ref{edgevalues1}) and
  (\ref{edgevalues2})) can be expressed as:
  \begin{equation}
    \label{sedgevalues}
    \phi_{i+1/2} = \left [7(\phi_{i+1} + \phi_{i}) - (\phi_{i+2} +
     \phi_{i-1}) \right ]/12.
  \end{equation}
    
  \subsubsection{The PPM\_01 monotonic scheme}  

  When sharp gradients or discontinuities are present in the scalar
  field, the piecewise continuous parabolas can have very steep
  gradients and produce non-physical values -- overshoots
  (e.g. concentrations higher than 100 \%) or undershoots (e.g. negative
  concentrations) somewhere inside the grid-cell. In those cases the
  calculated advective forcing (given by cell-edge fluxes (\ref{flux+})
  and (\ref{flux-})) would likely result in non-physical values in the
  scalar field after the advection step. To avoid the introduction of
  new extrema (overshoots or undershoots) in the domain, monotonicity
  constraints can be imposed on the advection scheme. In the PPM case,
  this can be achieved in two ways: modifying the parabolas or limiting
  the cell-edge fluxes.

  In the original PPM scheme (Colella 1984) the monotonicity is ensured
  by modifying the parameters of the parabolas. Lin and Rood (1996) have
  proposed simplified monotonicity constraints that require fewer
  floating point operations. The parabola parameters $\delta(\phi_{i})$,
  $\Delta(\phi)_{i}$, $\phi_{6,i}$, $\phi_{L,i}$ and $\phi_{R,i}$ are
  modified in the following way. The average slope $\delta(\phi_{i})$ in
  the $i^{\mathrm{th}}$ grid-cell is replaced by the modified version
  $\delta_{m}(\phi_{i})$:  
  \begin{eqnarray}
    \delta_{m}(\phi_{i}) &=& \mathrm{sign}(\delta(\phi_{i})) \times \\
    \nonumber
    & & \min \left [ |\delta(\phi_{i}) |, \,
      2(\phi_{i}-\phi^{\mathrm{min}}_{i}), \, 2(\phi^{\mathrm{max}}_{i} -
    \phi_{i}) \right ] \\    \nonumber
    \phi^{\mathrm{min}}_{i}&=& \min (\phi_{i-1}, \phi_{i},
    \phi_{i+1}) \\ \nonumber
    \phi^{\mathrm{max}}_{i} &=& \max (\phi_{i-1}, \phi_{i}, \phi_{i+1})
  \end{eqnarray}
  The first-guess parabola coefficients ($\Delta(\phi)_{i}$,
  $\phi_{6,i}$, $\phi_{L,i}$, $\phi_{R,i}$) are calculated using the
  modified $\delta_{m}(\phi_{i})$. The parameters are then adjusted to
  eliminate overshoots and undershoots through the following algorithm:
  \begin{description}
    \label{ppm01}
    \item \texttt{IF} \hspace{5mm} $\delta_{m}(\phi_{i}) = 0$ \\       
       $\hat{\phi}_{L,i} = \phi_{i}, \qquad \hat{\phi}_{R,i} = \phi_{i}, \qquad
       \hat{\phi}_{6,i} = 0$
    \item \texttt{ELSE IF} \hspace{5mm} $\phi_{6,i} \Delta(\phi)_{i} <
      -(\Delta(\phi)_{i})^{2}$ \\
       $\hat{\phi}_{6,i} = 3(\phi_{L,i} - \phi_{i}), \qquad \hat{\phi}_{R,i} =
       \phi_{L,i} - \hat{\phi}_{6,i}, \qquad \hat{\phi}_{L,i} = \phi_{L,i}$
    \item \texttt{ELSE IF} \hspace{5mm} $\phi_{6,i} \Delta(\phi)_{i} >
      (\Delta(\phi)_{i})^{2}$ \\
       $\hat{\phi}_{6,i} = 3(\phi_{R,i} - \phi_{i}), \qquad \hat{\phi}_{L,i} =
       \phi_{R,i} - \hat{\phi}_{6,i}, \qquad \hat{\phi}_{R,i} =
       \phi_{R,i}$
    \item \texttt{END}
  \end{description}
  The parameters of the monotonized parabolas (labeled with
  $\hat{\,}\,$) are used to calculate the advective fluxes defined in
  equations (\ref{flux+}) and (\ref{flux-}).

  \subsubsection{The PPM\_02 monotonic scheme}  

  An alternative way of ensuring monotonicity is by correcting the
  cell-edge fluxes as in the standard FCT approach. This limiter is
  based on the work of Skamarock (2006) and Blossey and Durran
  (2008). The flux at the cell-edge is composed of a
  monotonicity-preserving upwind flux
  \begin{equation}
    F^{\mathrm{up}}_{i+1/2} = \left\{ \begin{array}{ll}
    (\rho U)_{i+1/2} \phi^{t}_{i}, & \mathrm{for} \quad (\rho U)_{i+1/2}
    \ge 0 \\
    (\rho U)_{i+1/2} \phi^{t}_{i+1}, & \mathrm{otherwise} 
    \end{array}\right.
  \end{equation}
  and a higher order correction such that
  \begin{equation}
    F^{m}_{i+1/2} = F^{\mathrm{up}}_{i+1/2} + r_{i+1/2}F^{\mathrm{cor}}_{i+1/2}
  \end{equation}
  where $0 \le r_{i+1/2} \le 1$ and $F^{\mathrm{cor}}_{i+1/2} =
  F^{\mathrm{ppm}}_{i+1/2} - F^{\mathrm{up}}_{i+1/2}$ is the difference
  between the PPM flux (calculated using the expressions \ref{sflux+} or
  \ref{sflux-}) and the upstream flux. The resulting flux
  $F^{m}_{i+1/2}$ will produce a monotonic solution in the advection step.

  To evaluate $r_{i+1/2}$ we first calculate an approximate
  ``transported and diffused'' solution using the upwind flux
  \begin{equation}
    (\rho \phi)^{td}_{i} = (\rho \phi)^{t}_{i} - \frac{\Delta t}{\Delta
      x_{i}} \left (F^{up}_{i+1/2} - F^{up}_{i-1/2} \right )
  \end{equation}
  The sum of the correction fluxes directed out of the cell $i$ is
  computed as $F^{\mathrm{out}}_{i} = \max(F^{\mathrm{cor}}_{i+1/2},\, 0) -
  \min(F^{\mathrm{cor}}_{i-1/2},\, 0)$, and the sum of the fluxes
  directed into the cell $i+1$ is calculated as $F^{\mathrm{in}}_{i+1} =
  \max(F^{\mathrm{cor}}_{i+1/2},\, 0) - \min(F^{\mathrm{cor}}_{i+3/2},\,
  0)$. Finally, let
  \begin{equation}
    \phi^{\max,\min}_{i} = \max, \min(\phi^{t}_{i-1}, \, \phi^{t}_{i},
    \, \phi^{t}_{i+1}, \, \phi^{td}_{i-1}, \, \phi^{td}_{i},
    \, \phi^{td}_{i+1})
  \end{equation}
  The re-normalization factor for monotonicity preservation is defined
  as:
  \begin{equation}
    r_{i+1/2} = \max \left [0, \, \min \left(1, \, \frac{[(\rho
          \phi)_{i}^{td} - \hat{\rho}_{i}\phi_{i}^{\min} ]
          \Delta x_{i}}{\Delta t F^{\mathrm{out}}_{i} + \varepsilon}, \,
        \frac{[\hat{\rho}_{i+1} \phi^{\max}_{i+1} - (\rho
          \phi)_{i+1}^{td}] \Delta x_{i+1}}{\Delta t
          F^{\mathrm{in}}_{i+1} + \varepsilon} \right ) \right ]
  \end{equation}
  for $F^{\mathrm{cor}}_{i+1/2} \ge 0$,
  \begin{equation}
    r_{i+1/2} = \max \left [0, \, \min \left(1, \, \frac{[\hat{\rho}_{i}
          \phi^{\max}_{i} - (\rho \phi)_{i}^{td}] \Delta x_{i}}{\Delta t
          F^{\mathrm{in}}_{i} + \varepsilon}, \, \frac{[(\rho
          \phi)_{i+1}^{td} - \hat{\rho}_{i+1}\phi_{i+1}^{\min}
          ] \Delta x_{i+1}}{\Delta t F^{\mathrm{out}}_{i+1} +
          \varepsilon}  \right ) \right ]
  \end{equation}
  for $F^{\mathrm{cor}}_{i+1/2} < 0$. Here $\varepsilon$ is a small
  parameter chosen to avoid division by zero, and $\hat{\rho}$ is the
  density as updated in the current advection step. More details about
  the density correction associated with directional splitting in 3D
  applications will be discussed later. In the final step the cell
  averages are updated to time $t+\Delta t$
  \begin{equation}
    (\rho \phi)^{t+\Delta t}_{i} = (\rho \phi)^{td}_{i} -
    \frac{\Delta t}{\Delta x_{i}} \left
      (r_{i+1/2}F^{\mathrm{cor}}_{i+1/2} -
      r_{i-1/2}F^{\mathrm{cor}}_{i-1/2} \right ).
  \end{equation}

  The PPM\_{02} scheme described here is fully monotonic and can be
  modified to use semi-Lagrangian approximation to the flux divergence
  that extends the domain of dependence beyond the adjacent upstream
  grid-cell and allows stable computation for Courant numbers greater
  than unity. More details about the semi-Lagrangian extension can be
  found in e.g. Skamarock (2006) and Blossey and Durran (2008). For
  Eulerian integration with Courant number less than unity, the PPM\_01
  is more computationally efficient.

  \subsubsection{Time marching and extension to three dimensions}

  To extend the 1D scalar advection scheme (\ref{1DFIT}) to multiple
  dimensions, we follow the formulation of Easter (1993) where the mass
  conservation $\pl\rho/\pl t + \nabla \cdot (\rho \bvec{V})=0$ is
  simultaneously integrated with the discrete version of the scalar
  transport (\ref{1DFIT}). This sort of density correction procedure is
  necessary with the PPM schemes because the mass flux ($\rho U$) must
  be saved and updated for scalar advection at each grid-cell
  interface. The density correction is restricted to advection step at
  and the model overall still uses the selected anelastic
  approximation. The mass conservation is discretized within a 3D
  formulation as:
  \begin{equation}
   \label{mass}
    \rho^{t+\Delta t} = \rho^{t} - \mathcal{F}_{x}(I) -
    \mathcal{F}_{y}(I) - \mathcal{F}_{z}(I),
  \end{equation}
  where the vector $I \equiv 1$ and $\mathcal{F}_{x,y,z}$ denotes the
  discrete flux divergence in $x$, $y$ and $x$ direction, which is
  calculated using the non-monotonic PPM\_00 scheme. The full 3D
  algorithm for simultaneous transport of the scalar and mass
  conservation is:
  \begin{eqnarray}
   \label{easter}
   (\rho\phi)^{*} &=& (\rho \phi)^{t} - \mathcal{F}_{x}(\phi^{t}) \\
   \nonumber
   \rho^{*} &=& \rho^{t} - \mathcal{F}_{x}(I) \\ \nonumber
   \phi^{*} &=& (\rho\phi)^{*} / \rho^{*} \\ \nonumber
   (\rho\phi)^{**} &=& (\rho \phi)^{*} - \mathcal{F}_{y}(\phi^{*}) \\
   \nonumber
   \rho^{**} &=& \rho^{*} - \mathcal{F}_{y}(I) \\ \nonumber
   \phi^{**} &=& (\rho\phi)^{**} / \rho^{**} \\ \nonumber
   (\rho\phi)^{t+\Delta t} &=& (\rho \phi)^{**} -
   \mathcal{F}_{z}(\phi^{**}) \\ \nonumber
   \rho^{t+\Delta t} &=& \rho^{**} - \mathcal{F}_{z}(I) \\ \nonumber
   \phi^{t+\Delta t} &=& (\rho\phi)^{t+\Delta t} / \rho^{t+\Delta t}.
  \end{eqnarray}
  Here the flux-divergence operators $\mathcal{F}_{x,y,z}$ can be
  estimated using any of the described PPM schemes (PPM\_00, PPM\_01 or
  PPM\_02), however the density corrections are always calculated using
  the non-monotonic scheme. The PPM\_00 is the most computationally
  efficient and accuracy of the correction is sufficient even for the
  advection of scalars with the monotonic schemes. 

  It can easily be seen that (\ref{easter}) collapses to (\ref{mass}) for
  $\phi = I$ and it is consistent (if $\phi$ is constant at initial time
  it remains constant). To achieve second-order accuracy of the
  time-split scheme (\ref{easter}) a form of Strang splitting (Strang
  1968) is used. It consists of alternating the order of flux divergence
  operators between $x \rightarrow y \rightarrow z$ to $z \rightarrow y
  \rightarrow x$ at each time step.

  The PPM is finite-volume method and the cell-edge fluxes estimated at
  the current model time $t$ can only be used to calculate the scalar
  values at time $t+\Delta t$ and hence it intrinsically works with
  forward-in-time time-marching. Since the advection of a scalar
  variable is integrated from time $t$ to $t+\Delta t$, the wind-speed
  components used for calculating the fluxes (e.g. in equation
  (\ref{FIT})) should be given at time $t+\Delta t/2$ as an average
  $U^{t+\Delta t/2} = 1/2(U^{t}+U^{t+\Delta t})$. Wind components at
  time $t+\Delta t$ are not available when the scalar advection is
  performed so linearly extrapolated values are used: $U^{t+\Delta t/2}
  \approx 1/2(3U^{t} - U^{t-\Delta t})$.

\subsection{Discretization}

Before calling a PPM advection routine (\texttt{PPM\_MET} or
\texttt{PPM\_SCALAR}), extrapolated wind components are calculated:
\begin{eqnarray}
 \tilde{U} &=& 1/2(3 U^{t} - U^{t-\Delta t})\\
 \tilde{V} &=& 1/2(3 V^{t} - V^{t-\Delta t})\\
 \tilde{W} &=& 1/2(3 W^{t} - W^{t-\Delta t})
\end{eqnarray}
The extrapolated wind components are used to calculate the contravariant
winds ($\tilde{U}^{c}, \tilde{V}^{c}, \tilde{W}^{c}$) and respective
Courant numbers: 
\begin{equation}
Cr_{x} = \tilde{U}^{c} \Delta t, \quad Cr_{y} = \tilde{V}^{c} \Delta t,
\quad Cr_{z} = \tilde{W}^{c} \Delta t
\end{equation}
The Courant numbers are then used to calculate the density corrections
for all three steps in 3D directional splitting described in
(\ref{easter}). 

The scalar variable values at the grid-cell edges are calculated as in
equation (\ref{sedgevalues}). The scalar fluxes are determined from the
equations (\ref{sflux+}) and (\ref{sflux-}), depending on the sign of the
local Courant number:
\begin{equation}
f(\phi)_{x,i} = f(\phi)^{+}_{x,i} \left [ \frac{1}{2} +
  \frac{1}{2}\mathrm{sign}(Cr_{x,i}) \right ] + 
f(\phi)^{-}_{x,i} \left [ \frac{1}{2} -
  \frac{1}{2}\mathrm{sign}(Cr_{x,i}) \right ] 
\end{equation}
and similarly for the $y$ and $z$ directions. Here the index $i$
indicates a momentum grid point. The scalar fluxes $f(\phi)^{-}$ and
$f(\phi)^{+}$ are calculated depending on which PPM scheme is used
(PPM\_00, PPM\_01 or PPM\_02). The flux divergences are calculated using
Schuman operators ($\delta_{x}, \delta_{y}, \delta_{z}$):
\begin{eqnarray}
\mathcal{F}_{x} &=& \delta_{x}\left [\ol{\trho}^{x} Cr_{x} f(\phi)_{x}
\right] \\
\mathcal{F}_{y} &=& \delta_{y}\left [\ol{\trho}^{y} Cr_{y} f(\phi)_{y}
\right] \\
\mathcal{F}_{z} &=& \delta_{z}\left [\ol{\trho}^{z} Cr_{z} f(\phi)_{z}
\right]
\end{eqnarray}
At the end of an advection step, following the directional splitting
described in (\ref{easter}), we get the updated scalar variable
$\phi^{t+\Delta t}$. 
As the set of the model is based for the moment on the leap-frog scheme except 
for the PPM advection, the temporal discretization is given by :
\begin{eqnarray}
   (\rho\phi)^{t+\Delta t} &=& (\rho\phi)^{t-\Delta t} - \frac{1}{2} \left [\mathcal{F}_{x,y,z}(\phi^{t-\Delta t}) + \mathcal{F}_{x,y,z}(\phi^{t}) \right ]
  \end{eqnarray}

\noindent where $\mathcal{F}_{x,y,z}$ is the flux-divergence PPM operator and $\mathcal{F}_{x,y,z}(\phi^{t-\Delta t})$ is storaged from the previous time step.

% The advective forcing that is passed back to the
%model is calculated as:
%\begin{equation}
%\phi_{\mathrm{src}} = \trho (\phi^{t+\Delta t} - \phi^{t})/\Delta t
%\end{equation}

\section{References}
\decrefname
Blossey, P. N. and Durran, D. R., 2008: Selective Monotonicity
   Preservation in Scalar Advection, 
   {\it J. Comput. Phys.},  {\bf 227}, 5160–5183. 
\decrefname
Carpenter, Richard L., Droegemeier, Kelvin K.,
   Woodward, Paul R.\ and Hane, Carl E., 1990: Application of the
   Piecewise Parabolic Method (PPM) to Meteorological Modeling,
  {\it Mon. Wea. Rev.},  {\bf 118}, 586--612.
\decrefname
Colella, Phillip and Woodward, Paul R., 1984: The
   Piecewise Parabolic Method (PPM) for Gas-Dynamical Simulations,
   {\it J. Comput. Phys.},  {\bf 54}, 174--201.
\decrefname
 Easter, R. C., 1993: Two Modified Versions of Bott's Positive-Definite
   Numerical Advection Scheme, {\it Mon. Wea. Rev.},  {\bf 121}, 297-304.
\decrefname
Lin, S. and Rood, R. B., 1996: Multidimensional Flux-Form
   Semi-Lagrangian Transport Schemes, 
{\it Mon. Wea. Rev.},  {\bf 124}, 2046-2070.
\decrefname
Rood, R.B., 1987: Numerical advection algorithms and their role in
atmospheric transport and chemistry models.
{\it Rev. Geophys.},  {\bf 25}, 71-100.
\decrefname
Skamarock, W. C., 2006: Positive-Definite and Monotonic Limiters for
Unrestricted-Time-Step Transport Schemes,
{\it Mon. Wea. Rev.}, {\bf 134}, 2241--2250.
\decrefname
Smolarkiewicz, P. K., 1983: A simple positive definite advection scheme with
small implicit diffusion.
{\it Mon. Wea. Rev.},  {\bf 11}, 479-486.
\decrefname
Smolarkiewicz, P. K., 1984: 
A fully multidimensional positive definite advection
transport algorithm with implicit diffusion.
{\it J. Comput. Phys.},  {\bf 54}, 325-362.
\decrefname
Smolarkiewicz, P. K. and Clark T. L., 1986: 
The multidimensional positive definite advection
transport algorithm: further developments and applications.
{\it J. Comput. Phys.},  {\bf 67}, 396-438.
\decrefname
Smolarkiewicz, P. K. and Grabowski W.W., 1990: The multidimensional
positive definite advection transport algorithm: nonoscillatory option.
{\it J. Comput. Phys.},  {\bf 86}, 355-375.
\decrefname
Staniforth A., Cote J. and Pudykiewicz J., 1987: Comments on "Smolarkiewicz's
deformational flow". {\it Mon. Wea. Rev.},  {\bf 115}, 894-900.
\decrefname
Strang, G., 1968: On the Construction and Composition of Difference Schemes, 
{\it SIAM J. Numer. Anal.}, {\bf 5}, 506-517.
%\end{document}
