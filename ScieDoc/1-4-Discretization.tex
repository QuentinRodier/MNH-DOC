%%%%%%%%%%%%%%%%%%%%%%%%%%%%%%%%%%%%%%%%%%%%%%%%%%%%%%%%%%%%%%%%%%%%%%%%%%%%%
% March 2008, J.-P. Chaboureau, editorial corrections
%%%%%%%%%%%%%%%%%%%%%%%%%%%%%%%%%%%%%%%%%%%%%%%%%%%%%%%%%%%%%%%%%%%%%%%%%%%%%

\chapter{Discretization}
\minitoc

%{\em by V. Ducrocq and P. Bougeault}

\section{Stretching}

For many reasons, it may be desirable to stretch the computational grid. On
the vertical, it is customary to have a better resolution in the planetary
boundary layer than in the atmosphere. On the horizontal, the use of stretched
coordinates may allow to study the interaction between larger-scale and
smaller-scale processes, or simply avoid problems at the lateral boundary.
However, if the stretching is not done properly, it may result in a loss of
accuracy.

In order to stretch the coordinates, we introduce a new system
noted $(\overline{x},\overline{y},\overline{z})$, and related to
$\widehat{x},\widehat{y},\widehat{z}$ by

\begin{eqnarray}
d\overline{x} & = & \dfrac{d\widehat{x}}{ {\cal D}_x(\widehat{x})}
\\
& & \\
d \overline{y}& = & \dfrac{d\widehat{y}}{ {\cal D}_y(\widehat{y})}
\\
 & & \\
d\overline{z} & = &\dfrac{ d\widehat{z}}{ {\cal D}_z(\widehat{z})}
\end{eqnarray}

The stretching functions
${\cal D}_x(\widehat{x}), {\cal D}_y(\widehat{y}),
{\cal D}_z(\widehat{z})$ are defined independently
in the three directions.


The new metric coefficients $d_{ij}$ can be computed as
\begin{eqnarray}
d_{xx} &=& \widehat{d}_{xx} \;{\cal D}_x(\widehat{x}) \nonumber  \\
d_{yy} &=& \widehat{d}_{yy} \;{\cal D}_y(\widehat{y}) \nonumber  \\
d_{zz} &=& \widehat{d}_{zz} \;{\cal D}_z(\widehat{z})  \\
d_{zx} &=& \widehat{d}_{zx} \;{\cal D}_x(\widehat{x}) \nonumber \\
d_{zy} &=& \widehat{d}_{zy} \;{\cal D}_y(\widehat{y}) \nonumber   \\
\nonumber \\
J&=& \widehat{J} {\cal D}_x(\widehat{x})\;{\cal D}_y(\widehat{y})\; {\cal D}_z
(\widehat{z})
\end{eqnarray}
and we define
\begin{equation}
\tilde{\rho}  = \rho_{d\,ref} J
\end{equation}

In practice, the value of $\overline{x}$ is 1 on the first grid point,
2 to the second grid point, etc.. and ${\cal D}_x(\widehat{x})$
is equal to the local value of the grid distance on the surface of projection.
Therefore $d_{xx}$ and $d_{zz}$ are precisely the horizontal
and vertical grid distances in the physical space, and $J$ is the volume
of the grid box in the physical space. Thus, $\tilde{\rho}$ is the mass of
dry air within each grid box, for the reference state.


The model equations in this new system have exactly the same
expression as found in the previous chapter, substituting
 ($\widehat{x}$,$\widehat{y}$, $\widehat{z})$  by
 $( \overline{x},\overline{y},\overline{z})$.
This reads

\begin{itemize}
\item \underbar{Continuity equation}
\begin{equation}
\dfrac{\partial }{\partial \overline{x}} (\tilde{\rho} U^{c} \; )
+ \dfrac{\partial }{\partial \overline{y}} (\tilde{\rho} V^{c} \; )
+ \dfrac{\partial }{\partial \overline{z}} (\tilde{\rho} W^{c} \; ) =0.
\end{equation}

\item\underbar{Momentum equation}
\begin{eqnarray}
\dfrac{\partial}{\partial t}(\tilde{\rho}  u ) &= &
 \, - \, \dfrac{\partial }{\partial \overline{x}} (\tilde{\rho} U^{c} \;   u )
 \, - \, \dfrac{\partial }{\partial \overline{y}} (\tilde{\rho} V^{c} \;   u )
 \, - \, \dfrac{\partial }{\partial \overline{z}} (\tilde{\rho} W^{c} \;   u )
\nonumber \\
& & \nonumber \\
&  + & \delta _{2}\tilde{\rho}  u   v  \dfrac{ \cos\gamma}{r \cos\varphi} (\sin\varphi -K)
\,+ \,\delta _{2}\tilde{\rho}  v ^{2}  \dfrac{ \sin\gamma  }{r \cos \varphi} (\sin\varphi -K)
\nonumber \\
& & \nonumber \\
 & - &\delta _{2}\delta _{1} \tilde{\rho}\dfrac{ u  w}{r} \,  - \,
  \tilde{\rho}\dfrac{1}{d_{xx}}
\dfrac{\partial \Phi}{\partial \overline{x}}
\, + \,\tilde{\rho}\dfrac{d_{zx}}{d_{xx}d_{zz}}
\dfrac{\partial \Phi}{\partial \overline{z}} \nonumber \\
& & \nonumber \\
&  - &\delta _{1} \tilde{\rho} f^* \cos\gamma w \, +  \, \tilde{\rho} f  v
\, + \, \tilde{\rho} \vec{F}_{v} \cdot \vec{i}
\\
& & \nonumber \\
\dfrac{\partial}{\partial t}(\tilde{\rho}  v ) &= &
 \, - \, \dfrac{\partial }{\partial \overline{x}} (\tilde{\rho} U^{c} \;   v )
 \, - \, \dfrac{\partial }{\partial \overline{y}} (\tilde{\rho} V^{c} \;   v )
 \, - \, \dfrac{\partial }{\partial \overline{z}} (\tilde{\rho} W^{c} \;   v )
\nonumber \\ & & \nonumber \\
&  - & \delta _{2}\tilde{\rho}  u ^{2}  \dfrac{ \cos\gamma}{r \cos\varphi} (\sin\varphi -K)
 \, - \, \delta _{2}\tilde{\rho}  u   v   \dfrac{\sin\gamma}{r \cos \varphi} (\sin\varphi -K)
\nonumber \\ & & \nonumber \\
 & - & \delta _{2}\delta _{1} \tilde{\rho}\dfrac{ v  w}{r} \,  - \,
 \tilde{\rho}\dfrac{1}{d_{yy}}
\dfrac{\partial \Phi}{\partial \overline{y}}
 \, +  \, \tilde{\rho}\dfrac{d_{zy}}{d_{yy}d_{zz}}
\dfrac{\partial \Phi}{\partial \overline{z}} \nonumber \\
& & \nonumber \\
&  - &\delta _{1}\tilde{\rho} f^* \sin\gamma w  \, - \,  \tilde{\rho} f  u
 \, + \, \tilde{\rho} \vec{F}_{v} \cdot \vec{j}
\\
& & \nonumber \\
\dfrac{\partial}{\partial t}(\tilde{\rho} w) &= &
 \, - \, \dfrac{\partial }{\partial \overline{x}} (\tilde{\rho} U^{c} \;  w)
 \, - \, \dfrac{\partial }{\partial \overline{y}} (\tilde{\rho} V^{c} \;  w )
 \, - \, \dfrac{\partial }{\partial \overline{z}} (\tilde{\rho} W^{c} \;  w)
\nonumber \\ & & \nonumber \\
 & + & \delta _{2}\delta _{1}\tilde{\rho}\dfrac{ u ^{2}+ v ^{2} }{r}
 \,  - \, \tilde{\rho}\dfrac{1}{d_{zz}}
\dfrac{\partial \Phi}{\partial \overline{z}}
 \, + \,\tilde{\rho} g \dfrac{\theta_v ' }{\overline{\theta}_v}\nonumber \\
& & \nonumber \\
&  + &\delta _{1}\tilde{\rho} f^* (\sin\gamma  v  + \cos\gamma  u )
 \, + \, \tilde{\rho} \vec{F}_{v} \cdot \vec{k}
 \end{eqnarray}

\item\underbar{Thermodynamic equation}

\begin{eqnarray}
\dfrac{\partial}{\partial t}(\tilde{\rho}\theta) &=& \, - \,
 \dfrac{\partial }{\partial \overline{x}} (\tilde{\rho} U^{c} \;  \theta)
 \, - \, \dfrac{\partial }{\partial \overline{y}} (\tilde{\rho} V^{c} \;  \theta )
\, - \, \dfrac{\partial }{\partial \overline{z}} (\tilde{\rho} W^{c} \;  \theta)
\nonumber \\ & &
+\tilde{\rho} \left[ {R_d+r_vR_v\over R_d}{C_{pd} \over C_{ph}}-1 \right]
{\theta \over \Pi_{ref}} {w \over d_{zz} }
 {\partial \Pi_{ref} \over \partial \overline{z}}
\nonumber \\ & &
+ {\tilde{\rho}\over \Pi_{ref} C_{ph}} \left[
 L_m {D(r_i+r_s+r_g+r_h)\over Dt} - L_v {Dr_v\over Dt} + {\cal H}  \right]
\end{eqnarray}
\end{itemize}


\section{Location of the variables on the grid}

We use a C-grid in the Arakawa convention (Mesinger and Arakawa 1976),
both on the horizontal and on the vertical.

\begin{figure}[pbh]
\psfig{figure=\EPSDIR/horzgrid1.eps}
\vspace{1cm}
\psfig{figure=\EPSDIR/essai.eps}
\caption{Discretization on the horizontal \label{horizgrid}}
\end{figure}

The horizontal grid is shown on Fig. ~\ref{horizgrid}.
The "mass" points, located at the center of each grid element, are noted
by $\bigcirc$, the $u$ points by $\triangleright$, and the $v$ points
by  $\bigtriangleup$.
$\zeta $ is the vertical component of the vorticity.
The corners of the shaded square all have the same integer coordinates
$(i,j)$.

\begin{figure}[pbh]
\psfig{figure=\EPSDIR/vergrid1.eps}
\vspace{1cm}
\psfig{figure=\EPSDIR/essai4.eps}
\caption{Discretization on the vertical \label{vertgrid}}
\end{figure}

The vertical grid is shown on Fig.~\ref{vertgrid}.
Again the mass points are noted by
$\bigcirc$, the $u$ points by $\triangleright$, the $v$ points
by $\bigtriangleup$,
and the $w$ points by $\Box$. $\xi$ and $\eta$ are the vorticity components
along x and y. The corners of the shaded square all
have the same integer coordinates.

$\widehat{x},\widehat{y},\widehat{z}$
and the metric coefficients $d_{xx},d_{yy},d_{zz}$ are
respectively located on $u$, $v$, and $w$ points. On the other hand,
$d_{zx}$ and $d_{zy}$ are located on the $\eta$ and $\xi$ vorticity points.
The Jacobian $J$ is located on the mass points.

The orography $z_{s}(i,j)$ is defined on the lowest $w$ point.

\section{Schuman operators}
The discretization is based on second-order finite differences, and two point
averages. We will adopt the traditional notations of Schuman:
$\delta_{x}\alpha$, $\delta_{y}\alpha$ and $\delta_{z}\alpha$ for finite
differences in the directions $\overline{x}$,  $\overline{y}$, and
$\overline{z}$;  $\overline{\alpha}^{x}$, $\overline{\alpha}^{y}$, and
$\overline{\alpha}^{z}$ for averaging in the same directions.

For instance, $\delta_x\alpha (i,j,k)= (\alpha(i+1,j,k)-\alpha(i,j,k))$
or $(\alpha(i,j,k) -\alpha(i-1,j,k))$, depending of the location of
the variable $\alpha$, $\overline{\alpha}^{x}(i,j,k)=(\alpha(i+1,j,k)+
\alpha(i,j,k))/2$ or $(\alpha(i,j,k)+\alpha(i-1,j,k))/2$, and
similarly for the other operators.

For the time evolution,
we use the classical leap-frog explicit scheme. The operator of time
derivation also follows Schuman notation:
\begin{equation}
\delta _{t}\overline{ \alpha (i,j,k,t)}^{t} =
\dfrac{\alpha (i,j,k,t+\Delta t)-\alpha (i,j,k,t-\Delta t)}{2 \Delta t}
\end{equation}

\section{Grid generation}

The parameters needed to generate the grids are:
\begin{itemize}
\item the projection parameters $\lambda_{0}$, $\varphi_{0}$, $\beta$ and $K$.
\item the latitude and longitude of one mass point to start the integration,
i.e. the point $(i=1,j=1)$.
\item the series of values of $\widehat{x}$ at $u$ points, $\widehat{y}$
at $v$ points, and $\widehat{z}$ at w points.
\end{itemize}

The following computations are performed:

1) The positions $\widehat{x}_m(i)$, $\widehat{y}_m(j)$, $\widehat{z}_m(k)$
of the mass points are deduced:
\begin{eqnarray}
\widehat{x}_m(i) & = &0.5 \left[ \widehat{x}(i) +\widehat{x}(i+1) \right]
\nonumber \\
\widehat{y}_m(j) & = &0.5 \left[ \widehat{y}(j) +\widehat{y}(j+1) \right] \\
\widehat{z}_m(k) & = &0.5 \left[ \widehat{z}(k) +\widehat{z}(k+1) \right]
\nonumber
\end{eqnarray}

2) The latitude and longitude of each mass point is retrieved:
\\
\\
{\bf Case of the polar stereographic or Lambert projection}
\\
\\
Applying the general equations to the point (1,1) of the grid, one gets
\begin{eqnarray}
\rho (1,1) & = & \dfrac{a}{K} (\cos\varphi _{0})^{1-K}
 (1 + \sin\varphi _{0})^{K}
\left(\dfrac{\cos\varphi (1,1)}{1 + \sin\varphi (1,1)}\right)^{K} \nonumber\\
\gamma (1,1) & = &   K (\lambda (1,1) - \lambda _{0}) - \beta
\end{eqnarray}
The pole coordinates may be deduced as
\begin{eqnarray}
X_{p} & = &  \widehat{x}_m(1) -  \rho (1,1) \sin\gamma (1,1)  \nonumber \\
Y_{p} & = & \widehat{y}_m(1) + \rho (1,1) \cos\gamma (1,1) \nonumber \\
\end{eqnarray}
The longitude of any mass point (i,j) follows:
\begin{eqnarray}
\lambda (i,j) &=&  \dfrac{\beta + \arctan \left( -
\dfrac{ \widehat{x}_{m}(i)- X_{p}}{ \widehat{y}_{m}(j)- Y_{p}}
 \right) + \epsilon  \pi}{K} + \lambda_{0} \; \; \; \;  \left[ 2 \pi \right]
\end{eqnarray}
with
\begin{eqnarray*}
\epsilon & = & 0 \hbox{ if } \widehat{y}_{m}(j)- Y_{p} \leq 0 \\
\epsilon & = & 1 \hbox{ if } \widehat{y}_{m}(j)- Y_{p} > 0
\end{eqnarray*}
while the latitudes are given by
\begin{eqnarray}
\rho  (i,j)^{2} &=&  \left[ \widehat{x}_m(i) - X_{p} \right]^{2} +
 \left[ \widehat{y}_m(j) - Y_{p}\right] ^{2} \nonumber \\
\varphi (i,j) &=& \dfrac{\pi}{2} - \arccos \left\{ \dfrac{
\left[ a (\cos \varphi_{0})^{1-K} \right] ^{2/K} (1 + \sin\varphi _{0})^{2} -
\left[ K^{2} \rho  (i,j)^{2} \right] ^{1/K} }{
\left[ a (\cos \varphi_{0})^{1-K} \right] ^{2/K} (1 + \sin\varphi _{0})^{2}+
 \left[ K^{2} \rho  (i,j)^{2}  \right] ^{1/K} }
\right\}
\end{eqnarray}
\\
\\
{\bf Case of the Mercator projection}
\\
\\
The latitudes and longitudes of the mass points are given by
\begin{eqnarray}
\lambda (i,j) &=&  \dfrac{\left[ \widehat{x}_{m}(i)-\widehat{x}_{m}(1) \right]
\cos \gamma + \left[\widehat{y}_{m}(j)-\widehat{y}_{m}(1) \right]\sin \gamma  }
{a \cos \varphi _{0}}
  + \lambda(1,1) \nonumber \\
\varphi (i,j) & = & -\dfrac{\pi}{2} + 2 \arctan \left\{ \exp \left( -
\dfrac{\left[\widehat{x}_{m}(1)-\widehat{x}_{m}(i) \right]
\sin \gamma + \left[\widehat{y}_{m}(j)-\widehat{y}_{m}(1) \right]\cos \gamma  }{a \cos \varphi _{0} } \right. \right.  \nonumber \\
& & \left. \left. +  \ln \left[ \tan (\dfrac{\pi}{4} +\dfrac{ \varphi(1,1)}{2})
\right]
\right)
\right\}
\end{eqnarray}
\\
\\
3) Once the latitudes and longitudes are known, the map scale factor,
the angle $\gamma$, and the Coriolis parameters are easily
computed for each mass point:
\begin{eqnarray}
m&=&\left( \dfrac{\cos \varphi _{0}}{\cos \varphi}\right)^{1-K}
\left( \dfrac{1 + \sin\varphi _{0}}{1 + \sin\varphi} \right)^{K} \\
\gamma & = & K (\lambda - \lambda _{0}) - \beta \nonumber \\
f&=&2\Omega \sin\varphi \\
f_{*}&=&2\Omega \cos\varphi
\end{eqnarray}


\section{Metric coefficients and Jacobian}

\begin{eqnarray}
z & = & z_{s} + \widehat{z} \left( 1 - \dfrac{z_{s}}{H} \right) \\
 d_{xx} &  = & \overline{ \left[ \dfrac{\overline{\left(a+ \delta _{1}\delta _{2} z\right)}^{z}}{a}
\;  \dfrac{1}{m}\;  \delta_{x}\widehat{x} \right]}^{x} \\
 d_{yy}  & = & \overline{ \left[\dfrac{ \overline{\left(a+ \delta _{1}\delta _{2} z \right)}^{z}}{a}
 \dfrac{1}{m}\; \delta_y \widehat{y} \right] }^{y} \\
 d_{zx} & = & \delta_{x} z \\  d_{zy} & = & \delta_{y} z \\
 d_{zz} & = & \overline{\delta_{z} z}^z  \\
  J   &  = & \left(
 \dfrac{ \overline{ \left( a+ \delta _{1}\delta _{2} z \right)}^{z}}{a} \right)^{2}
\;  \left( \dfrac{1}{m} \right)^{2} \;
\delta_x \widehat{x} \; \delta_y \widehat{y} \; \delta_z z
\end{eqnarray}

\section{Contravariant velocity components}

\begin{eqnarray}
\overline{\tilde{\rho}}^{x} U^{c} \;  &  = & \overline{\tilde{\rho}}^{x} \dfrac{ u }{ d_{xx}} \\
\overline{\tilde{\rho}}^{y} V^{c} \;  &  = & \overline{\tilde{\rho}}^{y} \dfrac{ v }{ d_{yy}} \\
\overline{\tilde{\rho}}^{z} W^{c} \;  & = & \dfrac{1}{d_{zz}}
\left[\overline{\tilde{\rho}}^{z}  w -
\overline{\left( \overline{\left(
 \dfrac{\overline{\tilde{\rho}}^{x}  u }{d_{xx}} \right)}^{z}
d_{zx}\right)}^{x}
-  \overline{\left( \overline{\left(\dfrac{\overline{\tilde{\rho}}^{y}  v }{d_{yy}}
\right)}^{z}
d_{zy}\right)}^{y}\right]
\end{eqnarray}

\section{Time derivatives}

\begin{eqnarray}
\dfrac{\partial}{\partial t}(\tilde{\rho}  u ) & \Longrightarrow &
\delta _{t } \left[ \overline{ \left( \overline{\tilde{\rho}}^{x}  u
 \right)}^{t} \right]
 \\
\dfrac{\partial}{\partial t}(\tilde{\rho}  v ) & \Longrightarrow &
\delta _{t } \left[  \overline{ \left( \overline{ \tilde{\rho}}^{y}  v
 \right)}^{t}  \right]
 \\
  \dfrac{\partial}{\partial t}(\tilde{\rho} w) & \Longrightarrow &
\delta_{t } \left[
\overline{ \left( \overline{\tilde{\rho}}^{z} w  \right)}^{t} \right]
\end{eqnarray}

\section{Advection}

\begin{eqnarray}
- \dfrac{\partial }{\partial \overline{x}} (\tilde{\rho} U^{c} \;   u )
 & \Longrightarrow & - \delta_{x} \left[
\overline{  \left(
\overline{\tilde{\rho}}^{x} U^{c} \;  \right)}^{x}
\overline{  u  }^{x}
 \right]
  \\
  -  \dfrac{\partial }{\partial \overline{y}} (\tilde{\rho} V^{c} \;   u )
 & \Longrightarrow & - \delta_{y} \left[
\overline{  \left(
\overline{ \tilde{\rho}}^{y} V^{c} \; \right)}^{x} \overline{  u }^{y}
  \right]
\\
 -  \dfrac{\partial }{\partial \overline{z}} (\tilde{\rho} W^{c} \;   u )
 & \Longrightarrow & - \delta _{z} \left[ \overline{  \left(
\overline{\tilde{\rho}}^{z} W^{c} \;  \right)}^{x}
 \overline{ u }^{z}
 \right] \\ \nonumber \\
  -  \dfrac{\partial }{\partial \overline{x}} (\tilde{\rho} U^{c} \;   v )
& \Longrightarrow & - \delta_{x} \left[\overline{  \left(
\overline{\tilde{\rho}}^{x} U^{c} \; \right)}^{y} \overline{  v }^{x}
 \right]
\\
- \dfrac{\partial }{\partial \overline{y}} (\tilde{\rho} V^{c} \;   v )
& \Longrightarrow & - \delta_{y} \left[\overline{  \left(
\overline{ \tilde{\rho}}^{y} V^{c} \; \right)}^{y} \overline{   v }^{y}
\right] \\
   - \dfrac{\partial }{\partial \overline{z}} (\tilde{\rho} W^{c} \;   v )
& \Longrightarrow &  - \delta_{z} \left[ \overline{  \left(
\overline{ \tilde{\rho}}^{z} W^{c} \;  \right)}^{y} \overline{  v }^{z}
\right]  \\ \nonumber \\
  -  \dfrac{\partial }{\partial \overline{x}} (\tilde{\rho} U^{c} \;  w)
& \Longrightarrow &  - \delta_{x} \left[\overline{  \left(
\overline{ \tilde{\rho}}^{x}
 U^{c} \; \right)}^{z} \overline{ w}^{x} \right]
   \\
-  \dfrac{\partial }{\partial \overline{y}} (\tilde{\rho} V^{c} \;  w )
& \Longrightarrow &  - \delta_{y} \left[\overline{  \left(
\overline{ \tilde{\rho}}^{y} V^{c} \; \right)}^{z}\overline{
 w }^{y}\right] \\
 -  \dfrac{\partial }{\partial \overline{z}} (\tilde{\rho} W^{c} \;  w)
& \Longrightarrow &  - \delta_{z} \left[ \overline{  \left(
\overline{ \tilde{\rho}}^{z}  W^{c} \;  \right)}^{z}\overline{w}^{z} \right]
 \end{eqnarray}

\section{Curvature terms}

\begin{eqnarray}
 \delta _{2}\tilde{\rho}  u   v
 \dfrac{ \cos\gamma}{r \cos\varphi} (\sin\varphi -K) & \Longrightarrow &
 \delta _{2} \overline{\tilde{\rho}}^{x}   u
 \overline{\left(\overline{ v }^{y}
 \dfrac{ \cos\gamma}{ \cos\varphi} (\sin\varphi -K)
 \dfrac{1}{\overline{a+\delta _{1} z}^{z}}\right)}^{x}  \\
\delta _{2}\tilde{\rho}  v ^{2}  \dfrac{ \sin\gamma  }{r \cos \varphi} (\sin\varphi -K)
& \Longrightarrow & \delta _{2}  \overline{\left( \overline{\left(\overline{\tilde{\rho}}^{y}
 v ^{2}\right)}^{y}\dfrac{ \sin\gamma}{ \cos\varphi} (\sin\varphi -K)
 \dfrac{1}{\overline{a+\delta _{1} z}^{z}}\right)}^{x}
 \\
 - \delta _{2}\delta _{1} \tilde{\rho}\dfrac{ u  w}{r}
  & \Longrightarrow &
 - \delta _{2}\delta _{1} \overline{\tilde{\rho}}^{x}  u
 \overline{\left( \dfrac{\overline{ w}^{z}}{\overline{a+\delta _{1} z}^{z}} \right)}^{x}\\
 \nonumber \\
 -  \delta _{2}\tilde{\rho}  u ^{2}  \dfrac{ \cos\gamma}{r \cos\varphi} (\sin\varphi -K)
& \Longrightarrow &  -  \delta _{2}
 \overline{\left(\overline{\left(
\overline{\tilde{\rho}}^{x}
  u ^{2}\right)}^{x}  \dfrac{ \cos\gamma}{ \cos\varphi} (\sin\varphi -K)
 \dfrac{1}{\overline{a+\delta _{1} z}^{z}}\right)}^{y}\\
 - \delta _{2}\tilde{\rho}  u   v
   \dfrac{\sin\gamma}{r \cos \varphi} (\sin\varphi -K)
& \Longrightarrow & -  \delta _{2} \overline{\tilde{\rho}}^{y} v
 \overline{\left( \overline{  u }^{x} \dfrac{ \sin\gamma}{ \cos\varphi} (\sin\varphi -K)
 \dfrac{1}{\overline{a+\delta _{1} z}^{z}}\right)}^{y} \\
-  \delta _{2}\delta _{1} \tilde{\rho}\dfrac{ v  w}{r}
& \Longrightarrow & -  \delta_1\delta _{2}\delta _{1}\overline{\tilde{\rho}}^{y}
 v \overline{\left( \dfrac{\overline{ w}^{z}}{
\overline{a+\delta _{1} z}^{z}} \right)}^{y}\\
 \nonumber \\
 \delta _{2}\delta _{1}\tilde{\rho}\dfrac{ u ^{2}+ v ^{2} }{r}
& \Longrightarrow &    \delta _{2}\delta _{1}
\overline{\left(\dfrac{ \overline{\left(\overline{\tilde{\rho}}^{x} u ^{2}\right)}^{x}
+ \overline{\left( \overline{\tilde{\rho}}^{y} v ^{2}\right)}^{y}}
{\overline{a+\delta _{1} z}^{z}}\right)}^{z}
\end{eqnarray}

\section{Coriolis force}

\begin{eqnarray}
 - \delta _{1} \tilde{\rho} f \cos\gamma w & \Longrightarrow &
 - \delta _{1} \overline{ \left(\tilde{\rho} f \cos\gamma \overline{
 w}^{z}\right)}^{x}
 \\
 \tilde{\rho} f  v  & \Longrightarrow & \overline{ \left(\tilde{\rho} f
  \overline{ v  }^{y} \right)}^{x} \\
-\delta _{1}\tilde{\rho} f \sin\gamma w & \Longrightarrow &
-\delta _{1} \overline{ \left(\tilde{\rho} f \sin\gamma \overline{
 w}^{z}\right)}^{y} \\
 -   \tilde{\rho} f  u  & \Longrightarrow &
 - \overline{ \left(  \tilde{\rho} f \overline{  u }^{x}\right)}^{y} \\
\delta _{1}\tilde{\rho} f(\sin\gamma  v  + \cos\gamma  u )
 & \Longrightarrow & \delta _{1} \overline{ \left(\tilde{\rho} f
 \cos\gamma  \overline{   u  }^{x}\right)}^{z }\, + \, \delta _{1} \overline{ \left(\tilde{\rho} f
 \sin\gamma  \overline{   v  }^{y}\right)}^{z }
  \end{eqnarray}

\section{Pressure gradient}
\begin{eqnarray}
- \tilde{\rho}\dfrac{1}{d_{xx}}
\dfrac{\partial \Phi}{\partial \overline{x}}
 & \Longrightarrow & -  \overline{  \tilde{\rho}}^{x}\dfrac{\delta_{x} \Phi}{d_{xx}}
 \\
\tilde{\rho}\dfrac{d_{zx}}{d_{xx}d_{zz}}
\dfrac{\partial \Phi}{\partial \overline{z}}
 & \Longrightarrow &   \overline{ \tilde{\rho}}^{x} \dfrac{1}{d_{xx}}
\overline{ \left(\overline{ \left(\dfrac{\delta_{z} \Phi}{d_{zz}} \right)}^{x}d_{zx}\right)}^{z}\\
- \tilde{\rho}\dfrac{1}{d_{yy}}
\dfrac{\partial \Phi}{\partial \overline{y}}
 & \Longrightarrow & -   \overline{  \tilde{\rho}}^{y}
 \dfrac{\delta_{y} \Phi}{d_{yy}}
 \\
   \tilde{\rho}\dfrac{d_{zy}}{d_{yy}d_{zz}}
\dfrac{\partial \Phi}{\partial \overline{z}}
 & \Longrightarrow &  \overline{ \tilde{\rho}}^{y} \dfrac{1}{d_{yy}}
\overline{ \left(\overline{ \left(\dfrac{\delta_{z} \Phi}{d_{zz}}
\right)}^{y}d_{zy}\right)}^{z}\\
  -  \tilde{\rho}\dfrac{1}{d_{zz}}
  \dfrac{\partial \Phi}{\partial \overline{z}}
 & \Longrightarrow & - \overline{ \tilde{\rho}}^{z}  \dfrac{\delta_{z}
 \Phi}{d_{zz}}
   \end{eqnarray}

\section{Buoyancy term}
\begin{equation}
\tilde{\rho} g \dfrac{\theta_v ' }{\overline{\theta}_v}
\Longrightarrow g \overline{ \left(
\tilde{\rho}\dfrac{\theta_v ' }{\overline{\theta}_v}\right) }^{z}
\end{equation}

\section{Thermodynamic equation}

\begin{eqnarray}
\delta _{t } \left[ \overline{ \left( \tilde{\rho} \theta
\right)}^{t} \right]  & =  & - \delta_{x} \left[\overline{\left(
\tilde{\rho} \theta\right)}^{x} U^{c} \;
\right] - \delta_{y} \left[\overline{\left(
\tilde{\rho}\theta\right)}^{y}  V^{c} \;
\right] - \delta_{z} \left[\overline{ \left(
\tilde{\rho}\theta\right)}^{z} W^{c} \;
\right] \nonumber \\ & &
+\tilde{\rho} \left[ {R_d+r_vR_v\over R_d}{C_{pd} \over C_{ph}}-1 \right]
{\theta \over \Pi_{ref}} \overline{ {w \over d_{zz} } \delta_z \Pi_{ref} }^z
\nonumber \\ & &
+ {\tilde{\rho}\over \Pi_{ref} C_{ph}} \left[
 L_m {D(r_i+r_s+r_g+r_h)\over Dt} - L_v {Dr_v\over Dt} + {\cal H}  \right]
\end{eqnarray}


\section{Continuity equation}

\begin{equation}
\delta_{x} \left[ \overline{\tilde{\rho}}^{x} U^{c} \; \right] +
\delta_{y} \left[ \overline{\tilde{\rho}}^{y} V^{c} \; \right] +
\delta_{z} \left[ \overline{\tilde{\rho}}^{z} W^{c} \; \right] =0
\end{equation}

\section{Pressure equation}

For convenience, the discretization of the pressure equation is described
in Chapter \ref{PressureProblem}.

\section{References}
\decrefname
Mesinger, F., and A. Arakawa, 1976:
Numerical methods used in atmospheric models.
GARP Publications Series No. 17, WMO/ICSU Joint Organizing Committee, 64 pp.

%\end{document}

