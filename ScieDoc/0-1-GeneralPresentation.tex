%%%%%%%%%%%%%%%%%%%%%%%%%%%%%%%%%%%%%%%%%%%%%%%%%%%%%%%%%%%%%%%%%%%%%%%%%%%%%%
%
% THE MESONH BOOK1
% Author      : Philippe Bougeault, and collective contributions.
% Original    : September 1, 1995 (Ph. Bougeault)
% Last Update : May 21 2002  (P. Mascart, P. Bechtold new subcloud scheme)
%
% Copyright Meteo-France and Laboratoire d'Aerologie, 1996, 1999, 2000, 2001
%
%%%%%%%%%%%%%%%%%%%%%%%%%%%%%%%%%%%%%%%%%%%%%%%%%%%%%%%%%%%%%%%%%%%%%%%%%%%%%%
%

\chapter{General Presentation}

\minitoc

%{\em by P. Bougeault}

\section{Introduction}
 The Meso-NH Atmospheric Simulation System
is a joint effort of Centre National de
Recherches M\'et\'eorologiques (M\'et\'eo-France) and Laboratoire d'A\'erologie
(CNRS). It comprises several elements:
{\bf a numerical model} able to simulate the atmospheric
motions, ranging from the large meso-alpha scale down to the micro-scale,
with a {\bf comprehensive physical package}; a flexible {\bf file manager};
 an ensemble of facilities to {\bf prepare
initial states}, either idealized or interpolated from real meteorological
analyses or forecasts;
a flexible {\bf post-processing and graphical facility}
 to visualize the results;
an ensemble of {\bf unix procedures} to control these functions.

 Some of the distinctive features of this ensemble are the followings:
the model is based on an advanced set of anelastic systems.
 It allows for simultaneous
 simulation of several scales of motion, by the
so-called  "interactive grid-nesting technique". It allows for the in-line
computation and accumulation of various terms of the budget of several
quantities. It allows for the transport and diffusion of passive scalars,
to be coupled with a chemical module. It uses the relatively new Fortran 90
compiler. It is tailored to be easily implemented on any Unix machine.

The documentation of this system is divided in three "books". The present
"book 1" contains a {\bf scientific introduction} to Meso-NH, discussing the
system of equations, the geographical projections, the discretization, and
the physical parameterizations. The "book 2" is the {\bf algorithmic
documentation}
of the Fortran 90 code and of the Unix procedural package, with a summary
of code organization, as well as the inputs, outputs, and actions of each
subroutine. Finally, the "book 3" is a {\bf User Manual}, designed to allow
a quick acquaintance of new users with the control parameters of the system.

Meso-NH is designed as a research tool for small and meso-scale atmospheric
processes. It is freely accessible to any interested user, and we have
tried to make it as "user friendly" as possible, and as general as possible,
although these two goals sometimes appeared contradictory.

\section{The anelastic system\label{CHP1anel}}

It is well known that the fast propagation of acoustic waves in the
Euler equation system sets a very strong constraint on the computational
time step. Generally speaking, three different
methods have been developed to avoid this problem: (i)the use of two different
time steps for acoustic and non-acoustic types of motions; (ii)the implicit
treatment of some terms, which usually involves a linearization around some
mean atmospheric state; (iii)finally the various "anelastic" approximations.
In presence of steep orography, the separation of the terms responsible
for acoustic waves becomes much more difficult, and the problem is more
complex. An extensive comparison of these method has been done by Ikawa (1988).
\\

We have examined some of these method in great detail during the
preliminary studies of the Meso-NH project, and concluded in favor of the
"anelastic" approach. In this approach, the acoustic waves are eliminated
from the continuous set of equations by the use of a constant density profile
instead of the actual fluid density in the continuity equation and in the
momentum equation, except for the buoyancy term, which is the leading term
of the approximation. The fluid becomes therefore formally incompressible,
and the pressure is deduced from the solution of an elliptic equation. The
continuity equation sets a strong constraint on the 3D velocity field,
which is no longer synonymous of mass conservation. In fact the mass
conservation must be insured by an additional equation, in a similar way
as the surface pressure equation in hydrostatic models using the sigma
coordinate system. A beneficial consequence is that the system has no
external gravity wave, and the time discretization is entirely explicit,
without introducing an overly restrictive constraint on the computational
time step. This makes the code in general much more simple and conservative.
\\

Known detrimental effects of the anelastic approximation are that: (i)
the vertical velocities simulated by the model are slightly inaccurate
(because the continuity and vertical momentum equations are approximate);
(ii) pressure
perturbations linked to the filtered acoustic waves may be transmitted
instantaneously from one place to the other in the simulation domain, instead
of traveling at the speed of sound. At the present time, there does not
seem to exist any meteorological consequences of these effects, since
the inaccuracy on the vertical velocity is smaller than the one induced
by other sources (like uncertainty in the initial state and in the physical
parameterizations), and since the acoustic waves are largely decoupled
from meteorologically significant motions. We have therefore decided to
adopt this method as the most simple allowing for an effective filtering of
acoustic waves. However, at several places in the code,
we have made provisions for a future possible change back to the fully
compressible system, in case future research demonstrates some incapacity
of the anelastic system.
\\

% Modif pour MASTER 4 version, April 2000, P.M.
%Finally, among the various available
%versions of the anelastic approximation, we have
%adopted the Lipps and Hemler (1982) formulation, as the most simple allowing
%for a correct representation of the gravity waves momentum flux in the upper
%atmosphere.
Finally, among the various available versions of the anelastic approximation, we have
implemented three formulations: the Lipps and Hemler (1982) anelastic system,
the Wilhelmson an Ogura (1972) "modified anelastic equations", and the "psedo--incompressible"
system of Durran (1989), which offer some flexibility for the representation of meteorological
flow depending on horizontal and vertical scales. A brief examination of the respective merits
of these three systems is given in next chapter, but most of the rest of this manual simply
illustrates the case of the  Lipps and Hemler (1982) formalism for the sake of brevity.

\section{Thermodynamic Variables}

The exact conservation of energy in presence of water, and especially liquid
and ice phase, is a very difficult problem. The use of entropy as a prognostic
thermodynamic variable appears to be the ultimate solution, however the
computation of temperature, and the various mixing ratios of water from the
entropy appears technically difficult, and subject to approximations.
Alternative proposals by Pointin(1984), Hauf and H\"oller (1987), and
Marquet(1991), do not get rid of these problems. Therefore, although
we did not rule out the use of entropy in Meso-NH, we have decided that the
first version of the model would rather use a "dry" potential temperature as a
thermodynamic variable. The evolution
equation for this variable is an original one,
and takes into account the effect of water vapor, and water phase changes,
to a very good accuracy, we think.

\section{Water}

We adopted the use of mixing ratio, rather than specific content,
for the various water phases, and other scalars in the model.
The specific content is the ratio of the mass of a given substance to
the {\em total mass} within a given volume.
This total mass may vary because of the
loss of water by precipitation, making the use of specific content tricky.
The mixing ratio is the ratio of the mass of considered substance to the
mass of {\em dry air}
 within the considered volume. The later is a true invariant.
Therefore, we think that mixing ratio is a more appropriate quantity to
conserve.

For the water, up to seven different mixing ratios can be carried
by the model, corresponding to vapor, cloud liquid water, rain water, cloud ice,
snow, graupel, and hail. The user may choose to use only a few of these,
depending of his interest.

\section{Geographical Systems}
Here the classical alternative is between the latitude/longitude system on the
sphere, with possible stretching of the grid, and the conformal projection.
Although all options may be maintained simultaneously in the code, we
thought this would make it less easy to understand and maintain, and in view
of the major foreseen applications, decided to code only the conformal
projection system. This encompass a simple cartesian plan for very idealized
applications. Since the curvature terms are retained, and the thin layer
approximation may be relaxed, this does not preclude, in our opinion, the
use of Meso-NH for the study of very large scale (though not global)
processes. On the projection plane, the coordinates may be stretched,
which allows for the use of effective variable resolution grid, much like
the Canadian EFR System (Chouinard et al., 1994).
In fact the combination in the same code of the
grid-nesting capacity and the variable resolution option renders this tool
rather unique to adress problems of scale interactions.

\section{Vertical Coordinate}
Again, we were faced with a difficult choice between the classical Gal-Chen
and Sommerville (1975)
 system of vertical coordinates, featuring a purely vertical,
very simple transformation and allowing for an effective terrain following
coordinate system, and the much more complicated, generalized curvilinear
system, such as recently adopted in the ARPS model for instance. The major
benefit expected from the generalized curvilinear transformation
is to simplify the solution of the pressure problem in case of steep and
nearly angulous topography. We have developed a prototype 2D version of
the Meso-NH system, supporting both coordinate systems, in order to examine
this point. Stein (1995) has demonstrated that in most cases of interest for
the foreseen use of Meso-NH, the Gal-Chen and Sommerville approach would
do nearly as well as the generalized curvilinear system, for a much reduced
memory occupation and computation time. We therefore opted for the previous
one, resulting in a huge simplification of the code. The major limitation
of this system is not in terms of {\em slope}, since idealized flows over slopes
of $70\%$ have been simulated satisfactorily, but in terms of {\em slope
discontinuity}: the topography used in the model must avoid a cliff-type
 behaviour.
We feel this limitation would have been present with any other system anyway.


\section{References}
\por
Chouinard, C., J. Mailhot, H. L. Mitchell, A. Staniforth, and R. Hogue, 1994:
The Canadian Regional Data Assimilaton System: Operational and Research
Applications. Mon. Wea. Rev., 122, 1306-1325.
\por
Gal-Chen, T., and R. C. J. Sommerville, 1975: On the Use of a Coordinate
Transformation for the Solution of the Navier-Stokes Equations. J. Comput.
Phys., 17, 209-228.
\por
Hauf, T., and H. H\"oller, 1987: Entropy and Potential Temperature. J. Atmos.
Sci., 44, 2887-2901.
\por
Ikawa, M., 1988: Comparison of Some Schemes for Nonhydrostatic Models with
Orography. J. Meteor. Soc. Jap., 66, 753-776.
\por
Lipps, F. B., and R. S. Hemler, 1982: A scale analysis of deep moist convection
and some related numerical calculations. J. Atmos. Sci., 39, 2192-2210.
\por
Marquet, P., 1991: On the Concept of Exergy and Available Enthalpy: Application
to Atmospheric Anergetics. Quart. J. Roy. Meteor. Soc., 117, 449-475.
\por
Pointin, Y., 1984: Wet Equivalent Potential Temperature and Enthalpy as
Prognostic Variables in Cloud Modeling. J. Atmos., Sci., 15, 651-660.
\por
Stein, J., 1995: to appear.
\por
Durran D. R., 1989: Improving the anelastic approximation, J. Atmos. Sci., {\bf 46}, 1453--1461.
\por
Lafore J.P., J. Stein, N. Asencio, P. Bougeault, V. Ducrocq, J. Duron, C. Fischer, P. H�reil,
P. Mascart, V. Masson, J. P. Pinty, J. L. Redelsperger, E. Richard, J. Vil\'a-Guerau de
Arellano, 1998: The Meso-NH atmospheric simulation system. Part I: adiabatic formulation and
control simulations, Ann. Geophysicae, {\bf 16}, 90--109.
\por
Lipps F., and R. S. Hemler, 1982: A scale analysis of deep moist convection and some related
numerical calculations, J. Atmos. Sci., {\bf 39}, 2192--2210.
\por
Wilhelmson, R., and Y. Ogura,1972: The pressure perturbation and the numerical modelling of a
cloud, J. Atmos. Sci., {\bf 29}, 1295--1307.

