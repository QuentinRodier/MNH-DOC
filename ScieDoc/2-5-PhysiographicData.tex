%%%%%%%%%%%%%%%%%%%%%%%%%%%%%%%%%%%%%%%%%%%%%%%%%%%%%%%%%%%%%%%%%%%%%%%%%%%
%%%%%%%%%%%%%%%%%%%%%%%%%%%%%%%%%%%%%%%%%%%%%%%%%%%%%%%%%%%%%%%%%%%%%%%%%%%
% CONTRIBUTION TO THE MESONH BOOK1: "Physiographic Data and
%                                    Initial Fields for Real Flows "
% Author : S. Belair, V. Masson and F. Mereyde
% Original : September 07, 1997
% Update   : January 12, 1998
% Update   : January 17, 2000
%%%%%%%%%%%%%%%%%%%%%%%%%%%%%%%%%%%%%%%%%%%%%%%%%%%%%%%%%%%%%%%%%%%%%%%%%%%
%%%%%%%%%%%%%%%%%%%%%%%%%%%%%%%%%%%%%%%%%%%%%%%%%%%%%%%%%%%%%%%%%%%%%%%%%%%
%
%%%%%%%%%%%%%%%%%%%%%%%%%%%%%%%%%%%%%%%%%%%%%%%%%%%%%%%%%%%%%%%%%%%%%%%%%%%%%
\chapter{Physiographic Data\label{PGD}}
\minitoc

%{\em by S. Belair, V. Masson and F. Mereyde}

\section{Introduction}

As shown in many studies,
a good representation of land surface characteristics
is often necessary for numerical models to reproduce realistically certain
meteorological events and climatological patterns.
For example, particular atmospheric phenomena such as mesoscale waves, typical
precipitation patterns, blocking of synoptic systems, and cyclogenesis, can
be generated or induced by the topography, thereby indicating the need to use
realistic and high-resolution two-dimensional fields in numerical models to
represent this surface feature.
Similarly, the land-water mask, the soil-vegetation characteristics, and
the urbanized area locations are
required for the calculation of surface fluxes of heat, moisture, and
momentum over continental and oceanic surfaces.
Other surface fields, like the albedo and emissivity, have direct effects
on the radiation balance at the surface and in the atmosphere.


For the Meso-NH atmospheric model, the determination of the two-dimensional
surface fields reveals to be particularly crucial and difficult.
Crucial because many of the scientific purposes of this model are related
to the examination of mesoscale circulations generated by discontinuities
in the surface characteristics (e.g., mountain waves, sea and land breezes,
urban heat island, meso-$\gamma$ fluxes heterogeneities);
and difficult since this model can be run at horizontal resolutions ranging
from the micro-scale to the meso-$\alpha$ scales, thereby making the
automatic generation of surface fields bewildering.
Nevertheless, it is the aim of this part of the Meso-NH atmospheric modelling
system to produce an MNH file containing all the surface fields
necessary for an integration of the model. Such a file is called a
PGD file (for 'PhysioGraphic Data file'). This is done by a program
called PREP\_PGD, which averages
(or interpolates) the geographic data onto the MESONH grid.
The averaging is arithmetic, except for some fields, such as heat capacities
and stomatal resistances (where it is harmonic), or for roughness lengths.\\

Since many users will want to use their own physiographic data sources
(especially for very high resolution runs, some hundreds of meters or less),
we have designed a system which allows easy interfacing with the model.
This is based on "data files" having a very general and simple format.
How to use such data files is explained in the MESO-NH book3.

However, for those users that are not preoccupied by these problems,
we also provide default data files.
These have a very good resolution, of the order of 30 seconds (approximately
1 km), they are global, and thus assure the MNH users that
correct surface characteristics will be considered in their simulation,
regardless of the position of the model grid on the globe.
Over Europe, a surface cover type file with finer resolution (250m)
is available.

The default information comes from 4 distinct data files (see table{dataPGD}).

\begin{table}
\hspace*{2.cm}
\begin{tabular}{||l||c|c|c||}
\hline
\hline
type of data & resolution & area & file name \\
\hline
\hline
surface cover type & 1km & global & ecoclimats\_v1 \\
%                   & 250m & Europe & ecoclimats\_europe \\
\hline
orography          & 1km & global & gtopo30 \\
\hline
clay fraction & 10km & global & clay\_fao \\
\hline
sand fraction & 10km & global & sand\_fao \\
\hline
\hline
\end{tabular}
\caption{Default data files.
\label{dataPGD}}
\end{table}

All the physiographic data necessary to the different parameterizations
of the MESO-NH model are recovered from these data.
The next section presents the surface cover type data.
The following section
will present the fields computed from the surface cover type. The next
one the fields computed with the orographic data.

%%%%%%%%%%%%%%%%%%%%%%%%%%%%%%%%%%%%%%%%%%%%%%%%%%%%%%%%%%%%%%%%%%%%%%%%%%%%%%%%

\section{The surface cover types}

The most important data is the cover type. A surface cover type is the
type of landscape that is present at a certain location:
sea, river, forest, town, grassland, crops, desert...
Idealy, the more the cover types are, the more it is possible to refine
the description of each location. The default data file contains 250
cover types.
This is fine enough (as will be presented in next section) to derive
reasonable vegetation parameters all around the world. Here is explained
how this file was built:

\subsection{On the world, except Europe}

\begin{enumerate}
\item the coastline, lakes and river paths
(finer data, resolution 1km) come from the Digital Chart of the World
(DCW) data.
\item 15 main cover types come from the University of Maryland data
(Hansen et al. 2000) derived from satellite NOAA/AVHRR.
The two last cover types come from the IGBP-DIS data set (Belward 1996).
The resolution of these two datasets is 1km. However, we can consider
that the real resolution is about a few kilometers.
These types are:
\begin{enumerate}
\item evergreen needleleaf forest
\item evergreen broadleaf forest
\item deciduous needleleaf forest
\item deciduous broadleaf forest
\item mixed forest
\item woodland
\item wooded grassland
\item closed shrubland
\item open shrubland
\item grassland
\item cropland
\item bare soil
\item urbanized areas
\item permanent snow and ice
\item wetlands
\end{enumerate}
\item The problem is that this classification does take into account
only the most general pattern of the cover. In reality, an open
shrubland in polar regions (tundra) and in tropical aeras (savana)
will have a very different vegetative behaviour. The same stands
for all these main types. Therefore,
a map of climates of the world derived from Koeppe and De Long (1998)
was used to add a discrimination between these cover types. 16 climates
were used:
\begin{enumerate}
\item dry summer subtropical
\item tropical desert
\item semiarid tropical
\item wet and dry tropical
\item wet equatorial
\item trade wind littoral
\item humid subtropical
\item semiarid continental
\item intermediate desert
\item moderate polar
\item cool marine
\item polar
\item cool littoral
\item humid continental
\item extreme subpolar
\item marine subpolar
\end{enumerate}
\item The combination of the 15 cover types with the 16 climates
for the six continents (both for northern and southern hemisphere parts for
Africa, South-america and Oceania) led to a very important number
of possible 'ecosystems'. Some of them occupy a very small area
on the earth, and are not representatives. Others, even in different
climates or continents have a very similar behaviour. In order to
discriminate the representative cover-climate ecosystems,
the 1km NDVI composites from NOAA/AVHRR from April 1992 to March 1993
(Eidenshink and Faudeen 1994) were used.
Aggregation between different ecosystems into one representative ecosystem
was performed according to the following rules:
\begin{itemize}
\item an ecosystem with one cover type can not be aggregated with one
other ecosystem with another cover type (e.g. closed shrubland will not be
aggregated with open shrubland). This respects the main cover types above.
Exception: a forest could be aggregated with another forest, leading
either to a forest with one of their type (needleleaf or broadleaf,
evergreen or deciduous), or to a mixed forest.
\item the aggregation was performed with priority between climates
on one continent. This was perfomed comparing the NDVI annual profiles
(minimum, maximum, shape of the profile).
When more than one month seperates the minimum (or the maximum)
of two profiles, they were considered to be representative of two
different vegetations, and were not aggregated.
The ecosystems in the 'wet equatorial' climate must be the same on the two
sides of the equator.
\item if two NDVI profiles for two ecosystems of the same cover type are
found on several continents, they could also be aggregated.
\end{itemize}
Finally, this procedure lead to 125 representative cover types on the world
(except Europe). They are really
representative of one vegetation behaviour, allowing to derive
vegetation parameters for each of them (leaf are index, fraction of forest,
roughness,...).
\end{enumerate}


\subsection{On Europe, for the 1km map}

The same procedure was used over Europe, but with initial data at a
better resolution:

\begin{enumerate}
\item where available, the coastline is derived
from the CORINE land cover data set (resolution 250m), averaged at 1km.
\item the cover types are issued from CORINE land cover when available.
When not available, they come from the PELCOM data set
(Mucher, 2000). The main goal of PELCOM project, which was an European project,
was to obtain a 1km pan-european land cover database from NOAA/AVHRR.
\item the climate map is derived on Europe from the FIRS project.
\item this leads to about 100 ecosystems on Europe
\end{enumerate}

%%%%%%%%%%%%%%%%%%%%%%%%%%%%%%%%%%%%%%%%%%%%%%%%%%%%%%%%%%%%%%%%%%%%%%%%%%%%%%%%

\newpage

\section{Fields deduced from the surface cover types}

%%%%%%%%%%%%%%%%%%%%%%%%%%%%%%%%%%%%%%%%%%%%%%%%%%%%%%%%%%%%%%%%%%%%%%%%%%%%%%%%

\subsection{Sea, inland water, town and artificial surfaces, natural and
cultivated landscape}

\begin{table}[h]
\hspace*{2.cm}
\begin{tabular}{||l|c|c||}
\hline
\hline
field & notation &  unit \\
\hline
\hline
 fraction of sea & $f_{sea}$ & - \\
 fraction of artificial areas & $f_{town}$ &- \\
 fraction of inland water & $f_{water}$ & - \\
 fraction of natural and cultivated areas &$f_{nature}$ &  - \\
\hline
\hline
\end{tabular}
\caption{Main surface parameters.
\label{paramSURF}}
\end{table}

These four surface types
are the most important fields deduced from the
cover types. They define the partition of the grid mesh between four
very different surfaces.
During the MESO-NH run,
a particular surface scheme will be used for each of them to compute the
energy fluxes towards the atmosphere. Then, the four fluxes will be
averaged according to the surface occupied by each surface, in order to
retrieve the global energy fluxes to the atmosphere.

Default values for these four fields are prescribed for each of the
representative ecosystems: for example,
natural surface fraction is 1 for forest, but only 0.3 for urban areas
(garden), while the other 0.7 are for artificial areas. See appendix for
details.

%%%%%%%%%%%%%%%%%%%%%%%%%%%%%%%%%%%%%%%%%%%%%%%%%%%%%%%%%%%%%%%%%%%%%%%%%%%%%%%%

\subsection{Artificial parameters}


\begin{table}[h]

{\footnotesize{
\begin{tabular}{||l |c|c||}
\hline
\hline
field                         & notation & unit    \\
\hline
geometric parameters && \\
\hline
 fractional artificial area occupied by buildings & $a_{bld}$                 & - \\
 fractional artificial area occupied by roads     & $1 - a_{bld}$             &- \\
 building height                       & $h$                       &m \\
 building aspect ratio           & $h/l$                     & - \\
 canyon aspect ratio             & $h/w$                     &- \\
 dynamic roughness length for the building/canyon system & $z_{0_{town}}$            & m \\
\hline
radiative parameters && \\
\hline
 roof, road and wall albedos & $\alpha_R$, $\alpha_r$, $\alpha_w$              &- \\
roof, road and wall emissivities
              &$\epsilon_R$, $\epsilon_r$, $\epsilon_w$            & - \\
\hline
thermal parameters && \\
\hline
 thickness of the $k^{th}$ roof, road or wall layer &$d_{R_k}$, $d_{r_k}$, $d_{w_k}$ & m \\
 thermal conductivity of the $k^{th}$ roof, road or wall layer & $\lambda_{R_k}$, $\lambda_{r_k}$, $\lambda_{w_k}$          &W m$^{-1}$ K$^{-1}$\\
 heat capacity of the $k^{th}$ roof, road or wall layer & $C_{R_k}$, $C_{r_k}$, $C_{w_k}$                 &J m$^{-3}$ K$^{-1}$\\
\hline
\hline
\end{tabular}
}}

\caption{Parameters of the TEB scheme.  {\it{
Note that some surfaces
between the buildings, such as gardens or parks for example, are {\bf not}
treated by the TEB model, but modify the canyon width, $w$.}}
}
\label{paramTEB}
\end{table}

For each of the cover types partially or totally covered by
artificial areas (typically urban areas), some parameters describing the
surface are needed.
These parameters will be used in the 'Town Energy Budget' model (Masson 2000).

\newpage
%%%%%%%%%%%%%%%%%%%%%%%%%%%%%%%%%%%%%%%%%%%%%%%%%%%%%%%%%%%%%%%%%%%%%%%%%%%%%%%%

\subsection{Water parameters}

\begin{table}[h]
\hspace*{4.cm}
\begin{tabular}{||l|c|c||}
\hline
\hline
field & notation & value \\
\hline
\hline
water near-infra-red albedo & $\alpha_{{nir}_{wat}}$ & 0.20 \\
water visible albedo & $\alpha_{{vis}_{wat}}$ & 0.07 \\
water emissivity & $\epsilon_{_{wat}}$ & 0.98 \\
\hline
sea ice near-infra-red albedo & $\alpha_{{nir}_{ice}}$ & 0.85 \\
sea ice visible albedo & $\alpha_{{vis}_{ice}}$ & 0.85 \\
sea ice emissivity & $\epsilon_{_{ice}}$ & 1. \\
\hline
\hline
\end{tabular}
\caption{Water and Sea ice parameters
\label{water}}
\end{table}

At the time being, the physiographic fields (albedos and emissivity)
for water are uniform.

Occurence of sea ice is detected using the
Sea Surface Temperature (coming from the meteorological analyses).
If SST is below two degrees below 0$^\circ$ (273.15K), then the
sea surface is assumed to be frozen. Its radiative parameters correspond
to those of fresh snow.


%%%%%%%%%%%%%%%%%%%%%%%%%%%%%%%%%%%%%%%%%%%%%%%%%%%%%%%%%%%%%%%%%%%%%%%%%%%%%%%%

\subsection{Main natural landscape parameters parameters}

\begin{table}[h]
\hspace*{4.cm}
\begin{tabular}{||l|c|c||}
\hline
\hline
field & notation & unit \\
\hline
\hline
fraction of vegetation type &  & - \\
\hspace*{1.cm}$\begin{array}{l}
{\rm trees}\\
{\rm grassland}\\
{\rm C3 \; type \; crops}\\
{\rm C4 \; type \; crops}\\
{\rm bare \;  soil \; (smooth)}\\
{\rm rocks} \\
{\rm permanent  \; snow} \\
\end{array} $ &
$\begin{array}{l}
f_{trees}\\
f_{grass}\\
f_{C3}\\
f_{C4}\\
f_{bare}\\
f_{rocks} \\
f_{snow} \\
\end{array}$
&  \\
\hline
Leaf Area Index (monthly)   & LAI       &  $m^2/m^2$ \\
\hline
height of trees & $h$ &  $m$ \\
\hline
 root depth & $d_2$ & $m$ \\
\hline
 soil depth & $d_3$ &$m$ \\
\hline
\hline
\end{tabular}
\caption{Main ISBA parameters.
\label{paramISBA1}}
\end{table}

When a representative ecosystem contains some vegetated or bare soil part
(this is the most common case, the only exception
being for sea or inland water types), surface parameters are needed
to describe both the vegetation and its horizontal extension.

Some parameters are prescribed for each ecosystem.

\begin{itemize}
\item
The fractions of each type of vegetation are estimated from
the 15 main cover types (before climatic treatment). Small variations
are allowed for ecosystems coming from the same main cover type. For example,
some wooded grassland contains more bare soil than some other ones,
and then less trees or grass.

\item
The Leaf Area Index (LAI) is the surface of leaves per surface of ground
($m^2/m^2$). It describes the amount of {\bf green} vegetation over the
area. A monthly evolution of the LAI is necessary, to describe the annual
cycle of the vegetation.
For the default data files, LAI is deduced from measured NDVI profiles,
with consideration of the type of vegetation (there is less LAI over
wooded grassland than over forest, even if NDVI is close).
A LAI profile is defined for each representative ecosystem.

However, if a user
wants to use directly his (or her) LAI data, it is possible
to include it during the PREP\_PGD step, and this LAI will be used during the
MESONH run.

\item
The height of trees is defined for the representative ecosystems containing trees.
It is used to compute the roughness length of the vegetation.

\item
The root depth $d_2$ defines the area within the soil reached by the
vegetation roots. The transpiration of the vegetation takes water
in the soil by these roots.

\item
The total soil depth $d_3$ is not strictly a vegetation parameter.
This is an approximation to use the ecosystems to define this
depth, but it is legitimated by the fact that in the ISBA scheme, this is the
soil area which is in interaction with the root layer above (water
goes up by capillarity if transpiration occurs). Therefore, it depends
of the type of the vegetation above.
\end{itemize}




%%%%%%%%%%%%%%%%%%%%%%%%%%%%%%%%%%%%%%%%%%%%%%%%%%%%%%%%%%%%%%%%%%%%%%%%%%%%%%%%

\subsection{Other natural landscape parameters parameters}

\begin{table}[h]
\hspace*{0.cm}
\begin{tabular}{||l|c|c||}
\hline
\hline
field & notation &  unit \\
\hline
\hline
vegetation fraction (green veg. + dead biomass) & $veg$ & - \\
vegetation roughness length for momentum & $z_{0_{veg}}$ & $m$ \\
vegetation roughness length for heat & $z_{0h\, veg}$ & $m$ \\
emissivity of the ecosystem & $\epsilon_{eco}$ & - \\
near-infra-red albedo of the vegetation (only) &$\alpha_{{nir}_{veg}}$ &  - \\
visible albedo of the vegetation (only) &$\alpha_{{vis}_{veg}}$ &  - \\
minimum stomatal resistance &$r_{s_{min}}$ &  $sm^{-1}$ \\
coefficient for the calculation of the stomatal resistance &$\gamma$ &  - \\
maximum solar radiation usable in photosynthesis &$rgl$ & $Wm^{-2}$ \\
mesophyl conductance &$gm$ &  $ms^{-1}$ \\
biomass LAI ratio &$B/lai$ &  $kgm^{-2}$ \\
e-folding time for senescence &$e_{_{fold}}$&  days \\
$2\sqrt{\frac{\pi/\tau}{C_{veg}\lambda_{veg}}}$ &$C_v$& $Km^2J^{-1}$ \\
\hline
\hline
\end{tabular}
\caption{other ISBA parameters.
\label{paramISBA2}}
\end{table}

The other ISBA scheme parameters
are computed from the parameters described in previous section.
If a user wants to use its own data for some of these parameters,
it is still possible.

 Vegetation type, Leaf area index and height of vegetation (for trees)
 are used in these computations.\\
 Because the Lai is representative of the entire surface,
 it is not used directly, but is re-scaled on the potentially vegetated surface:\\
 $\hat{lai}$ is the leaf area index corresponding to the area
 occupied by trees, grassland, and C3 and C4 crops,
 excluding bare land, rocks and snow types:
 $\hat{lai} = \frac{lai}{f_{tree}+f_{grass}+f_{C3}+f_{C4}}$.
 Therefore, $\hat{lai}$ is greater than the area averaged $lai$
 ($lai$ is zero on bare surfaces).\\
 \medskip\\
 \underline{ vegetation fraction}\\
 \begin{tabular}{rll}
 veg = & $1-e^{-0.6 \hat{lai}}$ & for C3 cultures, C4 cultures\\
 veg = & 0.95 & for grassland\\
 veg = & $1-e^{-0.5 \hat{lai}}$ & for trees\\
 veg = & 0.   & for bare soil, snow and rocks
 \end{tabular}
 \smallskip\\
 When averaging is needed, it is performed arithmetically
 \medskip\\
 \underline{ roughness length for momentum}\\
 The height of the vegetation is computed as:\\
 \begin{tabular}{rll}
 $h_{veg}$ = & min $(1. , h_{allen})$ & for C3 cultures\\
 $h_{veg}$ = & min $(2.5, h_{allen})$ & for C4 cultures\\
 $h_{veg}$ = & $h$ & for trees\\
 $h_{veg}$ = & $\hat{lai}/6$ & for grassland\\
 $h_{veg}$ = & 0.01 m   & for bare soil and snow\\
 $h_{veg}$ = & 1.   m   & for rocks
 \end{tabular}
\smallskip\\
 where $h_{allen} = e^{(\hat{lai}-3.5)/1.3}$\\
 The roughness length is deduced: $z_{0_{veg}} = 0.13 h_{veg}$\\
 When averaging is needed, it is performed according to the
 $1/{\rm ln}^2(\frac{z_{0_{veg}}}{10})$ quantities.
 \medskip\\
 \underline{ roughness length for heat}\\
 $z_{0h\, veg}$ is equal to one  10 $^{th}$ of $z_{0\, veg}$\\
 \medskip\\
 \underline{ emissivity}\\
 Emissivity is equal to 0.99 on the vegetated part (veg),
 to 0.96 on bare soil and rocks, and to 1. on snow.
 Averaging is linear.
 \medskip\\
 \underline{ other vegetation parameters}\\
 Other vegetation parameters are computed from the vegetation types.
 The 'bare soil', 'rocks' and 'snow' vegetation types are not pertinent.
 When averaging is needed,
 it is performed linearly, except for the $C_v$ parameter, where it is harmonic.\\
 \medskip\\
 \begin{tabular}{||l||c|c|c|c|c|c|c|c|c||}
 \hline
 &$\alpha_{{nir}_{veg}}$&$\alpha_{{vis}_{veg}}$&$r_{s_{min}}$&$\gamma$&$rgl$&$gm$&$B/lai$&$e_{_{fold}}$&$C_v$
\\
 \hline
 trees      & .15 & .05 & 150 & .04 & 30 & .001 & .25 & 365. & 1. $10^{-5}$ \\
 C3 crops   & .30 & .10 &  40 & 0. & 100 & .003 & .06 &  60. & 2. $10^{-5}$ \\
 C4 crops   & .30 & .10 &  40 & 0. & 100 & .003 & .06 &  60. & 2. $10^{-5}$ \\
 grassland  & .30 & .10 &  40 & 0. & 100 & .020 & .36 &  90. & 2. $10^{-5}$ \\
 \hline
 \end{tabular}


%%%%%%%%%%%%%%%%%%%%%%%%%%%%%%%%%%%%%%%%%%%%%%%%%%%%%%%%%%%%%%%%%%%%%%%%%%%%%%%%
\newpage
\section{Soil characteristics}

\subsection{Composition of the soil}

\begin{table}[h]
\hspace*{4.cm}
\begin{tabular}{||l|c|c||}
\hline
\hline
field & notation &  unit \\
\hline
\hline
clay fraction & $f_{clay}$ &- \\
sand fraction & $f_{sand}$ &- \\
\hline
\hline
\end{tabular}
\caption{composition of the soil
\label{paramSOIL}}
\end{table}

The soil is supposed to be of uniform compostion over its total depth
(but it can vary horizontally).
It is defined by its sand and clay fractions. These come from the FAO
database.

\subsection{Color of the soil}

\begin{table}[h]
\hspace*{3.cm}
\begin{tabular}{||l|c|c||}
\hline
\hline
field & notation &  unit \\
\hline
\hline
near-infra-red albedo for dry bare soil & $\alpha_{{nir}_{dry}}$ &- \\
visible albedo for dry bare soil & $\alpha_{{vis}_{dry}}$ &- \\
near-infra-red albedo for wet bare soil & $\alpha_{{nir}_{wet}}$ &- \\
visible albedo for wet bare soil & $\alpha_{{vis}_{wet}}$ &- \\
\hline
\hline
\end{tabular}
\caption{composition of the soil
\label{paramCOLOR}}
\end{table}

At the time being, there is no soil color data incorporated in MESO-NH.
An estimation of the {\bf dry} bare soil albedo is computed
from the sand fraction, using linear formulae:\\

$
\begin{array}{ccc}
\alpha_{{vis}_{dry}}\; =\;& 0.05\; +\; 0.30 \; f_{sand}\\
&&\\
\alpha_{{nir}_{dry}}\; =\;& 0.15\; +\; 0.30 \; f_{sand}\\
&&\\
&&
\end{array}
$\\

If the soil is wet, it is darker.
This happens when soil water content reaches the field capacity
(capilarity and gravity are in equilibrium).
The albedos for wet soil are estimated as:\\

$
\begin{array}{ccc}
\alpha_{{vis}_{wet}}\; =\;&                        &  \;\frac{1}{2} \; \alpha_{{vis}_{dry}}\\
&&\\
\alpha_{{nir}_{wet}}\; =\;& \alpha_{{nir}_{dry}} \; - & \; \frac{1}{2} \; \alpha_{{vis}_{dry}}\\
&&\\
&&
\end{array}
$\\

%%%%%%%%%%%%%%%%%%%%%%%%%%%%%%%%%%%%%%%%%%%%%%%%%%%%%%%%%%%%%%%%%%%%%%%%%%%%%%%%
\newpage

\section{Orography}

\begin{table}[h]
\hspace*{3.cm}
\begin{tabular}{||l|c|c||}
\hline
\hline
field & notation &  unit \\
\hline
\hline
mean orography       & ${z_s}_{mean}$ & $m$ \\
envelop orography    & ${z_s}_{env} $ & $m$ \\
silhouette orography & ${z_s}_{sil} $ & $m$ \\
\hline
\hline
\end{tabular}
\caption{orography
\label{paramZS}}
\end{table}

Three types of orography can be computed by {\bf PREP\_PGD}: mean, envelop or
silhouette orography. Note that orography
is not computed on seas, where it is automatically set to zero.

\subsection{Mean orography}
This orography is simply the averaged of the subgrid-scale-orography in each
grid mesh.

\subsection{Envelope orography}
This envelope orography is computed as the mean orography in each grid mesh
${z_s}_{mean}$ plus a contribution from the smaller scales, proportional
to the subgrid scale orography standard deviation $\mu_{z_s}$. This reads
\begin{equation}
{z_s}_{env}={z_s}_{mean} + XENV * \mu_{z_s},
\end{equation}
where XENV is a parameter of the namelist. For XENV=0,
one obtains the mean orography.

\subsection{Silhouette orography}
The silhouette orography is computed as shown on figure \ref{silhouette},
from the maximum relief silhouette one could see in the grid mesh,
looking in x and y
directions. The 'resolution of the eye' is given by the parameter XSIL,
expressed in meters (also see book3):
\begin{itemize}
\item XSIL = grid mesh width :\\one obtains the maximum relief in the mesh.
\item XSIL = resolution of input data :\\one obtains the typical silhouette orography.
\item XSIL $\rightarrow$ 0 :\\one obtains the mean orography, since all the points will
be seen in the silhouettes\footnote{except if input data is exactly located following
x and y directions; it is the case in Mercator projection without rotation
with input data on a regular latitude-longitude grid. In this case, one obtains
again the typical silhouette orography}.
\end{itemize}
\begin{figure}
\hspace*{0.cm}
\psfig{figure=\EPSDIR/realcas.silhouette.eps,width=15.cm}
\caption{Computation of silhouette orography in one grid mesh\label{silhouette}}
\end{figure}

%%%%%%%%%%%%%%%%%%%%%%%%%%%%%%%%%%%%%%%%%%%%%%%%%%%%%%%%%%%%%%%%%%%%%%%%%%%%%%%%
\newpage

\section{Subgrid-scale orographic parameters}

\subsection{Roughness length for momentum}

\begin{table}[h]
\hspace*{0.cm}
\begin{tabular}{||l|c|c||}
\hline
\hline
field & notation &  unit \\
\hline
\hline
obstacle front area (A) & $\sum A_{i^+}/S$, $\sum A_{i^-}/S$ & -\\
over mesh surface (S) in each direction&$\sum A_{j^+}/S$, $\sum A_{j^-}/S$& \\
\hline
characteristic heights of obstacles encountered &
 $h_{i^+}$, $h_{i^-}$, $h_{j^+}$, $h_{j^-}$& $m$\\
in each direction &&\\
\hline
\hline
subgrid-scale-orography  dynamical roughness lengths &
${z_{0eff}}_{i^+}$, ${z_{0eff}}_{i^-}$,& $m$\\
in each direction &
${z_{0eff}}_{j^+}$,
${z_{0eff}}_{j^-}$&\\
\hline
subgrid-scale-orography dynamical roughness length & $z_{0rel}$& $m$\\
\hline
\hline
\end{tabular}
\caption{parameters for roughness lengths computations
\label{paramZ0}}
\end{table}

Roughness lenghts are computed along each model main axis ($x$ and $y$)
and in both directions, leading then to four roughness lengths.
They are computed according to Mason (1991), as in Georgelin {\it et al} (1994),
in each direction:

\begin{equation}
\frac{1}{ln^2\l \frac{h_{i^+}}{2{z_{0eff}}_{i^+}} \r} = \frac{0.5 C_D \sum
A_{i^+}}{S k^2}
+ \frac{1}{ln^2\l \frac{h_{i^+}}{2z_{0veg}} \r}
\end{equation}
where $C_D$ is a drag constant near unity, depending on obstacle shape
(equal to 0.8), and $k$ is the Von Karman constant, 0.4.

Note that the roughness lengths are computed both from subgrid-scale orographic
parameters and from vegetation roughness length for momentum.
The uni-directional roughness length $z_{0rel}$ is defined as the mean
of the four directional roughness lengths.


\subsection{Subgrid-scale orography structure}

\begin{table}[h]
\hspace*{2.cm}
\begin{tabular}{||l|c|c||}
\hline
\hline
field & notation &  unit \\
\hline
\hline
subgrid-scale-orography standard deviation &$\mu_{z_s}$&$m$\\
\hline
subgrid-scale-orography anisotropy&$\gamma_{z_s}$&-\\
\hline
direction of small axis of subgrid-scale-orography
&$\theta_{z_s}$&$^\circ$\\
 according to $x$ axis
&&\\
\hline
subgrid-scale-orography characteristic slope&$\sigma_{z_s}$&-\\
\hline
\hline
\end{tabular}
\caption{Subgrid-scale orography structure parameters
\label{paramSSO}}
\end{table}

The computations of these parameters is performed as
in Lott and Miller 1997. One uses the topographic correlation tensor,
defined according to the directions $i$ and $j$ of
the model domain axes on the conformal
plane (and not the input data axes, since data is not necessary
on a regular grid). As for the case of the silhouette orography,
a parameter is used to define the subgrid sampling
of the data, in order to compute the subgrid gradients (also see book3).
This allows to use any orographic data set directly in PREP\_PGD,
without any necessary prior computation.

The tensor is diagonalized to find the directions of the principal axes
and the degree of anisotropy. If

\begin{displaymath}
K=\frac{1}{2} \left\{ \overline{\left(\frac{\partial h}{\partial x}\right)^2}
+ \overline{\left(\frac{\partial h}{\partial y}\right)^2} \right\}
\hspace*{1.cm}
L = \frac{1}{2} \left\{ \overline{\left(\frac{\partial h}{\partial x}\right)^2}
- \overline{\left(\frac{\partial h}{\partial y}\right)^2} \right\}
\hspace*{1.cm}
M = \overline{\frac{\partial h}{\partial x}\frac{\partial h}{\partial y}}
\end{displaymath}

Then the angle of the main axis according to $i$ axis is:\\

$
\begin{array}{lcll}
\theta_{z_s} =& \frac{\pi}{4}                  &                 &\hspace*{2.cm} {\rm if} \; L=0\\
\theta_{z_s} =& \frac{1}{2} \; {\rm arctan} (M/L) &                 &\hspace*{2.cm} {\rm if} \; L>0\\
\theta_{z_s} =& \frac{1}{2} \; {\rm arctan} (M/L) & + \frac{\pi}{2 }&\hspace*{2.cm} {\rm if} \; L<0\\
&&&\\
&&&\\
\end{array}
$

The slope parameter is defined as:\\

$
\begin{array}{ll}
\sigma^2_{z_s} = K + \sqrt{L^2+M^2}&\\
&\\
&\\
\end{array}
$

The anisotropy is given by:\\

$
\begin{array}{lcl}
\gamma^2_{z_s} =& 1                                             & \hspace*{2.cm} {\rm if} \; \sigma_{z_s} = 0 \\
\gamma^2_{z_s} =& \frac{K - \sqrt{L^2+M^2}}{K + \sqrt{L^2+M^2}} & \hspace*{2.cm} {\rm if} \; \sigma_{z_s} > 0 \\
&&\\
&&\\
\end{array}
$

%%%%%%%%%%%%%%%%%%%%%%%%%%%%%%%%%%%%%%%%%%%%%%%%%%%%%%%%%%%%%%%%%%%%%%%%%%%%%%%%
\newpage
\section{Appendix: default parameters for the ecosystems}

%%%%%%%%%%%%%%%%%%%%%%%%%%%%%%%%%%%%%%%%%%%%%%%%%%%%%%%%%%%%%%%%%%%%%%%%%%%%%%%%

\newpage
\section{References}

\begin{description}

\item
Belward, A.S., ed., 1996, The IGBP-DIS global 1 km land cover data set
(DISCover)-proposal and implementation plans: IGBP-DIS  Working Paper No. 13,
Toulouse, France, 61 pp.

\item
Eidenshink, J.C. and Faundeen, J.L., 1994:
The 1 km AVHRR global land data
set-first stages in implementation. {\it International Journal
of Remote Sensing}, {\bf 15}, {\bf no. 17}, part. 3, 443-3,462.

\item
Georgelin, M. and E. Richard and M. Petitdidier and A. Druilhet, 1994:
Impact of subgrid scale orography parameterization on the
simulation of orographic flows. {\it Monthly Weather Review}, {\bf 122},
1509-1522.

\item
Hansen, M.C., DeFries, R.S., Townshend, J.R.G., and Sohlberg, R., 2000:
Global land  cover classification at 1km spatial resolution using
a classification tree  approach. {\it International Journal of Remote
Sensing}, (in press).

\item
Koeppe, C. E. and De Long G. C., 1998:
Weather and Climate. Mc Graw-Hill book company.

\item
Lott, F. and M. J. Miller, 1997:
A new subgrid-scale orographic drag parameterization: its formulation
and testing. {\it Q.J.R.M.S.}, {\bf 123}, 101-127.

\item
Mason, P. J., 1991:
Boundary-layer parameterization in heterogeneous terrain.
{\it Workshop Proc. on Fine Scale Modelling and the
Development of Parameterizations Schemes, ECMWF},
275-288.

\item
Masson, 2000:
A physically-based scheme for the urban energy budget in atmospheric models.
{\it Boundary Layer Meteorology}, in press.

\end{description}
%%%%%%%%%%%%%%%%%%%%%%%%%%%%%%%%%%%%%%%%%%%%%%%%%%%%%%%%%%%%%%%%%%%%%%%%%%%%%%
%%%%%%%%%%%%%%%%%%%%%  END OF "Physiographic Data" CHAPTER    %%%%%%%%%%%%%%%%
%%%%%%%%%%%%%%%%%%%%%%%%%%%%%%%%%%%%%%%%%%%%%%%%%%%%%%%%%%%%%%%%%%%%%%%%%%%%%%

