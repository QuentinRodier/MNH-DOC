%%%%%%%%%%%%%%%%%%%%%%%%%%%%%%%%%%%%%%%%%%%%%%%%%%%%%%%%%%%%%%%%%%%%%%%%%%%%%
% CONTRIBUTION TO THE MESONH BOOK1: "GRID NESTING"
% Author : P. Lafore
% Original : June, 1995
% Last Update   : March, 2000
% March 2008, J.-P. Chaboureau, editorial corrections
%%%%%%%%%%%%%%%%%%%%%%%%%%%%%%%%%%%%%%%%%%%%%%%%%%%%%%%%%%%%%%%%%%%%%%%%%%%%%

\chapter{Grid Nesting}
\minitoc

\section{Principles}

 There are many problems in atmospheric fluid dynamic in which higher spatial
resolution is required in only a limited portion of the computation domain.
 The two-way interactive grid nesting of Clark and Farley (1984)
is implemented in the Meso-NH code with some modifications
to allow one to focus on such desired regions and obtain higher
spatial resolution and greater computational efficiency (Stein et al. 2000).

 The main principle is to execute in parallel several simulations on a nested
grid whose a possible structure is illustrated in Fig. \ref{nesting-example}.
The only restriction concerns the temporal and spatial resolution ratio between
models, that should be an integer. In the present example we have 4 nested models,
with spatial resolution ratios of 2 and 4 between model 1 and model 2, and
between model 2 and model 3 or 4, respectively.


\begin{figure}[!ht]
\centerline{\includegraphics[]{\EPSDIR/nest-ex.eps}}
\caption{An example of nested grid-mesh horizontal and temporal structure.}
\label{nesting-example}
\end{figure}

The interactions between each fine mesh (FM) model
and its coarse mesh (CM) model are treated through either
one- or two-way interactive nesting.

%${\cal M}_{CM}$ ${\cal M}_{FM}$

\begin{itemize}
\item
In the simpler one-way approach, only CM waves are allowed to enter and affect
the FM model. It is simply performed by the use of interpolation operators
{\bf I} to derived boundary conditions of the FM model from its CM model.
\item
 In the two-way approach, waves resolved by the FM model can also affect
the CM model solution. It is done by operators {\bf S} averaging
FM data down to the CM resolution.
\end{itemize}


\section{Basic choices}

The following choices have been made:

\begin{itemize}
\item The nesting is restricted to the horizontal directions. Thus all models
have the same vertical resolution and level numbering.
\item The resolution ratio in the x- and y-direction can be different.
\item Differing temporal resolutions are allowed (Fig. \ref{nesting-example}).
\item A same CM model can monitor several FM models as illustrated by
Fig. \ref{nesting-example}, where model 2 monitors models 3 and 4.
\item No relative translation speed is allowed between CM and FM models.
\item All models have the same x- and y-axis orientation, and work therefore
in the same geographic projection system.
\item As developed later, the FM topography averaged at the CM scale must be
equal to the one of the CM model.
\item Different levels of coupling are possible between models:
\subitem {\bf level 2} for the two-way interaction of all variables,
\subitem {\bf level 1} for the one-way interaction,
\subitem {\bf level 0} for the no-interaction case (i.e. model is independent
from the others).
\end{itemize}

\section{Averaging and interpolation procedures}

The original grid-nesting method proposed by Clark and Farley (1984)
have been implemented with two major modifications concerning the
averaging and interpolation procedures. First we preferred to inject the FM
model results into the CM model through a relaxation term instead of a
simple substitution. It allows to strictly respect the anelastic constraint
independently of the averaging operator, and to get a fully consistent pressure
field. Second the Clark and Farley interpolation is replaced by an
algebraic interpolator, to avoid discontinuities from one grid box to the next
one. Contrary to results of Clark and Farley, our validation tests showed that
the reversibility condition of Kurihara et al. (1979) between averaging and
interpolation operators was not strictly necessary in our case.

\subsection{Interpolation operators from coarse to fine meshes}

\subsubsection{The Bikhardt spatial interpolation}

 This algebraic interpolator (cited by de Floriani and Dettori 1981)
uses 16 points.
The interpolation formula at a point of coordinate (a,b) can be
written as:

\begin{equation}
\label{Bik0}
g(a,b) \, =  \, \sum_{i=1}^4  \, \sum_{j=1}^4  \,
f_{ij} \,B_{i}(a) \,B_{j}(b)
\end{equation}

\noindent where $f_{ij}$ corresponds to values at points $(i,j)$
(Fig. 2) and the $B_{i}$ are the following third order polynomials:

\vspace{-.3cm}
\begin{eqnarray}
B_{1} = &-0.5 t^3 & + \,\,t^2 -0.5t   \nonumber \\
B_{2} = &+1.5 t^3 & - 2.5 t^2 +1 \\
B_{3} = &-1.5 t^3 & + \,2 t^2 +0.5t   \nonumber \\
B_{4} = &         & + 0.5 t^2 (t-1)   \nonumber
\end{eqnarray}

\noindent obeying the properties $B_{i}(0)=0$ except $B_{2}(0)=1$
and $B_{i}(1)=0$ except $B_{2}(1)=1$, so that the interpolation surface
passes exactly through the 4 most inner points (2,2), (2,3), (3,2) and (3,3).

 This Bikhardt operator is used to horizontally interpolate on
$\overline{z}$ surface the CM models variables and $LS$ fields to obtain the
corresponding $LB$ and $LS$ fields of the FM model. Finally a linear vertical
interpolation is performed on the $LB$ and $LS$ fields to account for the
topography  differences between CM and FM models.

\begin{figure}[!ht]
\centerline{\includegraphics[width=8cm]{\EPSDIR/Bikhardt.eps}}
\caption{Configuration of points involved in the Bikhardt interpolation.
The interpolation is valid in the shaded area.}
\end{figure}

\subsubsection{Temporal interpolation}

In case of differing time steps, the previous spatial interpolating is
performed only when the two models are synchronized in time. In between
a linear temporal interpolation is used to get $LB$ and $LS$ fields at the
current time.
In practice it is done by storing the $LB$ and $LS$ tendencies to allow its
forward temporal integration.

\subsubsection{Lateral boundaries conditions and other treatments}

The previous interpolations allow to get the $LB$ fields necessary to
treat lateral boundaries as described in Chapter \ref{LateralBoundCond}.
As for a single model
run different types of l.b.c. formulations (cyclic, rigid wall or open) can
be used. For instance for open l.b.c. normal velocities are treated owing
to the use of a modified Sommerfeld equation already presented
in Chapter \ref{LateralBoundCond}.

\begin{equation}
\label{carpenter2}
{\dr u_n \over \dr t} =             {\l{{\dr u_n \over \dr t}}\r}_{LB}
- C^*{\l  {\dr u_n \over \dr x}  -  {\l{{\dr u_n \over \dr x}}\r}_{LB}  \r}
- K \, \, {\l u_n - {u_n}_{LB}   \r}
\end{equation}

Also a lateral sponge zone may be applied to smoothly relax FM
fields towards CM corresponding fields stored in the $LB$ fields.

The $LS$ fields previously prepared are eventually used for the numerical
diffusion and the top absorbing layer (Chapter \ref{NumericalDiffusion}).

\subsection{Averaging operators}

Define $\Psi$ to be a CM variable and $\psi$ to be the equivalent
variable in the FM. We define the following top hat averaging operators
$S$, $S_u$ and $S_v$ for variables located at scalar, u and v points
respectively:

\begin{equation}
\label{averaging-op}
 S ( \psi) \, = \, \dfrac{
 \sum_{i=1}^{X_{ratio}} \sum_{j=1}^{Y_{ratio}} \,
\left( \tilde{\rho}_{FM} \psi \right)_{i,j}          }
{
 \sum_{i=1}^{X_{ratio}} \sum_{j=1}^{Y_{ratio}} \,
\left( \tilde{\rho}_{FM} \right)_{i,j}               }
\end{equation}


\begin{equation}
\label{averaging-opu}
 S_u ( u) \, = \, \dfrac{ \,
\sum_{j=1}^{Y_{ratio}} \, \left( \overline{\tilde{\rho}_{FM}}^{x} u \right)_{j}}
{ \,
\sum_{j=1}^{Y_{ratio}} \, \left( \overline{\tilde{\rho}_{FM}}^{x}   \right)_{j}}
\end{equation}

\begin{equation}
\label{averaging-opv}
 S_v ( v) \, = \, \dfrac{ \,
\sum_{i=1}^{X_{ratio}} \, \left( \overline{\tilde{\rho}_{FM}}^{y} v \right)_{i}}
{ \,
\sum_{i=1}^{X_{ratio}} \, \left( \overline{\tilde{\rho}_{FM}}^{y}   \right)_{i}}
\end{equation}

\noindent
where $X_{ratio}=\dfrac{\Delta \overline{x}}{\Delta \overline{X}}$
  and $Y_{ratio}=\dfrac{\Delta \overline{y}}{\Delta \overline{Y}}$ are the
resolution ratio on the conformal plan in the x- and y-direction respectively.

 The variables $\Psi$ of the CM model are relaxed towards the averaged
$S( \psi)$ values obtained from the FM model in the entire overlapping domain.
The corresponding additional source for any prognostic variable $\Psi$ is


\begin{equation}
\label{2w_relaxation}
S_{2W} = -K_{2W} \big[ \Psi^{t-\Delta T} - S ( \psi^{t-\Delta T}) \big]
\end{equation}

\noindent
where subscript $2W$ stands for two-way interactive grid nesting and $K_{2W}$
is the relaxation coefficient set by default to $1/(4\Delta T)$ with $\Delta T$
being the CM model time step.
The mass continuity remains satisfied owing to the pressure solver passage
although the averaging operators
(\ref{averaging-op}, \ref{averaging-opu} and \ref{averaging-opv}) do not
strictly guarantee this property.


\section{Implementation of the nesting}

In a first stage the grid-nesting technics has been implemented on a Cray 90,
owing to the use of the Cray macrotasking facility to execute each model as a
specific task. Nevertheless this approach appears to have several disadvantages;
first  macrotasking is specific to Cray computers, second the efficiency of
the nesting parallelization is weak, as each task is dependent from the others,
third the model synchronization is quite delicate to code.

We finally preferred a "sequential" approach, consisting in running at the same time
only one model for only one time step. The coarser resolution models are run first.
In case of multiprocessor run, each model time integration is distributed and
performed on each processor.


\section{Preparation of a nested simulation (spawning)}

 In order to conduct a grid-nesting simulation, each FM model must be first
initialized by its CM model.
This operation is performed by a program named SPAWNING,
designed to generate initial fields by horizontal interpolation
of fields provided by a given Meso-NH file.
The input CM model for this interpolation task, will be named model 1,
whereas the model 2 will refer to the one to be spawned (CM model).
The main possibilities and limitations of the SPAWNING program are:

\begin{itemize}
\item The model 2 horizontal domain must be included in the model 1 domain.
\item The resolution ratio between model 2 and 1 must be an integer, and
can be different in the $\bar x$ and $\bar y$ directions.
\item As regards other
characteristics, the two domains remain identical, such as vertical grid,
geometry or orientation.
\end{itemize}

\noindent {\bf NB:} A different program, part of the PREP\_REAL\_CASE facilities
(see the {\it "Initial fields for real case"} Chapter), can be used
on the FM-File generated by the spawning, to change the vertical grid and
modify the surface fields, including finer scale information.

\subsection{Horizontal grid configuration}

 Figure \ref{grid-conf} shows the horizontal grid structure of the models 1 and
2. The physical boundary of the two simulation domains (thick line on
Fig. \ref{grid-conf}) corresponds to localization of the velocity component
normal to the boundaries.
 The  model 2 horizontal grid is defined by 3 input parameters:

%\newpage
\begin{enumerate}
\item the resolution ratios between models in the x and y-directions
(DXRATIO and DYRATIO respectively) which must be integer,
\item the position of model 2 (XOR,YOR) relative to model 1
(see Fig. \ref{grid-conf})
\item and the number of points in each horizontal direction.
\end{enumerate}

\noindent The only constraint is that the number of points must be a multiple of the
resolution ratio for each horizontal direction.

\begin{figure}
\centerline{\includegraphics[width=13.5cm]{\EPSDIR/horzgrid2.eps}}
\caption{Horizontal structure of the grids for a resolution ratio of 2
between models 1 and 2.}
\label{grid-conf}
\end{figure}

\subsection{Interpolation operators from fine to coarse meshes}

 The basic task of the spawning program is to perform horizontal interpolations
(at constant $\bar{z}$ level). Two different interpolation operators are used:

\begin{enumerate}
\item a simple linear interpolator, mainly to interpolate the
$\widehat{x}$ and $\widehat{z}$ coordinates,
\item and the Bikhardt interpolation previously described for most variables.
\end{enumerate}

\subsection{Specific treatments for spawning}

\subsubsection{Grid structure and configuration}

 After initialization of model 1 from the corresponding FM-File,
the program first initializes the model 2 configuration and spatio-temporal
grid structure. The model configurations are identical except for the
radiation which is not treated by the spawning, and the lateral boundary
conditions which are deduced from the model 2 domain position.
Concerning grid structure, the topography $z_s$ is interpolated by the Bikhardt
operator, whereas the coordinates are linearly interpolated.
Some corrections are applied to $z_s$ depending on the land-sea mask and on
the model 1 topography, to be consistent, to avoid non-zero altitude at
sea level, and to keep negative altitudes over land where it is the case at
the model 1 scale.

\subsubsection{3D fields}

Each of the 3D prognostic or diagnostic fields of model 1 are interpolated
independently, with correction of negative values for variables which must be,
by definition, positive. Finally the pressure solver is used to enforce the
anelastic constraint, accounting for new surface conditions due
to the interpolated topography.

\subsubsection{Surface fields}

 The interpolation of surface variables is more complicated, mainly due
to the existence of non-physical values (flagged), soil characteristic over the
sea for instance, and of discontinuities (at the sea coast).
The method consists in first replacing the model 1 flagged values by
physical values of the nearest point, then the interpolation is
performed, and finally the flagged values are restored. Several corrections
are also necessary to avoid overshoots, and to enforce certain constraints
(consistency between the sea-land-lake masks or the clay-sand-silt indexes).
Using such methods it is easy to perform a simulation at finer scale keeping
the same surface characteristics (except the interpolation). If
available, finer scale characteristics can be introduced with the use of
the PREP\_REAL\_CASE facilities.

\section{References}
\decrefname
Clark, T. L., and R. D. Farley, 1984:
Severe downslope windstorm calculations in two and three
spatial dimensions using anelastic interactive grid nesting: a possible
mechanism for gustiness. {\it J. Atmos. Sci.},  {\bf 41}, 329-350.
\decrefname
Chen C. 1991:
A nested grid, non-hydrostatic, elastic model using a terrain-following
coordinate transformation: the radiative-nesting boundary conditions.
{\it Mon. Wea. Rev.},  {\bf 119}, 2852-2869.
\decrefname
De Floriani, A. and G. Dettori, 1981:
An interpolation method for surfaces with tension.
{\it Adv. Eng. Software},  {\bf  3}, 151-154.
\decrefname
Kurihara, Y., G. J. Tripoli and M. A. Bender, 1979:
Design of a movable nested-mesh primitive equation model.
{\it Mon. Wea. Rev.},  {\bf 107}, 239-249.
\decrefname
Stein J., E. Richard, J.-P. Lafore, J.-P. Pinty, N. Asencio
and S. Cosma, 2000:
High-resolution non-hydrostatic simulations of flash-flood
episodes with grid-nesting and ice-phase parametrization.
{\it Meteorol. Atmos. Phys.},  {\bf 72}, 101-110
%\end{document}

