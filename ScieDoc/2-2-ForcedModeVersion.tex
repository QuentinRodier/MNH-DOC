%%%%%%%%%%%%%%%%%%%%%%%%%%%%%%%%%%%%%%%%%%%%%%%%%%%%%%%%%%%%%%%%%%%%%%%%%%%%%%
% CONTRIBUTION TO THE MESONH BOOK1: "FORCED MODE VERSION"
% Author : M. Georgelin - K. Suhre - J. P. Pinty
% Original : October 8, 1997
% Update   : October 9, 1997
%%%%%%%%%%%%%%%%%%%%%%%%%%%%%%%%%%%%%%%%%%%%%%%%%%%%%%%%%%%%%%%%%%%%%%%%%%%%%%
%
% DEFINITIONS:
%
\def\souligne#1{$\underline{\smash{\hbox{#1}}}$}
%
%\begin{document}
%\begin{titlepage}
%\thispagestyle{empty}
%\title{\huge \bf FORCED MODE VERSION}
%\author{ Marc GEORGELIN \cr Revised by K. Suhre and J.-P. Pinty}
%\date{\today}
%\maketitle
%\vfill
%\hfill {\bf \LARGE BOOK 1}
%\end{titlepage}

\chapter{Forced Mode Version}
\minitoc

\section{Purpose}
%\section{Equations}
For some academic studies (1D PBL dynamics, 2D Eady wave problem, 2D mountain
flow, moving squall lines, 3D breeze studies...), it is necessary to impose a
large scale environment even crudely (geostrophic wind, subsidence, ... ), in
order to perform consistent simulations with Meso-NH. Practically, this means
that some additional forcing terms have to be included in the Meso-NH system of
equation.

Actually, only the most usual physical forcing terms have been taken into
account for simplicity. The forcing terms are prescribed empirically from
external sources of information (e.g. series of observation) and are
mutually independent when acting on the same prognostic variable. Generally
they are non-stationary and horizontally homogeneous, only the case of
flat terrain is addressed. Furthermore, forcing terms are considered for
the restricted Cartesian geometry of Meso-NH (constant Coriolis parameter).

The effects of several forcings are considered:

\begin{tabular}{l c}
                                               &\\
$\bullet$ Domain translation at a constant speed: & \\
(affects the surface fluxes and the Coriolis force) & \\
                                               & \\
                                               &\\
$\bullet$ Geostrophic wind (or pressure gradient) forcing:
\ & $\dfrac{\partial u}{\partial t}\Big|_{FRC}=+f v_{frc}$ \\
\ & $\dfrac{\partial v}{\partial t}\Big|_{FRC}=-f u_{frc}$  \\
                                               & \\
                                               &\\
\end{tabular}

\begin{tabular}[t]{l c}
                                               &\\
$\bullet$ Forced ascent or descent (vertical transport) of $\alpha$:
 & $\dfrac{\partial \alpha}{\partial t}\Big|_{FRC}=-w_{frc}\dfrac{\partial \alpha}{\partial z}$ \\
applied to \souligne{all} prognostic variables ($w$ excepted in the 1D case)
                                               &\\
                                               & \\
$\bullet$ Large scale horizontal transport of $\alpha \in [\theta, r_{v} ]$:
   & $\dfrac{\partial u}{\partial t}\Big|_{FRC}=-u (\dfrac{\partial \alpha}{\partial x})_{frc}$ \\
   & $\dfrac{\partial v}{\partial t}\Big|_{FRC}=-v (\dfrac{\partial \alpha}{\partial y})_{frc}$ \\
                                               &\\
                                               & \\
$\bullet$ Large scale variables $\alpha \in [u, v, \theta, r_{v} ]$:
   & $\dfrac{\partial \alpha}{\partial t}\Big|_{FRC}=- \dfrac{\alpha - \alpha_{frc}}{\tau_{frc}}$ \\
for a Newtonian relaxation with damping time $\tau_{frc}$
                                               &\\
                                               &\\
\end{tabular}

In the above definitions, the subscript $_{frc}$ tags the forcing fields or scalars
which must be provided by the user for a specific application. The forcing terms
are written in the form of a tendency of the relevant prognostic variable.

\section{Modified equations}
The inclusion of the forcing terms (single underbraced for the physical
terms and double underbraced for the Newtonian relaxation terms) in the
Meso-NH system of equation is made as follows:

\bigskip
\noindent {\bf Conservation of momentum}
\begin{eqnarray}
{\partial \over \partial t} (\rho_{dref} \vec U) & +
\vec \nabla \cdot (\rho_{dref} \vec U\otimes \vec U) + \rho_{dref} \vec \nabla \Phi + \rho_{dref} \vec g \dfrac{\theta_v - \theta_{vref}}{\theta_{vref}}+
2\rho_{dref} \vec \Omega \wedge (\vec U - \underbrace{\vec U_{frc}})
\nonumber \\
 &+ \underbrace{w_{frc} \dfrac{\partial u}{\partial z} \vec i}
  + \underbrace{w_{frc} \dfrac{\partial v}{\partial z} \vec j}
  = \rho_{dref} \vec {\cal F}
- \underbrace{\underbrace{\dfrac{u - u_{frc}}{\tau_{frc}} \vec i}}
- \underbrace{\underbrace{\dfrac{v - v_{frc}}{\tau_{frc}} \vec j}}
\end{eqnarray}

\noindent {\bf Thermodynamic equation}
\begin{eqnarray}\label{eqfrc1}
&\dfrac{\partial }{\partial t} (\rho_{dref} \theta)
+ \vec \nabla \cdot  (\rho_{dref} \theta \vec U)+ \rho_{dref} \Big(
  \underbrace{u (\dfrac{\partial \theta }{ \partial x})_{frc}}
+ \underbrace{v (\dfrac{\partial \theta }{ \partial y})_{frc}}
+ \underbrace{w_{frc} \dfrac{\partial \theta }{\partial z}}
\Big) =  \rho_{dref} \Big[ \dfrac{R_d + r_v R_v C_{pd}} {R_d    C_{ph}}
\nonumber \\
& -1 \Big]
 \dfrac{
\theta} {\Pi_{ref}} w \dfrac{\partial \Pi_{ref}} {\partial z}
 + \dfrac{\rho_{dref}} {\Pi_{ref} C_{ph}} \Big[ L_m
\dfrac{D(r_i + r_s + r_g + r_h)} {Dt} - L_v \dfrac{Dr_v} {Dt} + \cal H \Big]
- \underbrace{\underbrace{\dfrac{\theta - \theta_{frc}} {\tau_{frc}}}}
\end{eqnarray}

\noindent {\bf Conservation of the water vapor mixing ratio}
\begin{equation}
{\partial \over \partial t}(\rho_{dref} r_{v}) + \vec \nabla \cdot
(\rho_{dref} r_{v} \vec U)+ \rho_{dref} \Big(
  \underbrace{u (\dfrac{\partial r_{v} }{ \partial x})_{frc}}
+ \underbrace{v (\dfrac{\partial r_{v} }{ \partial y})_{frc}}
+ \underbrace{w_{frc} \dfrac{\partial r_{v} }{\partial z}}
\Big) = \rho_{dref} Q_{v}
- \underbrace{\underbrace{\dfrac{r_{v}-r_{v\ frc}}{\tau_{frc}}}}
\end{equation}

\noindent {\bf Conservation of the condensed water mixing ratios}
\begin{equation}
{\partial \over \partial t}(\rho_{dref} r_\star) + \vec \nabla \cdot
(\rho_{dref} r_\star \vec U)+ \rho_{dref}
\underbrace{w_{frc} \dfrac{\partial r_\star }{\partial z}} = \rho_{dref} Q_\star
\end{equation}

\bigskip
These forced terms are not independent from each other, the horizontal
gradient of $\theta$ (baroclinicity) is related to the geostrophic wind by
the "thermal wind" equation which takes the following form:

\begin{equation}
\dfrac{\partial u_{frc}}{\partial z} = + {g \over f \theta_{vref}} (\dfrac{\partial \theta_v}
{\partial y} )_{frc}
\end{equation}
\begin{equation}
\dfrac{\partial v_{frc}}{\partial z} = - {g \over f \theta_{vref}} (\dfrac{\partial \theta_v}
{\partial x} )_{frc}
\end{equation}
where $u_{frc}$ and $v_{frc}$ are the horizontal components of the geostrophic
wind and $f = 2 \vec \Omega \cdot \vec k$, the (constant) Coriolis parameter.

The forced temperature gradients are deduced from these relations but beware
that condensation modifies gradients of $\theta$ or $r_{v}$ and so forced
gradients of $\theta$ or forced advection of humidity might be excessive in
some circumstances.

The relaxation towards large scale variables is useful to constraint a free
atmosphere state to match series of observation above a specified height level.

\section{Interpolation}
The series of asynchronous 1D forcing data are provided by the Meso-NH user
in a special format (see the {\bf Forced version} described in the
{\bf Meso-NH user's guide}).
Then they are linearly interpolated to fill the Meso-NH vertical grid system
and stored in the LFI files.
In the case of orography, these fields are further projected on the horizontal
embedded grids of Meso-NH by linear interpolation.

The time-evolution of the forcings is obtained by linear interpolation between
two bounding forcing fields at each time step. When the last forcing data is
reached a stationary forcing is kept on. Conversely, no forcing is applied
until the current time of the Meso-NH integration has reached that of the first
forcing data. The following scheme summarizes how a set of 3 forcing data are
used during the model integration starting at time DTEXP.

\vskip 1.0cm
\begin{center}
\begin{tabular}{c}
%\setlength{\epsfxsize}{14cm}
%\epsfbox{eps/scheme_FRC.eps}
%\psfig{figure=\EPSDIR/scheme_FRC.eps,width=14cm}
\psfig{figure=\EPSDIR/scheme_FRC.pdf,width=14cm}
\end{tabular}
\end{center}
\vskip 1.0cm


\section{Discretization}
The vertical transport $w_{frc}(\dfrac{\partial \alpha}{\partial z})$ is lagged
and integrated with a first order upstream scheme to ensure positivity in some
situation, e.g. the case of large scale subsidence over a shallow
inversion in the stratus cloud modeling problem. The numerical scheme of the
large scale horizontal transport is time centered and the horizontal forcing
gradients
$(\dfrac{\partial \alpha}{\partial x})_{frc}$ and
$(\dfrac{\partial \alpha}{\partial y})_{frc}$ are assumed to be given on the
$\alpha$ (mass) grid.
The integration of the large scale relaxation terms is lagged for numerical
stability. This term applies either above a constant selected height or above
a time evolving thermal inversion (useful for PBL dynamics studies).
%\vfill
%\eject

%%%%%%%%%%%%%%%%%%%%%%%%%%%%%%%%%%%%%%%%%%%%%%%%%%%%%%%%%%%%%%%%%%%%%%%%%%%
%%%%%%%%%%%%%%%%%% END OF FORCED MODE VERSION SECTION %%%%%%%%%%%%%%%%%%%%%
%%%%%%%%%%%%%%%%%%%%%%%%%%%%%%%%%%%%%%%%%%%%%%%%%%%%%%%%%%%%%%%%%%%%%%%%%%%
