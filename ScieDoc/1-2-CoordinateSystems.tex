\chapter{Coordinate Systems}
\minitoc

%{\em by V. Ducrocq and P. Bougeault}

In general, it will not be possible to develop the equations of the previous
chapter in a simple cartesian coordinate system, because
of the earth sphericity, and of the underlying topography.
In meteorology, a natural coordinate system is defined by the longitude
$\lambda$, the latitude $\varphi$, and the distance from the earth center $r$
(or the altitude above sea surface $z=r-a$, where $a$ is the earth radius).
The vector basis associated with this natural system will be called
$(\vec{i}_0, \vec{j}_0,\vec{k})$. $\vec{i}_0$ points towards the east,
$\vec{j}_0$ towards the north, and $\vec{k}$ is vertical.
In Meso-NH, we prefer to work with a conformal
projection allowing for rotation with respect to this natural basis.
This allows more flexibility in the direction of the coordinate lines
to study particular processes. This also simplifies the initialization
of model runs with products of operational weather prediction systems,
like the Arpege system of Meteo-France, or the ECMWF system.

\section{Conformal Projections and Terrain Following Coordinates}

We use a system of curvilinear coordinates $\widehat{x},\widehat{y},
\widehat{z}$, defined in the following manner.

\subsection{Gal-Chen and Sommerville vertical coordinate}

$\widehat{z}$ is
the Gal-Chen and Sommerville (1975) vertical coordinate:

\begin{equation}
\widehat{z} =H  \dfrac{z - z_{s}}{H - z_{s} },
\end{equation}
where $H$ is the height of the model top, and  $z_{s}$ the height of the
local topography. Since the transformation is purely vertical, $\vec{k}$
is also parallel to the $\widehat{z}$ lines in the physical space
(see Fig.~\ref{galchen}).

\begin{figure}[hpbt]
\psfig{figure=\EPSDIR/fig3.1.eps}
\caption{The Gal-Chen and Sommerville vertical coordinate and the cartesian
($\vec{i},\vec{j}$), covariant ($\vec{e}_i$) and contravariant ($\vec{e}^i$)
bases \label{galchen}}
\end{figure}

On the other hand, $\widehat{x}$ and $\widehat{y}$
are the distances counted from an arbitrary origin in two arbitrary orthogonal
directions on a conformal surface of projection (Fig.~\ref{projection}).
The traces
on the sphere of these coordinate lines define two orthogonal directions
in each point.  We will call hereafter $\vec{i}$ and $\vec{j}$ the horizontal,
unit length vectors parallel to those directions.
We note $\gamma$ the angle between from $\vec{i}$ to
$\vec{i}_0$. In general, this angle will vary with $\widehat{x}$ and
$\widehat{y}$ because of the earth sphericity.

\begin{figure}[hpbt]
\psfig{figure=\EPSDIR/fig3.2.eps}
\caption{Principles of Projections and notations on the sphere
\label{projection}}
\end{figure}

Clearly, $\vec{i},\vec{j}$ and $\vec{k}$ form a local, cartesian basis
which is particularly interesting to develop the wind
velocity vector. In the following, we will call $ u ,  v , w$ the
components of ${\vec U}$ on this basis.
\begin{equation}
\vec{U}=u\vec{i}+v\vec{j}+w\vec{k}
\end{equation}
It is the most natural decomposition of the wind in a local basis.

Three types of conformal projections are supported in Meso-NH: Lambert, Polar
Stereographic, and Mercator. We recall here the formulae which allow to
compute the coordinates $\widehat{x},\widehat{y}$ from the latitude $\varphi$
and the longitude $\lambda$ of a given point, and inversely.

\subsection{Polar stereographic and Lambert Projections:}

The projection is defined by the conicity parameter $K$ ($0<K<1$), the
earth radius $a$,
a reference latitude $\phi_0$, a reference
longitude $\lambda_0$, an arbitrary angle of rotation $\beta$,
and the coordinates of the pole in the projection
$\widehat{x}_0, \widehat{y}_0$ (Fig.~\ref{polarstereo}).
The useful formulae are
\begin{eqnarray}
\gamma & = & K (\lambda - \lambda _{0}) - \beta \nonumber \\
R &  = & \dfrac{a}{K} (\cos\varphi _{0})^{1-K} (1 + \sin\varphi _{0})^{K}
\left(\dfrac{\cos\varphi }{1 + \sin\varphi}\right)^{K} \nonumber \\
\widehat{x}& = & \widehat{x}_0 + R sin\gamma \\
\widehat{y}& = & \widehat{y}_0 - R cos \gamma \nonumber
\end{eqnarray}

The polar stereographic projection corresponds to $K=1$. The more general case
of the Lambert conical projection is obtained for $0<K<1$.

The map scale factor, defined as the ratio of distances on the projection
surface to distances on the sphere, is given by

\begin{equation}
m=\left( \dfrac{\cos \varphi _{0}}{\cos \varphi}\right)^{1-K}
\left( \dfrac{1 + \sin\varphi _{0}}{1 + \sin\varphi} \right)^{K}
\end{equation}

\begin{figure}[hpbt]
\psfig{figure=\EPSDIR/fig3.3.eps}
\caption{Polar Sterographic and Lambert Projections: $M_0$ is the point
at the intersection of the reference latitude and longitude. In $M_0$, the
angle from $\vec{i}$ to $\vec{i}_0$ is $-\beta$. In any other point $M$,
this angle has a different value. Note that the North pole may be situated
inside the model domain in the case of the polar stereographic projection,
but not in the case of the Lambert projection \label{polarstereo}}
\end{figure}

\subsection{Mercator Projection:}

The projection is defined by a reference latitude $\phi_0$, a reference
longitude $\lambda_0$, an arbitrary angle of rotation $\beta$,
and the coordinates of the origin in the projection
$\widehat{x}_0, \widehat{y}_0$ (Fig.~\ref{mercator}). The useful formulae are

\begin{eqnarray}
\nonumber \\ \gamma & = & - \beta \nonumber \\
\widehat{x}' & = & \widehat{x}_0
+ a \cos \varphi _{0}(\lambda - \lambda _{0})  \nonumber \\
\widehat{y}' & = & \widehat{y}_0
 -a\cos\varphi _{0} \ln|\tan(\dfrac{\pi}{4}-\dfrac{\varphi}{2})|  \\
\widehat{x}  & = & \widehat{x}'\cos\gamma - \widehat{y}'\sin\gamma
\nonumber \\
\widehat{y}  & = & \widehat{x}'\sin\gamma + \widehat{y}'\cos\gamma
\nonumber
\end{eqnarray}

For this projection, if $\beta$ is chosen equal to zero, the local
cartesian basis coincides everywhere with the natural basis.

The map scale factor is given by

\begin{equation}
m = \dfrac{\cos\varphi _{0} }{\cos\varphi }
\end{equation}

\begin{figure}[hpbt]
\psfig{figure=\EPSDIR/fig3.4.eps}
\caption{Mercator Projection: the angle between $\vec{i}$ and $\vec{i}_0$
is independent of the position in the domain \label{mercator}}
\end{figure}

Note that allowing for $K=0$ in the expression of the map scale factor of
the Lambert projection will give exactly the same result. As far as the
local metrics is concerned, the only useful informations are therefore
$m$, $K$, and $\gamma$. These quantities
 will be used in the following equations, without further
necessity to distinguish between the three types of projection.

\section{Calculus in Non Orthogonal Coordinates}

In the physical space, the $\widehat{x}$ and $\widehat{z}$ coordinate lines
are not orthogonal, because of the underlying topography $z_s$. Therefore,
the coordinate system is not orthogonal and it will be necessary to introduce,
beside the cartesian basis, the covariant and contravariant bases to
develop the tensor operators of Chapter 2. In Meso-NH, we follow the ideas
of Viviand (1974), and Vinokur (1974), and use alternatively the
cartesian and covariant bases, in order to simplify the formulation of
these operators. We now recall the main classical formulae to work with
non orthogonal coordinate systems.

\subsection{Metric Coefficients}

Let us call $x,y,z$ the local distances on the sphere in the directions
$\vec{i},\vec{j},\vec{k}$.

We will use the classical notations $\widehat{d}_{\star \star}$
for the metric coefficients. Their values in the case of our transformation
are given by:
\begin{eqnarray}
\widehat{d}_{xx} & = & \dfrac{\partial x}{\partial \widehat{x} }
         = { r \over a m } \nonumber\\
\widehat{d}_{yy} & = & \dfrac{\partial y}{\partial \widehat{y} }
         = { r \over a m } \nonumber\\
\widehat{d}_{zz} & = & \dfrac{\partial z}{\partial \widehat{z} }
         =  1 - {z_s \over H} \nonumber \\
\widehat{d}_{zx} & = & \dfrac{\partial z}{\partial \widehat{x} }
         = {\partial z_s \over \partial \widehat{x} }(1-{\widehat{z}\over H})
\nonumber\\
\widehat{d}_{zy} & = & \dfrac{\partial z}{\partial \widehat{y} }
         = {\partial z_s \over \partial \widehat{y} }(1-{\widehat{z}\over H})
\\
\widehat{d}_{xz} & = & \dfrac{\partial x}{\partial \widehat{z} } = 0 \nonumber \\
\widehat{d}_{yz} & = & \dfrac{\partial y}{\partial \widehat{z} } = 0 \nonumber \\
\widehat{d}_{xy} & = & \dfrac{\partial x}{\partial \widehat{y} } = 0 \nonumber \\
\widehat{d}_{yx} & = & \dfrac{\partial y}{\partial \widehat{x} } = 0 \nonumber
\end{eqnarray}

\subsection{Covariant Basis}

The basis of covariant vectors is defined as

\begin{equation}
\vec{e}_{i} = \dfrac{\partial \vec{r}}{\partial \widehat{x}^{i}}
\end{equation}
(with $\widehat{x}^{i}=\widehat{x}$, $\widehat{y}$ or $\widehat{z}$).
Those vectors are tangent to the coordinate lines (see Fig. \ref{galchen}).
They can be projected to the cartesian basis, resulting in

\begin{eqnarray}
\vec{e}_{1} & = & \dfrac{\partial \vec{r}}{\partial \widehat{x}} =
\widehat{d}_{xx} \vec{i} + \widehat{d}_{zx} \vec{k} \nonumber \\
\vec{e}_{2} & = & \dfrac{\partial \vec{r}}{\partial \widehat{y}} =
\widehat{d}_{yy} \vec{j} + \widehat{d}_{zy} \vec{k} \nonumber \\
\vec{e}_{3} & = & \dfrac{\partial \vec{r}}{\partial \widehat{z}} =
\widehat{d}_{zz}\vec{k}
\end{eqnarray}

\subsection{Jacobian}
The Jacobian of the transformation from $\widehat{x},\widehat{y},\widehat{z}$
to $x,y,z$ is given by
\begin{equation}
\widehat{J}  = \vec{e}_{1} \cdot ( \vec{e}_{2} \wedge  \vec{e}_{3})
\end{equation}
or
\begin{equation}
\widehat{J}  = \widehat{d}_{xx} \widehat{d}_{yy} \widehat{d}_{zz} = \left( {r\over am} \right)^2(1 - {z_s \over H})
\end{equation}
It is the ratio of the volumes in the transformed and physical spaces.

\subsection{Contravariant Basis}

The basis of contravariant vectors is defined as
\begin{equation}
\vec{e}\, ^{i} = \vec{\nabla}\widehat{x}^{i}
\end{equation}
These vectors are orthogonal to the surfaces $\widehat{x}^{i}=$Const (see Fig.
\ref{galchen}).

Since $\vec{e}\, ^{i} \cdot \vec{e} _{j} = \delta_{ij}$, we also have:
\begin{eqnarray}
\vec{e}\, ^{1} & = & \dfrac{1}{\widehat{J}}(\vec{e} _{2} \wedge \vec{e} _{3}) =
\dfrac{1}{\widehat{d}_{xx}} \vec{i}
\nonumber \\
\vec{e}\, ^{2} & = & \dfrac{1}{\widehat{J}}(\vec{e} _{3} \wedge \vec{e} _{1}) =
\dfrac{1}{\widehat{d}_{yy}} \vec{j}
\\
\vec{e}\, ^{3} & = & \dfrac{1}{\widehat{J}}(\vec{e} _{1} \wedge \vec{e} _{2}) =
\dfrac{1}{J}\left(\widehat{d}_{xx} \widehat{d}_{yy} \vec{k}- \widehat{d}_{zx} \widehat{d}_{yy}\vec{i}
 - \widehat{d}_{zy} \widehat{d}_{xx}\vec{j}\right) \nonumber
\end{eqnarray}
These formulae allow to find the expression of the contravariant components
of any vector, as a function of its cartesian components
\begin{eqnarray}
\vec{A} &=& A^{c1} \vec{e} _{1}+ A^{c2} \vec{e} _{2} + A^{c3} \vec{e} _{3}
\nonumber \\
\vec{A} &=& a^{1} \vec{i}+ a^{2} \vec{j} + a^{3} \vec{k} \nonumber \\
A^{c1}&  =& \dfrac{ a^1 }{\widehat{d}_{xx}} \\
A^{c2}&  =& \dfrac{ a^2 }{\widehat{d}_{yy}} \nonumber \\
A^{c3}&  =& \dfrac{ a^3 }{\widehat{d}_{zz}}- \dfrac{ a^1 \widehat{d}_{zx}}{\widehat{d}_{xx}\widehat{d}_{zz}}-
\dfrac{ a^2 \widehat{d}_{zy}}{\widehat{d}_{yy}\widehat{d}_{zz}} \nonumber
\end{eqnarray}

\subsection{Spatial derivatives of the Cartesian basis}

We will also use the expression of the spatial derivatives of the
vectors of the cartesian basis, due to the earth sphericity:

\begin{eqnarray}
\dfrac{\partial \vec{i}}{\partial \widehat{x}} & = & \cos\gamma \;
\dfrac{\widehat{d}_{xx}}{r \cos\varphi}  \; (\sin\varphi - K)
\;\vec{j} -   \dfrac{\widehat{d}_{xx}}{r} \;  \vec{k} \nonumber \\
\dfrac{\partial \vec{i}}{\partial \widehat{y}} & = & \sin\gamma \;
\dfrac{\widehat{d}_{yy}}{r \cos\varphi}  \; (\sin\varphi - K)
\;\vec{j} \nonumber \\
\dfrac{\partial \vec{j}}{\partial \widehat{x}} & = & -\cos\gamma \;
\dfrac{\widehat{d}_{xx}}{r \cos\varphi}  \; (\sin\varphi - K)
\;\vec{i}  \\
\dfrac{\partial \vec{j}}{\partial \widehat{y}} & = & - \sin\gamma \;
\dfrac{\widehat{d}_{yy}}{r \cos\varphi}  \; (\sin\varphi - K)
\;\vec{i} -  \dfrac{\widehat{d}_{yy}}{r} \;  \vec{k} \nonumber \\
\dfrac{\partial \vec{k}}{\partial \widehat{x}} & = & \dfrac{\widehat{d}_{xx}}{r}
\;\vec{i} \nonumber \\
\dfrac{\partial \vec{k}}{\partial \widehat{y}} & = & \dfrac{\widehat{d}_{yy}}{r}
\;\vec{j} \nonumber
\end{eqnarray}


\subsection{Gradient}

For any scalar quantity $\alpha$,
\begin{equation}
\vec{\nabla} \alpha =
\dfrac{\partial \alpha }{\partial \widehat{x}}\vec{e} \, ^{1} +
\dfrac{\partial \alpha }{\partial \widehat{y}}\vec{e} \, ^{2} +
\dfrac{\partial \alpha }{\partial \widehat{z}}\vec{e} \, ^{3}
\end{equation}

This vector may be projected on $(\vec{i},\vec{j},\vec{k})$ resulting in
the cartesian components of the gradient

\begin{eqnarray}
{\partial \alpha \over \partial x }  =
\vec{i} \cdot \vec{\nabla} \alpha & = & \dfrac{1}{\widehat{d}_{xx}}
\dfrac{\partial \alpha }{\partial \widehat{x}}- \dfrac{\widehat{d}_{zx}}{\widehat{d}_{xx}\widehat{d}_{zz}}
\dfrac{\partial \alpha }{\partial \widehat{z}}
\nonumber \\ & &\nonumber  \\
{\partial \alpha \over \partial y }  =
\vec{j} \cdot \vec{\nabla} \alpha & = & \dfrac{1}{\widehat{d}_{yy}}
\dfrac{\partial \alpha }{\partial \widehat{y}}- \dfrac{\widehat{d}_{zy}}{\widehat{d}_{yy}\widehat{d}_{zz}}
\dfrac{\partial \alpha }{\partial \widehat{z}}
\nonumber \\ & &  \\
{\partial \alpha \over \partial z }  =
\vec{k} \cdot \vec{\nabla} \alpha & = & \dfrac{1}{\widehat{d}_{zz}}
\dfrac{\partial \alpha }{\partial \widehat{z}} \nonumber
\end{eqnarray}


\subsection{Divergence}

For any vector quantity $\vec{A}$, the contravariant components
\begin{equation}
\vec{A} =A^{c \, 1}\vec{e} _{1} + A^{c \, 2}\vec{e} _{2}
 +A^{c \, 3}\vec{e} _{3} = (\vec{A} \cdot\vec{e}\, ^{1})\vec{e} _{1}
 + (\vec{A} \cdot\vec{e} \,^{2})\vec{e} _{2}
+ (\vec{A} \cdot\vec{e}\, ^{3})\vec{e} _{3}
\end{equation}
allow to compute easily the divergence
\begin{equation}
\vec{\nabla}\cdot \vec{A} = \dfrac{1}{\widehat{J}}\left(
\dfrac{\partial}{\partial \widehat{x}}(\widehat{J} A^{c \, 1})
+\dfrac{\partial}{\partial \widehat{y}}(\widehat{J} A^{c\, 2})
+\dfrac{\partial}{\partial \widehat{z}}(\widehat{J} A^{c\,  3}) \right)
\end{equation}

\subsection{Divergence of a Tensor Product}

In order to avoid generating a large number of Christoffel symbols (quantities
involved in the derivatives of the basis vectors), we use the
contravariant components of $\vec{B}$ and the cartesian components of $\vec{A}$

\begin{eqnarray}
\vec{B} &=& B^{c \, 1}\vec{e} _{1} +
B^{c \, 2}\vec{e} _{2} +B^{c \, 3}\vec{e} _{3}\nonumber \\
\vec{A} &=& {\cal A}^{1} \vec{i} +
{\cal A}^{2}\vec{j}+{\cal A}^{3} \vec{k} \nonumber
\end{eqnarray}

The result projected on the cartesian basis reads

\begin{eqnarray}
\vec{i} \cdot \vec{\nabla}\cdot (\vec{B} \otimes \vec{A}) & = &
\dfrac{1}{\widehat{J}}\left(\dfrac{\partial}{\partial \widehat{x}}(\widehat{J}
B^{c \, 1} {\cal A}^{1})
+ \dfrac{\partial}{\partial \widehat{y}}(\widehat{J} B^{c \, 2} {\cal A}^{1})
+ \dfrac{\partial}{\partial \widehat{z}}(\widehat{J} B^{c \, 3}{\cal A}^{1})\right)
\nonumber \\
& &  -  B^{c \, 1} {\cal A}^{3}\dfrac{\widehat{d}_{xx}}{r}
 - B^{c \,1 } {\cal A}^{2} \cos \gamma \dfrac{\widehat{d}_{xx}}{r \cos\varphi}
 (\sin\varphi - K) \nonumber \\
& & -  B^{c \, 2}{\cal A}^{2} \sin \gamma \dfrac{\widehat{d}_{yy}}{r \cos\varphi}
 (\sin\varphi - K)
 \nonumber \\
\vec{j} \cdot \vec{\nabla}\cdot (\vec{B} \otimes \vec{A}) & = &
\dfrac{1}{\widehat{J}}\left(\dfrac{\partial}{\partial \widehat{x}}(\widehat{J}
B^{c \, 1} {\cal A}^{2})
+ \dfrac{\partial}{\partial \widehat{y}}(\widehat{J} B^{c \, 2}{\cal A}^{2})
 + \dfrac{\partial}{\partial \widehat{z}}(\widehat{J} B^{c\, 3}{\cal A}^{2})\right)
\nonumber \\
 & &  +  B^{c \, 2}{\cal A}^{3}\dfrac{\widehat{d}_{yy}}{r}
  +  B^{c \, 1}{\cal A}^{1} \cos \gamma \dfrac{\widehat{d}_{xx}}{r cos\varphi}
  (\sin\varphi - K) \nonumber \\
& &  +  B^{c \, 2}{\cal A}^{1} \sin \gamma \dfrac{\widehat{d}_{yy}}{r cos\varphi}
  (\sin\varphi - K)
  \nonumber \\
\vec{k} \cdot \vec{\nabla}\cdot (\vec{B} \otimes \vec{A}) & = &
\dfrac{1}{\widehat{J}}\left(\dfrac{\partial}{\partial \widehat{x}}(J B^{c \, 1}{\cal
A}^{3})
+ \dfrac{\partial}{\partial \widehat{y}}(\widehat{J} B^{c \, 2}{\cal A}^{3})
 + \dfrac{\partial}{\partial \widehat{z}}(\widehat{J} B^{c \, 3}{\cal A}^{3})\right)
 \nonumber \\
 & &  -  B^{c \, 1}{\cal A}^{1} \dfrac{\widehat{d}_{xx}}{r}
   -  B^{c \, 2}{\cal A}^{2}  \dfrac{\widehat{d}_{yy}}{r}
\end{eqnarray}

For each equation, the first line contains the dominant terms, and the
second and third lines show the additional terms due to the spatial
variation of the vectors of the local basis. These terms are the
{\em curvature terms}. The influence of non-orthogonality is also
present through the use of the contravariant components of $\vec{B}$.

\subsection{Coriolis Force}

Both vectors are expressed in basis $(\vec{i},\vec{j},\vec{k})$.
We call $f=2\Omega \sin\varphi$ and $f_{\star}=2\Omega \cos\varphi$.
\begin{eqnarray}
2\vec{\Omega}\wedge\vec{U}&=&(f_{\star} \cos\gamma w - f v ) \vec{i} \nonumber\\
& & + (f u + f_{\star} \sin \gamma w ) \vec{j} \nonumber \\
& & - (f_{\star} \cos\gamma u + f \sin \gamma v ) \vec{k}
\end{eqnarray}

\section{Model Equations in Non Orthogonal Coordinates}
\subsection{Prognostic Variables}
Given the form of the divergence operator, the most convenient choice for the
model prognostic variables is the product of the wind components, potential
temperature, and mixing ratios by the dry density of the reference state,
{\em and by the Jacobian of the coordinate system}.

To simplify the notations, we define $\widehat{\rho}=\rho_{d\,ref}\widehat{J}$.
The prognostic variables of the model are therefore $\widehat{\rho}u,
\widehat{\rho}v, \widehat{\rho}w, \widehat{\rho}\theta, \widehat{\rho}r_{*}$,
and $\widehat{\rho} s_{*}$.

For the water and passive scalars, the prognostic variable represents therefore
(to the extent that $\rho_{d\,ref}$ is a good approximation of $\rho_d$)
{\em the mass of substance within the grid volume}, which is a very simple and
safe quantity to carry in a model.

\subsection{The contravariant components of the wind}

The contravariant components of the velocity vector
 $\vec{U} = U^{c} \vec{e} _{1}+ V^{c} \vec{e} _{2}
 + W^{c} \vec{e} _{3} $ will be needed to express the advection operator.
They may be computed as

\begin{eqnarray}
U^{c} & = &  \vec{U} \cdot \vec{e}\, ^{1} \\
V^{c} & = &  \vec{U} \cdot \vec{e}\, ^{2} \\
W^{c} & = &\vec{U} \cdot \vec{e}\, ^{3}
\end{eqnarray}

Since $( u , v )$ are the horizontal wind components in the
basis  $( \vec{i}, \vec{j} )$ we have

\begin{eqnarray}
U^{c}&  =& \dfrac{ u }{\widehat{d}_{xx}} \\
V^{c}&  =& \dfrac{ v }{\widehat{d}_{yy}} \\
W^{c}&  =& \dfrac{ w}{\widehat{d}_{zz}}- \dfrac{ u \widehat{d}_{zx}}{\widehat{d}_{xx}\widehat{d}_{zz}}-
\dfrac{ v \widehat{d}_{zy}}{\widehat{d}_{yy}\widehat{d}_{zz}}
\end{eqnarray}

This simple computation is performed at the beginning of each model time
step.

\subsection{Momentum Equation}

\begin{eqnarray}
\dfrac{\partial}{\partial t}(\widehat{\rho}  u ) &= &
 \, - \, \dfrac{\partial }{\partial \widehat{x}} (\widehat{\rho} U^{c } \;   u )
 \, - \, \dfrac{\partial }{\partial \widehat{y}} (\widehat{\rho} V^{c } \;    u )
 \, - \, \dfrac{\partial }{\partial \widehat{z}} (\widehat{\rho} W^{c }  \;    u )
\nonumber \\
& & \nonumber \\
&  + &\widehat{\rho}  u   v  \dfrac{ \cos\gamma}{r\cos\varphi} (\sin\varphi -K)
\,+ \,\widehat{\rho}  v ^{2} \dfrac{ \sin\gamma  }{r\cos\varphi} (\sin\varphi -K)
\nonumber \\
& & \nonumber \\
 & - & \widehat{\rho}\dfrac{ u  w}{r} \,  - \, \widehat{\rho}\dfrac{1}{\widehat{d}_{xx}} \dfrac{\partial \Phi}{\partial \widehat{x}}
\, + \,\widehat{\rho}\dfrac{\widehat{d}_{zx}}{\widehat{d}_{xx}\widehat{d}_{zz}} \dfrac{\partial \Phi}{\partial \widehat{z}} \nonumber \\
& & \nonumber \\
&  - & \widehat{\rho} f^{*} \cos\gamma w \, +  \, \widehat{\rho} f  v
\,+ \, \widehat{\rho} \vec{F}_{v} \cdot \vec{i}
\end{eqnarray}

\begin{eqnarray}
\dfrac{\partial}{\partial t}(\widehat{\rho}  v ) &= &
 \, - \, \dfrac{\partial }{\partial \widehat{x}} (\widehat{\rho} U^{c }\;    v )
 \, - \, \dfrac{\partial }{\partial \widehat{y}} (\widehat{\rho} V^{c }\;    v )
 \, - \, \dfrac{\partial }{\partial \widehat{z}} (\widehat{\rho} W^{c } \;    v )
\nonumber \\ & & \nonumber \\
&  - & \widehat{\rho}  u ^{2}  \dfrac{ \cos\gamma}{r\cos\varphi} (\sin\varphi -K)
 \, - \, \widehat{\rho}  u   v \dfrac{\sin\gamma}{r\cos\varphi} (\sin\varphi -K)
\nonumber \\ & & \nonumber \\
 & - & \widehat{\rho}\dfrac{ v  w}{r} \,  - \, \widehat{\rho}\dfrac{1}{\widehat{d}_{yy}} \dfrac{\partial \Phi}{\partial \widehat{y}}
 \, +  \, \widehat{\rho}\dfrac{\widehat{d}_{zy}}{\widehat{d}_{yy}\widehat{d}_{zz}} \dfrac{\partial \Phi}{\partial \widehat{z}} \nonumber \\
& & \nonumber \\
&  - & \widehat{\rho} f^{*} \sin\gamma w  \, - \,  \widehat{\rho} f  u
 \,+ \, \widehat{\rho} \vec{F}_{v} \cdot\vec{j}
\end{eqnarray}


\begin{eqnarray}
\dfrac{\partial}{\partial t}(\widehat{\rho} w) &= &
 \, - \, \dfrac{\partial }{\partial \widehat{x}} (\widehat{\rho} U^{c } \;   w)
 \, - \, \dfrac{\partial }{\partial \widehat{y}} (\widehat{\rho} V^{c } \;   w )
 \, - \, \dfrac{\partial }{\partial \widehat{z}} (\widehat{\rho} W^{c } \;   w)
\nonumber \\ & & \nonumber \\
 & + & \widehat{\rho} u ^{2} \dfrac{1}{r}+
\widehat{\rho}  v ^{2}\dfrac{1}{r}
 \,  - \, \widehat{\rho}\dfrac{1}{\widehat{d}_{zz}} \dfrac{\partial \Phi}{\partial \widehat{z}}
\, + \,\widehat{\rho} g \dfrac{\theta_v ' }{\widehat{\theta}_v}\nonumber \\
& & \nonumber \\
&  + & \widehat{\rho} f^{*}(\sin\gamma  v  + \cos\gamma  u )
 \,+ \, \widehat{\rho} \vec{F}_{v} \cdot \vec{k}
\end{eqnarray}
The right hand side of these equations has been rewritten in Chapter 2 as
the sum of a dynamical source $\vec{S}_V$ and the pressure gradient. This
is still valid, except that the source now contains the Jacobian. Note
that those are still the cartesian components of $\vec{S}_V$,
i.e. the true horizontal and vertical components of the acceleration.

\subsection{Thermodynamic Equation}

\begin{eqnarray}
\dfrac{\partial}{\partial t}(\widehat{\rho} \theta) &=& \, - \,
 \dfrac{\partial }{\partial \widehat{x}} (\widehat{\rho} U^{c } \;   \theta)
 \, - \, \dfrac{\partial }{\partial \widehat{y}} (\widehat{\rho} V^{c }  \;  \theta )
 \, - \, \dfrac{\partial }{\partial \widehat{z}} (\widehat{\rho} W^{c }  \;
 \theta) \nonumber \\ & &
+\widehat{\rho} \left[ {R_d+r_vR_v\over R_d}{C_{pd} \over C_{ph}}-1 \right]
{\theta \over \Pi_{ref}} {w \over \widehat{d}_{zz} }
 {\partial \Pi_{ref} \over \partial \widehat{z}}
\nonumber \\ & &
+ {\widehat{\rho}\over \Pi_{ref} C_{ph}} \left[
 L_m {D(r_i+r_s+r_g+r_h)\over Dt} - L_v {Dr_v\over Dt} + {\cal H}  \right]
\end{eqnarray}

\subsection{Water and Other Scalars}
The equations have exactly the same form as the thermodynamic equation,
except for the source terms.

\subsection{Continuity Equation}
It has the simple form, using the contravariant velocity components
\begin{equation}
\dfrac{\partial }{\partial \widehat{x}} (\widehat{\rho} U^{c })
+ \dfrac{\partial }{\partial \widehat{y}} (\widehat{\rho} V^{c })
+ \dfrac{\partial }{\partial \widehat{z}} (\widehat{\rho} W^{c }) =0.
\end{equation}

\subsection{Pressure Equation}
The divergence operators appearing in the pressure equation may be
evaluated with the contravariant components of the vectors $\vec{\nabla}\Phi$
and $\vec{S}_V$. The result is an elliptic equation, involving a quasi-laplacian
operator. This is discussed in Chapter 7.

\section{Degenerated Forms}

\subsection{Thin shell approximation}

The depth of the atmosphere ($\simeq 30$ km) is much smaller than the earth
radius ($a \simeq 6000 $ km). Therefore, it is customary in meteorology
to assume that the atmosphere is a thin shell ($z \ll a$ and $r \simeq a$).
However, if this hypothesis is retained without care, the principle of
conservation of angular momentum may be violated (Phillips, 1966).
It is necessary to make the additional hypothesis that the horizontal
component of the earth rotation is negligible. In the Meso-NH model, we
decided not to retain the thin shell approximation. However, for the
purpose of comparison with models making this approximation,
we introduce a flag $\delta_1$, taking the value 1 in the general case
and 0 in the case of the thin shell approximation.

\subsection{Cartesian Coordinate System}

For many purposes, one may want to work in the much simpler
cartesian frame on a plane tangent to the sphere,
and neglect the curvature terms (e.g. study of very small
scale processes, or idealized studies of meso-scale processes). This may be
obtained easily by setting $f=2 \Omega \sin \varphi _{0}$ ,
$f^{*}=2 \Omega \cos \varphi _{0}$, $\widehat{d}_{xx}=1$, $\widehat{d}_{yy}=1$,
$m=1$, $r=a=\infty$.
Note that contrary to the previous case, we may keep here $f^{*} \neq 0$
without violating the conservation of angular momentum.
In order to obtain this very simple form, we introduce another flag
$\delta_2$ in the general system, taking the value 1 for the general case,
and 0 for the cartesian case.

The combination of the thin shell approximation and the cartesian frame
will supply the traditional f-plane approximation (no horizontal component
of the earth rotation).

\subsection{Flagged Equations}

The $\delta_1$ and $\delta_2$ flags modify the momentum equation in the
following way:

\begin{eqnarray}
\dfrac{\partial}{\partial t}(\widehat{\rho}  u ) &= &
 \, - \, \dfrac{\partial }{\partial \widehat{x}} (\widehat{\rho} U^{c }\;    u )
 \, - \, \dfrac{\partial }{\partial \widehat{y}} (\widehat{\rho} V^{c } \;      u )
 \, - \, \dfrac{\partial }{\partial \widehat{z}} (\widehat{\rho} W^{c }\;     u )
\nonumber \\
& & \nonumber \\
&  + & \delta _{2}\widehat{\rho}  u   v  \dfrac{ \cos\gamma}{r \cos\varphi} (\sin\varphi -K)
\,+ \,\delta _{2}\widehat{\rho}  v ^{2}  \dfrac{ \sin\gamma  }{r \cos \varphi} (\sin\varphi -K)
\nonumber \\
& & \nonumber \\
 & - &\delta _{2}\delta _{1} \widehat{\rho}\dfrac{ u  w}{r} \,  - \, \widehat{\rho}\dfrac{1}{\widehat{d}_{xx}} \dfrac{\partial \Phi}{\partial \widehat{x}}
\, + \,\widehat{\rho}\dfrac{\widehat{d}_{zx}}{\widehat{d}_{xx}\widehat{d}_{zz}} \dfrac{\partial \Phi}{\partial \widehat{z}} \nonumber \\
& & \nonumber \\
&  - &\delta _{1} \widehat{\rho} f^{*} \cos\gamma w \, +  \, \widehat{\rho} f  v
\, + \, \widehat{\rho} \vec{F}_{v} \cdot \vec{i}
\\
& & \nonumber \\
\dfrac{\partial}{\partial t}(\widehat{\rho}  v ) &= &
 \, - \, \dfrac{\partial }{\partial \widehat{x}} (\widehat{\rho} U^{c }\;     v )
 \, - \, \dfrac{\partial }{\partial \widehat{y}} (\widehat{\rho} V^{c }\;     v )
 \, - \, \dfrac{\partial }{\partial \widehat{z}} (\widehat{\rho} W^{c }\;     v )
\nonumber \\ & & \nonumber \\
&  - & \delta _{2}\widehat{\rho}  u ^{2}  \dfrac{ \cos\gamma}{r \cos\varphi} (\sin\varphi -K)
 \, - \, \delta _{2}\widehat{\rho}  u   v   \dfrac{\sin\gamma}{r \cos \varphi} (\sin\varphi -K)
\nonumber \\ & & \nonumber \\
 & - & \delta _{2}\delta _{1} \widehat{\rho}\dfrac{ v  w}{r} \,  - \, \widehat{\rho}\dfrac{1}{\widehat{d}_{yy}} \dfrac{\partial \Phi}{\partial \widehat{y}}
 \, +  \, \widehat{\rho}\dfrac{\widehat{d}_{zy}}{\widehat{d}_{yy}\widehat{d}_{zz}} \dfrac{\partial \Phi}{\partial \widehat{z}} \nonumber \\
& & \nonumber \\
&  - &\delta _{1}\widehat{\rho} f^{*} \sin\gamma w  \, - \,  \widehat{\rho} f  u
 \, + \, \widehat{\rho} \vec{F}_{v} \cdot \vec{j}
\\
& & \nonumber \\
\dfrac{\partial}{\partial t}(\widehat{\rho} w) &= &
 \, - \, \dfrac{\partial }{\partial \widehat{x}} (\widehat{\rho} U^{c }\;    w)
 \, - \, \dfrac{\partial }{\partial \widehat{y}} (\widehat{\rho} V^{c }\;   w )
 \, - \, \dfrac{\partial }{\partial \widehat{z}} (\widehat{\rho} W^{c }\;  w)
\nonumber \\ & & \nonumber \\
 & + & \delta _{2}\delta _{1}\widehat{\rho}\dfrac{ u ^{2}+ v ^{2} }{r}
 \,  - \, \widehat{\rho}\dfrac{1}{\widehat{d}_{zz}} \dfrac{\partial \Phi}{\partial \widehat{z}}
\, + \,\widehat{\rho} g \dfrac{\theta_v ' }{\widehat{\theta}_v}\nonumber \\
& & \nonumber \\
&  + &\delta _{1}\widehat{\rho} f^{*}(\sin\gamma  v  + \cos\gamma  u )
 \, + \, \widehat{\rho} \vec{F}_{v} \cdot \vec{k}
 \end{eqnarray}
with
\begin{eqnarray}
\widehat{d}_{xx}&=&\dfrac{a+\delta_1 \delta_2 z}{am} \nonumber \\
\widehat{d}_{yy}&=&\dfrac{a+\delta_1 \delta_2 z}{am} \nonumber \\
r &=& a + \delta_1 z \\
\widehat{J} &=& \left( \dfrac{a+\delta_1 z}{am} \right)^2 (1 - {z_s \over H }) \nonumber
\end{eqnarray}
The other equations of the model do not involve the flags.


\section{References}
\por
Gal-Chen, T., and R. C. J. Sommerville, 1975: On the Use of a Coordinate
Transformation for the Solution of the Navier-Stokes Equations. J. Comput.
Phys., 17, 209-228.
\por
Vinokur, 1974:?
\por
Viviand, H., 1974: Formes Conservatives des Equations de la Dynamique des Gaz.
La Recherche A\'erospatiale, 1974-1, 65-66.
%%%%%%%%%%%%%%%%%%%%%%%%%%%%%%%%%%%%%%%%%%%%%%%%%%%%%%%%%%%%%%%%%%%%%%%%%%%%%

