\chapter{The pressure problem}
\minitoc

%{\em by P. Hereil}

\section{Continuous pressure equation}
\subsection{Full elliptic problem}
\par In the  Meso-NH model formulation, the equations of motion are
expressed as:
\begin{eqnarray}
\label{motion1}
\frac {\partial{\tilde{\rho} { u }} } {\partial{t} } = S_x - \tilde{\rho}
\frac{\partial{\Phi} }{\partial{x} } \\
\label{motion2}
\frac {\partial{\tilde{\rho} { v }} } {\partial{t} } = S_y - \tilde{\rho}
\frac{\partial{\Phi} }{\partial y} \\
\label{motion3}
\frac {\partial{\tilde{\rho} {w}} } {\partial{t} } = S_z - \tilde{\rho}
\frac{\partial{\Phi} }{\partial z}
\end{eqnarray}
where $S_x$, $S_y$ and $S_z$ are the source terms apart from the
pressure terms; $\tilde{\rho} = \rho_{d\,ref} J$; $\Phi = C_p \,
\theta_{v\,ref}\,\Pi '$ is called here the "pressure" for short.
The anelastic constraint reads\\
\begin{equation}
\label{conti}
     \dfrac {\partial{\tilde{\rho} U^{c} } } {\partial{\overline{x} } }
   + \dfrac {\partial{\tilde{\rho} V^{c} } } {\partial{\overline{y} } }
   + \dfrac {\partial{\tilde{\rho} W^{c} } } {\partial{\overline{z} } } = 0
\end{equation}
If we combine the equations (\ref{motion1}-\ref{motion3}) in order to build and
nullify the divergence of the wind at the future instant in (\ref{conti}), we
obtain a diagnostic equation for the pressure:
\begin{equation}
\label{conteq}
GDIV(\tilde{\rho}  \overrightarrow{\nabla} \Phi) = GDIV \left(
\overrightarrow {S} -
\dfrac{\tilde{\rho} \vec{U}_{(t-\Delta t)} }
{2 \Delta t}
\right)
\end{equation}
where
\begin{displaymath}
GDIV(\vec{B}) = \dfrac {\partial{ B^{c1} } } {\partial{\overline{x} } }
               + \dfrac {\partial{ B^{c2} } } {\partial{\overline{y} } }
               + \dfrac {\partial{ B^{c3} } } {\partial{\overline{z} } }
\end{displaymath}
$B^{c1}, B^{c2}, B^{c3}$ are the contravariant components of
$\overrightarrow{B}$ and can be written as:
\begin{eqnarray}
 B^{c1} \;  &  = &
 \dfrac{{B_1}}{ d_{xx}} \\
 B^{c2} \;  &  = &
 \dfrac{{B_2}}{ d_{yy}} \\
 B^{c3} \;  & = &
\dfrac{1}{d_{zz}}
\left[  B_3 -
   \dfrac  {B_1} {d_{xx}}  d_{zx}
-  \dfrac  {B_2} {d_{yy}}  d_{zy}  \right]
\end{eqnarray}
where $(B_1, B_2, B_3)$ are the components of $\vec{B}$ along
$\left( \vec{i}, \vec{j}, \vec{k} \right)$.\\
$\overrightarrow{\nabla}$ is defined by:
$
\overrightarrow{\nabla}=
\vec{i} \dfrac {\partial{  } } {\partial{x} }  +
\vec{j} \dfrac {\partial{  } } {\partial{y} }  +
\vec{k} \dfrac {\partial{  } } {\partial{z} }
$
where
\begin{eqnarray*}
\dfrac {\partial{  } } {\partial{x} } &=&
\dfrac{1}{d_{xx}}
  \dfrac{\partial  }{\partial \overline{x}}- \dfrac{d_{zx}}{d_{xx}d_{zz}}
\dfrac{\partial  }{\partial \overline{z}}\\
\dfrac {\partial{  } } {\partial{y} } &=&
\dfrac{1}{d_{yy}}
  \dfrac{\partial  }{\partial \overline{y}}- \dfrac{d_{zy}}{d_{yy}d_{zz}}
   \dfrac{\partial  }{\partial \overline{z}}\\
\dfrac {\partial{  } } {\partial{z} } &=&
\dfrac{1}{d_{zz}}
  \dfrac{\partial  }{\partial \overline{z}}
\end{eqnarray*}
For convenience, we define the $QLAP$ operator by: $QLAP =
GDIV(\tilde{\rho} \overrightarrow{\nabla})$.

Note that because an iterative method is used to solve the pressure equation,
the residual term $GDIV(\tilde{\rho} \vec{U}_{(t-\Delta t)})$
must be kept and injected in the right hand side
of 6.5 to get the most accurate solution for $\Phi$.
%%%%%%%%%%%%%%%%%%%%%%%%%%%%%%%%%%%%%%%%%%%%%%%%%%%%%%%%%%%%%%%%%%%%%%%%%%%%%%%
\subsubsection{Top and Bottom Boundaries}
For the highest and the lowest vertical level, a Neumann condition is
imposed:\\
\begin{equation}
\label{vb1}
\tilde{\rho} \vec{n} \cdot \overrightarrow{\nabla}\Phi= \vec{n} \cdot
\overrightarrow{S}
\end{equation}
%%
%%%%%%%%%%%%%%%%%%%%%%%%%%%%%%%%%%%%%%%%%%%%%%%%%%%%%%%%%%%%%%%%%%%%%%%%%%%%%%%
%%
\subsubsection{Lateral Boundaries}
We allow for several possibilities:
\begin{itemize}

\item For a periodic simulation domain, the lateral boundary conditions are
a periodisation of the pressure field.
\item For an open area, the ${ u }$ and ${ v }$ evolution on the boundaries
are performed by the Sommerfeld's equations:
\begin{eqnarray}
\label{som1}
\frac {\partial{\tilde{\rho} { u }} } {\partial{t} } = -  c_x \frac
{\partial{\tilde{\rho} { u }} } {\partial{x} } \\
\label{som2}
\frac {\partial{\tilde{\rho} { v }} } {\partial{t} } = -  c_y \frac
{\partial{\tilde{\rho} { v }} } {\partial{y} }
\end{eqnarray}
where $c_x$ and $c_y$ are the phase velocities
of the fastest waves in  respectively x and y directions and determined by the
Orlanski's method (1976\nocite {orl76}) or fixed (Klemp and Wilhelmson, 1978
\nocite{kle78}). \\
\item If a Davies' scheme is used, the tendency on the boundary is given by:\\
\begin{eqnarray}
\frac {\partial{\tilde{\rho} { u }} } {\partial{t} } =
\left( \frac {\partial{\tilde{\rho} { u }} } {\partial{t} } \right)_{Large
\; Scale} \\
\frac {\partial{\tilde{\rho} { v }} } {\partial{t} } =
\left( \frac {\partial{\tilde{\rho} { v }} } {\partial{t} } \right)_{Large
\; Scale}
\end{eqnarray}
\item If we use a Carpenter's scheme (Carpenter, 1982\nocite{car82}),
the tendency on the boundary is expressed as:
\begin{eqnarray}
\frac {\partial{\tilde{\rho} { u }} } {\partial{t} } =
\left( \frac {\partial{\tilde{\rho} { u }} } {\partial{t} } \right)_{Large
\; Scale}-  c_x  \left[
 \frac {\partial{\tilde{\rho} { u }} } {\partial{x} }
- \left( \frac {\partial{\tilde{\rho} { u }} } {\partial{x} }\right)_{Large
\; Scale}
\right]\\
%
\frac {\partial{\tilde{\rho} { v }} } {\partial{t} } =
\left( \frac {\partial{\tilde{\rho} { v }} } {\partial{t} } \right)_{Large
\; Scale} -  c_y \left[
 \frac {\partial{\tilde{\rho} { v }} } {\partial{y} }
- \left( \frac {\partial{\tilde{\rho} { v }} } {\partial{y} }\right)_{Large
\; Scale}
\right]
\end{eqnarray}
%%
\end{itemize}
Therefore, in every case, the temporal evolution of the horizontal momentum is
known at the boundary by these equations and from (\ref{motion1}-
\ref{motion2}), we can infer the normal component of the pressure gradient
when the sources components $\vec{S} \cdot \vec{n}$ is known.
The result is a non-homogeneous Neumann condition for the full elliptic
problem. We can make this condition homogeneous if we replace in
(\ref{motion1}-\ref{motion2}) the source terms $S_x$ and $S_y$ at the
boundaries by the prescribed values of $\frac {\partial{\tilde{\rho} { u }}
} {\partial{t} }$ and $\frac {\partial{\tilde{\rho} { v }} }
{\partial{t} }$ respectively. This leads to an algebraic simplification
in the pressure equation (\ref{conteq}).The homogeneous boundary condition
for the pressure field is:
\begin{equation}
\label{hb1}
\dfrac{\partial{\Phi } } {\partial{n} } = 0
\end{equation}
%%%%%%%%%%%%%%%%%%%%%%%%%%%%%%%%%%%%%%%%%%%%%
%%%
\par We want to solve efficiently and accurately the linear system
(\ref{conteq}) where the source term $\overrightarrow {S}$ is known and with
the previous boundary conditions. As it is
difficult and time-consuming to invert such a large system, we have
implemented two iterative methods in order to offer a good compromise between
versatility and efficiency. Both methods are described
in Golub and Meurant (1983\nocite{gol83}): the first one consists in
a conjugate gradient algorithm and the second one is the Richardson's method.
The convergence of such algorithms is considerably speeded-up with the use
of preconditioning techniques. Here, we precondition with the "flat"
problem (Bernardet\nocite{ber94}, 1994), i.e. without orography, which
compresses the eigenvalue spectrum of the original problem, leaving only a few
eigenvalues significantly different from 1, resulting in much faster
convergence. For both methods, the first guess is also set equal to the
solution of the flat problem.
%%
%%%%%%%%%%%%%%%%%%%%%%%%%%%%%%%%%%%%%%%%%%%%%%%%%%%%%%%%%%%%%%%%%%%%%%%%%%%%%%%
%%
\subsection{Conjugate gradient algorithm}
\par The Golub and Meurant's version of the conjugate gradient (CG) is devoted
to symmetric matrices. As our problem is non symmetric, we use a
generalization of the CG called ORTHOMIN, detailled in Young and Jea
(1981\nocite{you81}). Considering the linear system:
\begin{equation}
Qx = y
\end{equation}
where $x$ is unknown. In our context, $Q$ is the $QLAP$ operator defined before.
Preconditionning the previous system with the "flat" problem gives:
\begin{equation}
F^{-1} Q x = F^{-1} y
\end{equation}
where $F^{-1}$ is the preconditioner with $F$ defined as the $QLAP$ operator
where the topography is neglected (flat problem).\\
The ORTHOMIN algorithm can be put in the following form (Kapitza and Eppel,
1992\nocite{kap92}):
\begin{eqnarray*}
%%
x^{(n+1)} &=& x^{(n)} + \lambda _n p^{(n)}\\
%%
\lambda _n &=& \frac {(\delta^{(n)}, F^{-1} Q p^{(n)} )}
  { \left( F^{-1} Q p^{(n)}, F^{-1} Q p^{(n)} \right) } \\
%%
\delta^{(n)} &=& F^{-1} ( y - Q x^{(n)}) \\
%%
p^{(0)} &=& \delta^{(0)}\\
%%
p^{(n)} &=& \delta^{(n)} + \alpha _n p^{(n-1)} \\
%%
\alpha _n &=& - \frac {\left( F^{-1} Q \delta^{(n)},  F^{-1} Q p^{(n-1)} )
\right)} {\left( F^{-1} Q p^{(n-1)}, F^{-1} Q p^{(n-1)} \right)} \\
%%
\end{eqnarray*}
Here $x^{(n)}$ is the n-th iteration of $x$ and (.,.) is the dot product of two
vectors. Concerning the computing time, at each iteration, we have to
invert three times the $F$ matrix.
\\
%%
%%%%%%%%%%%%%%%%%%%%%%%%%%%%%%%%%%%%%%%%%%%%%%%%%%%%%%%%%%%%%%%%%%%%%%%%%%%%%%%
%%
\subsection{Richardson's method}
\par This technique allows to divide by 3 the number of matrix inversion,
compared to the CG algorithm . Furthermore, this method, including a
relaxation factor $\alpha$, allows an under-relaxation  which is useful to
improve the solver convergence in the context of simulations with large slopes.
For example, for slopes greater than 1, preliminary tests have shown that the
optimal value of $\alpha$ is equal to 1. for such simulations. If we assume the
following linear system preconditioned with the inverse of the "flat" problem:
\begin{equation}
F^{-1} Q x = F^{-1} y
\end{equation}
the Richardson's method reads:
\begin{eqnarray*}
x^{(n+1)} &=& x^{(n)} + \alpha (F^{-1} y - F^{-1} Q x^{(n)} )
\end{eqnarray*}
where $x^{(n)}$ is the n-th iteration of $x$.\\
%%
%%%%%%%%%%%%%%%%%%%%%%%%%%%%%%%%%%%%%%%%%%%%%%%%%%%%%%%%%%%%%%%%%%%%%%%%%%%%%%%
%%
\subsection{Flat operator}
\par Both iterative methods require the inversion of the operator $QLAP$ for
the "flat" problem. "Flat" means the topography is neglected, which induces
that $x=\overline{x}$, $y=\overline{y}$, $z=\overline{z}$.
This involves that $d_{zx}$, $d_{zy}$ vanish. Here, we consider a regular grid
and the stretching functions ${\cal{D}}_x (\widehat{x})$,
${\cal{D}}_y (\widehat{y})$, ${\cal{D}}_z (\widehat{z})$ are neglected
in order to use a FFT decomposition and will be implemented later. Thus, we
obtain for the flat operator $F$:\\
\begin{equation}
\label{flatop1}
F \Phi =  \frac { \partial{} } { \partial{\overline{x}} } \left(
          \dfrac{1}{d_{xx}} \tilde{\rho} \dfrac{1}{d_{xx}}
          \frac { \partial{\Phi} } { \partial{\overline{x}} } \right)
+          \frac { \partial{} } { \partial{\overline{y}} } \left(
          \dfrac{1}{d_{yy}} \tilde{\rho} \dfrac{1}{d_{yy}}
          \frac { \partial{\Phi} } { \partial{\overline{y}} } \right)
+          \frac { \partial{} } { \partial{\overline{z}} } \left(
          \dfrac{1}{d_{zz}} \tilde{\rho} \dfrac{1}{d_{zz}}
          \frac { \partial{\Phi} } { \partial{\overline{z}} } \right)
\end{equation}
In the flat problem, $\tilde{\rho}$ only depends on $z$, (\ref{flatop1})
can be rewritten as:\\
\begin{equation}
\label{flatop}
F \Phi =  < \tilde{\rho}>_{x,y}
  \left[ \dfrac{1}{<{d_{xx}}^2>_{x,y}}
         \frac { \partial{^2\Phi} } { \partial{\overline{x}^2} }
+        \dfrac{1}{<{d_{yy}}^2>_{x,y}}
         \frac { \partial{^2\Phi} } { \partial{\overline{y}^2} }
  \right]
+          \frac { \partial{} } { \partial{Z} } \left(
           <\tilde{\rho}>_{x,y} \dfrac{1} {<{d_{zz}}^2>_{x,y}}
          \frac { \partial{\Phi} } { \partial{Z} } \right)
\end{equation}
where $Z = <\overline{z}>_{xy}$.\\
We have applied a spatial averaging on a whole $xy$ plan because
$\tilde{\rho}$ and $\overline{z}$ are functions of $z$. When
the orography is neglected, we must recover a function of only $\overline{z}$ (this
is also necesary to lead to a separable problem). A Fast Fourier Transform
(Schumann and Sweet, 1988\nocite{sch88}) is used for the inversion of the
horizontal part of $F$. The form of this FFT depends on the lateral boundary
conditions. The vertical part of $F$ leads to a tridiagonal matrix and thus a
classical inversion algorithm is applied (the double sweep method, see
"Numerical Recipes").
%%%%%%%%%%%%%%%%%%%%%%%%%%%%%%%%%%%%%%%%%%%%%%%%%%%%%%%%%%%%%%%%%%%%%%%%%%%%%%%
%%
%%%%%%%%%%%%%%%%%%%%%%%%%%%%%%%%%%%%%%%%%%%%%%%%%%%%%%%%%%%%%%%%%%%%%%%%%%%%%%%
%%
\section{Discrete pressure equation}
%%
%%%%%%%%%%%%%%%%%%%%%%%%%%%%%%%%%%%%%%%%%%%%%%%%%%%%%%%%%%%%%%%%%%%%%%%%%%%%%%%
%%
\subsection{Full elliptic problem}
\par The continuous equation (\ref{conteq}) is:\\
\begin{equation}
\label{pressure}
GDIV(\rho  \overrightarrow{\nabla} \Phi) = GDIV \left( \overrightarrow {S} -
\dfrac{\tilde{\rho} \vec{U}_{(t-\Delta t)} }
{2 \Delta t}
\right)
\end{equation}
where
$
\overrightarrow{\nabla}\Phi =
\dfrac {\partial{\Phi  } } {\partial{x} }  \vec{i} +
\dfrac {\partial{\Phi } } {\partial{y} }  \vec{j} +
\dfrac {\partial{\Phi  } } {\partial{z} }  \vec{k}
$
is discretized by:
\begin{eqnarray}
\label{gradx}
\dfrac { \partial {\Phi}} {\partial{x}} &\rightarrow&  \frac {1}
{d_{xx} } \left[ \delta_x \Phi -
\overline { d_{zx} \overline { \frac { \delta_z \Phi} {d_{zz} } }^x }
^z \right] \\
\label{grady}
\dfrac {\partial {\Phi}} {\partial{y}} &\rightarrow&  \frac {1}
{d_{yy} } \left[ \delta_y \Phi -
\overline { d_{zy} \overline { \frac { \delta_z \Phi} {d_{zz} } }^y } ^z
\right] \\
\label{gradz}
\dfrac {\partial {\Phi}} {\partial{z}} &\rightarrow& \frac {\delta_z \Phi}
{ d_{zz}}
\end{eqnarray}
where $\delta _x$, $\delta _y$, $\delta _z $ are finite difference operators.\\
The discrete form of the contravariant components of a vector $\vec{B}
(B_1, B_2, B_3)$ is:
\begin{eqnarray}
 B^{c1} \;  &  = &
 \dfrac{{B_1}}{ d_{xx}} \\
 B^{c2} \;  &  = &
 \dfrac{{B_2}}{ d_{yy}} \\
 B^{c3} \;  & = &
\dfrac{1}{d_{zz}}
\left[  B_3 -
\overline{\left( \overline{\left(
 \dfrac{ {B_1}}{d_{xx}} \right)}^{z}
d_{zx}\right)}^{x}
-  \overline{\left( \overline{\left(\dfrac{
{B_2}}{d_{yy}}\right)}^{z}d_{zy}\right)}^{y}\right]
\end{eqnarray}
$(B_1, B_2, B_3)$ are the components of $\vec{B}$ along
$\left( \vec{i}, \vec{j}, \vec{k} \right)$
and thus we have:
\begin{equation}
GDIV(\vec{B}) = \delta _x B^{c1} + \delta _y B^{c2} + \delta _z B^{c3}
\end{equation}
We note $QLAP$ the numerical discretization of $QLAP$ and we obtain:
\begin{eqnarray*}
QLAP (\Phi) = GDIV(\tilde{\rho} \vec{\nabla}  \Phi ) &  = &  \delta _{x}
\left[ \overline {\tilde{\rho} }^{x} \dfrac{\delta _{x} \Phi }{{d_{xx}}^2}
 \right]
  - \delta _{x} \left[
 \overline{\tilde{\rho} }^{x}
  \dfrac{1 }{{d_{xx}}^2} \overline{\left( d_{zx} \overline{\left(\dfrac
{\delta  _{z} \Phi }{d_{zz} }\right) }^{x}\right) }^{z}   \right]
  \nonumber \\ & & \nonumber \\
 & + &   \delta _{y} \left[ \overline{\tilde{\rho} }^{y}
 \dfrac{\delta _{y} \Phi }{{d_{yy}}^2}
 \right] - \delta _{y} \left[ \overline{\tilde\rho  }^{y}
  \dfrac{1 }{{d_{yy}}^2} \overline{\left( d_{zy} \overline{\left( \dfrac
{\delta  _{z} \Phi}{d_{zz} } \right) }^{y}\right) }^{z}   \right]
  +    \delta _{z} \left[ \overline{\tilde{\rho} }^{z}
 \dfrac{\delta _{z} \Phi }{{d_{zz}}^2} \right]  \nonumber \\ & & \nonumber \\
 & -  & \delta _{z} \left[
 \dfrac{1}{d_{zz}}
 \overline{
  \left\{ d_{zx}
 \overline{ \left\{ \overline{\tilde{\rho} }^{x}\dfrac{\delta _{x} \Phi }
{{d_{xx}}^2}
  -\overline{\tilde{\rho} }^{x}\dfrac{1 }{d_{xx}}
 \overline{ \left(  d_{zx} \overline{  \left( \dfrac{\delta
  _{z} \Phi}{d_{zz} }\right) }^{x}\right) }^{z}
\right\}  }^{z}
\right\}  }^{x}
 \right] \nonumber \\ & & \nonumber \\
& -  & \delta _{z} \left[
 \dfrac{1}{d_{zz}}
 \overline{
  \left\{ d_{zy}
\overline{\left\{\overline{\tilde{\rho}}^{y}\dfrac{\delta _{y}\Phi}{{d_{yy}}^2}
  -\overline{\tilde{\rho}}^{y}\dfrac{1 }{d_{yy}}
 \overline{ \left(  d_{zy} \overline{  \left( \dfrac{\delta
  _{z} \Phi}{d_{zz} }\right) }^{y}\right) }^{z}
\right\}  }^{z}
\right\}  }^{y}
 \right]
\end{eqnarray*}
%%%%
For the first inner points, the calculation of the pseudo-Laplacian
operator $QLAP$ requires the evaluation of pressure outside the physical
domain. \\
%%%%%%%%%%%%%%%%%%%%%%%%%%%%%%%%%%%%%%%%%%%%%%%%%%%%%%%%%%%%%%%%%
\subsubsection {Vertical boundaries }
The discrete form of the boundary equation (\ref{vb1}) is expressed as:
\begin{equation}
\left(\vec{S}\right)^{c3} = \left( \tilde{\rho} \overrightarrow{\nabla} \Phi
 \right) ^{c3}
\end{equation}
where
\begin{equation}
\label{vb3}
\left( \tilde{\rho}  \overrightarrow{\nabla} \Phi \right) ^{c3} \;   =
\dfrac{1}{d_{zz}}
\left[ \left( \tilde{\rho} \dfrac {\partial \Phi}{\partial z} \right)  -
\overline{\left( \overline{\left(
 \dfrac{ 1}{d_{xx}}\left(\tilde{\rho}  \dfrac {\partial \Phi}{\partial x}
\right) \right)}^{z} d_{zx}\right)}^{x}
-  \overline{\left( \overline{\left(\dfrac{ 1
}{d_{yy}}\right)\left(\tilde{\rho} \dfrac {\partial
\Phi}{\partial y} \right)}^{z}d_{zy}\right)}^{y}\right]
\end{equation}
and the pressure gradient expressions are given by (\ref{gradx}-\ref{gradz}).
In (\ref{vb3}), the vertical spatial average requires the points located at
$z = - \Delta z/2$. In this case, we copy the value of the first
pressure gradient over the surface:
\begin{eqnarray*}
\left( \dfrac {\partial \Phi} {\partial x} \right)_{-\frac{\Delta z}{2} } =
 \left( \dfrac {\partial \Phi} {\partial x} \right)_{\frac{\Delta z}{2} } \\
\left( \dfrac {\partial \Phi} {\partial y} \right)_{-\frac{\Delta z}{2} } =
 \left( \dfrac {\partial \Phi} {\partial y} \right)_{\frac{\Delta z}{2} } \\
\end{eqnarray*}
%%%%%%%%%%%%%%%%%%%%%%%%%%%%%%%%%%%%%%%%%%%%%%%%%
%%
\subsubsection {Horizontal boundaries }
Here, we have to consider how to evaluate the pressure gradient terms at the
boundary (wind point), i.e. $i = 1+jphext$, $i= imax+1+jphext$, $j = 1+jphext$,
$j = jmax+1+jphext$ and this for $kmax+1+jpvext > k > jpvext$. $jphext$ and
$jpvext$ are the variables used to extend the number of horizontal and vertical
outer points respectively ($jphext \geq 1$, $jpvext \geq 1$). We have to
distinguish three cases:\\
%%
%%%%%%%%%%%%%%%%%%%%%%%%%%%%%%%%%%%%%%%%%%%%%%%%%
%%
$\bullet$\underline{Cyclic case:}\\
The periodic condition reads:\\
\begin{eqnarray}
\label{cyclic1}
 \Phi (jphext, j, k) &=& \Phi (imax+jphext, j, k) \\
\nonumber
\Phi (imax+1+jphext, j, k) &=& \Phi (1+jphext, j, k)
\end{eqnarray}
 in the x direction and
\begin{eqnarray}
\label{cyclic2}
 \Phi (i, jphext, k) &=& \Phi (i, jmax+jphext, k) \\
\nonumber
\Phi (i, jmax+1+jphext, k) &=& \Phi (i, 1+jphext, k)
\end{eqnarray}
in the y direction.\\
%%
%%%%%%%%%%%%%%%%%%%%%%%%%%%%%%%%%%%%%%%%%%%%%%%%%
%%
$\bullet$\underline{Open and Davies' cases:}\\
The x lateral boundary condition $\dfrac{\partial{\Phi } } {\partial{x} } = 0
$ for $i = 1+jphext, i= imax+1+jphext$ ($jmax+1+jphext>j>jphext$) becomes:
\begin{equation}
\label{open1}
\dfrac{\delta_{x} \Phi}{d_{xx}}  \, +  \,  \dfrac{1}{d_{xx}}
\overline{ \left(\overline{ \left(\dfrac{\delta_{z} \Phi}{d_{zz}}
\right)}^{x}d_{zx}\right)}^{z} = 0
\nonumber \\
\end{equation}
The y lateral boundary condition $\dfrac{\partial{\Phi } } {\partial{y} } = 0 $
for $j = 1+jphext, j= jmax+1+jphext$ ($imax+1+jphext>i>jphext$)
becomes:
\begin{equation}
\label{open2}
 \dfrac{\delta_{y} \Phi}{d_{yy}}  \, + \,   \dfrac{1}{d_{yy}}
\overline{ \left(\overline{ \left(\dfrac{\delta_{z} \Phi}{d_{zz}}
\right)}^{y}d_{zy}\right)}^{z} = 0
\end{equation}
%%
%%%%%%%%%%%%%%%%%%%%%%%%%%%%%%%%%%%%%%%%%%%%%%%%%
$\bullet$\underline{Mixed boundary conditions:}\\
In this case, we have to combine the previous set of equations (\ref{cyclic1})-
(\ref{open2}) to be consistent with the specification of the horizontal
boundaries. For example, if we assume a periodic condition along $y$ and a
Davies' condition along $x$ we use (\ref{cyclic2}) and (\ref{open1}).\\
\\
%%
%%%%%%%%%%%%%%%%%%%%%%%%%%%%%%%%%%%%%%%%%%%%%%%%%%%%%%%%%%%%%%%%%%%%%%%%%%%%%%%
%%
\underline {\bf Edges:}\\
The edges points at $k = jpvext$, (i.e. $\Phi (jphext, j, jpvext)$,
$\Phi (imax+1+jphext, j, jpvext)$, $\Phi (i, jphext,$  $jpvext)$,
$\Phi (i, jmax+1+jphext, 1)$, where $imax+jphext+1> i >  jphext$ and
$jmax+jphext+1 > j > jphext$) and their corresponding points located at the
coordinate $k = kmax+1+jpvext$, are necessary for the calculation of the
pseudo-Laplacian of the first inner grid points. So, we have imposed boundary
conditions consistent with the Neumann staggered condition defined for the
horizontal boundaries. These conditions are expressed as:\\
- for $i=1+jphext$, $imax+1+jphext$ and $z = \Delta z/2, H-\Delta z/2$:
\begin{displaymath}
\dfrac{\partial{\Phi } } {\partial{x} } = 0
\end{displaymath}
- for $i=1+jphext$, $imax+1+jphext$ and $z = 0, H$:
\begin{displaymath}
\delta _z \dfrac{\partial{\Phi } } {\partial{x} } = 0
\end{displaymath}
- for $j=1+jphext$, $jmax+1+jphext$ and $z = \Delta z/2, H - \Delta z/2$:
\begin{displaymath}
\dfrac{\partial{\Phi } } {\partial{y} } = 0 \\
\end{displaymath}
- for $j=1+jphext$, $jmax+1+jphext$ and $z = 0, H$:
\begin{displaymath}
\delta _z \dfrac{\partial{\Phi } } {\partial{y} } = 0
\end{displaymath}

\par The corner points, i.e. $\Phi (jphext, jphext, jpvext)$,
$\Phi (imax+1+jphext, jphext, jpvext)$, $\Phi (jphext,$ $jmax+1+jphext, jpvext)$,
$\Phi (imax+1+jphext, jmax+1+jphext, jpvext)$ and
their corresponding points at $k = kmax+1+jpvext$, are not required for the
resolution of the pressure equation.
%%
%%%%%%%%%%%%%%%%%%%%%%%%%%%%%%%%%%%%%%%%%%%%%%%%%%%%%%%%%%%%%%%%%%%%%%%%%%%%%%%
%%%%%%%%%%%%%%%%%%%%%%%%%%%%%%%%%%%%%%%%%%%%%%%%%%%%%%%%%%%%%%%%%%%%%%%%%%%%%%%
%%
\subsection{Flat operator}
%%%
\par We want to invert the following system (\ref{flat1}) where $Y$ is known
and $F$ is the flat operator.
\begin{equation}
\label{flat1}
F \Phi = Y
\end{equation}
The discretized form of the flat operator $F$ (\ref{flatop1})  is written as:
\begin{equation}
\label{flatop2}
F \Phi = <\tilde{\rho}>_{x, y} \left[ \dfrac {1} {<{d_{xx}}^2>_{xy}} \delta _{x^2}
 \Phi
+ \dfrac{1}{<{d_{yy}}^2>_{xy}}\delta _{y^2} \Phi \right]
+ \delta _{Z}  \overline{<\tilde{\rho} >_{x, y}}^{z} \dfrac {1}
{<{d_{zz}}^2>_{xy} } \delta _{Z}\Phi
\end{equation}
The method consists in treating the horizontal part of F in the Fourier space.
First, we compute the FFT of $Y$ noted $\hat{Y}$. Then, we introduce the
horizontal Fourier decomposition of $F$ completed by its classical vertical
part. This results in $imax \times jmax$ tridiagonal matrices where every
matrix corresponds to a different horizontal mode. A classical tridiagonal
matrix inversion is performed for each horizontal mode ($m, n$), which allows
us to compute the solution of (\ref{flat1}) in the Fourier space and denoted
$\hat{\Phi}$. Finally, we apply an inverse FFT to obtain $\Phi$ in the physical
space. As it is mentionned in the FFT documentation, the numbers of horizontal
modes $imax$ and $jmax$ should be a composite number
factorizable as a product of powers of 2, 3 and 5. Vectorization is achieved by
doing parallel transforms.
%%%%%%%%%%%%%%%%%%%%%%%%%%%%%%%%%%%%%%%%%%%%
\subsubsection{Horizontal discretization}
The FFT form depends on the lateral boundary conditions:\\
%%%%%%%%%%%%%%%%%%%%%%%%%%%%%%%%%%%%%%%%%%%%
$\bullet$ \underline{Cyclic case:}\\
%%%%%%%%%%%%%%%%%%%%%%%%%%%%%%%%%%%%%%%%%%%%
In the aim of no confusion with the complex number $i$ ($i^2 = -1$), we adopt
for a while the convention $(i, j, k) \rightarrow (I, J, K)$. With the boundary
conditions (\ref{cyclic1})-(\ref{cyclic2}), we have:
\begin{equation}
\label{foux1}
\Phi (I, J, K) =   \sum_{m=0}^{imax-1} \sum_{n=0}^{jmax-1}
\widehat{\Phi}_{m\, n}(K) e^ {i\frac {2\pi}{imax} m I \, + \,
i\frac {2\pi}{jmax} n J }
\end{equation}
We denote $\tilde{\Phi} _{m n}(K) = \widehat{\Phi} _{m n}(K)
e^ {i\frac {2\pi}{imax} m I \, + \, i\frac {2\pi}{jmax} n J } $ where
$\tilde{\Phi}_{m\, n}(K)$ is complex. Using the FFT decomposition
(\ref{foux1}), we have:\\
\begin{eqnarray*}
\delta _{x^2} \tilde{\Phi} _{m n} = - 4 \sin ^2 \left( \frac {\pi} {imax}
 m \right)     \tilde{\Phi} _{m n}  \\
\delta _{y^2} \tilde{\Phi}_{m n}  = - 4 \sin ^2 \left( \frac {\pi} {jmax}
 n \right)      \tilde{\Phi} _{m n}
\end{eqnarray*}
The horizontal part of the operator $F$ can now be written as:
\begin{equation}
\label{bmn}
<\tilde{\rho}>_{x, y} \left[ \dfrac {1} {<{d_{xx}}^2>_{xy}} \delta _{x^2}
+ \dfrac{1}{<{d_{yy}}^2>_{xy}}\delta _{y^2} \right]
\tilde{\Phi} _{m n} = b_{mn}  \tilde{\Phi}_{m n}
\end{equation}
where the eigenvalues are defined by:
\begin{displaymath}
b_{mn} = - 4 <\tilde{\rho}>_{x,y} \left[
\dfrac{\sin ^2 \left( \frac {\pi} {imax} m \right)} {<{d_{xx}}^2>_{xy}}   +
\dfrac{\sin ^2 \left( \frac {\pi} {jmax} n \right)} {<{d_{yy}}^2>_{xy}}
\right]
\end{displaymath}
where $m\,=\,0, \, 1...imax-1$ and $n\,=\,0,\, 1,...jmax-1$.\\
%%%%%%%%%%%%%%%%%%%%%%%%%%%%%%%%%%%%%%%%%%%%
$\bullet$ \underline{Open and Davies' cases:}\\
%%%%%%%%%%%%%%%%%%%%%%%%%%%%%%%%%%%%%%%%%%%%
The degeneracy of the full discrete operator with orography gives:
\begin{eqnarray*}
\dfrac{\partial \Phi}{\partial x} \rightarrow \delta _x \Phi \\
\dfrac{\partial \Phi}{\partial y} \rightarrow \delta _y \Phi
\end{eqnarray*}
We obtain an homogeneous problem by changing our variable $\Phi$ at the last
point to have an homogeneous condition.
 Then, we employ a FFT cosine decomposition expressed as:
\begin{equation}
\label{foux2}
\Phi (I, J, K) =    \sum_{m=0}^{imax-1} \sum_{n=0}^{jmax-1}
\widehat{\Phi}_{m \, n}(K)
\cos \left( \dfrac {(2I - 1) \pi}{2 \, imax} m\right) \times
\cos \left( \dfrac {(2J - 1) \pi}{2 \, jmax} n\right)
\end{equation}
In this case, the eigenvalues of $F$ are expressed as:
\begin{displaymath}
b_{mn} = - 4 <\tilde{\rho}>_{x,y} \left[
\dfrac{\sin ^2 \left( \frac { \pi} {2 \, imax}  m\right)} {<{d_{xx}}^2>_{xy}}
+
\dfrac{\sin ^2 \left( \frac { \pi} {2 \, jmax}  n\right)} {<{d_{yy}}^2>_{xy}}
\right]
\end{displaymath}
where $m\,=\,0, \, 1...imax-1$ and $n\,=\,0,\, 1,...jmax-1$.\\
%%%%%%%%%%%%%%%%%%%%%%%%%%%%%%%%%%%%%%%%%%%%
$\bullet$ \underline{Mixed boundary conditions:}\\
%%%%%%%%%%%%%%%%%%%%%%%%%%%%%%%%%%%%%%%%%%%%
Here, the decomposition of $\Phi$ results from a combination of (\ref{foux1})
and (\ref{foux2}). If we have a cyclic condition along $x$ and an open
condition along $y$, we have:
\begin{equation}
\Phi (I, J, K) =   \sum_{m=0}^{imax-1} \sum_{n=0}^{jmax-1}
\widehat{\Phi}_{m\, n}(K)
\left[ e^ {i\frac {2\pi}{imax} m I } \times
\cos \left( \dfrac {(2J - 1) \pi}{2 \, jmax} n\right) \right]
\end{equation}
and the eigenvalues are defined by:
\begin{equation}
b_{mn} = - 4 < \tilde{\rho} >_{x,y} \left[
\dfrac{\sin ^2 \left( \frac {\pi} {imax} m \right)} {<{d_{xx}}^2>_{xy}} +
\dfrac{\sin ^2 \left( \frac {\pi} {2 \, jmax}  n\right)} {<{d_{yy}}^2>_{xy}}
\right]
\end{equation}
In the other case, if we have an open condition along $x$ and a cyclic
condition along $y$, the decomposition of $\Phi$ reads:
\begin{equation}
\Phi (I, J, K) =   \sum_{m=0}^{imax-1} \sum_{n=0}^{jmax-1}
\widehat{\Phi}_{m\, n}(K)
\cos \left( \dfrac {(2I - 1) \pi}{2 \, imax} m\right) \times
e^ { i\frac {2\pi}{jmax} n J }
\end{equation}
and the eigenvalues are expressed by:
\begin{equation}
b_{mn} = - 4 < \tilde{\rho} >_{x,y} \left[
\dfrac{\sin ^2 \left( \frac { \pi} {2 \, imax}  m\right)} {<{d_{xx}}^2>_{xy}}
 +
\dfrac{\sin ^2 \left( \frac { \pi} {jmax}  n\right)} {<{d_{yy}}^2>_{xy}}
 \right]
\end{equation}
%%
%%%%%%%%%%%%%%%%%%%%%%%%%%%%%%%%%%%%%%%%%%%%%%%%%%%%%%%%%%%%%%%%%%%%%%%%%%%%%%%
%%
\subsubsection{Vertical discretization}
%%%%%%%%%%%%%%%%%%%%%%%%%%%%%%%%%%%%%%%%%%%%%%%%%%%%%%%%%%%%%%%%%%%%%%%%%%%%%%
The discretized vertical operator is applied after the horizontal operator with
boundary conditions derived from the full problem with orography:
\begin{equation}
\left( \tilde{\rho}  \overrightarrow{\nabla} \Phi \right) ^{c3}
\rightarrow \dfrac{\overline{< \tilde{\rho} >_{x,y}}^z} {<{d_{zz}}^2>_{xy}}
\delta _z \Phi
\end{equation}
Thus, we have:
\begin{displaymath}
F \tilde{\Phi} _{m n} (k) =
a(k) \tilde{\Phi} _{m n }  (k-1) + b(k) \tilde{ \Phi} _{m n} (k)  +  c(k)
\tilde{\Phi} _{m n} (k+1)
\end{displaymath}
where
\begin{eqnarray*}
a(k) &=& \frac {< \tilde{\rho} >_{x,y}(k-1) + < \tilde{\rho} >_{x,y}(k)}
 {2 \,\Delta z(k)^2} \\
b(k) &=& -  \dfrac {< \tilde{\rho} >_{x,y}(k-1) +  < \tilde{\rho} >_{x,y}(k)}
{2 \, \Delta z(k)^2}
-  \dfrac { < \tilde{\rho} >_{x,y}(k)+< \tilde{\rho} >_{x,y}(k+1)}
{2 \, \Delta z(k+1)^2}  + b_{mn} \\
c(k) &=& \frac {< \tilde{\rho} >_{x,y}(k) + < \tilde{\rho} >_{x,y}(k+1) }
 {2 \, \Delta z(k+1)^2}
\end{eqnarray*}
%%%%%%%%%%%%%%%%%%%%%%%%%%%%%%%%%%%%%%%%%%%%%%%%%%%%%%%%%%%%%%%%%%%%%%%%%%%%
\subsubsection{Complete flat operator}
%%%%%%%%%%%%%%%%%%%%%%%%%%%%%%%%%%%%%%%%%%%%%%%%%%%%%%%%%%%%%%%%%%%%%%%%%%%%%
Thus, the flat operator is expressed under the form of $m n$ tridiagonal
matrices of $kmax+2 \times kmax+2$ elements with the following form:
\begin{displaymath}
{\bf F_{mn}}=
\left(
\begin{array}{cccccc}
b(jpvext) & c(jpvext)         &   &   &   &   \\
a(1+jpvext) & b(1+jpvext) & c(1+jpvext)          &   &   &   \\
     &      &                &   &   &   \\
     &      &             &   &   & \huge{0}  \\
     &      &                 &   &   &   \\
     &      &                 &   &   &   \\
\huge{0}     &      &               &   &   &   \\
     &  a(kmax+jpvext)    &  b(kmax+jpvext)         &  c(kmax+jpvext)&   &   \\
     &      &                      a(kmax+1+jpvext) &  b(kmax+1+jpvext) & &\\
\end{array}
\right)
\end{displaymath}
The $ F_{mn}$ components do not depend on time and are
stored in arrays of:\\
- $imax \times jmax \times (kmax+2)$ elements for the $b$ coefficients;\\
- $kmax+2$ elements for the $a$ and $c$ coefficients.\\
%%
All the matrices $F_{mn}$ are invertible except for
($m, n$) = (0, 0) (the uniform mode) because the vertical Neumann conditions
involve an evaluation of the pressure terms to within a constant.
For this reason, this particular mode is inverted apart. Moreover, as all the
$F_{mn}$ vertical matrices are independent,
this computation can easily be vectorized for the $m \, n$ different modes.

{\bf Algorithm Performance}

Both iterative methods presented above showed a good convergence for the
solution of the elliptic equation. All in all, the Richardson method
requires less computation time than the CG method, because it needs less
matrix inversions.

{\bf Current Limitations}

At the present time, the algorithm is not adapted for large stretching
of the horizontal grid. A practical maximum ratio of the larger to the
smaller grid size is about two, which is sufficient to allow simulations
on small domains in conformal projection.
