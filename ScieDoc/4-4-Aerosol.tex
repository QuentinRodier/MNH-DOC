\chapter{Aerosol Schemes}
%
% Mise à jour: Celine Mari, CNRS/LA (20/03/08)
%
\minitoc
Aerosols are treated in two different ways in Meso-NH. 
First, aerosols are treated independently from 
the gas phase chemistry with a prescribed chemical composition. This is the 
case for dust or sea-salt aerosols. 
These aerosols however can interact with the radiative code and/or the cloud 
dynamics. Second, organic and inorganic aerosols (anthropogenic aerosols) can be fully coupled with the
gas phase chemistry allowing subsequent interactions with gaseous 
source precursors. 
%%%%%%%%%%%%%%%%%%%%%%%
\section{Dust aerosols and sea salt}
%\author{Alf Grini, P. Tulet CNRM/GMEI}
%%%%%%%%%%%%%%%%%%%%%%%
\subsection{General theory}

Dust is mobilized from dry desert surfaces when the wind friction 
speed reaches a threshold wind friction speed of approximately
0.2 m~s$^{-1}$. Dust is an important aerosol with annual global
 emissions ranging from 1000 to 3000 Tg~yr$^{-1}$ and average global
load around 10-30 Tg \citep{Zender2004}.

Dust aerosols both scatter and absorb solar radiation, and they absorb and
re-emit terrestrial radiation. Since the dust is emitted to the atmosphere
when there are large winds, dust emission often occur in the form of
{\it dust storms} and optical depths near the source can be very large.

Sea salt aerosols are produced as film and jet droplets when bubbles, entrained in the water by breaking waves, disrupt the sea surface \citep{Blanchard1983}, and at winds speeds exceeding about 9~m~s$^{-1}$, by direct disruption of the wave tops (spume droplets) \citep{Monahan1983}.
Sea salt play a major role acting as a marine CCN and modifying the marine clouds properties.


\subsection{Sources and transport}

A detailed description of the dust emissions processes is provided in a 
preceding chapter "Surface Processes Scheme".

Sea salt and dust are produced depending on the wind speed. 
The externalized surface
of Meso-NH takes care of the mobilization, and sends moments of the size 
distribution
to the atmospheric model. The size distributions of sea salt and dust are assumed to consist
of three lognormal modes each, and these modes are described by their $0^{th}$, $3^{rd}$ 
and
$6^{th}$ moment. The moments can be transformed as diagnostic variables of 
the size distribution ($\sigma$ (dispersion coefficient), $r$ (median radius) 
and $N$ (number concentration) following \citet{Tulet2005}.
The dust aerosols are parameterized following \citet{Grini2006}. For emission processes, dust is mobilized using the Dust Entrainment And Deposition model (DEAD) \cite[]{Zender2003} as a function of saltation and sandblasting as defined by \cite{Marticorena1995}.
Sea salt emission is parameterized upon either the formulation of \citet{Vignati2001}
(effective source function) or a lookup table defined by \citet{Schulz2004}.
Dust and sea salt are currently lost through sedimentation and rain out in convective clouds.
The sedimentation takes into account the size of the aerosol where large 
particles have larger sedimentation velocity 
(see Sedimentation - Dry deposition in next section).
The convective rainout is similar to gas chemistry and assumes that the 
aerosols are soluble in rain water with a solubility factor equal to 0.3. 
Explicit aqueous mass transfer of aerosol within clouds and rain droplets have been parameterized for dust, sea salt and anthropogenic aerosols (see section "Scavenging and aqueous mass transfer").

\subsection{Radiation and optical properties}

The dust influences the radiation scheme of Meso-NH.
Optical properties in the 
short wave are calculated using the Mie code SHDOM \citep{Evans1998} 
{\it http://nit.colorado.edu/~evans/shdom.html}
They are calculated off line, assuming a refractive index of:
\begin{itemize}
\item Ri[1]="(1.448,-0.00292)" for  wavelengths comprises between 0.185 $\mu$m and 0.25 $\mu$m
\item Ri[2]="(1.448,-0.00292)" for  wavelengths comprises between 0.25 $\mu$m and 0.44 $\mu$m
\item Ri[3]="(1.448,-0.00292)" for  wavelengths comprises between 0.44 $\mu$m and 0.69 $\mu$m
\item Ri[4]="(1.44023,-0.00116)" for  wavelengths comprises between 0.69 $\mu$m and 1.19 $\mu$m
\item Ri[5]="(1.41163,-0.00106)" for  wavelengths comprises between 1.19 $\mu$m and 2.38 $\mu$m
\item Ri[6]="(1.41163,-0.00106)" for  wavelengths comprises between 2.38 $\mu$m and 4.0 $\mu$m
\end{itemize}
See \citet{Tulet2008} for more explanations.

 The results of the Mie calculations are stored in a look up
table as {\it asymetry factor $[no~unit]$}, 
{\it single scattering albedo $[no~unit]$} and 
{\it extinction coefficient $[m^2~g^{-1}]$}.
These values are looked up as a function of wavelength, number median radius 
and dispersion coefficient of the aerosols. 

The long wave code is not modified for special treatment of dust. 
There is a default {\it Longwave absorption coefficient} defined in 
{\it yoesw.f90} of the radiation code. The coefficient is used
to calculated absorption of terrestrial radiation from optical depth at 
550~nm. Although this treatment could be refined, the longwave effect of 
dust is supposed to be smaller than the shortwave effect \citep{Myhre2003}. 

%-----------------------------------------------------

%%%%%%%%%%%%%%%%%%%%%%%
\section{Organic and inorganic log-normal aerosols model (ORILAM)}
%\author{Pierre Tulet, CNRM/GMEI}
%%%%%%%%%%%%%%%%%%%%%%%

The prognostic evolution of the aerosol size distribution is determined by a 
general dynamical
equation \citep{Friedlander1977,Seinfeld1997} without analytical 
solution: 

\begin{equation}
\frac{\partial n(r_p)}{\partial t} = f(n(r_p))
\label{eqanal}
\end{equation}
where $n$ is the function of aerosol size distribution (particles cm$^{-3}$) 
and $r_p$ is the aerosol radius ($\mu$m).
This equation can be integrated to obtain an equation system such as: 
\begin{equation}
\frac{\partial M_k}{\partial t} = f(M_k)
\label{genmoment}
\end{equation}
where the $k^{th}$ moment is given by $M_k = \int_{0}^{\infty} r_{p}^{k} n(r_p) 
dr_p$ ($\mu$m$^k$ cm$^{-3}$).
Using several assumptions (choice of the aerosol spectral distribution) the 
equation (\ref{genmoment})
can be closed giving $f(M_k)$ in terms of moments.
Three modes have been implemented; the first one to represent the new particles 
formed (nuclei mode); the second one
for bigger and evoluted particles (accumulation mode); the last one is devoted to 
dust particles and introduced as a passive mode (no interaction with the nuclei and 
accumulation mode). Each mode is represented by a log-normal distribution as:
\begin{equation}
 n(\ln D) = \frac{N}{\sqrt{2 \pi} \ln{\sigma_g}} \exp \left(- \frac{\ln^2 (\frac{D}{D_g})}{2 \ln^2(\sigma_g)}\right) \label{distribution}
\end{equation}
where $N$ is the particle number concentration (in particles cm$^{-3}$), $D$ is the particle diameter ($\mu$m) and $D_g$
, $\sigma_g$ are respectively the number median diameter and the geometric 
standard deviation of the modal distribution.
The $k^{th}$ moment of the mode i is defined as :
\begin{equation}
M_{k,i} = \int_{0}^{\infty} r^k n_i(r) dr
\label{defmomentint}
\end{equation}
After integration (variable change as $ x = \ln (r/D_g) / \ln(\sigma)$ ) we 
obtain:

\begin{equation}
M_{k,i} =  N R^{k}_g \exp \left(\frac{k^2}{2} \ln^2 (\sigma)\right)
\label{defmoment}
\end{equation}

A simple combination with equation (\ref{defmoment}) gives a relationship between 
$M_k$ and the log-normal parameters $\sigma_g$, $R_g$ and $N$:
\begin{equation}
N = M_0 \\
\label{number}
\end{equation}
\begin{equation}
R_g = \left( \frac{M_3^4}{M_6 M_0^3} \right)^{1/6} \\
\label{radius}
\end{equation}
\begin{equation}
\sigma_g = \exp \left( \frac{1}{3} \sqrt{\ln \left( \frac{M_0 M_6}{M_3^2} 
\right)} \right)
\label{sigma}
\end{equation}


For aerosol with many monomers, it is possible to modelize the aerosol size 
distribution by 
a continuous function $n$ relative to mean radius $r_p$. The general dynamical 
equation is given by \citet{Friedlander1977}:
\begin{equation}
\frac{\partial n(r_p)}{\partial t} = (f_{convection} + f_{diffusion} + 
f_{coagulation} +
f_{growth} + f_{sources/sink} + f_{external})(n(r_p))
\label{GDE}
\end{equation}
This equation can be integrated in terms of moments of the distribution as:
\begin{equation}
\frac{\partial M_k}{\partial t} = (f_{convection} + f_{diffusion} + 
f_{coagulation} +
f_{growth} + f_{sources/sink} + f_{external})(M_k)
\label{GDE_M}
\end{equation}

The aerosol dynamics are modeled  as described by
\citet{Whitby1991,Binkowski1995,Ackermann1998,Binkowski2003} 
with notable differences: (1) We chose to integrate 3 moments (0, 3 and 6th) as 
prognostic variables. This
procedure permits us to keep all parameters of the modal distribution variable.
(2) Different sets of parameterization of sulfate nucleation and inorganic 
chemistry solvers are given.
(3) The aerosol module is coupled on-line with meteorological fields and chemical 
species (organic condensation).
(4) Sedimentation is integrated analytically for all moments.
(5) Surface exchange is coupled to a mesoscale 
atmosphere/biosphere model.

One can note that the moments of order 0 and 3 are well known ; the integration 
of $M_k = \int_{0}^{\infty} r_{p}^{k} n(r_p) dr_p$ gives for $M_{0,i}=N_i$ where 
$N_i$ is the total concentration of particles for mode i and $M_{3,i}= 
\frac{3}{4 \Pi} V_i$  is a direct function of the total volume of mode i.

\subsection{Coagulation}
\subsubsection*{General description}
Aerosol size distribution evolves by collision between particles, leading to 
coagulation process. Numerical cost of coagulation treatment is expensive but less expensive when
a log-normal approach is used. 

Several assumptions has been made to solve the binary coagulation:
(1) A collision between two particles forms a new particle
(2) The new particle is spherical
(3) The new volume is equal to the sum of both initial particle volumes.
Moments of the new particle formed after collision of particles of respective radius 
$r_{p1}$ and $r_{p2}$ is 
\begin{equation}
r_{p12}^k = (r_{p1}^3 + r_{p2}^3)^{\frac{k}{3}}
\label{rp12}
\end{equation}
It is necessary to consider coagulation as a transfer process  of particles in 
the lognormal distribution: to update the moment evolution due to coagulation, 
 we first consider the loss in moment due to extinction of 
both initial particles 
 $r_{p1}$ and $r_{p2}$ (term ($r_{p1}^k + r_{p2}^k$ )), and the supply of moment 
due to the
 creation of a new particle  $r_{p12}$ (term  $r_{p12}^k$). 
 The coagulation process can be integrated:
 \begin{eqnarray}
\frac{\partial M_k}{\partial t} = \frac{1}{2} \int_{0}^{\infty} 
\int_{0}^{\infty} r_{p12}^k
\beta(r_{p1},r_{p2}) n(r_{p1}) n(r_{p2}) dr_{p1} dr_{p2} \nonumber \\
- \frac{1}{2} \int_{0}^{\infty} \int_{0}^{\infty} (r_{p1}^k + r_{p2}^k)
\beta(r_{p1},r_{p2}) n(r_{p1}) n(r_{p2})  dr_{p1} dr_{p2}
\label{coag1}
\end{eqnarray}
with $\beta(r_{p1},r_{p2})$ representing the coagulation rate between particles 
$r_{p1}$ and 
$ r_{p2}$ in cm$^3$ s$^{-1}$.

For the particular case of $N=2$, different modes i and j  ($N > 2$ is an 
extrapolation of results below),  we 
obtain from (\ref{coag1}) and (\ref{rp12}):

 \begin{eqnarray}
\frac{\partial M_k}{\partial t} = \frac{1}{2} \int_{0}^{\infty} 
\int_{0}^{\infty} (r_{p1}^3 + r_{p2}^3)^{\frac{k}{3}}
\beta(r_{p1},r_{p2}) (n_i + n_j)(r_{p1}) (n_i + n_j)(r_{p2}) dr_{p1} dr_{p2} 
\nonumber \\
- \frac{1}{2} \int_{0}^{\infty} \int_{0}^{\infty}
(r_{p1}^k + r_{p2}^k)  \beta(r_{p1},r_{p2}) (n_i + n_j)(r_{p1}) (n_i + 
n_j)(r_{p2}) dr_{p1} dr_{p2}
\label{coag2}
\end{eqnarray}
It is easy to solve terms of equation (\ref{coag2}) with the following convention:
(1) When two particles collide in the same mode (intra-modal coagulation), the 
new one stays in this mode.
(2) When two particles of different modes collide (inter-modal coagulation), the 
new one 
is in the mode with largest radius (here j).
The second convention implies for the inter-modal coagulation that each 
particle of the lowest
mode is transferred into the largest one.
Nevertheless, the intra-modal coagulation of the lowest mode (i) is able to 
reach beyond the largest mode (j). As a consequence, we use an hybrid approach 
discussed by \citet{Ackermann1998}. First, the model resolves the intersection 
diameter ($d_{eq}$) of both modes, with the equation:
 \begin{equation}
 \ln \left(\frac{N_i \ln \sigma_i}{N_j \ln \sigma_j}\right) = \frac{(\ln d_{eq} 
- \ln d_{pi})^2}{2\ln^2 \sigma_{gi}} - \frac{(\ln d_{eq} - \ln d_{pj})^2}{2\ln^2 
\sigma_{gj}}
\label{coag3}
\end{equation}
At the end of the time-step, particles of the lowest mode (Aitken mode) with diameter greater than  $d_{eq}$
are transferred into the largest one (Accumulation mode).

\subsubsection*{Coagulation rate}
Coagulation is represented by an harmonic function which is an
average between both coagulation limit regimes : 'free-
molecular' and 'near continuum' \citep{Whitby1991}. Knudsen number defined by  $Kn = 
\frac{\lambda}{r_p}$ permits to define
the nature of the relationship between atmospheric gas and the particle. For $Kn < 
0.1$, the particle is in a continuous fluid (continuum regime), whereas for $Kn > 
10$  the particle moves as a gas molecule (free molecular regime).

\begin{itemize}
\item Free molecular regime $Kn > 10$

In this regime, \citet{Friedlander1977} gave the expression of coagulation rate 
as:
\begin{equation}
\beta^{fm}(r_{p1},r_{p2}) = \left(\frac{6 k 
T}{\rho_p}\right)^{\frac{1}{2}}\left(\frac{1}{r^3_{p1}} 
+ \frac{1}{r^3_{p2}}\right)^{\frac{1}{2}} (r_{p1} + r_{p2})^2 
\label{coagfm}
\end{equation}
To integrate this equation, we need to make the same assumption as in 
\citet{Lee1984}:
 \begin{equation}
\left(\frac{1}{r^3_{p1}} + \frac{1}{r^3_{p2}}\right)^{\frac{1}{2}} \approx 
\left(\frac{1}{r^{\frac{3}{2}_{p1}}} 
                                                               + 
\frac{1}{r^{\frac{3}{2}_{p2}}}\right)
\label{coag_ap}
\end{equation}

Now the equation (\ref{coagfm}) is written on the form:
\begin{equation}
\widetilde{\beta}^{fm}(r_{p1},r_{p2})
= \left(\frac{6 k_B T}{\rho_p}\right)^{\frac{1}{2}}
\left(r^{\frac{1}{2}}_{p1} + 2 \frac{r_{p2}}{r^{\frac{1}{2}}_{p1}} + 
\frac{r_{p2}^2}{r^{\frac{3}{2}}_{p1}} 
+ \frac{r_{p1}^2}{r^{\frac{3}{2}}_{p2}} +  2\frac{r_{p1}}{r^{\frac{1}{2}}_{p2}} 
+ r^{\frac{1}{2}}_{p2}\right)
\end{equation}

With this assumption, we can compute the variation of the moment due to 
coagulation process. But it is clear that this approximation is valid only in a 
short range of particle radius. To minimize the limitation of the 
approximation, a correction factor is defined 
as (for the mode i and $k^{th}$ moment) :


\begin{equation}
%\label{coag_betafm}
b_{(6,i),intra} = \frac{\int_{0}^{\infty} \int_{0}^{\infty} r_{p1}^6 \beta^{fm} 
n_i(r_{p1}) n_i(r_{p2}) dr_{p1} dr_{p2}}
{\int_{0}^{\infty} \int_{0}^{\infty} r_{p1}^6 \widetilde{\beta}^{fm} n_i(r_{p1}) 
n_i(r_{p2}) dr_{p1} dr_{p2}}
\end{equation}

These factors are tabulated  in function of the log-normal 
parameters.
Finally, we obtain:

\begin{equation}
(\frac{\partial M_{6,i}}{\partial t})_{intra}  \approx - b_{6,i,intra} 
\int_{0}^{\infty} \int_{0}^{\infty} r_{p1}^6 \widetilde{\beta}^{fm} n_i(r_{p1}) 
n_i(r_{p2}) dr_{p1} dr_{p2}
%\label{coag_betafm}
\end{equation}


\item Near-continuum regime $Kn < 1$

For particles larger than their free mean path $\lambda$,
\citet{Friedlander1977} suggested
the following  expression for the coagulation rate:
\begin{equation}
\widetilde{\beta}^{nc}(r_{p1},r_{p2}) = 4 \Pi (D_{p1} + D_{p2}) (r_{p1} + 
r_{p2})
\label{coagne}
\end{equation}

where $D_{p} = (k_B T Cc / 6 \Pi \mu r_p)$,\\
the Cunningham coefficient $Cc=1 + 
Kn(a+b \exp(-c/Kn))$, with  a=1.126; b=0.42 and c= 0.87. \\
Equation (\ref{coagne}) cannot be integrated analytically. If we consider only the 
continuum/near-continuum regime, we can approximate Cc as:
\begin{equation}
Cc \approx 1 + A^{nc'} Kn
\label{Cc}
\end{equation}

with $ A^{nc'} = 1.392 Kn^{0.0783}$. After substitution, equation (\ref{coagne}) 
becomes:
\begin{equation}
\widetilde{\beta}^{nc}(r_{p1},r_{p2}) = \frac{2 k_B T}{3 \mu}\left(2+\lambda 
A^{nc'}_i 
\left(\frac{1}{r_{p1}} + \frac{r_{p2}}{r_{p1}^2}\right) + \lambda A^{nc'}_j 
\left(\frac{1}{r_{p2}} + \frac{r_{p1}}{r_{p2}^2}\right) + \frac{r_{p1}}{r_{p2}}
+ \frac{r_{p2}}{r_{p1}}\right)
\label{coagne_cor}
\end{equation}
As previously, correction factor can be considered. Nevertheless, the 
approximation used is
precise enough in the continuum/near-continuum regime to be exempted.

\item Generalization of Brownian Coagulation:\\
\citet{Seinfeld1997} developed a general formulation for the coagulation rate $
\beta(r_{p1},r_{p2})$ using \citet{Fuchs1964} formulation and a Cunningham 
coefficient
due to \citet{Philips1975}. But the expression is too complicated to be 
integrated in
a log-normal distribution approach.
That is why \citet{Whitby1991} proposed an alternative solution to compute all 
coagulation coefficients averaging previously expression of free-molecule and 
near continuum regime as:

\begin{equation}
\frac{\partial M_k}{\partial t}  \approx \frac{ (\frac{\partial M_k}{\partial 
t})^{fm} (\frac{\partial M_k}{\partial t})^{nc}}
{(\frac{\partial M_k}{\partial t})^{fm} +  (\frac{\partial M_k}{\partial 
t})^{nc}}
%\label{coag_betafm}
\end{equation}

\end{itemize}

\subsection{Gas-particles conversion}

Pre-existing particles grow by gaseous transfer upon their surface. A second way 
of gaseous-particles transfer is related to the formation of new particles by 
nucleation.

Just after the emission of a combustion particle, some of the gaseous species 
fix on the aerosol surface as an adsorption process. When these atmospheric 
molecules are either in sufficient number or on aerosol site with low curve 
radius, a phase change appears. At this stage, adsorption classical formalism on
dry surface is not applicable. If the surface film is composed by an unique 
constituent in balance with the gas phase, when the atmospheric concentration of 
the constituent increases,
molecules condense upon the aerosol surface to restore
the thermodynamic balance.\\
The condensation process is discontinuous: the partial pressure needs to exceed 
a critical step to allow the phase change. In this case, the aerosol 
surface is crucial to transfer gaseous molecules into aerosol by condensation.
The absorption process needs to have a pre-existing liquid film at the aerosol 
surface.
The problem becomes different: as soon as a quantity of a species appears in 
gaseous
phase, some molecules are transferred into particle phase by thermodynamical 
balance. In this model, we assume the aerosol is old enough to have a short
liquid film at the surface. 
So, absorption has been retained as the dominant process
of aerosol growth.\\
Gaseous species that interact with aerosol phase are from two different 
categories: mineral and organic species. Mineral species are fundamental to 
predict the condensation of water $H_2O$ and thus the aerosol growth.
Organic species include a large number of different 
species with particular specificity of solubility, saturation vapor pressure and hearthless.
The organic aerosol fraction is able to modify the aerosol hygroscopic 
specificity. It is necessary to distinguish the organic matter issued from urban 
and rural areas. The first one  is mainly primary (emitted) and hydrophobic 
whereas the second one is mainly secondary (condensed) and hydrophilic 
\citep{Saxena1986}.

\subsubsection*{Growth processing}
Several parameters, such as temperature, relative humidity, total aerosol surface and 
the condensation matter rate determine which one is the principal growth factor 
of the aerosol.  
\citet{Whitby1991} gave the growth rate of the $k^{th}$ moment relative to $i$ 
mode as:
\begin{equation}
\frac{\partial M_{k,i}}{\partial t} = \frac{2 k}{\Pi} \int_{0}^{\infty} r_p^{k-
3} \Psi_p(r_p)n_i(r_p) dr_p
\label{growth_gen}
\end{equation}
where $\Psi_p$ is the condensation law on particles ($\mu$m$^3$ s$^{-1}$). $\Psi_p$ 
can be separated in a $\Psi_T$ and $\Psi$ respectively independent and dependant 
on particle size.
\begin{equation}
\Psi_p = \Psi_T . \Psi
\label{psi}
\end{equation}

Equation (\ref{growth_gen}) can be written as:
\begin{equation}
\frac{\partial M_{k,i}}{\partial t} = \frac{2 k}{\Pi} \Psi_T I_{k,i}
\label{growth_sec}
\end{equation}

with 
\begin{equation}
I_{k,i} = \int_{0}^{\infty} r_p^{k-3} \Psi(r_p) n_i(r_p) dr_p
\label{ik}
\end{equation}
and 
\begin{equation}
\Psi_T = \frac{m_w . (P_l - P_{surf,l})}{\rho_l R T}
\label{psit}
\end{equation}

where $m_w$ is the molar mass of species $l$, $P_l$ the partial pressure of 
species l, $P_{surf,l}$ the pressure of species l at the particle surface,  
$\rho_l$ the volumic mass of $l$ and $T$ the ambient temperature of 
the system (the aerosol is supposed to be in thermal equilibrium with its 
environment).
$\Psi(r_p)$ is the size contribution with two asymptotic forms:
\begin{itemize}
\item Free molecular regime;
\begin{equation}
\Psi^{fm} = \pi \alpha \overline{c} r_p^2
\label{psifm}
\end{equation}
where $\alpha$ is the accommodation coefficient, $\overline{c}$ the kinetic 
velocity of vapor molecules ($\overline{c} = \sqrt{8 RT/\pi m_w}$).
Integration of equation (\ref{psifm}) gives:
\begin{equation}
I^{fm}_{k,i} = \pi \alpha \overline{c} M_{k-1,i}
\label{psifmint}
\end{equation}
\item Near continuum regime;
\begin{equation}
\Psi^{nc} = 4 \pi D_v r_p
\label{psinc}
\end{equation}
where $D_v$ the diffusivity of species $l$ in the air.
Integration of equation (\ref{psinc}) gives:
\begin{equation}
I^{nc}_{k,i} = 4 \pi D_v M_{k-2,i}
\label{psincint}
\end{equation}
\end{itemize}

Finally, with the same average as for coagulation we can approximate the general
form of $I_{k,i}$ as:
\begin{equation}
I_{k,i} = \frac{I^{fm}_{k,i} I^{nc}_{k,i}}{I^{fm}_{k,i} + I^{nc}_{k,i}}
\label{ik_fin}
\end{equation}
\citet{Pratsini1988} estimated that this kind of procedure to average 
growth process is a very good approximation in the transitional regime.
In our model, we assume thermodynamical equilibrium ($P_l = P_{surf,l}$) in (\ref{psit}).
We calculate $\delta \Psi_p$ as a diagnostic at the end of the timestep for use in (\ref{growth_gen}).

\subsubsection*{Nucleation}

To activate the nucleation process of aerosol, it is necessary to 
have
the partial vapor pressure of gas species greater than associated saturated 
vapor pressure.
Nevertheless  there is few knowledge about nucleation of organic matter. In this 
model, only the sulfur nucleation is considered. We choose the \citet{Kulmala1998} 
parameterization for its consistence with the classical theory of binary 
homogeneous
nucleation \citep{Wilemski1984} and for taking into account the hydratation 
effect.
The nucleation rate is parameterized as: 
\begin{equation}
\begin{split}
J = \exp & \left( 25.1289 N_{sulf}  - 4890.8 \frac{N_{sulf}}{T} -\frac{1743.3}{T} \right.  \\ 
& \left. -2.2479 \delta N_{sulf} RH + 7643.4 \frac{x_{al}}{T} - 1.9712 \frac{x_{al}}{RH} 
\right)
\end{split}
\label{nucleation}
\end{equation}

with  
$x_{al}=1.2233 - \frac{0.0154 RA}{RA+RH} + 0.0102 \ln(N_{av})-0.0415\ln(N_{wv}) 
+ 0.0016 T$ the molar fraction of $H_2SO_4$ in the critical nucleus (stable);\\
$N_{av}$ and $N_{wv}$ respectively the concentration  of sulfuric acid vapor and water vapor in 
cm$^{-3}$; \\
$T$ the atmospheric temperature (K), RA and RH absolute and relative humidity. 
$N_{sulf} = \ln\left(N_a / exp(-14.5125+0.1335T-10.5462 RH+1958.4(RH/T)) 
\right)$ 
is the logarithm ratio between $N_a$ (ambient concentration of sulfur acid in 
cm$^{-3}$) and $N_{a,c}$ the concentration sulfur acid need to reach a nucleation 
rate of $J= 1$ cm$^{-3}$ s$^{-1}$; 
and $\delta = 1 + \frac{T - 273.15}{273.15}$.
$N_{a,c}$ can be given by:
\begin{equation}
N_{a,c} = \exp (-14.5125 + 0.1335 T - 10.5462 RH + 1958.4 (RH/T)) 
\label{nac}
\end{equation}

\subsubsection*{Mineral thermodynamic balance}
In this version, two sets of mineral thermodynamical equilibrium has been 
introduced for 
the prediction of the balance between aerosol and gas phases of the  system $NH_3$-
$SO_4$-$HNO_3$-$H_2O$.
The first parameterization is ARES, a revised version of MARS \citep{Saxena1986}, 
developed by \citet{Binkowski1995}.
The second parameterization introduced is ISORROPIA from \citet{Nenes1998}. 

\subsubsection*{Heterogenous chemistry}
The aerosol phase can modify the gaseous composition by heterogeneous and 
multiphase reactions
\citep{Ravi1997}. Following \citet{Jacob2000}, we introduced the 
minimal set of reactions which is presented here with their associated uptake 
coefficients $\lambda$ to improve the  ozone model:

\begin{equation}
HO_2 \rightarrow 0.5 H_2O_2 \qquad   \lambda = 0.2 
\label{hetero1}
\end{equation}
\begin{equation}
NO_3 \rightarrow HNO_3   \qquad   \lambda = 10^{-3} 
\label{hetero2}
\end{equation}
\begin{equation}
NO_2 \rightarrow 0.5 HNO_3 + 0.5 HONO  \qquad    \lambda = 10^{-4}  
\label{hetero3}
\end{equation}
\begin{equation}
N_2O_5 \rightarrow 2 HNO_3 \qquad    \lambda \in [0.01,1] 
\label{hetero4}
\end{equation}
The first order rate constant for gas heterogeneous loss onto particles is given 
by:\\
\begin{equation}
ka = \sum_{k}\left(\frac{d_k}{2D_g} +  \frac{4}{\nu \lambda} \right)^{-1} A_k
\label{ka}
\end{equation}
with $d_k$ the particle diameter (m), $D_g$ the reacting gas molecular 
diffusivity (m$^2$ s$^{-1}$), $\mu$ 
the mean molecular velocity (m s$^{-1}$), $A_k$ the total surface area of mode 
$k$ and $\lambda$ the uptake coefficient of reactive species.

\subsubsection*{Organic condensation}

In the troposphere, Volatil Organic Compounds (VOCs) are mainly oxidized by $OH$ 
radical, $NO_3$ and $O_3$. Some of these products have a very low 
saturated vapor pressure to be absorbed and form SOA (secondary organic aerosol).
To take into account correct condensation process we need to restore 
thermodynamical balance for all species. 
With this aim, some new chemical schemes, such as CACM \citep{Griffin2002} 
distinguish VOCs products in
accordance with their capability to condense on the aerosol phase.
The default scheme (ReLACS) do not allow these secondary organic aerosol parent:
in this case the use of ORILAM do not compute organic condensation.
Otherwise, using CACM scheme (or it's reduced version ReLACS2), one can active two
different set of organic thermodynamic balance which are MPMPO \citep{Griffin2005} and the AER module of \citet{Pun2002}.  The SOA precursor lumped groups are partitioned to distinguish hydrophobic, hydrophilic structural characteristics \citet{Pun2002} together with sources (biogenic versus anthropogenic), volatility and potential dissociation  \citep{Griffin2005},  chemical and structural characteristics defined by \citet{Pun2002} in previous applications of CACM.  The predictions of lumped SOA precursors can be coupled with thermodynamic equilibrium modules for aerosol prediction.  SOA are group into ten different aerosol class.
To active SOA, it is necessary to compile CACM.chf or ReLACS.chf using the preprocessor m9 (or m10).

%%%%%%%%%%%%%%%%%%%%%%%%%%%%%%%%%%%%%%%%%%%
\subsection{Dry deposition and sedimentation}
%%%%%%%%%%%%%%%%%%%%%%%%%%%%%%%%%%%%%%%%%%%

Dry deposition and sedimentation of aerosols are driven by the Brownian 
diffusivity:
\begin{equation}
D_p = \left(\frac{k T}{6 \pi \nu \rho_{air} r_p} \right) C_c
\label{diff_bro}
\end{equation}
and by  the gravitational velocity:
\begin{equation}
V_g = \left(\frac{2 g}{9 \nu}\left(\frac{\rho_{p,i}}{\rho_{air}}\right) r_p^2 
\right) C_c
\label{vitesse_grav}
\end{equation}
where $k$ is the Bolzmann constant, $T$ the ambient temperature, $\nu$ the air 
kinematic velocity, $\rho_{air}$ the air density, $g$ the gravitational 
acceleration, $\rho_{p,i}$ the aerosol density of mode $i$, and $C_c = 1 + 
1.246\frac{\lambda_{air}}{r_p}$ the gliding coefficient. 
These expressions need to be averaged on the $k^{th}$ moment and mode $i$ as:
\begin{equation}
\hat{X} = \frac{1}{M_{k,i}} \int_{-\infty}^{\infty} X r^k_p n_i(\ln r_p)d(\ln r_p)
\label{moyenne}
\end{equation}
where $X$ represents either $D_p$ or $v_g$.
After integration, we obtain for  Brownian diffusivity:
\begin{equation}
\hat{D}_{p_{k,i}} = \tilde{D}_{p_{g,i}}\left[\exp\left(\frac{-2k+1}{2} 
\ln^2\sigma_{g,i} \right) + 1.246 Kn_g 
\exp \left(\frac{-4k+4}{2} \ln^2\sigma_{g,i}\right) \right]
\label{diff_brow}
\end{equation}
with $\tilde{D}_{p_{g,i}} = \left(\frac{kT}{6 \pi \nu \rho_{air} R_{g,i}} 
\right)$

and for gravitational velocity: 
\begin{equation}
\hat{Vg}_{p_{k,i}} = \tilde{Vg}_{p_{g,i}}\left[\exp\left(\frac{4k+4}{2} 
\ln^2\sigma_{g,i} \right) + 1.246 Kn_g \exp \left(\frac{2k+4}{2} 
\ln^2\sigma_{g,i}\right) \right]
%\label{gravi_vel}
\end{equation}
with $\tilde{Vg}_{p_{g,i}} = \left(\frac{2g \rho_{p,i}}{9 \nu \rho_{air}} 
R_{g,i}^2 \right)$

\subsubsection*{Dry deposition}
According to \citet{Seinfeld1997} and using the resistance concept of 
\citet{Wesely1989}, aerosol dry deposition
velocity for the $k^{th}$ moment and mode $i$ is:
\begin{equation}
\hat{v}_{d_{k,i}} = ( r_a + \hat{r}_{d_{k,i}} + r_a  \hat{r}_{d_{k,i}} 
\hat{Vg}_{p_{k,i}})^{-1} + \hat{Vg}_{p_{k,i}}
%\label{gravi_vel}
\end{equation}
where surface resistance $\hat{r}_{d_{k,i}}$ is given by
\begin{equation}
\hat{r}_{d_{k,i}} = \left[(\hat{Sc}_{k,i}^{-2/3} + 10^{-3/\hat{St}_{k,i}}) 
\left(1+ 0.24 \frac{w_*^2}{u_*^2} \right) u_* \right]^{-1}
\label{surfres}
\end{equation}
Schmidt and Stokes number are respectively equal to $\hat{Sc}_{k,i} = \nu / 
\hat{D}_{p_{k,i}}$ and
$\hat{St}_{k,i}= (u_*^2/g\nu)\hat{v}_{d_{k,i}}$.
One can observe that the friction velocity $u_*$ and the convective velocity 
$w_*$ depend on meteorological
and surface conditions. 

\subsubsection*{Sedimentation}
For sedimentation process, we can use the above parameterization of 
$\hat{Vg}_{p_{k,i}}$. When
vertical resolution is high, it is necessary to use 
a classical time splitting to compute sedimentation fluxes.
It can be noted that the sedimentation / deposition processing modifies the 
particle distribution
with  an important loss of large particles in comparison to the small ones. 
After integration 
of the three moments, the distribution does not preserve the log-normal shape. 
If we consider after 
sedimentation processing the distribution as log-normal, the  
reconstruction of
log-normal parameters $\sigma$, $R_g$ induces a decrease of $\sigma$ and an 
increase of $R_g$. 
The variation of $\sigma$ is stronger than the $R_g$ one. Nevertheless, in nature
sedimentation process must decrease simultaneously  $\sigma$ and $R_g$.
 Therefore, we cannot consider the integration of the 
three moments to solve this problem. Two choices are possible:\\
\begin{itemize}
\item Sedimentation process with $\sigma$ fixed :
$M_{6,i}$ can be computed by maintaining $\sigma$ equal to the previous values:
\begin{equation}
M_{6,i} = M_{0,i} \left(\frac{M_{3,i}}{M_{0,i}} \right)^{1/3}  \exp \left( -3/2 
\log(\sigma_g)^2\right)^6  \exp\left(18 \log(\sigma_g)^2\right)
\label{sigmafix}
\end{equation}

\item Sedimentation process with $R_g$ fixed :
A simple combination of $M_{k,i}$ gives $M_{6,i}$ in function of $M_{0,i}$, 
$M_{3,i}$, and $R_{g,i}$ as:

\begin{equation}
M_{6,i} = \frac{M_{3,i}^4}{R_{g,i}^6 M_{0,i}^3}
\label{rgfix}
\end{equation}

\end{itemize}

To decrease $\sigma$ and $R_g$, a solution is to consider alternatively
both treatment of $M_{6,i}$ for each time step.

%%%%%%%%%%%%%%%%%%%%%%%%%%%%%%%%%%%%%%%%%%%
\subsection{Scavenging and aqueous mass transfer}
%%%%%%%%%%%%%%%%%%%%%%%%%%%%%%%%%%%%%%%%%%%
Washout and cloud aerosol collection are calculated in the MesoNH based upon first order principals.  In- and below-cloud impaction scavenging by cloud droplets and raindrops uses a kinetic approach to calculate the aerosol mass transfer as:
\begin{equation}
\frac{dM_p}{dt}= - \Lambda_M M_p
\label{eq:scav}
\end{equation}  
where 
$dMp/dt$ represents the aerosol dry mass transfer in the acquous phase, $M_p$ the aerosol dry mass and $\Lambda_M$ the path normalized scavenging coefficient (in s$^{-1}$).
Upon \citet{Pruppacher2000} the normalized scavenging coefficient for Brownian motion between dry aerosol and cloud droplets is determined by the semi-empirical formulation as:
\begin{equation}
\Lambda_M M_p  = \frac{1.35 LWC D_p}{r^2_{cloud}}
\label{eq:cloud}
\end{equation}  
where $LWC$ is the Liquid Water Content in g cm$^{-3}$, $D_p$ is the diffusivity of the particle in m$^2$ s$^{-1}$ and $r_{cloud}$ the
cloud droplet radius in m.
Whereas the normalized scavenging coefficient between dry aerosol and rain droplets can be calculated as \citet{Seinfeld1997}:

\begin{equation}
\Lambda_M M_p  = \frac{3}{2} \frac{E}{r_{rain}} \, F_{rain}
\label{eq:rain}
\end{equation} 
where $E$ is the collection efficiency fully described in \citep{Seinfeld1997,Tost2006}, $r_{rain}$ the radius of the rain droplets (in mm) and $F_{rain}$ the effective precipitation flux in kg m$^{-2}$ s$^{-1}$.

Within this scheme, efficiencies for three types of collections are calculated.  Small particles are collected efficiently through Brownian diffusion, but the collection efficiency decreases with increasing particle size.  Inertial impaction is important for large particles, with collection efficiencies approaching one for particles of diameter greater than 20 $\mu$m.  Washout is the least efficient for particles with diameters ranging from 0.2 to 2.0 $\mu$m, where interception in difficult due to particles following the streamlines of air around the falling droplet.  
The in-cloud mass aerosol transfers into rain droplets by autoconversion and accretion processes have been introduced as described by \citet{Pinty1998}. The sedimentation of aerosol mass included in raindrops is solved using a time splitting technique with an upstream differencing scheme of the vertical flux as:

\begin{equation}
 P_{asr}  = \frac{m_{aero}}{m_{rain}\rho} \frac{d}{d z}(V_r \, \rho \, r_{rain})
\label{eq:sedim}
\end{equation} 

where $P_{asr}$ is the raindrop aerosol mass sedimentation rate, $m_{aero}$ the aerosol mass included in raindrops (in kg kg$^{-1}$ of air), $m_{rain}$ the rain water (in  kg kg$^{-1}$ of air), $\rho$ the air density and $V_r$ the raindrop sedimentation velocity (in m s$^{-1}$).
The re\-release of aerosols into the air due to rain evaporation is assumed to be proportional to the water evaporated \citep{Chin2000}.  However, this is likely to overestimate the release of aerosols due to evaporation as some evaporation of the rain results in smaller raindrops that still contain the aerosols.



%%%%%%%%%%%%%%%%%%%%%%%%%%%% BIBLIOGRAPHY %%%%%%%%%%%%%%%%%%%%%%%%
\begin{btSect}{4-4-Aerosol}
\section{References}
\btPrintCited
\end{btSect}
%%%%%%%%%%%%%%%%%%%%%%%%%%%% BIBLIOGRAPHY %%%%%%%%%%%%%%%%%%%%%%%%
