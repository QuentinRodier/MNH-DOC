%%%%%%%%%%%%%%%%%%%%%%%%%%%%%%%%%%%%%%%%%%%%%%%%%%%%%%%%%%%%%%%%%%%%%%%%%%%%%%
% CONTRIBUTION TO THE MESONH BOOK1: "ADVECTION SCHEMES FOR SCALAR VARIABLES"
% Author : J. Vila-Guerau de Arellano and J. P. Lafore
% Original : Juin, 1995
% Update   : Janvier, 1998
%%%%%%%%%%%%%%%%%%%%%%%%%%%%%%%%%%%%%%%%%%%%%%%%%%%%%%%%%%%%%%%%%%%%%%%%%%%%%%
%
%\begin{document}
%\setlength{\baselineskip}{15pt}
%\begin{center}
%\large
%%{\bf PROJET MESO-NH}\\[0.6cm]
%%\normalsize
%{\bf ADVECTION SCHEMES}\\[.3cm]
%%{\bf FOR SCALAR VARIABLES}\\[0.3cm]
%J. Vil\`a-Guerau de Arellano and J. P. Lafore \\[0.3cm]
%M\'et\'eo-France \\[0.3cm]
%{\today}\\[1.7cm]
%\end{center}
%\normalsize

\chapter{Positive Advection Schemes For Scalar Variables}
\minitoc

\section{Introduction}

Two positive definite schemes are presented in this section. The first one, namely the
flux-corrected transport scheme (FCT), is a centred scheme (spatially and
temporally) in which
the advective fluxes are corrected (if necessary) by a limiter factor to insure positiveness.
The second one, namely Multidimensional Positive Definite Advection Transport Algorithm
(MPDATA), is based on the upstream scheme. An iterative method based on the antidiffusive
velocities is applied to correct the excessive numerical diffusion of such scheme.
FCT and MPDATA are second-order accurate on space and time.

Both schemes are recommended to be used to simulate the advective transport of
quantities such as
microphysical variables
and atmospheric chemistry variables. Both schemes can be applied to describe
the advection of the following quantities: temperature, water substances,
turbulent kinetic energy, dissipation of turbulent kinetic energy and scalars
(chemical species). At Meso-NH the two schemes can be simultaneously apply to
different variables. However, in order to gain consistency, the above mentioned variables
have been classified in two groups: meteorological variables (temperature, water substances,
turbulent kinetic energy, dissipation of turbulent kinetic energy) and scalar variables.

Two 2D-horizontal tests have been carried out to evaluate these schemes. A scalar with
a cone distribution was advected in a rotating constant velocity field (Test 1) and
in a deformational velocity field (Test 2). The evaluation of the tests includes the
performance of the advection schemes as well as the computational costs.

\section{Flux-Corrected Transport}

\subsection{Description}

The advective part of the governing equation of a scalar quantity $\psi$
(temperature, water substances, TKE, dissipation of TKE and scalar (chemical species))
can be written

\begin{equation}
\label{cond1}
\dfrac{\partial}{\partial t}(\tilde{\rho}\psi) \, =
 \, - \, \dfrac{\partial }{\partial \overline{x}} (F_1)
 \, - \, \dfrac{\partial }{\partial \overline{y}} (F_2)
 \, - \, \dfrac{\partial }{\partial \overline{z}} (F_3),
\end{equation}

\noindent where

\begin{eqnarray}
\label{cond2}
F_1\,=\,\tilde{\rho} U^{c} \;  \psi \\
F_2\,=\,\tilde{\rho} V^{c} \;  \psi \\
F_3\,=\,\tilde{\rho} W^{c} \;  \psi,
\end{eqnarray}

\noindent and $U^{c},V^{c}$ and $W^{c}$ are the respective velocities.

At Meso-NH, the advection of a quantity is solved with a second order centred advection
scheme. The solution of equation (\ref{cond1}) is

\begin{equation}
\dfrac{\tilde{\rho} \psi^{t+\Delta t}} {2 \Delta t} \, = \,
\dfrac{\tilde{\rho} \psi^{t-\Delta t}} {2 \Delta t} \,
 \, - \, \sum_{I=1}^{3} \,  \dfrac{\partial }{\partial \overline{x}_I}
F_{I}^{t}
\end{equation}

\noindent where $2\Delta t$ is the leapfrog time step and $F_I$ are the three fluxes defined
in (\ref{cond2}) and calculated at the instant $t$.

The numerical solution of the advective part might produce oscillations and
negative values. This occurs if
the flux calculated is overestimated with respect the analytical value, i.e. the
monotonicity condition is not fulfilled at the lower limit. In order to obtain a
positive definite advection scheme, a limiter factor for the advective flux is
defined below. The development of the expression of the limiter factor follows
the approach suggested by Smolarkiwecickz and Grabowski (1990). In the previous
study, a nonoscillatory option was developed to correct the oscillations produced by
the MPDATA scheme.

Assuming that the Courant number (CFL) is
smaller than one, the scalar variable computed at $t+\Delta t$ must satisfy the
following condition

\begin{equation}
\label{cond3}
\psi_{MIN}^{t-\Delta t}\,\leq \,
\psi^{t+\Delta t} \, = \,
\dfrac{-2 \Delta t}{\tilde{\rho}\,\Delta \overline{x}_{I}}\, (F^{OUT}-F^{IN}) \,+\,
\psi^{t-\Delta t},
\end{equation}

\noindent where $\psi_{MIN}$ is the local minimum of the quantity and $F^{OUT}$ and $F^{IN}$ are
the total
outgoing flux and incoming flux, respectively. A sufficient condition for (\ref{cond3}) is

\begin{equation}
\label{cond4}
\psi^{t-\Delta t}  \, - \,
\psi_{MIN}^{t-\Delta t}\,\geq \,
\dfrac{2 \Delta t}{\tilde{\rho}\,\Delta \overline{x}_{I}}\, F^{OUT}
\end{equation}

The previous condition (\ref{cond4}) can be written as a flux limiter ratio ($\beta^{OUT}$)

\begin{equation}
\beta^{OUT}=
 \dfrac{ \psi^{t-\Delta t}- \psi_{MIN}^{t-\Delta t} }
{ \dfrac{2 \Delta t}{\tilde{\rho}\,\Delta \overline{x}_{I}} {F}^{OUT}}.
\end{equation}

If $\beta^{OUT} \geq 1$, the flux is calculated correctly.
If $\beta^{OUT}<1$, the outgoing flux is overestimated and therefore the flux limiter
ratio must be applied.
The corrected flux are

\begin{equation}
\label{cond5}
{F}_{I} = min(1,\beta^{OUT}_{I-1})[\,F_{I}]^{+}\,+\,
min(1,\beta^{OUT}_{I})[\,F_{I}]^{-}
\end{equation}

\noindent where $[\,.\,]^{+}\,\equiv\, max(0,.)$ and $[\,.\,]^{-}\,\equiv\,min(0,.)$.

\subsection{Discretization}

For the centred advection scheme (ADVECSCALAR routine in Meso-NH) the advective
flux are discretized according to

\begin{eqnarray}
F_1\,=\,
\overline{\tilde{\rho}}^{x}U^{c} \overline{\psi}^{x} \\
F_2\,=\,
\overline{\tilde{\rho}}^{y}V^{c} \overline{\psi}^{y} \\
F_3\,=\,
\overline{\tilde{\rho}}^{z}W^{c} \overline{\psi}^{z}.
\end{eqnarray}

The outgoing flux for the grid cell ($F^{OUT}$) (located at mass point) is

\begin{eqnarray}
F^{OUT}_{i,j,k}\,=\,[F^1_{i+1,j,k}]^+\,-\,[F^1_{i,j,k}]^-
\nonumber \\
\,+\,[F^2_{i,j+1,k}]^+\,-\,[F^2_{i,j,k}]^-
\nonumber \\
\,+\,[F^3_{i,j,k+1}]^+\,-\,[F^3_{i,j,k}]^-.
\end{eqnarray}

Only on the above expression and in order to clarify the notation, the fluxes
components ($F_1,F_2,F_3$) have been written as ($F^1,F^2,F^3$).

An absolute minimum ($\psi_{MIN}$) must be prescribed to calculate the limiter factor.
This mimimum can be, for instance, a
background value or as a default the value equal to zero.
Alternatively, one can prescribed a local minimum ($\psi_{MIN}^{t-\Delta {t}}$)
which is defined

\begin{equation}
\psi_{MIN}^{t-\Delta {t}}=
min[\psi^{t-\Delta {t}}_{i,j,k},
\psi^{t-\Delta {t}}_{i-1,j,k}, \psi^{t-\Delta {t}}_{i+1,j,k},
\psi^{t-\Delta {t}}_{i,j-1,k}, \psi^{t-\Delta {t}}_{i,j+1,k},
\psi^{t-\Delta {t}}_{i,j,k-1}, \psi^{t-\Delta {t}}_{i,j,k+1}].
\end{equation}

If one knows $F^{OUT}$ and $\psi^{MIN}$, the $\beta^{OUT}-$ratio
can be calculated

\begin{equation}
\beta^{OUT}=
 \dfrac{ \psi^{t-\Delta {t}}- \psi_{MIN}^{t-\Delta {t}}   }
{\dfrac{ {2 \Delta t} \, {F}^{OUT}}{\tilde{\rho}} +\varepsilon },
\end{equation}

\noindent where $\varepsilon$ is a small value, for example $10^{-15}$. Finally,
the discretization of the three components of the corrected flux yields

\begin{eqnarray}
{F_1} = min(1,\beta_{i-1,j,k}^{OUT})\,[\,F_1]^{+}\,+\,
min(1,\beta^{OUT}_{i,j,k})\,[\,F_1]^{-}\\
{F_2} = min(1,\beta_{i,j-1,k}^{OUT})\,[\,F_2]^{+}\,+\,
min(1,\beta^{OUT}_{i,j,k})\,[\,F_2]^{-}\\
{F_3} = min(1,\beta^{OUT}_{i,j,k-1})\,[\,F_3]^{+}\,+\,
min(1,\beta^{OUT}_{i,j,k})\,[\,F_3]^{-}.
\end{eqnarray}

Once the advective fluxes are calculated and corrected, the advection of the
scalar can be calculated (r.h.s) of equation (\ref{cond1}). Its discretize form
form is

\begin{equation}
-\,\delta_{x}F_1\,-\,\delta_{y}F_2\,-\,\delta_{z}F_3.
\end{equation}

\subsection{Boundary Conditions}

The following boundary conditions are prescribed for the limiter factor $\beta^{OUT}$

- Cyclic boundary conditions

\begin{equation}
\beta^{OUT}_{b-1}=\beta^{OUT}_{e}
\end{equation}

- Open boundary conditions

\begin{equation}
\beta^{OUT}_{b-1}=1.
\end{equation}

\section{Multidimensional Positive Definite Advection Transport Algorithm (MPDATA)}

The MPDATA can be summarized as follows.
First, the advection of a quantity is solved by means of an upstream scheme
(Rood, 1987). This is the first iteration in the MPDATA scheme.
Second, the excessive numerical diffusion produced by such scheme is
corrected reapplying the scheme, but now one substitutes the velocity field by
introducing an
antidiffusive velocity field. The antidiffusive velocity is derived analytically
based on the truncation error analysis of the upstream scheme.
A nonoscillatory option can be applied to the MPDATA scheme to assure monotonicity.
Such procedure may be repeated an optional number
of times. For two iterations, the  MPDATA is a second-order-accurate in time
and space for any
advective velocity field.
The properties
of this scheme are: stability, consistency and conservation of positive
definitivess.

For more detailed information, the user is addressed to the following articles:
Smolarkiewicz (1983) (an introduction to the scheme in 1-D form), Smolarkiewicz
(1984) (extension of the scheme to a fully multidimensional form), Smolarkiewicz
and Clark (1986) (application to the scheme to a time-dependent velocity field
and generalized form of the scheme
to the anelastic continuity equation) and Smolarkiewicz and
Grabowski (1990) (nonoscillatory option in order to preserve monotonicity).

The formulation and notation presented in this manual closely follow (when
it is possible) the one
presented in the last two papers above mentioned. However, the MPDATA scheme
has been adapted to the following Meso-NH requirements: leap-frog time stepping and
code optimisation.

\subsection{MPDATA Description}

The advection equation of a quantity $\psi$ written on the computational
grid, takes the following form:

\begin{equation}
\dfrac{\partial}{\partial t}(\tilde{\rho}\psi) \, =
 \, - \, \dfrac{\partial }{\partial \overline{x}} (\tilde{\rho} U^{c} \;  \psi)
 \, - \, \dfrac{\partial }{\partial \overline{y}} (\tilde{\rho} V^{c} \;  \psi)
 \, - \, \dfrac{\partial }{\partial \overline{z}} (\tilde{\rho} W^{c} \;  \psi)
\end{equation}

\noindent For compactness it may be written:

\begin{equation}
\dfrac{\partial}{\partial t}(\tilde{\rho}\psi) \, =
 \, - \, \sum_{I=1}^{3} \, \dfrac{\partial }{\partial \overline{x}_I}
 (u_I \; \psi) \, =
 \, - \, \sum_{I=1}^{3} \, \dfrac{\partial F_I }{\partial \overline{x}_I}
\end{equation}

\noindent
 where $u_I = \tilde{\rho} U_{I}^{c}$ is the Ith component of the non-divergent
contravariant velocity and $F_I = u_I \psi = \tilde{\rho} U_{I}^{c} \psi$
is a flux of $\psi$ in that direction.

 The basic MPDATA
iteratively solves the advection equation (at the $l$th iteration) in the
following way

\begin{equation}
\dfrac{\tilde{\rho} \psi^{(*)^{l}}} {2 \Delta t} \, = \,
\dfrac{\tilde{\rho} \psi^{(*)^{l-1}}} {2 \Delta t} \,
 \, - \, \sum_{I=1}^{3} \,  \dfrac{\partial }{\partial \overline{x}_I}
F_I (\psi^{(*)^{l-1}} , \tilde{u}_I^{l} )
\end{equation}

\noindent
where $2\Delta t$ is the leapfrog time step and $l=1,...,L$.
 L is the total number of iterations.
$F_I$ is the donor-cell advective flux evaluated by means of the
upstream scheme.
% Note that when L=1 the algorithm
%results in the first-order accurate upstream scheme.

\begin{itemize}

\item {\em The initial and final values} are given by

\begin{equation}
\psi^{(*)^{0}}\equiv \psi^{t-\Delta t} \, \, \, and \, \, \, \, \, \, \, \,
\psi^{t+\Delta t}\equiv \psi^{(*)^{L}}
\end{equation}

\item {\em For the first iteration $(l=1)$} the scalar field is advected using
the velocity at the central time $t$, resulting in the the first-order accurate
upstream scheme.

\begin{equation}
\tilde{u}_I^{1}\equiv {u}_{I}^{t}
\end{equation}

\item {\em The next iterations $(l>1)$} increase the accuracy and
reduce the strong diffusive character of the upstream scheme. This
is done by introducing an
{\em antidiffusive velocities} $\tilde{u}_I^{l}$

If $\psi > 0$
\begin{eqnarray}
\tilde{u}_{I}^{l}  =
0.5 \left[ |{\tilde{u}_I^{l-1}}| \Delta \overline{x}_I
       -{2 \Delta t}\dfrac{(\tilde{u}_I^{l-1})^{2}} {\tilde{\rho}}  \right]
\dfrac{1}{\psi^{(*)^{l-1}}}
\dfrac{\partial \psi^{(*)^{l-1}}}{\partial \overline{x}_I} \,
\nonumber \\
\nonumber \\
- \sum_{J=1,J\neq I}^{3}  0.5({2 \Delta t})
\dfrac{ \tilde{u}_I^{l-1}\tilde{u}_J^{l-1} } {\tilde{\rho} \psi^{(*)^{l-1}}}
\dfrac{\partial \psi^{(*)^{l-1}}}{\partial \overline{x}_J}
\end{eqnarray}
If $\psi = 0$
\begin{equation}
\tilde{u}_I^{l}  = 0.
\end{equation}

 Note that the antidiffusive velocities
$\tilde{u}_I^{l}$ ($l>1$) are dependent on the $\psi$ variable, contrary to
the first iteration ($l=1$) for which all variables are advected by
the same wind field ${u}_{I}$. This feature has
important impact on the code, as it suggests to perform the iterative cycle
of correction ($l=2,...,$ $ L$) variable by variable, in order to save memory.

\end{itemize}

\subsection{Discretization}

\begin{itemize}

\item {\em Fluxes } are estimated by the following upstream algorithm

\begin{eqnarray}
\label{flux1}
F_1(\psi^{(*)^{l-1}}, \,\tilde{u}^{l}) = [\,\tilde{u}^{l}]^{+}
\psi^{(*)^{l-1}}_{i-1,j,k}\,
\, + \ [\,\tilde{u}^{l}]^{-}
\psi^{(*)^{l-1}}_{i,j,k}\\
\label{flux2}
F_2(\psi^{(*)^{l-1}}, \,\tilde{v}^{l}) = [\,\tilde{v}^{l}]^{+}
\psi^{(*)^{l-1}}_{i,j-1,k}\,
\, + \ [\,\tilde{v}^{l}]^{-}
\psi^{(*)^{l-1}}_{i,j,k}\\
\label{flux3}
F_3(\psi^{(*)^{l-1}}, \tilde{w}^{l}) = [\tilde{w}^{l}]^{+}
\psi^{(*)^{l-1}}_{i,j,k-1}\,
\, + \ [\tilde{w}^{l}]^{-}
\psi^{(*)^{l-1}}_{i,j,k}
\end{eqnarray}

\noindent with triplets $(i,j,k)$ corresponding to the variables position
on the computational grid.

\item {\em The "antidiffusive" velocities $(l>1)$}. In order to insure
that $\dfrac{1}{\psi} \dfrac{\partial \psi}{\partial \overline{x}_J}$
converges when $\psi$ tends to zero, the antidiffusive velocities are
discretized as

\begin{eqnarray}
\tilde{u}^{l}  =  0.5(2\Delta t) \left[{
\left[ \overline{\tilde{\rho}}^{x} \dfrac{sign(\,\tilde{u}^{l-1})}{2\Delta t}
                                      - \,\tilde{u}^{l-1}  \right] \, A \,
\, - \, {\,\tilde{u}^{l-1}}\,
\left[\overline{ \overline{B}^y \,\,+ \, \overline{C}^z }^{x} \right]}\right] \\
\tilde{v}^{l}  =  0.5(2\Delta t) \left[{
\left[ \overline{\tilde{\rho}}^{y} \dfrac{sign(\,\tilde{v}^{l-1})}{2\Delta t}
                                      - \,\tilde{v}^{l-1}  \right] \, B \,
\, - \, {\,\tilde{v}^{l-1}}\,
\left[\overline{ \overline{A}^x \,\,+ \, \overline{C}^z }^{y} \right]}\right] \\
\tilde{w}^{l}  =  0.5(2\Delta t) \left[{
\left[ \overline{\tilde{\rho}}^{z} \dfrac{sign(  \tilde{w}^{l-1})}{2\Delta t}
                                      -   \tilde{w}^{l-1}  \right] \, C \,
\, - \, {  \tilde{w}^{l-1}}\,
\left[\overline{ \overline{A}^x \,\,+ \, \overline{B}^y }^{z} \right]}\right]
\end{eqnarray}


\noindent where

\begin{eqnarray}
\label{equA}
A =
 \dfrac{ \, \tilde{u}^{l-1} \delta_{x} \psi^{(*)^{l-1}}   }
{ \overline{\tilde{\rho} \psi^{(*)^{l-1}}}^x +\varepsilon }   \\
\label{equB}
B =
 \dfrac{ \, \tilde{v}^{l-1} \delta_{y} \psi^{(*)^{l-1}}   }
{ \overline{\tilde{\rho} \psi^{(*)^{l-1}}}^y +\varepsilon }   \\
\label{equC}
C =
 \dfrac{    \tilde{w}^{l-1} \delta_{z} \psi^{(*)^{l-1}}   }
{ \overline{\tilde{\rho} \psi^{(*)^{l-1}}}^z +\varepsilon }
\end{eqnarray}

\noindent where $\varepsilon$ is a small value, for example $10^{-15}$, to
ensure $\tilde{u}^{l}=\tilde{v}^{l}=\tilde{w}^{l}=0$ when
$\delta_{x} \psi^{(*)^{l-1}}=\delta_{y} \psi^{(*)^{l-1}}=
\delta_{z} \psi^{(*)^{l-1}}=0$ or $\overline{\tilde{\rho} \psi^{(*)^{l-1}}}^x=
\overline{\tilde{\rho} \psi^{(*)^{l-1}}}^y=
\overline{\tilde{\rho} \psi^{(*)^{l-1}}}^z=0$.

It is important here to note that the discretization proposed for Meso-NH
differs from the discretization suggested by Smolarkiewicz and Clark (1986). As a
result, we have not observed the formation of oscillation on the tests carried
out with the MPDATA scheme. Consequently, the nonoscillatory option, has not been
included in the MPDATA scheme.

\item {\em The $CFL<1$ condition } limits the values of the antidiffusive velocities
in the following way

\begin{equation}
\mid\tilde{u}^{l}\mid\leq \dfrac{\tilde{\rho}}{2 \Delta t} \, \, \, , \, \, \, \, \, \, \, \,
\mid\tilde{v}^{l}\mid\leq \dfrac{\tilde{\rho}}{2 \Delta t} \, \, \, and \, \, \, \, \, \, \,
\mid\tilde{w}^{l}\mid\leq \dfrac{\tilde{\rho}}{2 \Delta t}
\end{equation}

\end{itemize}

\subsection{Boundary Conditions}

\begin{itemize}
\item{\em {First iteration ($l=1$)}}

 The scalar values $\psi$ are taken at the $t-\Delta t$ instant ($n-1$),
whose boundary values are prepared by routine BOUNDARIES depending on the
boundary condition type:

- Cyclic if prescribed.

- At lateral boundaries for the "wall" type and at upper and lower boundaries
\begin{equation}
  \dfrac{\partial \psi }{\partial n } = 0 \;\;\;\;\;\; either \;\;\;\;
\psi^{n-1}_{b+1} = \psi^{n-1}_{b}
\end{equation}

- And for open lateral b.c.
\begin{equation}
  \dfrac{\partial \psi - \psi_{LS} }{\partial n } = 0 \;\;\;\;\;\; either \;\;\;\;
\psi^{n-1}_{b+1} = \psi^{n-1}_{b-1}
 + {\psi_{LS}}^{n-1}_{b+1} -{\psi_{LS}}^{n-1}_{b-1}
\end{equation}

\noindent These boundary conditions tested for the numerical diffusion, are also
well suitable for the upstream scheme. Vertically the contravariant velocities
being zero, the b.c. has no impact.

\item{\em {Next iterations ($l>1$)}}

 There is no reason to apply the antidiffusive procedure at boundaries, as it
is already the case for the 2nd-order advection scheme. Indeed a diffusive
scheme allows to reduce spurious reflections that may occur at boundaries.
We therefore impose at all boundaries a zero antidiffusive velocity:

\begin{equation}
\tilde{u}^{l}_{{I }_b} = 0
\end{equation}

\noindent except for cyclic boundaries. In that later case cyclic boundaries
must applies to both $\tilde{u}_I^{l}$ and $\psi^{(*)^{l-1}}$.

\item{\em {Fluxes determination}}

 The fluxes expressions (Eq. \ref{flux1}, \ref{flux2} and \ref{flux3}) are well defined except at the first
point, but which does not correspond to a physical point. We can fill up this
point by the dummy value -999.

\end{itemize}

\section{Evaluation of FCT and MPDATA advection schemes}

Two 2D-horizontal tests were carried out to validate FCT and MPDATA. Test 1 is the
advection of a cone in a rotating field with constant velocity. A full description
of the test can be found in Smolarkiewicz (1984). The initial maximum
concentration $(C_{max})$ was 4. The solution after six full rotations are summarized in Table 1.
The computing time is calculated relative to the computer time used by the centred
second-order
scheme. Note that the leapfrog scheme produces negative values.


\begin{table}[htpb]
  \centering
  \caption{Results of Test 1}
  \begin{tabular}{p{3cm}p{3cm}p{3cm}p{3cm}p{3cm}}
    Advection scheme  &  $C_{max}$  &  $C_{min}$  & $C_{max}$(Smol84)  &  Computing time  \\
    \hline
    \hline
    Centred      &   2.65  &  -0.32 &  -    & 1.      \\
    FCT          &   2.64  &   0.   &  -    & 3.4     \\
    MPDATA(L=2)  &   2.015 &   0.   & 2.16  & 7.       \\
    MPDATA(L=3)  &   2.85  &   0.   & 3.17  &11.8      \\
  \end{tabular}
\end{table}

Test 2 is the advection of the same cone but now in a deformational flow field. The
description and the analytical solution of the test were done by Staniforth et al. (1987).
FCT and MPDATA performed very well in spite of it being an extremely stressful test.

From the evaluation and intercomparison of the two tests
the following conclusions can be drawn:

\begin{itemize}

\item{The results obtained with MPDATA are in close agreement with the ones obtained
by Smolarkiewicz (1984). However, and due to the different discretization, MPDATA
at Meso-NH is a little bit more diffusive. On the other hand, we have not found oscillations
on the scalar field distribution.}

\item{FCT is less diffusive than MPDATA. Carrying out more iterations with MPDATA the
numerical diffusion can be corrected but at expense of an increase of the computer time.}

\item{FCT is less computer expensive than MPDATA}

\item{MPDATA conserves better the original distribution of the scalar, i.e. more symmetric}

\end{itemize}

\section{References}

\parindent 0cm

Rood, R.B., 1987: Numerical advection algorithms and their role in
atmospheric transport and chemistry models.
{\it Reviews of Geophysics},  {\bf 25}, 71-100.
\por

Smolarkiewicz, P. K., 1983: A simple positive definite advection scheme with
small implicit diffusion.
{\it Mon. Wea. Rev.},  {\bf 11}, 479-486.
\por

Smolarkiewicz, P. K., 1984: A fully multidimensional positive definite advection
transport algorithm with implicit diffusion.
{\it J. Comput. Phys.},  {\bf 54}, 325-362.
\por

Smolarkiewicz, P. K. and Clark T. L., 1986: The multidimensional positive definite
advection
transport algorithm: further developments and applications.
{\it J. Comput. Phys.},  {\bf 67}, 396-438.
\por

Smolarkiewicz, P. K. and Grabowski W.W., 1990: The multidimensional
positive definite
advection
transport algorithm: nonoscillatory option.
{\it J. Comput. Phys.},  {\bf 86}, 355-375.
\por

Staniforth A., Cote J. and Pudykiewicz J., 1987: Comments on "Smolarkiewicz's
deformational flow". {\it Mon. Wea. Rev.},  {\bf 115}, 894-900.
\por

\end{document}
