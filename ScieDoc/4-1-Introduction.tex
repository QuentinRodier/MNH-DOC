%
%* ABBREVIATIONS
\newcommand{\ddt}{\frac{\partial}{\partial t}}
\newcommand{\rhod}{\rho_{\rm dref}}
\newcommand{\scalar}[1]{{s}_\ast^{(#1)}} 
\newcommand{\source}[1]{{\cal S}_\ast^{(#1)}}
\newcommand{\divergence}{{\vec\nabla\cdot}}
\newcommand{\NEQ}{N_{\rm eq}}
\newcommand{\NREAC}{N_{\rm reac}}
\newcommand{\avg}[1]{\overline{#1}}
\newcommand{\Pmuavg}{{P_\mu^{\rm avg}}}
\newcommand{\Pmusub}{{P_\mu^{\rm sub}}}
\newcommand{\fmuavg}{{f_\mu^{\rm avg}}}
\newcommand{\fmusub}{{f_\mu^{\rm sub}}}
%
% for reference list
\newcommand{\refs}[1]{{\small\par%
\noindent\global\hangindent=1cm\global\hangafter=1{#1}\par}}
%
% for dry deposition
\newcommand{\ul}[1]{$\underline{\mbox{#1}}$}
\newcommand{\oxy}{\mbox{O($^{1}$D)}}
\newcommand{\ogr}{\mbox{O($^{3}$P)}}
\newcommand{\oxyd}{\mbox{O$_2$}}
\newcommand{\ozo}{\mbox{O$_3$}}
\newcommand{\oxo}{\mbox{O$_4$}}
\newcommand{\sul}{\mbox{S$_2$}}
\newcommand{\hyd}{\mbox{H$_2$}}
\newcommand{\oio}{\mbox{O${_{4}}{^{2-}}$}}
\newcommand{\met}{\mbox{CH$_4$}}
\newcommand{\hem}{\mbox{N$_2$O}}
\newcommand{\dio}{\mbox{CO$_2$}}
\newcommand{\atm}{\mbox{atmosphere}}
\newcommand{\mol}{\mbox{molecules}}
\newcommand{\dep}{\mbox{deposition}}
\newcommand{\res}{\mbox{resistance}}
\newcommand{\vd}{\mbox{$v_d$}}
\newcommand{\degree}{\mbox{^{\circ}}}
%
%%%%%%%%%%%%%%%%%%%%%
%%% DOCUMENT BODY %%%
%%%%%%%%%%%%%%%%%%%%%
%
%\begin{document}
%
\chapter{Basics for the chemistry and aerosols}
\minitoc
%\chapter{Scientific documentation}
%
%%%%%%%%%%%%%%%%%%%%%%%%%%%%%%%%%%%%%%%%%%%%%%%%%%%%%%%%%%%%%%%%%%%%%%%%%%%%
\section{Overview}
%%%%%%%%%%%%%%%%%%%%%%%%%%%%%%%%%%%%%%%%%%%%%%%%%%%%%%%%%%%%%%%%%%%%%%%%%%%%
%
The present document describes the scientific aspects of the 
Meso-NH-chemistry code (MNHC in the following), i.e.~the basic equations and
the algorithms used to solve them,
the initialization of the chemical variables, 
the treatment of the boundary conditions, 
the chemical reaction system presently implemented and
how to modify it, and finally the zero dimensional version
of Meso-NH-chemistry: the time-dependant box model.
%
The objective of MNHC
is to serve as a research tool for comprehensive atmospheric
chemistry modeling at the mesoscale, the cloud scale and the
turbulence scale (LES-chemistry), rather than as an operative
model to forecast air quality.
To meet this goal, it was essential to keep the chemical reaction
scheme flexible and the formulations of rate constants general.
Priority has thus been given to flexible solutions rather than
to optimal computational performance.
This version treats gas phase chemistry, 
bulk cloud chemistry and aerosol dynamics and chemistry.

A brief overview of the current state of MNHC is given here:

\begin{itemize}
\item MNHC is consistent with the following features of Meso-NH:
  \begin{itemize}
    \item 0-D, 1-D, 2-D, and 3-D versions are supported in one code;
    \item Meso-NH scalar variables are used to host the chemical variables,
          which means that boundary conditions, advection and
          turbulent transport are treated in the same way as for the
          Meso-NH water variables (refer to the corresponding documentation
          of these parts for details),
          i.e.~full profit is taken from the positive advections schemes
          as well as the different parameterizations of turbulent mixing;
    \item the chemistry part is coded such as to be coherent
          with Meso-NH grid-nesting and parallelization algorithms;
    \item as the chemical calculations are mostly local, vectorization
          has been introduced on a line-by-line basis; that is
          every instruction is done in vectorized mode for a large
          number of grid points, thus yielding optimal performance
          on vectorizing architectures;
    \item the box model (0-D) version is entirely compatible
          with the Meso-NH procedures
          and uses the same code as the multi-dimensional version.
  \end{itemize}
\item At each time step of Meso-NH and for each physical grid point
      a set of stiff differential equations describing the chemical 
      evolution of the chemical variables is solved.
      Also at each point and each time step, the reaction
      rates are calculated as a function of the meteorological variables
      temperature, pressure, humidity, etc.
\item The following chemical solvers are presently available:
      SIS, LinSSA, Cranck-Nicholson, EXQSSA (QSSA with extrapolation),
      SVODE (Gear-type). Stiff solvers from the NAG library can be easily
      added, but are not activated for portability reasons (NAG is
      not available on all platforms).
\item The reaction scheme 
      that is presently implemented treats 73 chemical
      species and 237 photo-chemical reactions 
      (RACM, \citet{Stockwell1997})
      and represents the state of the art in 3-D atmospheric chemistry
      modeling  (see Annex~\ref{RACM}).
\item The implemented reaction scheme may be easily replaced by a
      completely different one (see for example a reduced scheme described
      in \citet{Crassier2000}).
      Starting from a file defining the reaction mechanism
      using a simple syntax, the corresponding Fortran90 subroutines
      are automatically generated. New reactions may thus be added easily.
\item The following points are treated in a basic way,
      but need further development in the future:
  \begin{itemize}
    \item Initialization of horizontally homogeneous fields using 1-D vertical 
        profiles for the different chemical species is available
        either at the preparation of the dynamical fields
        (PREP\_IDEAL\_CASE program), or when the model is (re)started.
        The initialization with more realistic heterogeneous fields 
        for real cases studies is provided by 
        the large scale chemical model MOCAGE developed at M\'et\'eo-France.
        Initialization of stratospheric ozone using an empirical
        PV-$\rm O_3$ relationship have been tested successfully in
        the past but is not part of the present version of Meso-NH.
    \item Photolysis rates can be read from a look-up table file,
        or may be calculated on-line with a radiative transfer model.
        In the 1-D version, modeled clouds may also be taken into
        account in the radiative algorithm. For the 3-D version, a 
        parameterization is used to correct the clear-sky photolysis rates for
        cloud cover. Future developments will include 
        coupling with the radiation scheme from Meso-NH for the effects of
        clouds on radiation.
    \item Lateral boundary conditions are treated by Meso-NH.
        There are thus no problems
        running with "cyclic" or "wall" boundary conditions, whereas
        the use of "open" boundaries requires extra coding by the user,
        that is the specification of the "large scale" forcing fields
        for the chemical variables.
    \item Temporal integration by Meso-NH is done using the "leapfrog" scheme
        (integration from time step $t-dt$ to $t+dt$, using time $t$
        to calculate the temporal derivative),
        whereas most stiff solvers work by definition in "forward" mode
        (integrating from time $t$ to $t+dt$,
        using time $t$ to advance in time).
        Three different methods to couple the temporal advance of
        Meso-NH with chemistry are implemented, using either a "split"
        technique or "centered" or "lagged" tendencies.
        An optimal coupling with respect to computing time would be
        the use of "forward" scalar variables in Meso-NH, which has not been
        implemented so far. The gain of a factor of nearly
        two in computing time should be possible when using this technique.
  \end{itemize}
\end{itemize}
%%%%%%%%%%%%%%%%%%%%%%%%%%%%%%%%%%%%%%%%%%%%%%%%%%%%%%%%%%%%%%%%%%%%%%%%%%%%
%* BASIC EQUATIONS
%%%%%%%%%%%%%%%%%%%%%%%%%%%%%%%%%%%%%%%%%%%%%%%%%%%%%%%%%%%%%%%%%%%%%%%%%%%%
\section{Basic equations}
%
The dynamical part of Meso-NH
can already carry an arbitrary number of passive scalars,
following the equation:
%
\begin{equation}
\ddt \big(\rhod \scalar{i}\big) + \divergence \big(\rhod \scalar{i} \vec U \big)
= \rhod \source{i}
\,,
\label{basic1}
\end{equation}
%
where $\source{i}$  stands for the effect of a diabatic or chemical process 
of the $i$-th scalar (chemical) variable.
The treatment of advection and turbulent diffusion of the chemical
species is treated elsewhere in this document,
in this chapter only the chemical source/sink terms $\source{i}$ are calculated.

Given a reaction mechanism with $\NREAC$ reactions for $\NEQ$ prognostic
chemical species,
each reaction $\alpha$ between $n_\alpha$
reactants $R_i$ and leading to the formation
of $m_\alpha$ products $P_j$ will be given as:
%
\begin{equation}
R_1 + R_2 + ... + R_{n_\alpha} \stackrel{k_\alpha}{\longrightarrow}
P_1 + P_2 + ... + P_{m_\alpha}
\,,
\end{equation}
%
where $k_\alpha$ is the $n_\alpha$-th order reaction rate for reaction $\alpha$.
This defines for each reaction a term
%
\begin{equation}
  T_\alpha = k_\alpha \prod_{l=1}^{n_\alpha} [R_l]
  \,.
\end{equation}
%
The source terms $\source{i}$ is then given by the expression:
%
\begin{equation}
  \source{i} = \sum_{\alpha=1}^{\NREAC} \sigma_{i\alpha}  T_\alpha
  \,,
\end{equation}
%
with
%
\begin{equation}
  \sigma_{i\alpha} = \left\{
    \begin{array}{rl}
      +1 & \mbox{if species $i$ is a product in reaction $\alpha$} \\
      -1 & \mbox{if species $i$ is a reactant in reaction $\alpha$} \\
      0 & \mbox{otherwise}
    \end{array}
  \right.\,.
\end{equation}
%
Note that species like $\rm O_2$, $\rm N_2$ and $\rm H_2O$
are usually treated as being  constant and are thus taken into account
in the expression of the reaction rate.
They do not appear as prognostic variables.
Unfortunately, due to big differences in the timescales of the
chemical reactions, the system:
%
\begin{equation}
  \ddt \scalar{i}\big|_{\rm chem} = \source{i}
\end{equation}
%
usually turns out to be what is called a ``stiff'' differential system
(the lifetime of the different chemical species may vary by up to
ten orders of magnitude). 
It is impossible to solve such a system with explicit methods, like
Runge-Kutta or Euler-explicit.
The required time step would be so small and the number
of steps to be taken hence so big that the accumulation of numerical
noise would destroy the physical solution.
Special solvers like Gear's or other implicit methods have to be used.
Also a number of fast hybrid methods, adapted to the specific problem
of chemistry have been developed.
Since there is no universal solver for those problems,
a number of different solvers have been implemented that may
be chosen as a function of the problem treated.

In order to solve the coupled set of equations (dynamics + chemistry), 
two possibilities exist: (a) use of operator splitting between transport
and chemistry, and (b) use of tendencies.
Both methods have been implemented.
The algorithm for the "split" option is as follows:
\begin{enumerate}
  \item integrate all physical contributions in variable $\tt XRSVS$
        and calculate from that an intermediate solution for time step $t+dt$;
  \item integrate with the stiff solver from $t-dt$ to $t+dt$,
        using that intermediate variable as initial value;
  \item overwrite the variable $\tt XRSVS$ with the result from the solver.
\end{enumerate}
The algorithm using tendencies 
(which is similar to the treatment of slow microphysics)
allows for two options: using scalar variables at $t$ ("centered") or
at $t-dt$ ("lagged") as input for the solver. The algorithm is as 
follows:
\begin{enumerate}
  \item integrate with the stiff solver from $t-dt$ to $t+dt$,
        using the variable $\tt XSVT$ ("centered") or $\tt XSVM$ ("lagged")
        as initial value;
  \item add the tendency between the initial and the final value
        given by the solver to the contents of variable $\tt XRSVS$.
\end{enumerate}
The temporal integration of chemistry could be speed up by a factor of two
if the scalar variables would be integrated using a forward scheme,
which would have to be applied only every two time steps of the model.
In this case, chemistry would also be called only every two time steps.
%%%%%%%%%%%%%%%%%%%%%%%%%%%%%%%%%%%%%%%%%%%%%%%%%%%%%%%%%%%%%%%%%%%%%%%%%%%%
% STIFF SOLVERS
%%%%%%%%%%%%%%%%%%%%%%%%%%%%%%%%%%%%%%%%%%%%%%%%%%%%%%%%%%%%%%%%%%%%%%%%%%%%
\section{Stiff solvers}
%
A solution of the system
\begin{equation}
  \frac{d}{dt} y_i(t) = f_i (y,t)
\end{equation}
is seek,
where $y$ denotes the vector of chemical concentrations ($i=1,..,\NEQ$) and
$f$ the first derivative of the chemical system at every grid point of 
the dynamical  model.
Most of the computing time consumed by the chemistry part of Meso-NH
is used by the stiff solver.
The following methods are presently implemented: 

\paragraph*{SIS (Semi-implicit-symmetric):} \citet{Ramaroson1992}
\par\noindent
The method is linearized versions of the semi-implicit Euler method,
meaning that for each time-step it needs a single inversion of a matrix
of the dimension of the number of prognostic chemical species.
The method SIS may be written as:
\begin{equation}
y_i^{n+1} = y_i^n + \Delta t\left({I}
          - \frac{\Delta t}2 {J}^n\right)_{ij}^{-1}f_j^n \,,
\end{equation}
where $J$ is the Jacobian matrix of the system and
$y$ are the chemical concentrations at time $n\Delta t$
and $(n+1)\Delta t$.

\paragraph*{LinSSA (linearized steady state approximation):}
\citet{Suhre1994}
\par\noindent
The method LinSSA is a hybrid of SIS and QSSA, given as:
\begin{equation}
y_i^{n+1} = y_i^n + \Delta t\left({P}
          - \frac{\Delta t}2 {J}^n\right)_{ij}^{-1}
          \frac12({P}+{I})_{jk}f_k^n \,,
\end{equation}
with the projector ${P}$ onto the prognostic variables defined as
\begin{equation}
P_{ij} =
\left\{
  \begin{array}{ll}
    0\quad&i\ne j\,\, \hbox{\rm or}\,\, i=j\not\in {\cal I}\cr
    1\quad&i=j\in {\cal I}\cr
  \end{array}
\right.  \,.
\end{equation}
$\cal I$ is the index-set of the prognostic (not steady-state) variables.

\paragraph*{Cranck-Nicholson:} \citet{Stoer1978}
\par\noindent
This is a standard implicit one-step method:
\begin{equation}
y^{n+1} = y_i^n + \Delta t\left( 
                      \alpha * f_i^n  + (1-\alpha) f_i^{n+1}\right)
\,.
\end{equation}
For $\alpha=1$ it reduces to Euler-explicit and for $\alpha=0$ to 
Euler-implicit. For $\alpha=0.5$ this method can be shown to be of
highest possible order for a one-step method.
Its disadvantage is that a non-linear implicit equation has to
be solved at each time step. Tests with typical chemical systems
have shown that on average 4 matrix inversions are necessary in order
to complete this task. Thus, the method is about 4 times more expensive
than SIS and LinSSA, but therefore more precise.

\paragraph*{QSSA (Quasi-Steady-State approximation:} \citet{Hesstvedt1978}
\par\noindent
The differential equation for chemistry can be rewritten as
\begin{equation}
  \frac{d}{dt} y_i(t) = f_i (y,t) = P_i(y,t) - L_i(y,t) y_i
  \,,
  \label{toto}
\end{equation}
where $P_i$ and $L_i$ are the production 
(where species $i$ appears in the right hand side of a reaction)
and loss terms (where species $i$ appears in the left 
hand side of a reaction), respectively.
Depending on the "lifetime" of species $i$, defined as
$\tau_i = 1/L_i$, its integration is either performed with a
simple Euler-explicit scheme (long-lived species), or
using an analytic solution of equation (\ref{toto}) assuming
$P_i$ and $L_i$ constant during a time step (life time of the order of the
time step), or assuming a steady-state between production and loss
(short-lived species). This can be expressed as follows:
\begin{equation}
  y_i^{(n+1)} = y_i^n + \Delta t ( P_i^n - L_i^n y_i^n ) \qquad \mbox{for }
  \tau_i > 100 \Delta t;
\end{equation}
\begin{equation}
  y_i^{(n+1)} = P_i^n / L_i^n \qquad \mbox{for }
  \tau_i < \Delta t / 10;
\end{equation}
\begin{equation}
  y_i^{(n+1)} = P_i^n / L_i^n +(y_i^n- P_i^n / L_i^n) \exp(-L_i^n\Delta t)
  \qquad \mbox{otherwise.}
\end{equation}

%
%%%%%%%%%%%%%%%%%%%%%%%%%%%%%%%%%%%%%%%%%%%%%%%%%%%%%%%%%%%%%%%%%%%%%%%%%%%%
%* MODIFICATION OF THE REACTION SCHEME
%%%%%%%%%%%%%%%%%%%%%%%%%%%%%%%%%%%%%%%%%%%%%%%%%%%%%%%%%%%%%%%%%%%%%%%%%%%%
\section{Changing or adding reaction schemes}
%
The chemical reaction scheme of MNHC is not coded directly in Fortran90,
but using a "chemical definition file", which contains all necessary
information in order to create the Fortran90 subroutines that calculate
essentially the first derivative, the Jacobian of the system,
the subroutines that calculate the reaction and photolysis rates,
and some other utility subroutines.
Here we give a brief summary of how this is done.

Suppose you wish to implement the following reaction scheme, which is
actually an extract of the present reaction scheme RACM:

{\small
\begin{tabular}{l@{\,\,:\,\,}l@{\protect$\quad\longrightarrow\quad\protect$}l}
$K_{001} = J_1$ & $NO_{2}$ & $O({}^3P)+NO$ \\
$K_{024} = M \cdot 6.00\cdot10^{-34} \left( T / 300 \right)^{-2/3} $ 
         & $O({}^3P)+O_{2}$ & $O_{3}$ \\
$K_{048} = 2.00\cdot10^{-12} \exp\left( -1400/T \right)$ 
         & $O_{3}+NO$ & $NO_{2}+O_{2}$ 
\end{tabular}
}

Then you have to write the following lines in your
"chemical definition file" (for details of the syntax and additional 
options refer to book 3):

{\small
\begin{verbatim}
/begin_reactions/
K001 = !ZRATES(:,001)                     :: NO2      --> O3P + NO
K024 = TPK%M*6.00E-34*(TPK%T/300)**(-2.3) :: O3P + O2 --> O3
K048 = 2.00E-12*exp(-(1400.0/TPK%T))      :: O3  + NO --> NO2 + O2
/end_reactions/
\end{verbatim}
}
The first line is a photolysis reaction, where {\tt ZRATES} contains
the photolysis rates at a given time and grid point.
{\tt TPK\%M} is the concentration of air molecules, and {\tt TPK\%T} is
the temperature.  With respect to the earlier version of Meso-NH,
the syntax of the chemical definition file has become somewhat more
complex: The use of Fortran~90 types (indicated by {\tt \%}) is
required by the grid-nesting, and the use of Fortran~90 vectors
(indicated by {\tt :}) is needed for the vectorization.

The preprocessor then translates these lines into Fortran90 code.
An extract of  the first derivative would look like this:
{\small
\begin{verbatim}
!PPROD(O3) = +K024*<O3P>*<O2>+...
 PPROD(:,1) = +TPK(:)%K024*PCONC(:,13)*TPK(:)%O2+...
!PLOSS(O3) = +K048*<NO>+...
 PLOSS(:,1) = +TPK(:)%K048*PCONC(:,3)+...
\end{verbatim}
}

An extract of the Jacobian is:
{\small
\begin{verbatim}
!O3/O3=-K002-K003-K025*<O3P>-K029*<OH>-K030*<HO2>-K048*<NO>-K049*<NO2>-K106*<ET
!E>-K107*<OLT>-K108*<OLI>-K109*<DIEN>-K110*<ISO>-K111*<API>-K112*<LIM>-K113*<MA
!CR>-K114*<DCB>-K115*<TPAN>-K120*<ADDT>-K123*<ADDX>-K126*<ADDC>
 PJAC(:,1,1)=-TPK(:)%K002-TPK(:)%K003-TPK(:)%K025*PCONC(:,13)-TPK(:)%K029*PCONC&
&(:,15)-TPK(:)%K030*PCONC(:,16)-TPK(:)%K048*PCONC(:,3)-TPK(:)%K049*PCONC(:,4)-T&
&PK(:)%K106*PCONC(:,22)-TPK(:)%K107*PCONC(:,23)-TPK(:)%K108*PCONC(:,24)-TPK(:)%&
&K109*PCONC(:,25)-TPK(:)%K110*PCONC(:,26)-TPK(:)%K111*PCONC(:,27)-TPK(:)%K112*P&
&CONC(:,28)-TPK(:)%K113*PCONC(:,38)-TPK(:)%K114*PCONC(:,37)-TPK(:)%K115*PCONC(:&
&,43)-TPK(:)%K120*PCONC(:,59)-TPK(:)%K123*PCONC(:,60)-TPK(:)%K126*PCONC(:,61)
!
!O3/NO=-K048*<O3>
 PJAC(:,1,3)=-TPK(:)%K048*PCONC(:,1)
\end{verbatim}
}

The subroutine that sets up the reaction and photolysis rates would
contain the following lines:
{\small
\begin{verbatim}
TPK%M          = 1E-3*TPM%XMETEOVAR(2) * 6.0221367E+23 / 28.9644
TPK%T          = TPM%XMETEOVAR(3)
...
TPK(:)%K024=TPK%M*6.00E-34*(TPK%T/300)**(-2.3)
TPK(:)%K048=2.00E-12*exp(-(1400.0/TPK%T))
\end{verbatim}
}
where the variable {\tt TPM\%XMETEOVAR} contains the
meteorological fields, density (element 2), temperature (element 3), etc.
A change of the reaction scheme is thus relatively "easy" and requires
no extra Fortran90 coding by the user.
%%%%%%%%%%%%%%%%%%%%%%%%%%%%%%%%%%%%%%%%%%%%%%%%%%%%%%%%%%%%%%%%%%%%%%%%%%%%
%* BOX MODEL
%%%%%%%%%%%%%%%%%%%%%%%%%%%%%%%%%%%%%%%%%%%%%%%%%%%%%%%%%%%%%%%%%%%%%%%%%%%%
\section{0-D box model}
%
A time-dependant box model, using the same chemical reaction scheme
and the same solver than MNHC is also available.
In this case, the calculations that are done at every grid-point
of Meso-NH are only done once, and all meteorological parameters
that are used by the chemical system a read from a special file
and may evolve in time.
Conceptually, such a simplified model without dynamics can be obtained by
integration of equation (\ref{basic1}) over a fixed volume $V$:
\begin{equation}
\ddt \left<\rhod\scalar{i}\right> =  \oint \rhod\vec F \cdot d\vec A
+ \left< \rhod \source{i} \right>
\,,
\end{equation}
%
where $ \left< X \right> = \frac1V \int X \, dV $ denotes volume averaging 
of the variable $X$, $\vec F$ is the flux through the boundaries {\bf into} the
volume $V$ and $d\vec A$ the {\bf outward} pointing surface element vector.
Three cases are of interest: 
(a) the volume $V$ represents the whole
boundary layer, in this case the fluxes are the horizontal advection,
the surface emission and dry deposition, and the boundary-layer/free-troposphere
exchange flux;
(b) the volume $V$ represents a cylindric column of the boundary layer
which will be moved around by the wind field, here no horizontal advection terms
will be accounted for; and 
(c) the volume $V$ represents a small air parcel, in this case no
fluxes will be taken into account, the particle will be considered
as being inert.
For cases (a) and (b) instant and complete mixing of the whole boundary
layer is assumed. Cases (b) and (c) are examples for
so called "Lagrangian box models", external parameters like temperature,
pressure, humidity, chemical concentrations etc.\ will be 
varied during the simulation.
These parameters can be extracted from Meso-NH.
A possible application is to run the 3D Meso-NH-chemistry model using
a restricted reaction mechanism, then to trace Lagrangian trajectories in the
model, and finally force the box-model with these parameters, allowing 
for the use of 
a more complex chemical reaction mechanism and possibly to drive an aerosol
model with that data.
Note that all external parameters of the box model can be time-dependant.

Such a simplified version of MNHC using only chemistry can also be used
in order to test new reaction schemes, parameters for the solvers and
chemical initial conditions prior to their use in a multi-dimensional
version of Meso-NH.
%
\section*{The Regional Atmospheric Chemistry Mechanism RACM}
\label{RACM}
%
% file RACM.tex created by chf2tex , date: Wed Sep 22 09:30:16 METDST 1999
%
{\small
\newlength{\chfwidth}
\setlength{\chfwidth}{1\textwidth}
\noindent
\begin{tabular}{l@{\,:\,}p{0.2\chfwidth}@{$\quad\longrightarrow\quad$}p{0.6\chfwidth}}
$K_{001}$ & $NO_{2}$ & $O({}^3P)+NO$ \\
$K_{002}$ & $O_{3}$ & $O({}^1D)+O_{2}$ \\
$K_{003}$ & $O_{3}$ & $O({}^3P)+O_{2}$ \\
$K_{004}$ & $HONO$ & $OH+NO$ \\
$K_{005}$ & $HNO_{3}$ & $OH+NO_{2}$ \\
$K_{006}$ & $HNO_{4}$ & $0.65*HO_{2}+0.65*NO_{2}+0.35*OH+0.35*NO_{3}$ \\
\end{tabular}

\begin{tabular}{l@{\,:\,}p{0.2\chfwidth}@{$\quad\longrightarrow\quad$}p{0.6\chfwidth}}
$K_{007}$ & $NO_{3}$ & $NO+O_{2}$ \\
$K_{008}$ & $NO_{3}$ & $NO_{2}+O({}^3P)$ \\
$K_{009}$ & $H_{2}O_{2}$ & $OH+OH$ \\
$K_{010}$ & $HCHO$ & $H_{2}+CO$ \\
$K_{011}$ & $HCHO$ & $HO_{2}+HO_{2}+CO$ \\
$K_{012}$ & $ALD$ & $MO_{2}+HO_{2}+CO$ \\
\end{tabular}

\begin{tabular}{l@{\,:\,}p{0.2\chfwidth}@{$\quad\longrightarrow\quad$}p{0.6\chfwidth}}
$K_{013}$ & $OP_{1}$ & $HCHO+HO_{2}+OH$ \\
$K_{014}$ & $OP_{2}$ & $ALD+HO_{2}+OH$ \\
$K_{015}$ & $PAA$ & $MO_{2}+OH$ \\
$K_{016}$ & $KET$ & $ACO_{3}+ETHP$ \\
$K_{017}$ & $GLY$ & $0.13*HCHO+1.87*CO+0.87*H_{2}$ \\
$K_{018}$ & $GLY$ & $0.45*HCHO+1.55*CO+0.80*HO_{2}+0.15*H_{2}$ \\
\end{tabular}

\begin{tabular}{l@{\,:\,}p{0.2\chfwidth}@{$\quad\longrightarrow\quad$}p{0.6\chfwidth}}
$K_{019}$ & $MGLY$ & $ACO_{3}+HO_{2}+CO$ \\
$K_{020}$ & $DCB$ & $TCO_{3}+HO_{2}$ \\
$K_{021}$ & $ONIT$ & $0.20*ALD+0.80*KET+HO_{2}+NO_{2}$ \\
$K_{022}$ & $MACR$ & $HCHO+ACO_{3}+CO+HO_{2}$ \\
$K_{023}$ & $HKET$ & $ACO_{3}+HCHO+HO_{2}$ \\
$K_{024}$ & $O({}^3P)+O_{2}$ & $O_{3}$ \\
\end{tabular}

\begin{tabular}{l@{\,:\,}p{0.2\chfwidth}@{$\quad\longrightarrow\quad$}p{0.6\chfwidth}}
$K_{025}$ & $O({}^3P)+O_{3}$ & $2.0*O_{2}$ \\
$K_{026}$ & $O({}^1D)+N_{2}$ & $O({}^3P)+N_{2}$ \\
$K_{027}$ & $O({}^1D)+O_{2}$ & $O({}^3P)+O_{2}$ \\
$K_{028}$ & $O({}^1D)+H_{2}O$ & $OH+OH$ \\
$K_{029}$ & $O_{3}+OH$ & $HO_{2}+O_{2}$ \\
$K_{030}$ & $O_{3}+HO_{2}$ & $OH+2.0*O_{2}$ \\
\end{tabular}

\begin{tabular}{l@{\,:\,}p{0.2\chfwidth}@{$\quad\longrightarrow\quad$}p{0.6\chfwidth}}
$K_{031}$ & $OH+HO_{2}$ & $H_{2}O+O_{2}$ \\
$K_{032}$ & $H_{2}O_{2}+OH$ & $HO_{2}+H_{2}O$ \\
$K_{033}$ & $HO_{2}+HO_{2}$ & $H_{2}O_{2}+O_{2}$ \\
$K_{034}$ & $HO_{2}+HO_{2}+H_{2}O$ & $H_{2}O_{2}+H_{2}O+O_{2}$ \\
$K_{035}$ & $O({}^3P)+NO$ & $NO_{2}$ \\
$K_{036}$ & $O({}^3P)+NO_{2}$ & $NO+O_{2}$ \\
\end{tabular}

\begin{tabular}{l@{\,:\,}p{0.2\chfwidth}@{$\quad\longrightarrow\quad$}p{0.6\chfwidth}}
$K_{037}$ & $O({}^3P)+NO_{2}$ & $NO_{3}$ \\
$K_{038}$ & $OH+NO$ & $HONO$ \\
$K_{039}$ & $OH+NO_{2}$ & $HNO_{3}$ \\
$K_{040}$ & $OH+NO_{3}$ & $NO_{2}+HO_{2}$ \\
$K_{041}$ & $HO_{2}+NO$ & $NO_{2}+OH$ \\
$K_{042}$ & $HO_{2}+NO_{2}$ & $HNO_{4}$ \\
\end{tabular}

\begin{tabular}{l@{\,:\,}p{0.2\chfwidth}@{$\quad\longrightarrow\quad$}p{0.6\chfwidth}}
$K_{043}$ & $HNO_{4}$ & $HO_{2}+NO_{2}$ \\
$K_{044}$ & $HO_{2}+NO_{3}$ & $0.3*HNO_{3}+0.7*NO_{2}+0.7*OH$ \\
$K_{045}$ & $OH+HONO$ & $H_{2}O+NO_{2}$ \\
$K_{046}$ & $OH+HNO_{3}$ & $NO_{3}+H_{2}O$ \\
$K_{047}$ & $OH+HNO_{4}$ & $NO_{2}+H_{2}O+O_{2}$ \\
$K_{048}$ & $O_{3}+NO$ & $NO_{2}+O_{2}$ \\
\end{tabular}

\begin{tabular}{l@{\,:\,}p{0.2\chfwidth}@{$\quad\longrightarrow\quad$}p{0.6\chfwidth}}
$K_{049}$ & $O_{3}+NO_{2}$ & $NO_{3}+O_{2}$ \\
$K_{050}$ & $NO+NO+O_{2}$ & $NO_{2}+NO_{2}$ \\
$K_{051}$ & $NO_{3}+NO$ & $NO_{2}+NO_{2}$ \\
$K_{052}$ & $NO_{3}+NO_{2}$ & $NO+NO_{2}+O_{2}$ \\
$K_{053}$ & $NO_{3}+NO_{2}$ & $N_{2}O_{5}$ \\
$K_{054}$ & $N_{2}O_{5}$ & $NO_{2}+NO_{3}$ \\
\end{tabular}

\begin{tabular}{l@{\,:\,}p{0.2\chfwidth}@{$\quad\longrightarrow\quad$}p{0.6\chfwidth}}
$K_{055}$ & $NO_{3}+NO_{3}$ & $NO_{2}+NO_{2}+O_{2}$ \\
$K_{056}$ & $OH+H_{2}$ & $H_{2}O+HO_{2}$ \\
$K_{057}$ & $OH+SO_{2}$ & $SULF+HO_{2}$ \\
$K_{058}$ & $CO+OH$ & $HO_{2}+CO_{2}$ \\
$K_{059}$ & $ISO+O({}^3P)$ & $0.86*OLT+0.05*HCHO+0.02*OH+0.01*CO+0.13*DCB+0.28*HO_{2}+0.15*XO_{2}$ \\
$K_{060}$ & $MACR+O({}^3P)$ & $ALD$ \\
\end{tabular}

\begin{tabular}{l@{\,:\,}p{0.2\chfwidth}@{$\quad\longrightarrow\quad$}p{0.6\chfwidth}}
$K_{061}$ & $CH_{4}+OH$ & $MO_{2}+H_{2}O$ \\
$K_{062}$ & $ETH+OH$ & $ETHP+H_{2}O$ \\
$K_{063}$ & $HC_{3}+OH$ & $0.583*HC_{3}P+0.381*HO_{2}+0.335*ALD+0.036*ORA_{1}+0.036*CO+0.036*GLY+0.036*OH+0.010*HCHO+H_{2}O$ \\
$K_{064}$ & $HC_{5}+OH$ & $0.750*HC_{5}P+0.250*KET+0.250*HO_{2}+H_{2}O$ \\
$K_{065}$ & $HC_{8}+OH$ & $0.951*HC_{8}P+0.025*ALD+0.024*HKET+0.049*HO_{2}+H_{2}O$ \\
$K_{066}$ & $ETE+OH$ & $ETEP$ \\
\end{tabular}

\begin{tabular}{l@{\,:\,}p{0.2\chfwidth}@{$\quad\longrightarrow\quad$}p{0.6\chfwidth}}
$K_{067}$ & $OLT+OH$ & $OLTP$ \\
$K_{068}$ & $OLI+OH$ & $OLIP$ \\
$K_{069}$ & $DIEN+OH$ & $ISOP$ \\
$K_{070}$ & $ISO+OH$ & $ISOP$ \\
$K_{071}$ & $API+OH$ & $APIP$ \\
$K_{072}$ & $LIM+OH$ & $LIMP$ \\
\end{tabular}

\begin{tabular}{l@{\,:\,}p{0.2\chfwidth}@{$\quad\longrightarrow\quad$}p{0.6\chfwidth}}
$K_{073}$ & $TOL+OH$ & $0.90*ADDT+0.10*XO_{2}+0.10*HO_{2}$ \\
$K_{074}$ & $XYL+OH$ & $0.90*ADDX+0.10*XO_{2}+0.10*HO_{2}$ \\
$K_{075}$ & $CSL+OH$ & $0.85*ADDC+0.10*PHO+0.05*HO_{2}+0.05*XO_{2}$ \\
$K_{076}$ & $HCHO+OH$ & $HO_{2}+CO+H_{2}O$ \\
$K_{077}$ & $ALD+OH$ & $ACO_{3}+H_{2}O$ \\
$K_{078}$ & $KET+OH$ & $KETP+H_{2}O$ \\
\end{tabular}

\begin{tabular}{l@{\,:\,}p{0.2\chfwidth}@{$\quad\longrightarrow\quad$}p{0.6\chfwidth}}
$K_{079}$ & $HKET+OH$ & $HO_{2}+MGLY+H_{2}O$ \\
$K_{080}$ & $GLY+OH$ & $HO_{2}+2.0*CO+H_{2}O$ \\
$K_{081}$ & $MGLY+OH$ & $ACO_{3}+CO+H_{2}O$ \\
$K_{082}$ & $MACR+OH$ & $0.51*TCO_{3}+0.41*HKET+0.08*MGLY+0.41*CO+0.08*HCHO+0.49*HO_{2}+0.49*XO_{2}$ \\
$K_{083}$ & $OH+DCB$ & $0.50*TCO_{3}+0.50*HO_{2}+0.50*XO_{2}+0.35*UDD+0.15*GLY+0.15*MGLY$ \\
$K_{084}$ & $OH+UDD$ & $0.88*ALD+0.12*KET+HO_{2}$ \\
\end{tabular}

\begin{tabular}{l@{\,:\,}p{0.2\chfwidth}@{$\quad\longrightarrow\quad$}p{0.6\chfwidth}}
$K_{085}$ & $OP_{1}+OH$ & $0.65*MO_{2}+0.35*HCHO+0.35*OH$ \\
$K_{086}$ & $OP_{2}+OH$ & $0.44*HC_{3}P+0.08*ALD+0.49*OH+0.07*XO_{2}+0.41*KET$ \\
$K_{087}$ & $PAA+OH$ & $0.65*ACO_{3}+0.35*HO_{2}+0.35*HCHO+0.35*XO_{2}$ \\
$K_{088}$ & $PAN+OH$ & $HCHO+NO_{3}+XO_{2}+H_{2}O$ \\
$K_{089}$ & $TPAN+OH$ & $0.60*HKET+0.60*NO_{3}+0.40*PAN+0.40*HCHO+0.40*HO_{2}+XO_{2}$ \\
$K_{090}$ & $ONIT+OH$ & $HC_{3}P+NO_{2}+H_{2}O$ \\
\end{tabular}

\begin{tabular}{l@{\,:\,}p{0.2\chfwidth}@{$\quad\longrightarrow\quad$}p{0.6\chfwidth}}
$K_{091}$ & $HCHO+NO_{3}$ & $HO_{2}+HNO_{3}+CO$ \\
$K_{092}$ & $ALD+NO_{3}$ & $ACO_{3}+HNO_{3}$ \\
$K_{093}$ & $GLY+NO_{3}$ & $HNO_{3}+HO_{2}+2.0*CO$ \\
$K_{094}$ & $MGLY+NO_{3}$ & $HNO_{3}+ACO_{3}+CO$ \\
$K_{095}$ & $MACR+NO_{3}$ & $0.20*TCO_{3}+0.20*HNO_{3}+0.80*OLNN+0.80*CO$ \\
$K_{096}$ & $DCB+NO_{3}$ & $0.50*TCO_{3}+0.50*HO_{2}+0.50*XO_{2}+0.25*GLY+0.25*ALD+0.50*NO_{2}+0.03*KET+0.25*MGLY+0.50*HNO_{3}$ \\
\end{tabular}

\begin{tabular}{l@{\,:\,}p{0.2\chfwidth}@{$\quad\longrightarrow\quad$}p{0.6\chfwidth}}
$K_{097}$ & $CSL+NO_{3}$ & $HNO_{3}+PHO$ \\
$K_{098}$ & $ETE+NO_{3}$ & $0.80*OLNN+0.20*OLND$ \\
$K_{099}$ & $OLT+NO_{3}$ & $0.43*OLNN+0.57*OLND$ \\
$K_{100}$ & $OLI+NO_{3}$ & $0.11*OLNN+0.89*OLND$ \\
$K_{101}$ & $DIEN+NO_{3}$ & $0.90*OLNN+0.10*OLND+0.90*MACR$ \\
$K_{102}$ & $ISO+NO_{3}$ & $0.90*OLNN+0.10*OLND+0.90*MACR$ \\
\end{tabular}

\begin{tabular}{l@{\,:\,}p{0.2\chfwidth}@{$\quad\longrightarrow\quad$}p{0.6\chfwidth}}
$K_{103}$ & $API+NO_{3}$ & $0.10*OLNN+0.90*OLND$ \\
$K_{104}$ & $LIM+NO_{3}$ & $0.13*OLNN+0.87*OLND$ \\
$K_{105}$ & $TPAN+NO_{3}$ & $0.60*ONIT+0.60*NO_{3}+0.40*PAN+0.40*HCHO+0.40*NO_{2}+XO_{2}$ \\
$K_{106}$ & $ETE+O_{3}$ & $HCHO+0.43*CO+0.37*ORA_{1}+0.26*HO_{2}+0.13*H_{2}+0.12*OH$ \\
$K_{107}$ & $OLT+O_{3}$ & $0.64*HCHO+0.44*ALD+0.37*CO+0.14*ORA_{1}+0.10*ORA_{2}+0.25*HO_{2}+0.40*OH+0.03*KET+0.03*KETP+0.06*CH_{4}+0.05*H_{2}+0.03*ETH+0.006*H_{2}O_{2}+0.19*MO_{2}+0.10*ETHP$ \\
$K_{108}$ & $OLI+O_{3}$ & $0.02*HCHO+0.99*ALD+0.16*KET+0.30*CO+0.011*H_{2}O_{2}+0.14*ORA_{2}+0.07*CH_{4}+0.22*HO_{2}+0.63*OH+0.23*MO_{2}+0.12*KETP+0.06*ETH+0.18*ETHP$ \\
\end{tabular}

\begin{tabular}{l@{\,:\,}p{0.2\chfwidth}@{$\quad\longrightarrow\quad$}p{0.6\chfwidth}}
$K_{109}$ & $DIEN+O_{3}$ & $0.90*HCHO+0.39*MACR+0.36*CO+0.15*ORA_{1}+0.09*O({}^3P)+0.30*HO_{2}+0.35*OLT+0.28*OH+0.05*H_{2}+0.15*ACO_{3}+0.03*MO_{2}+0.02*KETP+0.13*XO_{2}+0.001*H_{2}O_{2}$ \\
$K_{110}$ & $ISO+O_{3}$ & $0.90*HCHO+0.39*MACR+0.36*CO+0.15*ORA_{1}+0.09*O({}^3P)+0.30*HO_{2}+0.35*OLT+0.28*OH+0.05*H_{2}+0.15*ACO_{3}+0.03*MO_{2}+0.02*KETP+0.13*XO_{2}+0.001*H_{2}O_{2}$ \\
$K_{111}$ & $API+O_{3}$ & $0.65*ALD+0.53*KET+0.14*CO+0.20*ETHP+0.42*KETP+0.85*OH+0.10*HO_{2}+0.02*H_{2}O_{2}$ \\
$K_{112}$ & $LIM+O_{3}$ & $0.04*HCHO+0.46*OLT+0.14*CO+0.16*ETHP+0.42*KETP+0.85*OH+0.10*HO_{2}+0.02*H_{2}O_{2}+0.79*MACR+0.01*ORA_{1}+0.07*ORA_{2}$ \\
$K_{113}$ & $MACR+O_{3}$ & $0.40*HCHO+0.60*MGLY+0.13*ORA_{2}+0.54*CO+0.08*H_{2}+0.22*ORA_{1}+0.29*HO_{2}+0.07*OH+0.13*OP_{2}+0.13*ACO_{3}$ \\
$K_{114}$ & $DCB+O_{3}$ & $0.21*OH+0.29*HO_{2}+0.66*CO+0.50*GLY+0.62*MGLY+0.28*ACO_{3}+0.16*ALD+0.11*PAA+0.11*ORA_{1}+0.21*ORA_{2}$ \\
\end{tabular}

\begin{tabular}{l@{\,:\,}p{0.2\chfwidth}@{$\quad\longrightarrow\quad$}p{0.6\chfwidth}}
$K_{115}$ & $TPAN+O_{3}$ & $0.70*HCHO+0.30*PAN+0.70*NO_{2}+0.13*CO+0.04*H_{2}+0.11*ORA_{1}+0.08*HO_{2}+0.036*OH+0.70*ACO_{3}$ \\
$K_{116}$ & $PHO+NO_{2}$ & $0.10*CSL+ONIT$ \\
$K_{117}$ & $PHO+HO_{2}$ & $CSL$ \\
$K_{118}$ & $ADDT+NO_{2}$ & $CSL+HONO$ \\
$K_{119}$ & $ADDT+O_{2}$ & $0.98*TOLP+0.02*CSL+0.02*HO_{2}$ \\
$K_{120}$ & $ADDT+O_{3}$ & $CSL+OH$ \\
\end{tabular}

\begin{tabular}{l@{\,:\,}p{0.2\chfwidth}@{$\quad\longrightarrow\quad$}p{0.6\chfwidth}}
$K_{121}$ & $ADDX+NO_{2}$ & $CSL+HONO$ \\
$K_{122}$ & $ADDX+O_{2}$ & $0.98*XYLP+0.02*CSL+0.02*HO_{2}$ \\
$K_{123}$ & $ADDX+O_{3}$ & $CSL+OH$ \\
$K_{124}$ & $ADDC+NO_{2}$ & $CSL+HONO$ \\
$K_{125}$ & $ADDC+O_{2}$ & $0.98*CSLP+0.02*CSL+0.02*HO_{2}$ \\
$K_{126}$ & $ADDC+O_{3}$ & $CSL+OH$ \\
\end{tabular}

\begin{tabular}{l@{\,:\,}p{0.2\chfwidth}@{$\quad\longrightarrow\quad$}p{0.6\chfwidth}}
$K_{127}$ & $ACO_{3}+NO_{2}$ & $PAN$ \\
$K_{128}$ & $PAN$ & $ACO_{3}+NO_{2}$ \\
$K_{129}$ & $TCO_{3}+NO_{2}$ & $TPAN$ \\
$K_{130}$ & $TPAN$ & $TCO_{3}+NO_{2}$ \\
$K_{131}$ & $MO_{2}+NO$ & $HCHO+HO_{2}+NO_{2}$ \\
$K_{132}$ & $ETHP+NO$ & $ALD+HO_{2}+NO_{2}$ \\
\end{tabular}

\begin{tabular}{l@{\,:\,}p{0.2\chfwidth}@{$\quad\longrightarrow\quad$}p{0.6\chfwidth}}
$K_{133}$ & $HC_{3}P+NO$ & $0.047*HCHO+0.233*ALD+0.623*KET+0.063*GLY+0.742*HO_{2}+0.150*MO_{2}+0.048*ETHP+0.048*XO_{2}+0.059*ONIT+0.941*NO_{2}$ \\
$K_{134}$ & $HC_{5}P+NO$ & $0.021*HCHO+0.211*ALD+0.722*KET+0.599*HO_{2}+0.031*MO_{2}+0.245*ETHP+0.334*XO_{2}+0.124*ONIT+0.876*NO_{2}$ \\
$K_{135}$ & $HC_{8}P+NO$ & $0.150*ALD+0.642*KET+0.133*ETHP+0.261*ONIT+0.739*NO_{2}+0.606*HO_{2}+0.416*XO_{2}$ \\
$K_{136}$ & $ETEP+NO$ & $1.6*HCHO+HO_{2}+NO_{2}+0.2*ALD$ \\
$K_{137}$ & $OLTP+NO$ & $0.94*ALD+HCHO+HO_{2}+NO_{2}+0.06*KET$ \\
$K_{138}$ & $OLIP+NO$ & $HO_{2}+1.71*ALD+0.29*KET+NO_{2}$ \\
\end{tabular}

\begin{tabular}{l@{\,:\,}p{0.2\chfwidth}@{$\quad\longrightarrow\quad$}p{0.6\chfwidth}}
$K_{139}$ & $ISOP+NO$ & $0.446*MACR+0.354*OLT+0.847*HO_{2}+0.847*NO_{2}+0.153*ONIT+0.606*HCHO$ \\
$K_{140}$ & $APIP+NO$ & $0.80*HO_{2}+0.80*ALD+0.80*NO_{2}+0.80*KET+0.20*ONIT$ \\
$K_{141}$ & $LIMP+NO$ & $0.65*HO_{2}+0.40*MACR+0.25*OLI+0.25*HCHO+0.65*NO_{2}+0.35*ONIT$ \\
$K_{142}$ & $TOLP+NO$ & $0.95*NO_{2}+0.95*HO_{2}+1.20*GLY+0.50*DCB+0.05*ONIT+0.65*MGLY$ \\
$K_{143}$ & $XYLP+NO$ & $0.95*NO_{2}+0.95*HO_{2}+0.35*GLY+0.95*DCB+0.05*ONIT+0.60*MGLY$ \\
$K_{144}$ & $CSLP+NO$ & $GLY+MGLY+NO_{2}+HO_{2}$ \\
\end{tabular}

\begin{tabular}{l@{\,:\,}p{0.2\chfwidth}@{$\quad\longrightarrow\quad$}p{0.6\chfwidth}}
$K_{145}$ & $ACO_{3}+NO$ & $MO_{2}+NO_{2}$ \\
$K_{146}$ & $TCO_{3}+NO$ & $NO_{2}+ACO_{3}+HCHO$ \\
$K_{147}$ & $KETP+NO$ & $0.54*MGLY+0.46*ALD+0.23*ACO_{3}+0.77*HO_{2}+0.16*XO_{2}+NO_{2}$ \\
$K_{148}$ & $OLNN+NO$ & $ONIT+NO_{2}+HO_{2}$ \\
$K_{149}$ & $OLND+NO$ & $0.287*HCHO+1.24*ALD+0.464*KET+2.00*NO_{2}$ \\
$K_{150}$ & $MO_{2}+HO_{2}$ & $OP_{1}$ \\
\end{tabular}

\begin{tabular}{l@{\,:\,}p{0.2\chfwidth}@{$\quad\longrightarrow\quad$}p{0.6\chfwidth}}
$K_{151}$ & $ETHP+HO_{2}$ & $OP_{2}$ \\
$K_{152}$ & $HC_{3}P+HO_{2}$ & $OP_{2}$ \\
$K_{153}$ & $HC_{5}P+HO_{2}$ & $OP_{2}$ \\
$K_{154}$ & $HC_{8}P+HO_{2}$ & $OP_{2}$ \\
$K_{155}$ & $ETEP+HO_{2}$ & $OP_{2}$ \\
$K_{156}$ & $OLTP+HO_{2}$ & $OP_{2}$ \\
\end{tabular}

\begin{tabular}{l@{\,:\,}p{0.2\chfwidth}@{$\quad\longrightarrow\quad$}p{0.6\chfwidth}}
$K_{157}$ & $OLIP+HO_{2}$ & $OP_{2}$ \\
$K_{158}$ & $ISOP+HO_{2}$ & $OP_{2}$ \\
$K_{159}$ & $APIP+HO_{2}$ & $OP_{2}$ \\
$K_{160}$ & $LIMP+HO_{2}$ & $OP_{2}$ \\
$K_{161}$ & $TOLP+HO_{2}$ & $OP_{2}$ \\
$K_{162}$ & $XYLP+HO_{2}$ & $OP_{2}$ \\
\end{tabular}

\begin{tabular}{l@{\,:\,}p{0.2\chfwidth}@{$\quad\longrightarrow\quad$}p{0.6\chfwidth}}
$K_{163}$ & $CSLP+HO_{2}$ & $OP_{2}$ \\
$K_{164}$ & $ACO_{3}+HO_{2}$ & $PAA$ \\
$K_{165}$ & $ACO_{3}+HO_{2}$ & $ORA_{2}+O_{3}$ \\
$K_{166}$ & $TCO_{3}+HO_{2}$ & $OP_{2}$ \\
$K_{167}$ & $TCO_{3}+HO_{2}$ & $ORA_{2}+O_{3}$ \\
$K_{168}$ & $KETP+HO_{2}$ & $OP_{2}$ \\
\end{tabular}

\begin{tabular}{l@{\,:\,}p{0.2\chfwidth}@{$\quad\longrightarrow\quad$}p{0.6\chfwidth}}
$K_{169}$ & $OLNN+HO_{2}$ & $ONIT$ \\
$K_{170}$ & $OLND+HO_{2}$ & $ONIT$ \\
$K_{171}$ & $MO_{2}+MO_{2}$ & $1.33*HCHO+0.66*HO_{2}$ \\
$K_{172}$ & $ETHP+MO_{2}$ & $0.75*HCHO+HO_{2}+0.75*ALD$ \\
$K_{173}$ & $HC_{3}P+MO_{2}$ & $0.810*HCHO+0.992*HO_{2}+0.580*ALD+0.018*KET+0.007*MO_{2}+0.005*MGLY+0.085*XO_{2}+0.119*GLY$ \\
$K_{174}$ & $HC_{5}P+MO_{2}$ & $0.829*HCHO+0.946*HO_{2}+0.523*ALD+0.240*KET+0.014*ETHP+0.245*XO_{2}+0.049*MO_{2}$ \\
\end{tabular}

\begin{tabular}{l@{\,:\,}p{0.2\chfwidth}@{$\quad\longrightarrow\quad$}p{0.6\chfwidth}}
$K_{175}$ & $HC_{8}P+MO_{2}$ & $0.753*HCHO+0.993*HO_{2}+0.411*ALD+0.419*KET+0.322*XO_{2}+0.013*ETHP$ \\
$K_{176}$ & $ETEP+MO_{2}$ & $1.55*HCHO+HO_{2}+0.35*ALD$ \\
$K_{177}$ & $OLTP+MO_{2}$ & $1.25*HCHO+HO_{2}+0.669*ALD+0.081*KET$ \\
$K_{178}$ & $OLIP+MO_{2}$ & $0.755*HCHO+HO_{2}+0.932*ALD+0.313*KET$ \\
$K_{179}$ & $ISOP+MO_{2}$ & $0.550*MACR+0.370*OLT+HO_{2}+0.080*OLI+1.09*HCHO$ \\
$K_{180}$ & $APIP+MO_{2}$ & $2.00*HO_{2}+ALD+HCHO+KET$ \\
\end{tabular}

\begin{tabular}{l@{\,:\,}p{0.2\chfwidth}@{$\quad\longrightarrow\quad$}p{0.6\chfwidth}}
$K_{181}$ & $LIMP+MO_{2}$ & $2.00*HO_{2}+0.60*MACR+0.40*OLI+1.40*HCHO$ \\
$K_{182}$ & $TOLP+MO_{2}$ & $HCHO+HO_{2}+0.65*GLY+DCB+0.35*MGLY$ \\
$K_{183}$ & $XYLP+MO_{2}$ & $HCHO+HO_{2}+0.37*GLY+DCB+0.63*MGLY$ \\
$K_{184}$ & $CSLP+MO_{2}$ & $GLY+MGLY+2.00*HO_{2}+HCHO$ \\
$K_{185}$ & $ACO_{3}+MO_{2}$ & $HCHO+HO_{2}+MO_{2}$ \\
$K_{186}$ & $ACO_{3}+MO_{2}$ & $HCHO+ORA_{2}$ \\
\end{tabular}

\begin{tabular}{l@{\,:\,}p{0.2\chfwidth}@{$\quad\longrightarrow\quad$}p{0.6\chfwidth}}
$K_{187}$ & $TCO_{3}+MO_{2}$ & $ACO_{3}+2.00*HCHO+HO_{2}$ \\
$K_{188}$ & $TCO_{3}+MO_{2}$ & $HCHO+ORA_{2}$ \\
$K_{189}$ & $KETP+MO_{2}$ & $0.40*MGLY+0.30*ALD+0.30*HKET+0.88*HO_{2}+0.08*XO_{2}+0.12*ACO_{3}+0.75*HCHO$ \\
$K_{190}$ & $OLNN+MO_{2}$ & $0.75*HCHO+HO_{2}+ONIT$ \\
$K_{191}$ & $OLND+MO_{2}$ & $0.960*HCHO+0.500*HO_{2}+0.640*ALD+0.149*KET+0.500*NO_{2}+0.500*ONIT$ \\
$K_{192}$ & $ETHP+ACO_{3}$ & $ALD+0.500*HO_{2}+0.5*MO_{2}+0.5*ORA_{2}$ \\
\end{tabular}

\begin{tabular}{l@{\,:\,}p{0.2\chfwidth}@{$\quad\longrightarrow\quad$}p{0.6\chfwidth}}
$K_{193}$ & $HC_{3}P+ACO_{3}$ & $0.724*ALD+0.127*KET+0.488*HO_{2}+0.508*MO_{2}+0.499*ORA_{2}+0.091*HCHO+0.006*ETHP+0.071*XO_{2}+0.004*MGLY+0.100*GLY$ \\
$K_{194}$ & $HC_{5}P+ACO_{3}$ & $0.677*ALD+0.330*KET+0.438*HO_{2}+0.554*MO_{2}+0.495*ORA_{2}+0.076*HCHO+0.018*ETHP+0.237*XO_{2}$ \\
$K_{195}$ & $HC_{8}P+ACO_{3}$ & $0.497*ALD+0.581*KET+0.489*HO_{2}+0.507*MO_{2}+0.495*ORA_{2}+0.015*ETHP+0.318*XO_{2}$ \\
$K_{196}$ & $ETEP+ACO_{3}$ & $0.8*HCHO+0.6*ALD+0.5*HO_{2}+0.5*MO_{2}+0.5*ORA_{2}$ \\
$K_{197}$ & $OLTP+ACO_{3}$ & $0.859*ALD+0.501*HCHO+0.501*HO_{2}+0.501*MO_{2}+0.499*ORA_{2}+0.141*KET$ \\
$K_{198}$ & $OLIP+ACO_{3}$ & $0.941*ALD+0.510*HO_{2}+0.510*MO_{2}+0.569*KET+0.490*ORA_{2}$ \\
\end{tabular}

\begin{tabular}{l@{\,:\,}p{0.2\chfwidth}@{$\quad\longrightarrow\quad$}p{0.6\chfwidth}}
$K_{199}$ & $ISOP+ACO_{3}$ & $0.771*MACR+0.229*OLT+0.506*HO_{2}+0.494*ORA_{2}+0.340*HCHO+0.506*MO_{2}$ \\
$K_{200}$ & $APIP+ACO_{3}$ & $HO_{2}+ALD+MO_{2}+KET$ \\
$K_{201}$ & $LIMP+ACO_{3}$ & $HO_{2}+0.60*MACR+0.40*OLI+0.40*HCHO+MO_{2}$ \\
$K_{202}$ & $TOLP+ACO_{3}$ & $MO_{2}+HO_{2}+DCB+0.65*GLY+0.35*MGLY$ \\
$K_{203}$ & $XYLP+ACO_{3}$ & $MO_{2}+HO_{2}+DCB+0.37*GLY+0.63*MGLY$ \\
$K_{204}$ & $CSLP+ACO_{3}$ & $GLY+MGLY+HO_{2}+MO_{2}$ \\
\end{tabular}

\begin{tabular}{l@{\,:\,}p{0.2\chfwidth}@{$\quad\longrightarrow\quad$}p{0.6\chfwidth}}
$K_{205}$ & $ACO_{3}+ACO_{3}$ & $2.0*MO_{2}$ \\
$K_{206}$ & $TCO_{3}+ACO_{3}$ & $MO_{2}+ACO_{3}+HCHO$ \\
$K_{207}$ & $KETP+ACO_{3}$ & $0.54*MGLY+0.35*ALD+0.11*KET+0.12*ACO_{3}+0.38*HO_{2}+0.08*XO_{2}+0.50*MO_{2}+0.50*ORA_{2}$ \\
$K_{208}$ & $OLNN+ACO_{3}$ & $ONIT+0.50*MO_{2}+0.50*ORA_{2}+0.50*HO_{2}$ \\
$K_{209}$ & $OLND+ACO_{3}$ & $0.207*HCHO+0.650*ALD+0.167*KET+0.516*MO_{2}+0.484*ORA_{2}+0.484*ONIT+0.516*NO_{2}$ \\
$K_{210}$ & $OLNN+OLNN$ & $2.00*ONIT+HO_{2}$ \\
\end{tabular}

\begin{tabular}{l@{\,:\,}p{0.2\chfwidth}@{$\quad\longrightarrow\quad$}p{0.6\chfwidth}}
$K_{211}$ & $OLNN+OLND$ & $0.202*HCHO+0.640*ALD+0.149*KET+0.500*HO_{2}+0.500*NO_{2}+1.50*ONIT$ \\
$K_{212}$ & $OLND+OLND$ & $0.504*HCHO+1.21*ALD+0.285*KET+ONIT+NO_{2}$ \\
$K_{213}$ & $MO_{2}+NO_{3}$ & $HCHO+HO_{2}+NO_{2}$ \\
$K_{214}$ & $ETHP+NO_{3}$ & $ALD+HO_{2}+NO_{2}$ \\
$K_{215}$ & $HC_{3}P+NO_{3}$ & $0.048*HCHO+0.243*ALD+0.670*KET+0.063*GLY+0.792*HO_{2}+0.155*MO_{2}+0.053*ETHP+0.051*XO_{2}+NO_{2}$ \\
$K_{216}$ & $HC_{5}P+NO_{3}$ & $0.021*HCHO+0.239*ALD+0.828*KET+0.699*HO_{2}+0.040*MO_{2}+0.262*ETHP+0.391*XO_{2}+NO_{2}$ \\
\end{tabular}

\begin{tabular}{l@{\,:\,}p{0.2\chfwidth}@{$\quad\longrightarrow\quad$}p{0.6\chfwidth}}
$K_{217}$ & $HC_{8}P+NO_{3}$ & $0.187*ALD+0.880*KET+0.155*ETHP+NO_{2}+0.845*HO_{2}+0.587*XO_{2}$ \\
$K_{218}$ & $ETEP+NO_{3}$ & $1.6*HCHO+HO_{2}+NO_{2}+0.2*ALD$ \\
$K_{219}$ & $OLTP+NO_{3}$ & $0.94*ALD+HCHO+HO_{2}+NO_{2}+0.06*KET$ \\
$K_{220}$ & $OLIP+NO_{3}$ & $HO_{2}+1.71*ALD+0.29*KET+NO_{2}$ \\
$K_{221}$ & $ISOP+NO_{3}$ & $0.600*MACR+0.400*OLT+HO_{2}+NO_{2}+0.686*HCHO$ \\
$K_{222}$ & $APIP+NO_{3}$ & $HO_{2}+ALD+NO_{2}+KET$ \\
\end{tabular}

\begin{tabular}{l@{\,:\,}p{0.2\chfwidth}@{$\quad\longrightarrow\quad$}p{0.6\chfwidth}}
$K_{223}$ & $LIMP+NO_{3}$ & $HO_{2}+0.60*MACR+0.40*OLI+0.40*HCHO+NO_{2}$ \\
$K_{224}$ & $TOLP+NO_{3}$ & $NO_{2}+HO_{2}+0.50*DCB+1.30*GLY+0.70*MGLY$ \\
$K_{225}$ & $XYLP+NO_{3}$ & $NO_{2}+HO_{2}+DCB+0.74*GLY+1.26*MGLY$ \\
$K_{226}$ & $CSLP+NO_{3}$ & $GLY+MGLY+NO_{2}+HO_{2}$ \\
$K_{227}$ & $ACO_{3}+NO_{3}$ & $MO_{2}+NO_{2}$ \\
$K_{228}$ & $TCO_{3}+NO_{3}$ & $NO_{2}+ACO_{3}+HCHO$ \\
\end{tabular}

\begin{tabular}{l@{\,:\,}p{0.2\chfwidth}@{$\quad\longrightarrow\quad$}p{0.6\chfwidth}}
$K_{229}$ & $KETP+NO_{3}$ & $0.54*MGLY+0.46*ALD+0.23*ACO_{3}+0.77*HO_{2}+0.16*XO_{2}+NO_{2}$ \\
$K_{230}$ & $OLNN+NO_{3}$ & $ONIT+NO_{2}+HO_{2}$ \\
$K_{231}$ & $OLND+NO_{3}$ & $0.280*HCHO+1.24*ALD+0.469*KET+2.00*NO_{2}$ \\
$K_{232}$ & $XO_{2}+HO_{2}$ & $OP_{2}$ \\
$K_{233}$ & $XO_{2}+MO_{2}$ & $HCHO+HO_{2}$ \\
$K_{234}$ & $XO_{2}+ACO_{3}$ & $MO_{2}$ \\
\end{tabular}

\begin{tabular}{l@{\,:\,}p{0.2\chfwidth}@{$\quad\longrightarrow\quad$}p{0.6\chfwidth}}
$K_{235}$ & $XO_{2}+XO_{2}$ &  \\
$K_{236}$ & $XO_{2}+NO$ & $NO_{2}$ \\
$K_{237}$ & $XO_{2}+NO_{3}$ & $NO_{2}$
\end{tabular}
}
%
%\end{appendix}
%
%\end{document}
%%%%%%%%%%%%%%%%%%%%%%%%%%%% BIBLIOGRAPHY %%%%%%%%%%%%%%%%%%%%%%%%
\begin{btSect}{4-1-Introduction}
\section{References}
\btPrintCited
\end{btSect}
%%%%%%%%%%%%%%%%%%%%%%%%%%%% BIBLIOGRAPHY %%%%%%%%%%%%%%%%%%%%%%%%
