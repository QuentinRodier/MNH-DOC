%%%%%%%%%%%%%%%%%%%%%%%%%%%%%%%%%%%%%%%%%%%%%%%%%%%%%%%%%%%%%%%%%%%%%%%%%%%%%
\chapter{Initial Fields for Idealized Flows}
\minitoc

%{\em by J. Stein}

In many occasions, we want to run simulations starting from highly idealized
conditions, such as an atmosphere in geostrophic equilibrium, with simple
vertical and horizontal structure, and simple orography, or no orography
at all.  This is useful
for theoretical research on geophysical flows, or simply to test if recent
changes in the code have not been detrimental to the quality of previous
results.  Meso-NH possesses a general program to prepare initial data for
such experiments, named PREP\_IDEAL\_CASE.

This program reads data on the
vertical structure of the atmosphere from an input file, that may take a
variety of formats for convenience, and prepares a file containing a 2D or 3D
initial state satisfying the geostrophic and anelastic balances,
according to various specifications for the geographical setting.
It is also possible to prepare non balanced fields following simple mathematical
formulas, in order to test the good functioning of the post-processing
and graphical facility (see last section).

The fields are produced either for a Cartesian geometry, or for a conformal
projection.  A stretched vertical grid may be used, but only regular horizontal
grids are supported in this program. The orography can be zero everywhere,
or defined by simple bell-shaped or sine-shaped mountains.

\section{Input atmospheric profile}
The input atmospheric profile may be specified in several ways, depending
on the willingness of the user to use observations (or pseudo observations)
in various formats, or to achieve a highly idealized experiment.

{\bf Sounding formats } (Option: 'RSOU')

If one wishes to use observations or pseudo-observations, the most convenient
way to communicate with the program is to use one of several available
"sounding" formats. The data consist of surface level information, followed
by $nu$ levels of temperature and moisture data, and $nm$ levels of wind data.
The date and time of the observation must also be supplied.

The following  data formats are supported, and may be selected by specification
of the appropriate value for the parameter YKIND:
\begin{itemize}
\item 'STANDARD'  :
\begin{itemize} \item altitude, pressure, temperature and dew point temperature at
ground level,
\item $nu$ levels with pressure, wind direction and wind force,
\item and $nm$
levels with pressure, temperature and dew point temperature.
\end{itemize}
\item 'PUVTHVMR'  : \begin{itemize} \item altitude, pressure, virtual potential temperature
and vapor mixing ratio at ground level,\item   $nu$ levels with pressure, zonal wind
component, meridian wind component,\item  $nm$ levels with pressure, virtual potential temperature
and vapor mixing ratio.\end{itemize}
\item 'PUVTHVHU' : \begin{itemize} \item altitude, pressure, virtual potential temperature
and relative humidity at ground level , \item $nu$ levels with pressure, zonal wind
component, meridian wind component,\item  $nm$ levels with pressure, virtual potential temperature
and relative humidity \end{itemize}
\item 'ZUVTHVHU'  : \begin{itemize} \item altitude, pressure, virtual potential temperature
and relative humidity at ground level,\item    $nu$ levels with altitude, zonal wind
component, meridian wind component, \item  $nm$ levels with altitude, virtual potential temperature
and relative humidity.\end{itemize}
\item 'ZUVTHVMR'  :  \begin{itemize} \item  altitude, pressure, virtual potential temperature
and vapor mixing ratio at ground level, \item  $nu$ levels with altitude, zonal wind
component, meridian wind component, \item  $nm$ levels with   altitude, virtual potential temperature
and vapor mixing ratio. \end{itemize}
\item 'PUVTHDMR'  : \begin{itemize} \item   altitude, pressure, dry potential temperature
and vapor mixing ratio at ground level, \item  $nu$ levels with pressure, zonal wind
component, meridian wind component, \item  $nm$ levels with pressure,
dry potential temperature and vapor mixing ratio.\end{itemize}
\item 'PUVTHDHU'  : \begin{itemize} \item altitude, pressure, dry potential temperature
and relative humidity at ground level, \item  $nu$ levels with pressure, zonal wind
component, meridian wind component, \item  $nm$ levels with pressure,
dry potential temperature and relative humidity.\end{itemize}
\item 'ZUVTHDMR'  : \begin{itemize} \item   altitude, pressure, dry potential temperature
and vapor mixing ratio at ground level, \item  $nu$ levels with  altitude, zonal wind
component, meridian wind component, \item  $nm$ levels with  altitude,
dry potential temperature and vapor mixing ratio.\end{itemize}
\end{itemize}

{\bf More idealized format} (Option: 'CSTN')

A more idealized manner to communicate with the program is to assume that
the atmosphere is composed of an arbitrary number $n-1$ of layers defined
by their altitudes $Z(k)$. Within each layer,
the stability is constant, and the wind and relative humidity vary linearly
with height. In that case, the input data consist in the surface level
information, followed by the moist V\"ais\"al\"a frequency of each layer
$N_v$, and the wind components and relative humidity at the layer
interfaces.
The virtual potential temperature on the $n$ input levels is retrieved
from the $N_{v}$ data by
\begin{equation}
\theta_{v}(k)  = \theta_{v}(k-1) \exp\left\{ \dfrac{N_{v}^{2}(k-1)}{g}
\left( Z(k)-Z(k-1) \right) \right\}
\end{equation}
for $k=2,n$. The situation concerning input data is then the same as for
the 'ZUVTHVHU' case in the 'RSOU' option, and subsequent treatment is similar.

{\bf Specifying a jet structure}

For other applications, one may want to have a jet structure $U(y,z)$ instead
of a horizontally uniform wind $U(z)$. Then we directly prescribe through
a Fortran function the following  value of the x-component wind U(y,z):

\begin{equation}
U(\hat{y},\hat{z})= { 1 \over \cosh \left(
  \left( { \hat{y} - \hat{y}_0 \over zwidthy } \right) ^2 +
  \left( { \hat{z} - \hat{z}_0 \over zwidthz } \right) ^2
 \right) }
\end{equation}

where $ zwidthy$ and  $ zwidthz $ are the characteristic sizes of the jet along
the horizontal and vertical directions. $\hat{y}_0$ and $\hat{z}_0$ are the
curvilinear coordinates  for the jet center.

An intermediate case is also available where the function U(y,z) is a separable
function of y and z:

$$ U(y,z) = U(z) * f(y)$$
 The vertical part $ U(z)$  is still obtained from the input atmospheric profile and
the horizontal part $f(y)$ is given by a Fortran function:
\begin{equation}
 f(\hat{y}) = { 1 \over \cosh \left(
   { \hat{y} - \hat{y_0} \over zwidth } \right) }
\end{equation}
with the same notations as for the jet case.

The jet structure or the separable function, can also be taken into account for
the V(x,z). All combinations between U(y,z) and V(x,z) (function of only z,
separable or non-separable function) are possible.


All these options lead to the prescription of the wind components along vertical
planes (I,K) for V(x,z) and (J,K) for U(y,z) and subsequent treatments
to balance the wind and the mass fields are the same.

\section{Vertical profile on the model grid}

For all the above options, the program will first convert the vertical profile
 in a common format, featuring only the surface altitude, pressure, virtual
temperature and vapor mixing ratio, together with the altitudes of the
data points, and the corresponding values of $\theta_v, r_v$, and wind
components.

The correspondance between pressure and altitude is achieved, when necessary,
by integration of the hydrostatic relation, starting from the surface level:
\begin{equation}
d \Pi = - \dfrac{g}{C_{pd} \theta_{v}}  dz
\end{equation}
with $P = P_{00} \Pi^{C_{pd}/R_{d}} $.

The virtual temperature is obtained as
\begin{equation}
T_{v} = \theta_{v} \Pi = \theta_{v} \left( \dfrac{P}{P_{00}} \right) ^{R_{d}/C_{pd}}
\end{equation}



In some cases, this requires the use of an iterative procedure
to derive the absolute temperature and the vapor mixing ratio from
available information (virtual temperature and relative humidity).
This is done independantly for each level, and
comprises the following steps, starting with $T=T_v$ as first guess:
\begin{enumerate}
\item The saturation vapor pressure $e_{s}(T)$ at temperature T is computed:
$$
e_{s}(T) =  \exp\left( \alpha_{w} - \dfrac{\beta_{w}}{T} - \gamma_{w}
\ln (T)\right)
$$

with
\begin{eqnarray*}
\alpha_{w} & = &  \ln (e_{s}(T_{t}) ) + \dfrac{\beta_{w}}{T_{t}} + \gamma_{w}
\ln (T_{t}) \\
\beta_{w} & = & \dfrac{L_{v}(T_{t})}{R_{v}} + \gamma_{w} T_{t} \\
\gamma_{w} & = & \dfrac{C_{l}-C_{pv}}{R_{v}}
\end{eqnarray*}
where $T_t$ is the triple point temperature.

\item  The partial pressure of the vapor $e= e_{s}(T_d)$
is computed from the relative humidity $Hu$ (in percents) and the saturation
vapor pressure $e_{s}(T)$:
$$
e= \dfrac{Hu}{100} e_{s}(T)
$$
\item  The vapor mixing ratio is evaluated by:
$$
r_{v} = \dfrac{R_{d}}{R_{v}} \;\; \dfrac{e}{P-e}
$$
\item   The temperature is deduced:
$$
T = T_{v} \dfrac{(1+r_{v})}{(1 + \dfrac{R_{v}}{R_{d}}r_{v})}
$$
\end{enumerate}
The iteration continues until convergence is achieved.

Once the variables are known on data points, they are linearly
interpolated to the vertical grid of the model without orography.
Here $\theta_v$ and $r_v$ are located on the mass points of the grid,
but the wind components are located on $w$ points for more accurate
estimate of the thermal wind balance. They will be called hereafter
$U$ and $V$.  The
next sections explain how the 3D fields are produced, starting from this
single profile.

\section{Wind and mass fields in the absence of orography}

The vertical profile obtained above is supposed to be valid at the point
$ILOC,JLOC$ of the horizontal model grid, and we want to reconstitute the
whole 3D fields.  As a first step in this computation, the program computes
the wind and mass fields on a 3D grid ignoring the orography (this grid is
exactly the model grid if there is no orography at all).

The wind is first projected on to model axes. At this stage,
the desired jet structure is introduced.

The virtual potential temperature on the model grid without
orography is then derived using the thermal wind balance, assuming the above
mentionned
$U$ and $V$ components of the wind can be taken as the geostrophic wind.

The continuous equation of the thermal wind balance is:
\begin{equation}
\dfrac{\partial \vec{V_{g}}}{\partial z} =
\dfrac{\vec{k}}{f} \wedge \dfrac{1}{\theta_{v\,ref}} g \vec{\nabla}_{z}
\theta_{v}
\end{equation}
where
\begin{equation}
\dfrac{\partial \Phi}{\partial z} = g \dfrac{ \theta_v - \theta_{v\,ref} }
{\theta_{v\,ref} }
\end{equation}
has been used. It follows
\begin{eqnarray}
\dfrac{\partial \theta_{v} }{\partial \overline{x}} & = &
d_{xx} \dfrac{f}{g}  \theta_{v\,ref}
\dfrac{1}{\widehat{d}_{zz}} \dfrac{\partial v_{g} }{\partial \widehat{z}}
\\
\dfrac{\partial \theta_{v} }{\partial \overline{y}} &  = & - d_{yy}
\dfrac{f}{g} \theta_{v\,ref}
\dfrac{1}{\widehat{d}_{zz}} \dfrac{\partial u_{g} }{\partial \widehat{z}}
\end{eqnarray}

This is discretized as
\begin{eqnarray}
\left( \delta _{x} \theta_{v}\right)_{(i,j,k)} & = & d_{xx}
 \dfrac{\overline{f}^x}{g}
\theta_{v\,ref}(k) \dfrac{1}{d_{zz}} \overline{\delta_{z} V}^x  \\
\left( \delta _{y} \theta_{v}\right)_{(i,j,k)} & = & - d_{yy}
 \dfrac{\overline{f}^y}{g}
\theta_{v\,ref}(k) \dfrac{1}{d_{zz}} \overline{\delta_{z} U }^y
\end{eqnarray}
where $U$ and $V$ are used as geostrophic wind components.

If the Coriolis parameter is equal to zero, $\theta_{v}$ is defined everywhere
as $\theta_{v}(k)$ at point $(ILOC,JLOC)$.

Note that at this stage $\theta_{v\,ref}$ is unknown, and an iterative
procedure must be used to enforce an exact
 thermal wind balance. For the first iteration, we initialize
$\theta_{v\,ref}$ by the value of $\theta_{v}$ at the point
$(ILOC,JLOC)$. At the following steps,
the virtual potential temperature for the anelastic reference state
is defined as the horizontal average of $\theta_{v}$:

$$
\theta_{v\,ref}(k) = \sum _{i,j} \theta_{v}(i,j,k)
$$

When the iteration is converged,
the remaining properties of the reference state can be computed.
The Exner function for the
reference state may then be obtained by integration of the hydrostatic relation
from $z=0$ ($\Pi_{ref} (k=KB) =1$):

$$
d \Pi_{ref}(k) = - \dfrac{g}{C_{pd} \theta_{v\,ref}(k)}  dz
$$
and the computation of the reference state characteristics is finalized
by the density of dry air:
$$
\rho_{d\,ref}(k) = \dfrac{P_{00} (\Pi)^{C_{pd}/R_d - 1} }{R_{d}
\theta_{v\,ref}(k) \left(1+r_{v}(k) \right) }
$$
\section{Add a perturbation}

In order to start a non-trivial experiment, it is often useful to add
a perturbation to the previously defined balanced fields. Two possibilities
are presently available:

\begin{itemize}
\item
add a perturbation to the $\theta $ field (convective bubble)
\begin{eqnarray}
&&Dist = \sqrt{ \left( \left( { \hat{x} - \hat{x}_0    \over  Rad_x } \right) ^2 +
                       \left( { \hat{y} - \hat{y}_0    \over  Rad_y } \right) ^2 +
                       \left( { \hat{z} - \hat{z}_0    \over  Rad_z } \right) ^2
                \right)       }  \nonumber \\
&&\theta ' =  Ampli \cos ^2  \left( {\pi \over 2} Dist  \right)  \ \ \
for \ Dist < 1  \nonumber \\
&&\theta ' = 0  \ \ \  \ \ \  \ \ \  \mbox{elsewhere}  \nonumber
\end{eqnarray}

\item
add a non-divergent perturbation to the horizontal wind, through the use of a
streamfunction $\Phi '$:
\begin{equation}
\Phi ' = Ampli \ \  e^{\left( { \hat{y} - \hat{y}_0    \over  Rad_y } \right) ^2 }
\  \cos \left( {2 \pi  \hat{x} \over  Rad_x } \right)   \nonumber
\end{equation}

This streamfunction is located at the vertical vorticity point and the $u'$
and $v'$ perturbations are computed at the right locations by:
\begin{eqnarray}
u' = - { \partial \Phi ' \over \partial \hat{y} } \nonumber \\
v' = + { \partial \Phi ' \over \partial \hat{x} } \nonumber
\end{eqnarray}
The discretizations use the grid-point curvilinear $ \hat{x}, \hat{y}$ because
at this point, the orography has  not yet  being  introduced in  the model.
This guarantees that the perturbation does not change the  divergence of the
previously obtained fields.
\end{itemize}



\section{Wind and mass fields with orography}

An orography following simple form can be specified:

It may be sine shaped:
$$
z_{s}(i,j) = h_{max} \left\{  \sin \left( \dfrac{\pi i}{IU-1}\right)^{2}
\right\}^{exp_{x}}\left\{  \sin \left( \dfrac{\pi j}{JU-1}\right)^{2}
\right\}^{exp_{y}}
$$

or bell-shaped :
$$
z_{s}(i,j) = h_{max} / \left\{ 1 + \left(\dfrac{x(i) - x_{s}}{a_{x}}\right)^{2}
+ \left(\dfrac{y(j) - y_{s}}{a_{y}}\right)^{2}
\right\}^{1.5}
$$
$IU$ and $JU$ are the total number of points along x and y directions (i.e.
including the external points). $h_{max}$, $exp_{x}$, $exp_{y}$, $a_{x}$,
$a_{y}$,$x_{s}$ and $y_{s}$ are the  degrees of freedom. $h_{max}$ is the maximum
height, $a_{x}$ and
$a_{y}$ represent widths along x and y directions, $x_{s}$ and $y_{s}$  are
the center
positions of the mountain.

In that case, an additional step is performed
to compute the 3D field on the model grid following the terrain.
The previous fields of $\theta_v$, $r_v$, $u$, and $v$
are linearly interpolated to the new grid. We also set the vertical component
$w$ equal to zero.
Then, $\theta$ is recovered by:
\begin{equation}
\theta = \theta_{v}\dfrac{(1+r_{v})}{(1 + \dfrac{R_{v}}{R_{d}}r_{v})}
\end{equation}
Finally, the  3D anelastic reference state (
{\it i.e.} on the grid with orography) is determined
 by vertical interpolation for  $\theta_{v\,ref}$ and
$\rho_{d\,ref}$ and by integration of the hydrostatic relation for
$\Pi_{ref}$
defined on the model grid without orography.

\section{Enforcement of the anelastic constraint}

Obviously, wind field obtained at this stage does not fulfil the anelastic constraint
%(\ref{anelcons})
(Eq. 2.14 in Part I: Dynamics)
and the free-slip boundary conditions. Therefore, a
last adjustement is necessary so that this wind field can be used as initial
field for the MESONH model. Thus, a perturbation deduced from a potential $Pot$
gradient  is added to the wind component $\vec{U}_0$ so that the resulting
 wind field $\vec{U}$ satisfies the anelastic constraint.

Let us write this constraint:

\begin{eqnarray}
\vec{U} = \vec{U} _0 - \vec {\nabla } Pot  \nonumber \\
\vec { \nabla } . \left( \rho _{dref}  \vec{U}  \right) = 0 \nonumber
\end{eqnarray}

From these 2 equations, we deduce an equation for the $Pot$ field given by:

\begin{eqnarray}
\vec { \nabla } .  \left( \rho _{dref} \vec {\nabla } Pot \right) = \vec { \nabla } .
 \left( \rho _{dref}  \vec{U} _0 \right) \label{eq:potential}
\end{eqnarray}

This equation is equivalent to the pressure equation solved in the temporal loop
of the model to determine the pressure function. Therefore, the same iterative
solver is used to solve the equation (\ref{eq:potential}) and the gradient of this
potential is added to find the final field $\vec{U}$ satifying  the anelastic constraint.


\section{Analytical 3D fields}

For the purpose of testing the post-processing and graphical facility, it
is also possible to generate a family of simple 3D analytical fields.
Three types of such fields are possible: spherical fields, plane
fields and wave fields.
\begin{itemize}
\item \underbar{spherical fields}:
\begin{eqnarray}
\overline{\rho^{*}}^{x} {\cal U} & = &  \overline{\rho^{*}}^{x}
\left\{ f_{0} + 0.5  (f_{1}- f_{0}) \sqrt{\dfrac{x-x_{0}}{L_{x}}
+ \dfrac{y_{m}-y_{0}}{L_{y}} + \dfrac{z_{m}-z_{0}}{L_{z}}}    \right\}
\\
\overline{\rho^{*}}^{y}{\cal V} & = &    \overline{\rho^{*}}^{y}
\left\{ f_{0} + 0.5  (f_{1}- f_{0}) \sqrt{\dfrac{x_{m}-x_{0}}{L_{x}}
+ \dfrac{y-y_{0}}{L_{y}} + \dfrac{z_{m}-z_{0}}{L_{z}}}    \right\}
\\
\overline{ \rho^{*}}^{z} w & = &  \overline{\rho^{*}}^{z}
\left\{ f_{0} + 0.5  (f_{1}- f_{0}) \sqrt{\dfrac{x_{m}-x_{0}}{L_{x}}
+ \dfrac{y_{m}-y_{0}}{L_{y}} + \dfrac{z-z_{0}}{L_{z}}}    \right\}
 \\
\rho^{*} \theta(k) & = &   \rho^{*} \left\{ f_{0}
+ 0.5  (f_{1}- f_{0}) \sqrt{\dfrac{x_{m}-x_{0}}{L_{x}}
+ \dfrac{y_{m}-y_{0}}{L_{y}} + \dfrac{z_{m}-z_{0}}{L_{z}}}    \right\}
\\
\rho^{*} r_{n} & = & 0.
\end{eqnarray}

\item \underbar{plane fields} :
\begin{eqnarray}
\overline{\rho^{*}}^{x} {\cal U} & = & \overline{\rho^{*}}^{x}
\left\{ f_{0} + (f_{1}- f_{0}) \left( \dfrac{x-x_{0}}{L_{x}} N_{x}
+ \dfrac{y_{m}-y_{0}}{L_{y}} N_{y} + \dfrac{z_{m}-z_{0}}{L_{z}} N_{z} \right)\right\}\\
\overline{\rho^{*}}^{y}{\cal V} & = &    \overline{\rho^{*}}^{y}
\left\{ f_{0} + (f_{1}- f_{0}) \left( \dfrac{x_{m}-x_{0}}{L_{x}} N_{x}
+ \dfrac{y-y_{0}}{L_{y}} N_{y} + \dfrac{z_{m}-z_{0}}{L_{z}} N_{z}\right)\right\} \\
\overline{ \rho^{*}}^{z} w & = &  \overline{\rho^{*}}^{z}
\left\{ f_{0} + (f_{1}- f_{0}) \left( \dfrac{x_{m}-x_{0}}{L_{x}} N_{x}
+ \dfrac{y_{m}-y_{0}}{L_{y}} N_{y} + \dfrac{z-z_{0}}{L_{z}} N_{z}\right)\right\} \\
\rho^{*} \theta(k) & = &   \rho^{*}
\left\{ f_{0} + (f_{1}- f_{0}) \left( \dfrac{x_{m}-x_{0}}{L_{x}} N_{x}
+ \dfrac{y_{m}-y_{0}}{L_{y}} N_{y} + \dfrac{z_{m}-z_{0}}{L_{z}} N_{z} \right)\right\}\\
\\
\rho^{*} r_{n} & = & 0.
\end{eqnarray}
where
\begin{eqnarray}
N_{x} & = & \cos(\alpha_{z})\sin(\alpha_{x}) \\
N_{y} & = & - \cos(\alpha_{z})\cos(\alpha_{x}) \\
N_{y} & = & - \sin(\alpha_{z})
\end{eqnarray}
\item \underbar{wave  fields}:
\begin{eqnarray}
\overline{\rho^{*}}^{x} {\cal U} & = & \overline{\rho^{*}}^{x}
\left\{ f_{0} + (f_{1}- f_{0}) \left( \dfrac{x-x_{0}}{L_{x}} N_{x}
+ \dfrac{y_{m}-y_{0}}{L_{y}} N_{y} + \dfrac{z_{m}-z_{0}}{L_{z}} N_{z}
\right) \right. \\
& + & \left. 0.125 (f_{1}+ f_{0}) \sin \left(2 \pi f_{2} (\dfrac{x-x_{0}}{L_{x}} C_{s}
+ \dfrac{y_{m}-y_{0}}{L_{y}} S_{n}) \right) \right\}
 \\
\overline{\rho^{*}}^{y}{\cal V} & = &    \overline{\rho^{*}}^{y}
\left\{ f_{0} + (f_{1}- f_{0}) \left( \dfrac{x_{m}-x_{0}}{L_{x}} N_{x}
+ \dfrac{y-y_{0}}{L_{y}} N_{y} + \dfrac{z_{m}-z_{0}}{L_{z}} N_{z} \right)\right.
\\ & + & \left.0.125 (f_{1}+ f_{0}) \sin \left( 2 \pi f_{2}( \dfrac{x_{m}-x_{0}}{L_{x}} C_{s}
+ \dfrac{y-y_{0}}{L_{y}} S_{n} ) \right) \right\}\\
\overline{ \rho^{*}}^{z} w & = &  \overline{\rho^{*}}^{z}
\left\{ f_{0} + (f_{1}- f_{0}) \left( \dfrac{x_{m}-x_{0}}{L_{x}} N_{x}
+ \dfrac{y_{m}-y_{0}}{L_{y}} N_{y} + \dfrac{z-z_{0}}{L_{z}} N_{z}\right)\right. \\
& + & \left. 0.125 (f_{1}+ f_{0}) \sin \left( 2 \pi f_{2} (\dfrac{x_{m}-x_{0}}{L_{x}} C_{s}
+ \dfrac{y_{m}-y_{0}}{L_{y}} S_{n} ) \right)\right\} \\
\rho^{*} \theta(k) & = &   \rho^{*}
\left\{ f_{0} + (f_{1}- f_{0}) \left( \dfrac{x_{m}-x_{0}}{L_{x}} N_{x}
+ \dfrac{y_{m}-y_{0}}{L_{y}} N_{y} + \dfrac{z_{m}-z_{0}}{L_{z}} N_{z} \right)\right.
\\ & + & \left. 0.125 (f_{1}+ f_{0}) \sin \left( 2 \pi f_{2} (\dfrac{x_{m}-x_{0}}{L_{x}} C_{s}
+ \dfrac{y_{m}-y_{0}}{L_{y}} S_{n} ) \right) \right\}\\
\\
\rho^{*} r_{n} & = & 0.
\end{eqnarray}
where
\begin{eqnarray*}
C_{s} & = &  \cos(\alpha_{x}) \\
S_{n} & = & \sin(\alpha_{x})
\end{eqnarray*}

$f_{0},f_{1},f_{2},\alpha_{x},\alpha_{z}$ are the degrees of freedom. $L_{x}$,
$L_{y}$,$L_{z}$ are the lengths of the physical part of the domain. $x_{0}$,
$y_{0}$ and $z_{0}$ represent the coordinates of the center point of  the
physical part of the domain.
\end{itemize}

