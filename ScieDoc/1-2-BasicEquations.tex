%%%%%%%%%%%%%%%%%%%%%%%%%%%%%%%%%%%%%%%%%%%%%%%%%%%%%%%%%%%%%%%%%%%%%%%%%%%%%%%
% CONTRIBUTION TO THE MESONH BOOK1: "Basic Equations"
% Author        : Ph. Bougeault, V. Ducrocq, J. Ph. Lafore, and P. J. Mascart
% Original      : September 1, 1995, Ph. Bougeault and V. Ducrocq (LH system only)
% Update        : March 1997, J. Ph. Lafore (side paper for Durran an MAE systems)
% Last Update   : June 2001, P. Mascart (merging with J. Ph. Lafore corrections)
% March 2008, J.-P. Chaboureau, editorial corrections
%%%%%%%%%%%%%%%%%%%%%%%%%%%%%%%%%%%%%%%%%%%%%%%%%%%%%%%%%%%%%%%%%%%%%%%%%%%%%%%

\chapter{Basic Equations}
\minitoc

\section{A family of anelastic approximations}

%A beneficial effect of the anelastic approximation is that it has no acoustic modes, and
%the time discretization may be entirely explicit without introducing an overly restrictive
%constraint on the time-step.
As mentioned above in section~\ref{CHP1anel},
% Provisoire avant d'integrer avec gros BOOK1
%1.2
% A utiliser ensuite
the size of time step that produces a stable numerical solution for explicit
finite-difference approximation to the Navier-Stokes equations is restricted by the rapid
propagation of sound waves. This time step limitation may be overcome by replacing the full
Navier-Stokes equations by an approximation that filters out the sound waves.
In search of such sound--proof systems, a wide family of anelastic approximation has been
proposed, and all these systems assume that thermodynamic state variables exhibit only
small departures from a suitably defined static reference state. The original form of the
anelastic approximation was proposed by Batchelor (1953)
and later refined by Ogura and Phillips (1962) as an improvement over the Boussinesq
approximation more suitable for studying convection within deep atmospheric layers. This
original anelastic system conserved modified forms of both total energy and Ertel potential
vorticity (Nance and Durran 1994; Bannon 1995). Unfortunately, this original anelastic
approximation only holds for an isentropic reference state, and is not really applicable to
realistic meteorological
investigation. Subsequently, this restriction was removed by Wihelmson and Ogura (1972) which
replaced the constant reference--state potential temperature with a vertically varying
reference--state potential temperature. This modified anelastic equation  set improves the
representation of deep convection but sacrifices potential vorticity and energy conservations
(Bannon 1995). A major improvement was provided by the Lipps and Hemler (1982) equation set
which allows an accurate representation of deep tropospheric phenomena, and
simultaneously conserves modified forms of Ertel potential vorticity and total energy for finite
amplitude disturbances (Bannon 1995). Finally, Durran (1989) has introduced the
pseudo-incompressible approximation  to obtain a better description of
upper--level isothermal layers, while retaining a good conservation of appropriate forms of
Ertel potential vorticity and total energy for finite amplitude displacements (Bannon 1995).

The current formulation of Meso--NH model allows for the use of three of these anelastic
systems: the Lipps and Hemler (1982) system (LH hereafter), the traditional Wihelmson and
Ogura (1972), a.k.a. the "Modified Anelastic Equations" (MAE hereafter ), and a simplified
implementation of the pseudo-incompressible system of Durran (1989, DUR hereafter). The
present chapter tries to describe these three implementations in an orderly way.

\section{Preliminary definitions}

We will use the following notations:  \\

the rotation velocity of the earth $\vec \Omega$,

the temperature $T$,

the pressure $P$,

the reference value of the pressure $P_{00}=100000 Pa$,

the total density of the moist air $\rho$,

the density of the dry fraction of the air $\rho_d$,

the specific heat at constant pressure of dry air $C_{pd}$,

the specific heat at constant pressure of water vapor $C_{pv}$,

the specific heat of liquid water $C_l$,

the specific heat of ice water $C_i$,

the gas constant for dry air  $R_d$,

the gas constant for water vapor $R_v$,

the latent heat of vaporization $L_v$,

the latent heat of sublimation $L_s$,

the latent heat of melting $L_m$.
\\


Various substances are measured by their mixing ratio $r_\bullet$, which is the
mass of the substance within a given volume, divided by the mass of dry air
within the same volume. In the current anelastic formulation, we assimilate
the dry air mass and the reference state dry air mass (see below), thus
the volumic mass may be recovered as $\rho_{d\,ref} r_{\bullet}$. \\

The treatment of the different phases of water is flexible.
Up to seven forms of water may be considered if the user wishes so, the
mixing ratios being $r_v$ for vapor, $r_c$ for cloud liquid water, $r_r$ for
liquid rain, $r_i$ for cloud ice, $r_s$ for snow, $r_g$ for graupels, and
$r_h$ for hail. The same mnemonics are used in the code, inasmuch as possible.
If a user is not interested by such computations, he may decide to use only a
subset of these variables. The computations concerning the remaining
water variables will not be performed, resulting in lesser cost. The
resulting equations may be obtained by setting the corresponding variables
to zero in the following of this chapter.\\

Whenever necessary, we use the mixing ratio of total water substance
\begin{equation}
r_w=r_v+r_c+r_r+r_i+r_s+r_g+r_h
\end{equation}

The specific heat at constant pressure of the moist air takes the general form
\begin{equation}
C_{ph}=C_{pd}+r_v C_{pv} + (r_c+r_r) C_l + (r_i+r_s+r_g+r_h) C_i .
\end{equation}


An important quantity is the Exner function
\begin{equation}
\Pi  =  ( P/P_{00} )^{R_d / C_{pd}} .
\end{equation}

We use the "dry" potential temperature
\begin{equation}
\theta  =  {T \over \Pi}.
\end{equation}

The virtual temperature is defined as
\begin{equation}
T_v= T \, (1+r_v R_v/R_d)/(1+r_w),
\end{equation}
and the virtual potential temperature as
\begin{equation}
\theta_v= \theta \, (1+r_v R_v/R_d)/(1+r_w).
\end{equation}

$\vec{U}$ is the air velocity, and whenever necessary, its Cartesian
components are called $u$, $v$, and $w$.

\section{Reference state}

All anelastic approximation systems rely on the hypothesis
that the atmosphere will not depart very far from a "reference
 state", defined as
an atmosphere at rest, in hydrostatic equilibrium, with horizontally
uniform profiles of temperature $T_{ref}(z)$ and water vapor
$r_{v\,ref}(z)$. No condensed water is considered in the reference state.
For our application, the reference
profiles are often chosen as the initial horizontal averages of actual fields
 over the expected domain of
simulation. Any profile, however, may be used, but the inaccuracy of the
computation increases if the reference state is far from the actual mean state.
In the following the dependency in $z$ only will be assumed for all quantities
subscripted with $()_{ref}$.\\

The hydrostatic relation and the equation of state are used to derive the
profiles of the virtual temperature and the Exner function of the reference
state $T_{v\,ref}$, $\Pi_{ref}$:

\begin{equation}
T_{v\,ref}=T_{ref}(1+r_{v\,ref} R_v/R_d)/(1+r_{v\,ref} )
\end{equation}

\begin{equation}
\dfrac{d(Log\Pi_{ref})}{dz} = -\dfrac {g}{C_{pd}T_{v\,ref}}
\end{equation}

with the upper boundary condition $\Pi_{ref} = \Pi_{ref}^{top}$
at the model top $z=H$.

\vspace{0.2cm}

$\theta_{v\,ref}$ is then deduced from $\Pi_{ref} $  and
$T_{v\,ref}$  as

\begin{equation}
\theta_{v\,ref} = \dfrac{T_{v\,ref}}{\Pi_{ref}}
\end{equation}

\vspace{0.2cm}

Finally, $\rho_{ref}$ is deduced from the equation of state
\begin{equation}
\rho_{ref} =
\dfrac{\Pi_{ref}^{C_{vd}/R_d}P_{00} }{R_d \theta_{v\,ref}} \label{state}
\end{equation}
and the density of the dry air fraction $\rho_{d\,ref}$ is deduced by
\begin{equation}
\rho_{d\,ref}={ \rho_{ref} \over (1+r_{v\,ref}) }
\end{equation}

Note that both the dry air and the water vapor must be considered to build the
profile of the Exner function of the reference state, since each gas
is subject to the partial pressure of the other one.

\section{Equation of state}

A primary difference between the three anelastic systems used in Meso--NH is the form
adopted for the equation of state.
Both the modified anelastic (MAE) and Lipps-Hemler (LH) systems follow
the usual framework of the Boussinesq approximation by linearizing the equation of state. On
the contrary, the Durran system (DUR) keeps the full equation of state, without any
linearization.

\subsubsection{Modified--anelastic and Lipps--Hemler systems}

For the MAE and LH cases, the equation of state $p=\rho R_d T_v$ is replaced by
the linearized form:

\begin{equation}
\rho ' =  \rho_{ref} \left( \dfrac{C_{vd}}{R_d}\dfrac{\Pi
'}{\Pi_{ref}} - \dfrac{\theta_v '}{\theta_{v\,ref}} \right),
\label{StateEq}
\end{equation}

where $\rho '= \rho - \rho_{ref}, \Pi '=\Pi - \Pi_{ref}$, and
$\theta_v '= \theta_v - \theta_{v\,ref}$.

\subsubsection{Durran system}

In the DUR case, contrary to most anelastic systems, the equation of state
\begin{equation}
p=\rho R_d T_v
\end{equation}
is not linearized, and thus is used without any approximation.

\section{Anelastic constraint}

For all three anelastic system used in Meso--NH, the continuity equation is may be
written as the approximated form:
\begin{equation}
\vec{ \nabla} \cdot
(\rho_{d\,eff} \vec{U}) = 0, \label{anelcons}
\end{equation}
\\
where $\rho_{d\,eff}$ is a convenient compact notation for:

%
\parbox{15cm}{
\begin{displaymath}
\rho_{d\,eff} = \left\{
\begin{array}{r@{\quad}cc}
\rho_{d\,ref}, \,\,\,\,\,\,\,\,\,\, \,\,\,\,\,\,
              & \quad &\mbox{(Lipps--Hemler)} \\
\rho_{d\,ref}, \,\,\,\,\,\,\,\,\,\, \,\,\,\,\,\,
              & \quad &\mbox{(Mod. Anelastic Eq.)}\\
\rho_{d\,ref} \theta_{v\,ref} (1+r_{v\,ref}).
              & \quad  &\mbox{(Durran)} 
\end{array}
\right.
\end{displaymath}
}
\hfill
\parbox{1cm}{
\begin{eqnarray}
\ \label{Lhconti} \\ \label{MAEconti}   \\ \label{DUconti}
\end{eqnarray}
}
%


Strictly speaking, the Durran (1989) form of the anelastic constraint (equation \ref{DUconti}) 
only holds in the adiabatic limit, and the rhs of (\ref{DUconti}) should be written 
$\dfrac{\cal H}{C_{pd}\Pi_{ref}}$ in the general case where $\cal H$ is the heating rate per 
unit volume. However, Durran (1989, p. 1454) argues that this generalized form reduces to 
(\ref{DUconti}) in the usual case where the anelastic assumption itself holds.\\


Also note that the  dimensional definition of $\rho_{d\,eff}$ is slightly inconsistent. It
corresponds to the dry air density of the reference state when the Lipps--Hemler deep
convection anelastic  (LH) or the Modified Anelastic Equation (MAE) approximations are used;
but it includes an extra potential temperature factor when the pseudo-incompressible anelastic
approximation (DUR) of Durran (1989) is used.\\

Also note that equation (\ref{anelcons}) represents a strong {\em geometrical} constraint on
the wind field, called the {\em anelastic constraint}. This constraint is enforced by
solving an elliptic equation for some pressure deviation function, that
results from the combination of the continuity and momentum equations (see below).

\section{Conservation of momentum}

The system is written in a referential frame linked to the earth, with
a rotation velocity $\vec{\Omega}$, and includes a yet un--specified momentum
source term $\vec{\cal F}$. In Meso-NH, unlike most atmospheric
models, the total density of the
air will vary in function of the precipitation and evaporation of water.
Therefore a consistent flux form of the equations can only be obtained by
combining the Lagrangian forms and the equation of continuity for dry air.
Furthermore, we want to allow for an easy transition to the compressible
system, should this one be adopted in the future. Therefore, we use an
equation of conservation for {\em the momentum of the dry air fraction} of the
fluid:

\begin{equation}
\dfrac{\partial}{\partial t}(\rho_{d\,eff} \vec{U})
+ \vec{ \nabla} \cdot (\rho_{d\,eff} \vec{U}\otimes \vec{U} )
%+\rho_{d\,eff} C_{pd} \theta_{v}\vec{ \nabla} \Pi '
+\rho_{d\,eff} \vec{{\cal F}_{\Pi}}
+ \rho_{d\,eff}  \vec{g} \dfrac{\theta_v-\theta_{v\,ref}}{\theta_{v\,ref}}
+ 2\rho_{d\,eff} \vec{\Omega} \wedge\vec{U}= \rho_{d\,eff} \vec{\cal F},
\label{EqMomentum}
\end{equation}

where $\rho_{d\,eff} \vec{{\cal F}_{\Pi}}$ is the pressure gradient force, which
takes different forms in the three systems:

%
\parbox{15cm}{
\begin{displaymath}
\rho_{d\,eff} \vec{{\cal F}_{\Pi}} =  \left\{
\begin{array}{l@{\quad}cc}
\rho_{d\,eff}  \vec{ \nabla}(C_{pd} \theta_{v\,ref} \Pi '),
                                        & \quad &\mbox{(Lipps-Hemler)}\\
\rho_{d\,eff} C_{pd} \theta_{v\,ref} \vec{ \nabla} \Pi ',
                                                & \quad &\mbox{(Mod. Anelastic  Eq.)} \\
\rho_{d\,eff} C_{pd} \theta_{v} \vec{ \nabla} \Pi '. & \quad  &\mbox{(Durran)}
\end{array}
\right.
\end{displaymath}
}
\hfill
\parbox{1cm}{
\begin{eqnarray}
\ \label{LhgradP} \\ \label{MAEgradP}   \\ \label{DUgradP}
\end{eqnarray}
}
%

A distinctive property of the Lipps and Hemler (1982) system is that the reference state virtual
potential temperature is pulled inside the pressure gradient term (\ref{LhgradP}), whereas
this factor appears outside of the gradient for both the MAE and Durran (1989) systems
((\ref{MAEgradP}) and (\ref{DUgradP})). For the MAE system also note that the virtual potential
temperature of the reference state is used, whereas the Durran (1989) system directly
retains the un--approximated virtual potential temperature in (\ref{DUgradP}).\\

It is also worth noticing that
the form adopted for the Durran (1989) system is identical to the momentum equation in the
complete compressible system. Although the hydrostatic reference state have been removed,
the pressure-gradient term has not been linearized. The Durran (1989) system is therefore
expected to provide a particularly accurate description of the (compressible) momentum
balance.\\

Apart from the differences in expressing the pressure gradient term, the other terms
retain their usual form, respectively representing the time evolution, the advection,
the buoyancy force, the Coriolis force, and the diabatic effects. Also note that the flux
form of the equation is  consistently kept throughout the model, insuring an integral
conservation of momentum to a very good
accuracy. For convenience, the model further uses $\theta_v$ rather than $\theta_v '$,
therefore, in the following equations, we will use $\theta_v-\theta_{v\,ref}$ to denote
the fluctuation of buoyancy.\\

Finally, enforcing the anelastic constraint (\ref{anelcons}) leads to the improperly
called "pressure problem" to determine $\Pi '$ (DUR and MAE) or $C_{pd} \theta_{v}  \Pi ' $
(LH), respectively. In order to simplify the expression of this problem, we introduce the
dynamical source of momentum:

\begin{equation}
\vec{S} = - \vec{ \nabla} \cdot
(\rho_{d\,eff} \vec{U}\otimes \vec{U} )
 - \rho_{d\,eff}  \vec{g} \dfrac{\theta_v-\theta_{v\,ref}}{\theta_{v\,ref}}
- 2\rho_{d\,eff} \vec{\Omega} \wedge\vec{U}  + \rho_{d\,eff} \vec{\cal F} ,
\end{equation}

whereby the momentum conservation equation may be rewritten:

%
\parbox{9cm}{
\begin{displaymath}
\dfrac{\partial}{\partial t}(\rho_{d\,eff} \vec{U})   =
\vec{S} - \left\{
\begin{array}{c}
\rho_{d\,eff}  \vec{ \nabla}(C_{pd} \theta_{v\,ref} \Pi ')\\
\rho_{d\,eff} C_{pd} \theta_{v\,ref} \vec{ \nabla} \Pi '  \\
\rho_{d\,eff} C_{pd} \theta_{v} \vec{ \nabla} \Pi '
\end{array}
\right\}.
\end{displaymath}
}
\hfill
\parbox{5cm}{
\begin{displaymath}
\begin{array}{c@{\quad}c}
\quad &\mbox{(Lipps-Hemler)} \\
\quad &\mbox{(Mod. Anelastic Eq.)} \\
\quad &\mbox{(Durran)}
\end{array}
\end{displaymath}
}
\hfill
\parbox{1cm}{
\begin{eqnarray}
\ \label{LhMom} \\ \label{MAEMom}   \\ \label{DuMom}
\end{eqnarray}
}
%

\section{Hydrostatic equation}

When an hydrostatic approximation of the vertical velocity
equation is required, for instance  within initialization routines or to compute
diagnostic quantities,
the hydrostatic form to be used also differs for the three equations systems.\\

With the pseudo--incompressible system of Durran (1989), as the vertical momentum equation
is written without any approximation, the hydrostatic relation is exact and reads:

\begin{equation}
\dfrac{\partial \Pi '}{\partial z} = \dfrac{g} {C_{pd} \theta_{v}}
\dfrac{\theta_v-\theta_{v\,ref}} {\theta_{v\,ref}}.
\label{hydrodur}
\end{equation}

With both the MAE and Lipp--Hemler systems, the usual form of the hydrostatic
relation $dp=-\rho g dz$ should not be used to establish diagnostic quantities
related to this model. For the Lipps--Hemler (1982) system, by consistency with
(\ref{EqMomentum}--\ref{LhgradP}), one should use instead:

\begin{equation}
\dfrac{\partial (C_{pd} \theta_{v\,ref} \Pi ')}{\partial z} = g \dfrac{\theta_v-\theta_{v\,ref}}
{\theta_{v\,ref}}.
\end{equation}

Whereas the MAE system leads to a formula similar to (\ref{hydrodur}), except for the use
of the reference state virtual potential temperature instead of the virtual potential
temperature:

\begin{equation}
\dfrac{\partial \Pi '}{\partial z} = \dfrac{g} {C_{pd} \theta_{v\,ref}}
\dfrac{\theta_v-\theta_{v\,ref}} {\theta_{v\,ref}}.
\end{equation}

\section{Thermodynamic equation}
For the time being, the prognostic energy variable is the dry potential
temperature $\theta$. In order to account for the effects of moisture, its
equation takes the following original form

\begin{eqnarray}
\dfrac{\partial}{\partial t}(\rho_{d\,eff} \theta)  + \vec{ \nabla} \cdot
(\rho_{d\,eff} \theta \;\vec{U})
= \rho_{d\,eff} \left[ {R_d+r_vR_v\over R_d}{C_{pd} \over C_{ph}}-1 \right]
{\theta \over \Pi_{ref}} w {\partial \Pi_{ref} \over \partial z}
\nonumber \\
+ {\rho_{d\,eff}\over \Pi_{ref} C_{ph}} \left[
 L_m {D(r_i+r_s+r_g+r_h)\over Dt} - L_v {Dr_v\over Dt} + {\cal H}  \right].
\end{eqnarray}

The terms on the right-hand side represent respectively the moist correction
in absence of any phase change (derived from the conservation of total
energy, with some approximations), the effects of phase changes, and the other
diabatic effects (radiation and diffusion). The use of the reference state
Exner function instead of the total one in the above equation allows for an
effective decoupling of the pressure problem from the thermodynamical
problem, leading to a great simplification of the resolution, while
retaining an excellent accuracy.

\section{Conservation of moisture}

For any of the water substances $r_{\star}$, the conservation equation
is written:

\begin{equation}
\dfrac{\partial}{\partial t}(\rho_{d\,eff} r_{\star})  + \vec{ \nabla} \cdot
(\rho_{d\,eff} r_{\star} \vec{U}) = \rho_{d\,eff} {\cal Q}_{\star},
\end{equation}

where ${\cal Q}_{\star}$ stands for the effects of phase changes, sedimentation
and diffusion. Note the flux form, and the use of $\rho_{d\,eff}$
(not $\rho_{ref}$),
insuring the existence of integral conservation properties.

\section{Conservation of passive scalars}

Similarly, the model can carry an arbitrary number of passive scalars,
following the equation:

\begin{equation}
\dfrac{\partial}{\partial t}(\rho_{d\,eff} s_{\star})  + \vec{ \nabla} \cdot
(\rho_{d\,eff} s_{\star} \vec{U}) = \rho_{d\,eff} {\cal S}_{\star},
\end{equation}

where ${\cal S}_{\star}$ stands for the effects of  diabatic and chemical
processes.

\section{Conservation of total mass}

The total mass inside the model domain is the sum of the mass of dry air
and the mass of water (other substances are neglected for the time
being)
\begin{equation}
{\cal M}={\cal M}_d + {\cal M}_w
\end{equation}

The anelastic constraint (see above)
does not supply the variation of the total mass of dry air ${\cal M}_d (t)$
inside the model domain, since it uses the reference
density profile instead of the actual density. It is therefore necessary
to use an additional equation. This one depends on the general conditions
of the experiment:
\begin{itemize}
\item If periodic or wall boundary conditions are assumed, the total mass
of dry air may not vary during a simulation. It is therefore specified
once and for all at the beginning (${\cal M}_d(t)={\cal M}_d(t_0)$).
\item If the model is driven by larger-scale meteorological information
(forecasts or analyses), the time variation of the total mass of dry
air is assumed to be entirely governed by the larger-scale fields. This
is consistent with procedures currently used in other limited area models. In
that case, the value ${\cal M}_d(t)$ of the total mass of dry air within the
simulation volume is regularly updated by linear interpolation in time
between the values deduced from the larger-scale fields.
\end{itemize}

On the other hand, the total mass of water ${\cal M}_w(t)$ can be computed
at each time during the simulation, by a simple volume integration

\begin{equation}
{\cal M}_w = \int_V \rho_{d\,ref} r_w dV
\end{equation}
where $V$ is the total volume of the model domain. Note that this results
in the total mass varying with precipitation and evaporation of water,
as it should, contrary to what was assumed in most models in the past.
Also, when the model is forced by larger-scale fields, the evolution of
the total mass of water in the model may be different from
that in the forcing fields, since the representation
of physical processes is different.

\section{Pressure equation}

For all three anelastic system implemented in Meso-NH, combining the anelastic constraint
(\ref{anelcons}) and the momentum equations (\ref{LhMom}--\ref{DuMom}), some appropriate
"pressure function" can be retrieved by solving an elliptic problem:

%
\parbox{2cm}{
\quad
}
\hfill
\parbox{6cm}{
\begin{displaymath}
\vec{\nabla} \cdot  \left\{
\begin{array}{c}
\rho_{d\,eff}  \vec{ \nabla}(C_{pd} \theta_{v\,ref} \Pi ')\\
\rho_{d\,eff} C_{pd} \theta_{v\,ref} \vec{ \nabla} \Pi '  \\
\rho_{d\,eff} C_{pd} \theta_{v} \vec{ \nabla} \Pi '
\end{array}
\right\}  = \vec{\nabla} \cdot \vec{S},
\end{displaymath}
}
\hfill
\parbox{4cm}{
\begin{displaymath}
\begin{array}{c@{\quad}c}
\quad &\mbox{(Lipps-Hemler)} \\
\quad &\mbox{(Mod. Anelastic Eq.)} \\
\quad &\mbox{(Durran)}
\end{array}
\end{displaymath}
}
\hfill
\parbox{1cm}{
\begin{eqnarray}
\ \label{LhPres} \\ \label{MAEPres}   \\ \label{DuPres}
\end{eqnarray}
}
%

the details of the procedure, however, are system--dependent.\\

\subsubsection{Lipps--Hemler system}

With the Lipps--Hemler (1982) system, it is convenient to define the pressure
function in (\ref{LhPres}), as:

\begin{equation}
\Phi= C_{pd} \theta_{v\,ref} \Pi ',
\end{equation}

and to solve directly (\ref{LhPres}) for $\Phi$, which can be interpreted as
a geopotential perturbation.\\

As the boundary condition for the elliptic equation (\ref{LhPres}) are
Neuman ones (see below and Chapter 4) the solution $\Phi$, hence the pressure, is determined
only to an arbitrary constant. This is sufficient for the dynamical problem;
however, for thermodynamic computations, it is often necessary to use the
absolute value of the pressure.
Let us assume that $\Phi_{fg}$ is a particular solution of the above elliptic
problem. We want to
determine the additional constant $\Phi_0$ that will insure the correct
absolute value of the pressure.  This is achieved by using the total
mass as defined above, which may be developed following:

\begin{equation}
{\cal M} = \int _{V} \rho \;  dV =  \int _{V} \rho_{ref} \;  dV +
\int _{V} \rho '  \; dV
 =   {\cal M}_{ref}  +
\int _{V} \rho ' \;  dV,
\label{MeanMassEq}
\end{equation}
where ${\cal M}_{ref}$ is the mass of the reference state (including both
dry air and water), computed once and for all at the beginning of each
experiment.

As the linearized equation of state (\ref{StateEq}) replaces the usual ideal--gas law for
density fluctuations in the Lipps--Hemler system, we have then
\begin{equation}
{\cal M}(t)  =    {\cal M}_{ref}  + \int _{V} \rho_{ref} \;  \left(
 \dfrac{C_{vd}}{R_d} \;  \dfrac{1}{C_{pd}\theta_{v\,ref}}
 \; \dfrac{\Phi_{fg} +\Phi_{0} }{ \Pi_{ref}} \;
- \;  \dfrac{\theta_v -\theta_{v\,ref}}{\theta_{v\,ref}}
\right) dV
\end{equation}
which is solved as
\begin{equation}
\Phi_{0}   =  \dfrac{{\cal M}(t) - 2 \times {\cal M}_{ref}
 \; +  \; \int _{V} \dfrac{\rho_{ref}}{\theta_{v\,ref}}
\left(\theta_v \;  -  \dfrac{C_{vd}}{R_d}
 \; \dfrac{\Phi_{fg}}{C_{pd}\Pi_{ref}}\right) \;  dV  }{ \int _{V} \;
 \dfrac{\rho_{ref}}{\theta_{v\,ref}}
 \;  \dfrac{C_{vd}}{R_d}\dfrac{1}{C_{pd}\Pi_{ref}}  \;  dV }
\end{equation}

The resulting field of $\Phi_{fg} + \Phi_0$ is used to retrieve the
absolute pressure
\begin{equation}
P= P_{00} \left(\Pi_{ref} + \dfrac{\Phi_{fg} + \Phi_0}{C_{pd}
\theta_{v\,ref} }\right)^{C_{pd}/R_d}
\end{equation}

\subsubsection{Modified anelastic system}

With the MAE system, solving the elliptic equation (\ref{MAEPres}) directly provides
the Exner function deviation $\Pi '$ from the reference state, instead of the above
defined  $\Phi$ function. \\

However, the rest of the solution procedure follows the principles outlined for
the Lipps--Hemler system. As the boundary condition for the elliptic equation (\ref{LhPres})
are Neuman ones, the Exner function $\Pi '$ is determined to an arbitrary constant only, and
the absolute value of the pressure is obtained as above. Assuming that $\Pi '_{fg}$ is
a particular solution of the elliptic problem, the additional constant $\Pi_0$ that
will insure the correct absolute value of the pressure is derived from the total mass budget,
written using (\ref{MeanMassEq}) without changes.\\

%\begin{equation}
%{\cal M} = \int _{V} \rho \;  dV =  \int _{V} \rho_{ref} \;  dV +
%\int _{V} \rho '  \; dV
% =   {\cal M}_{ref}  +
%\int _{V} \rho ' \;  dV
%\end{equation}
%where ${\cal M}_{ref}$ is the mass of the reference state (including both
%dry air and water), computed once and for all at the beginning of each
%experiment.

As the linearized equation of state (\ref{StateEq}) replaces the usual ideal--gas law
for density fluctuations in the MAE system, we have then
\begin{equation}
{\cal M}(t)  =    {\cal M}_{ref}  + \int _{V} \rho_{ref} \;  \left(
 \dfrac{C_{vd}}{R_d}
% \;  \dfrac{1}{C_{pd}\theta_{v\,ref}}
 \; \dfrac{\Pi '_{fg} +\Pi_{0} }{ \Pi_{ref}} \;
- \;  \dfrac{\theta_v -\theta_{v\,ref}}{\theta_{v\,ref}}
\right) dV
\end{equation}
which is solved as
\begin{equation}
\Pi_{0}   =  \dfrac{{\cal M}(t) - 2 \times {\cal M}_{ref}
 \; +  \; \int _{V} \rho_{ref}
 %\dfrac{\rho_{ref}}{\theta_{v\,ref}}
\left( \dfrac{\theta_v}{\theta_{v\,ref}} \;  -  \dfrac{C_{vd}}{R_d}
 \; \dfrac{\Pi '_{fg}}{\Pi_{ref}}\right) \;  dV  }{ \int _{V} \;
 \dfrac{\rho_{ref}}{\Pi_{ref}}
 \;  \dfrac{C_{vd}}{R_d}  \;  dV }
\end{equation}

The resulting field of $\Pi '_{fg} + \Pi_0$ is used to retrieve the
absolute pressure
\begin{equation}
P= P_{00} \left(\Pi_{ref} + \Pi '_{fg} + \Pi_0 \right)^{C_{pd}/R_d}
\end{equation}

\subsubsection{Durran system}

Finally, with the Durran (1989) system, solving the elliptic equation (\ref{DuPres})
directly provides the Exner function deviation $\Pi '$ from the reference state.
As it was the case for other systems in previous sections, the Neuman boundary
conditions only determine $\Pi '$  to an arbitrary constant, and this constant has to be
obtained from a total mass budget to retrieve the absolute value of the pressure field.
However, as the Durran (1989) system use the equation of state without linearization, a
modified procedure is followed. \\

Let us assume that ${\Pi '}_{fg}$ is a particular solution of the above elliptic
problem. We want to determine the additional constant $\Pi_0$ that will insure the correct
absolute value of the pressure.  This is achieved by using the total mass as defined above:
\begin{equation}
{\cal M} = \int _{V} \rho \;  dV.
\end{equation}

Combining the ideal--gas equation of state and the Exner function definition:
\begin{equation}
\rho =
\dfrac{\Pi^{C_{vd}/R_d}P_{00} }{R_d \theta_{v}} \label{DURstate}
\end{equation}

the total mass of the model can be written as:
\begin{equation}
{\cal M}(t)  =
  \int _{V}  \dfrac{P_{00} }{R_d \theta_{v}}
  \left( \Pi_{ref} + {\Pi '}_{fg} + \Pi_0 \right)^{C_{vd}/R_d} dV.
\end{equation}

As $\Pi_{ref} + {\Pi '}_{fg} \gg \Pi_0$, a linearization of the $\Pi_0$
term allows to solve the above equation as

\begin{equation}
\Pi_{0}   = \left(
  \dfrac{ {\cal M}_{d}(t) + {\cal M}_{w}(t)
 \; +  \; \int _{V}  \dfrac{P_{00} }{R_d \theta_{v}}
  \left( \Pi_{ref} + {\Pi '}_{fg} \right)^{C_{vd}/R_d} dV }
{ \int _{V}  \dfrac{P_{00} C_{vd}} {R_d^2 \theta_{v}} dV }
                                  \right)^{R_d/C_{vd}}
\end{equation}

Eventually, a second iteration is performed using the same equation to get
an accurate value of $\Pi_0$.
The resulting constant is used to retrieve the absolute pressure:
\begin{equation}
P= P_{00} \left( \Pi_{ref} + {\Pi '}_{fg} + \Pi_0 \right)^{C_{pd}/R_d}
\end{equation}

\section{Boundary conditions}

{\bf Top boundary}
\\
\\
The model is assumed to be limited by a rigid horizontal lid, exerting a
free slip condition on the atmosphere. The height $H$ of this lid may be
chosen by the user. In order to obtain realistic results, it is recommended
to place $H$ somewhere in the stratosphere. At $z=H$, the conditions imposed
on the model are:
\begin{equation}
\vec{U}\cdot\vec{n} = 0
\end{equation}
%
\parbox{2cm}{
\quad
}
\hfill
\parbox{5cm}{
\begin{displaymath}
\vec{\nabla} \left\{
\begin{array}{c}
\Phi \\
\Pi '  \\
\Pi '
\end{array}
\right\} \cdot \vec{n} = \vec{S}\cdot \vec{n}
\end{displaymath}
}
\hfill
\parbox{5cm}{
\begin{displaymath}
\begin{array}{c@{\quad}c}
\quad &\mbox{(Lipps-Hemler)} \\
\quad &\mbox{(Mod. Anelastic Eq.)} \\
\quad &\mbox{(Durran)}
\end{array}
\end{displaymath}
}
\hfill
\parbox{1cm}{
\begin{eqnarray}
 \\   \\
\end{eqnarray}
}
%

where $\vec{n}$ is a vertical unit vector. \\

In order to prevent the reflexion of gravity waves on this lid, an absorbing
layer may be activated, where the model prognostic variables are relaxed
towards the large-scale values.
\\
\\
{\bf Bottom boundary}
\\
\\
In the adiabatic formulation of the model, the lower boundary condition is
defined as an insulated rigid lid, with free slip. This may be again
formulated as
\begin{eqnarray}
\vec{U}\cdot\vec{n}&=&0  \\
\end{eqnarray}

%
\parbox{2cm}{
\quad
}
\hfill
\parbox{5cm}{
\begin{displaymath}
\vec{\nabla} \left\{
\begin{array}{c}
\Phi \\
\Pi '  \\
\Pi '
\end{array}
\right\} \cdot \vec{n} = \vec{S}\cdot \vec{n}
\end{displaymath}
}
\hfill
\parbox{5cm}{
\begin{displaymath}
\begin{array}{c@{\quad}c}
\quad &\mbox{(Lipps-Hemler)} \\
\quad &\mbox{(Mod. Anelastic Eq.)} \\
\quad &\mbox{(Durran)}
\end{array}
\end{displaymath}
}
\hfill
\parbox{1cm}{
\begin{eqnarray}
 \\   \\
\end{eqnarray}
}
%

but $\vec{n}$, a unit vector normal to the earth surface, may not be vertical.\\

In the physical package of the model, the turbulent fluxes of heat, moisture,
and momentum are computed in the usual way.
\\
\\
{\bf Lateral boundary}
\\
\\
This is a more complicated problem. We allow for several types of conditions:

\begin{itemize}
\item Periodic conditions
\item Rigid wall
\item Free propagation of waves out of the simulation domain
\item Evolution  of the prognostic variables imposed by external information
from larger scale
\item In some cases, a mixture of the previous conditions can be used
\end{itemize}

This will be discussed in detail in Chapter \ref{LateralBoundCond}.

\section{Energy and potential vorticity conservation}

A thorough discussion of the conservation properties for all three anelastic systems
implemented in Meso-NH is given in Bannon (1995). Only a summary of the main conclusions
is therefore given below, readers being referred to the original paper for
derivations and details. Following Bannon (1995), we restrict the scope of this discussion to a
{\em dry} atmosphere, and consider a finite amplitude  flow with $\cal H$ representing
the heating rate per unit volume, and $\vec{\cal F}$ the frictional forces. When a conservation
property is existing, it will be given in a flux form, assuring that the volume integral of the
relevant quantity is conserved for adiabatic inviscid flows in a closed domain for which the
normal component of the velocity field vanishes on the boundaries.

\subsubsection{Modified anelastic system}

When the potential temperature of the reference state varies with height, neither a closed form
of the energy, nor a consistent statement of potential vorticity conservation are possible.
The case of the constant reference state potential temperature corresponds to the original
anelastic  equations  (Ogura and Phillips 1962), which conserve both an energy form and
an Ertel potential vorticity form (see Bannon 1995, and Durran 1989, p 1459).

\subsubsection{Lipps--Hemler system}

This system conserves both a form of energy, and an Ertel--like potential vorticity.
Bannon (1995, p 2305) writes the total energy equation in flux form  as:

\begin{equation}
\dfrac{\partial}{\partial t}(E_{LH})
 + \vec{ \nabla} \cdot [(E_{LH} + P ') \vec{U}]
  = {\cal H} + \rho_{d\,ref} \vec{U} \cdot  \vec{\cal F},
\end{equation}

where $P '$ is the pressure deviation from reference state,
and $E_{LH}$ is the appropriate form of total energy in the LH system:
\begin{equation}
E_{LH} = \rho_{d\,ref} \left\{ { \frac{1}{2}\, \vec{U} \cdot \vec{U}}
         + g z  + C_{pd} \theta \right\}
\end{equation}

Note the use of the reference state density. Bannon (1995) further notes that the
presence of $C_{pd}$ instead of $C_{vd}$ in the
internal energy term reveals that in the LH system heating occur at
constant pressure, as is required by this system's approximations. \\

The Lipps and Hemler (1982) anelastic form of Ertel's conservation theorem is:

\begin{equation}
\dfrac{D}{Dt}\left({PV}_{LH}\right) =
\dfrac{\vec{\omega}_a \cdot \vec{\nabla} \dot{\Theta}}{\rho_{d\,ref}}
+ \dfrac{\vec{\nabla}\theta}{\rho_{d\,ref}}
\left(\vec{\nabla}\times\dfrac{\vec{\cal F}}{\rho_{d\,ref}}\right)
\end{equation}

where $\vec{\omega_a}  = 2 \vec{\Omega} + \vec{\nabla}\times\vec{U}$ is the absolute
vorticity, $\dot{\Theta}$ the diabatic warming rate:

\begin{equation}
\dot{\Theta} = \dfrac{\theta\; \cal H}{\rho_{d\,ref} C_{pd} T}
\end{equation}

and the appropriate form of the potential vorticity is (Bannon 1995, p 2306):

\begin{equation}
{PV}_{LH} = \dfrac{\vec{\omega_a} \cdot \vec{\nabla} \theta}{\rho_{d\,ref}}.
\end{equation}

Note that the reference state density has to be used, not the true density.

\subsubsection{Durran system}

The pseudo--incompressible system of Durran (1989) also exhibits conservation properties
for both a total energy and an Ertel--like potential vorticity. \\

Defining the pseudo--incompressible density:

\begin{equation}
\rho^{\star} = \rho_{d\,ref} \dfrac{\theta_{ref}}{\theta}
\end{equation}

The flux form of the total energy equation is (Durran, 1989):

\begin{equation}
\dfrac{\partial}{\partial t}(E_{D})
 + \vec{ \nabla} \cdot [(E_{D} + P^{\star}) \vec{U}]
    = {\cal H} + \rho^{\star} \vec{U} \cdot  \vec{\cal F},
\end{equation}

where $P^{\star} = P_{ref} +  \rho_{d\,ref} C_{pd} \theta_{ref} \Pi ' $,
and $E_{D}$ is the appropriate form of total energy in the Durran (1989) system:
\begin{equation}
E_{D} = \rho^{\star} \left\{ { \frac{1}{2}\, \vec{U} \cdot \vec{U}}
           + g z \right\} + \rho_{d\,ref} C_{vd} T_{ref}
\end{equation}


The pseudo--incompressible version of Ertel's conservation theorem is:

\begin{equation}
\dfrac{D}{Dt}\left({PV}_{D}\right) =
\dfrac{\vec{\omega}_a \cdot \vec{\nabla} \dot{\Theta}}{\rho^{\star}}
+ \dfrac{\vec{\nabla}\theta}{\rho^{\star}}
\left(\vec{\nabla}\times\dfrac{\vec{\cal F}}{\rho}\right)
\end{equation}

where ${PV}_{D}$ is the appropriate form of the potential vorticity
(Bannon 1995, p 2308):

\begin{equation}
{PV}_{D} = \dfrac{\vec{\omega_a} \cdot \vec{\nabla} \theta}{\rho^{\star}}.
\end{equation}

Note that the pseudo--incompressible density has to be used, not the reference
state one.


\section{References}
\decrefname
Bannon, P. R., 1995:
Potential vorticity conservation, hydrostatic adjustment,
and the anelastic approximation.
{\it J. Atmos. Sci.}, {\bf 52}, 2302--2312
\decrefname
Batchelor, G. K., 1953:
The condition for dynamical similarity of motions of frictionless
perfect-gas atmosphere.
{\it Quart. J. Roy. Meteor. Soc.}, {\bf 79}, 224--235.
\decrefname
Durran, D. R., 1989:
Improving the anelastic approximation.
 {\it J. Atmos. Sci.}, {\bf 46}, 1453--1461.
\decrefname
Lipps, F., and R. S. Hemler, 1982:
A scale analysis of deep moist convection
and some related numerical calculations.
{\it J. Atmos. Sci.}, {\bf 39}, 2192--2210.
\decrefname
Lipps, F., 1990:
On the anelastic approximation for deep convection.
{\it J. Atmos. Sci}., {\bf 47}, 1794--1798.
\decrefname
Nance, L. B., and D. R. Durran, 1994:
A comparison of the accuracy of three anelastic systems and
the pseudo--incompressible system.
{\it J. Atmos. Sci.}, {\bf 51}, 3549--3565.
\decrefname
Ogura, Y., and N. A. Phillips, 1962:
Scale analysis of deep and shallow convection in the atmosphere.
{\it J. Atmos. Sci.}, {\bf 19}, 173--179.
\decrefname
Wilhelmson, R., and Y. Ogura, 1972:
The pressure perturbation and the numerical modelling of a cloud,
{\it J. Atmos. Sci.}, {\bf 29}, 1295--1307.

%%%%%%%%%%%%%%%%%%%%%%%%%%%%%%%%%%%%%%%%%%%%%%%%%%%%%%%%%%%%%%%%%%%%%%%%%%%%%%%
%  END OF CONTRIBUTION: "Basic Equations"
%%%%%%%%%%%%%%%%%%%%%%%%%%%%%%%%%%%%%%%%%%%%%%%%%%%%%%%%%%%%%%%%%%%%%%%%%%%%%%%
