%%%%%%%%%%%%%%%%%%%%%%%%%%%%%%%%%%%%%%%%%%%%%%%%%%%%%%%%%%%%%%%%%%%%%%%%%%%%%%%
%%%%%%%%%%%%%%%%%%%%%%%%%%%%%%%%%%%%%%%%%%%%%%%%%%%%%%%%%%%%%%%%%%%%%%%%%%%%%%%
% CONTRIBUTION TO THE MESONH BOOK1: "The radiation parameterization"
% Author : Jean-Pierre Pinty
% Original : October 15, 1997
% Update   : October 15, 1997
%%%%%%%%%%%%%%%%%%%%%%%%%%%%%%%%%%%%%%%%%%%%%%%%%%%%%%%%%%%%%%%%%%%%%%%%%%%%%%%
%%%%%%%%%%%%%%%%%%%%%%%%%%%%%%%%%%%%%%%%%%%%%%%%%%%%%%%%%%%%%%%%%%%%%%%%%%%%%%%
%%%%%%%%%% Definition of new commands for LATEX :
% definition of a new environment mpa and a new command \nr for the footnote that must appear for the
%  variables in namelist EXSEG which are not read.
% This command is not at present used.
%\renewcommand{\thefootnote}{\fnsymbol{footnote}}
%\renewcommand{\thempfootnote}{\fnsymbol{mpfootnote}}
%\renewcommand{\thefootnote}{\alph{footnote}}
%%%%%%%%%%%%%%%%%%%%%%%%%%%%%%%%%%%%%%%%%%%%%%%%%%%%%%%%%%%%%%%%%%%%%%%%%%%%%%%
%
%
\chapter{The radiation parameterization}\label{RADIATIONS}
\minitoc
%
\section{Introduction to the radiation code}
%
\subsection{Purpose}
%

This section describes in details the radiation scheme which has been interfaced
with {\bf M\'eso-NH}.

The radiative scheme\footnotemark
%
\footnotetext{Notice that the FORTRAN code is NOT at the {\bf M\'eso-NH} norm.
The package is not written in F90 and furthermore the physical constants used
in the different routines (COMMON \/YOMCST\/) may differ from those used
elsewhere in {\bf M\'eso-NH}}
%
is used at ECMWF and so is why a large documentation can be found about it.
It was described, in particular, in the Technical Memorandum 165 "Description of
the radiation scheme in the ECMWF model" by Jean-Jacques Morcrette
(1989).  The present documentation is directly based on this paper and on the
documentation of ARPEGE-Climat at M\'et\'eo-France (Dandin, 1995). Recent
modifications have not been taken into account ({\it e.g.} iced clouds, LW
effects of some trace gases, climatological ozone and aerosol initialisation)
in the present code version but will be included in the next release of the
code.\\

The package\footnotemark
%
\footnotetext{based upon the Fouquart-Morcrette radiation code}
%
calculates the radiative fluxes taking into account
absorption-emission of longwave radiation and reflection, scattering and
absorption of solar radiation by the earth's atmosphere and surfaces.
Outline of the code, in the format of Stephens's (1984) paper, is given
in the following table. Notice that although the radiation parameterization
can deliver a great deal of fluxes (up/down LW/SW in several bands), the
package and its interface with {\bf M\'eso-NH} provides only an update of the
surface shortwave and longwave fluxes (with slope angle corrections) and the 3D
net radiative tendency of the thermodynamical variable $\theta$ to the regular
user. In addition, some specific use of the scheme are available to save
computation ressources when a great accuracy of the radiative effects is
superfluous. This includes the possibility of calling the package at a different
rate of the model timestep, with the "clear sky" approximation, for the "cloud
only" air columns or for a larger step of the horizontal sweep out of the air
columns. All the details of the parameterization and calling interface will be
exposed in this section.\\

The aim of the routine RADIATIONS is to produce the net total radiative heat
flux $F$ used to evaluate the potential temperature tendency term:

$$\frac{\partial \theta}{\partial t} = \frac{g}{C_{ph}}\Pi \frac{\partial F}{\partial p}, $$

\noindent where $F$ is a net flux: {\em i.e.} $F = F^{\!\uparrow} + F^{\!\downarrow}$
sum of the upward $F^{\!\uparrow}$ and downward $F^{\!\downarrow}$ fluxes,
and a total flux: {\em i.e.}  $F = F_{LW} + F_{SW}$ sum of the solar or
shortwave $F_{SW}$ and atmospheric or longwave $F_{LW}$ fluxes. Also in order
to drive the surface processes scheme, the routine RADIATIONS provides the
total downward surface flux in the shortwave (PSRFSWD) and longwave (PSRFLWD)
part of the spectrum. All these fluxes are positive when they contribute to a
heating of the surface; they are expressed in $W m^{-2}$ unit.

%
\subsection{The Fouquart-Morcrette radiation code and its {\bf M\'eso-NH}
interface}
%

The code treats successively the longwave and shortwave radiative transfers for
independent air columns. The routine named ECMWF\_RADIATIONS is the interface
between the physical variables available in {\bf M\'eso-NH} through routine
RADIATIONS and the Fouquart-Morcrette parameterization. This routine is used
also to initialize some radiative properties that do not directly belong to the radiative scheme, namely those for the clouds, aerosols, surface properties and
ozone. It calls the subroutine LW(SW) which handles the infra-red(solar) part,
respectively. In addition, external routines are available to compute the
temporal solar angles and to initialize some surface radiative properties.\\

\subsection{Outline of the scheme}
%

\begin{enumerate}

% -------------------- clear sky -----------------------------------------------

\item \underline{Clear sky}

\begin{enumerate}
\item Shortwave: Two-stream formulation employed together with photon path
distribution method (Fouquart and Bonnel, 1980) in two spectral intervals
(0.25-0.68 $\mu m$ and 0.68-4.0 $\mu m$).\footnotemark
%
\footnotetext{The first interval is further referenced as the VISible domain
while the second one is the Near Infra-Red domain for the surface albedo
nomenclature}
\begin{itemize}
               \item Rayleigh scattering (Parametric expression of the Rayleigh optical thickness)
               \item Aerosol scattering and absorption (Mie parameters for 5 types of aerosols based on climatological models (WMO-ICSU 1984))
               \item $H_{2}O$ (Two intervals)
               \item Uniformly mixed gases (One interval)
              \item $O_{3}$ (Two intervals)
\end{itemize}

\item Longwave: Broad band flux emissivity method with six intervals covering the spectrum between 0 and 2620 $cm^{-1}$. Temperature and pressure dependence of absorption following Morcrette {\em et al.} (1986). Absorption coefficients fitted from AFGL 1982.

\begin{itemize}
            \item $H_{2}O$ (Six spectral intervals, {\em e}- and {\em p}-type continuum absorption included between 350 and 1250 $cm^{-1}$)
            \item $CO_{2}$ (Overlap between 500 and 1250 $cm^{-1}$ in three intervals by multiplication of transmission)
            \item $O_{3}$ (Overlap between 970 and 1110 $cm^{-1}$)
            \item $CH_{4}$ (Overlap over intervals 3 and 6, not yet implemented)
            \item $N_{2}O$ (Overlap over intervals 3 and 6, not yet implemented)
            \item $CFC-11$ (Overlap over interval 6, not yet implemented)
            \item Aerosols (Absorption effects using an emissivity formulation)
\end{itemize}

\end{enumerate}

% -------------------- cloudy sky ----------------------------------------------

\item \underline{Cloudy sky}

\begin{enumerate}
\item Shortwave:

\begin{itemize}
            \item Droplet absorption and scattering (Employs a Delta-Eddington
method with $\tau$, $\omega$ and $g$ determined from the liquid water path
$u_{LWP}$ and a parameterization of the effective radius $r_{e}$). A recent
extension to icy clouds has been included. It is based upon the Sun and Shine
(1995)  parameterization.
            \item Gas absorption (Included separately through the photon path distribution method)
\end{itemize}

\item Longwave:

\begin{itemize}
             \item Scattering   (Neglected)
             \item Droplet and Ice crystals  absorption and scattering (From
$u_{LWP}$ and $u_{IWP}$ using an emissivity formulation)
             \item Gas absorption (As in (1.b))
\end{itemize}


\end{enumerate}

\end{enumerate}






%----------------------------------------------------------------------

\section[Longwave radiations]{Longwave radiation\footnotemark}
%
\footnotetext{Longwave, infra-red or thermal radiations will be used.}
\label{section2}

%
\subsection{First glance}
%

The rate of net atmospheric cooling by emission-absorption of longwave radiation
({\tt PDTLOG} is given in Kelvin/hour in the code) is:

\medskip
\be
\frac{\partial T}{\partial t} = \frac{g}{C_{ph}} \frac{\partial F}{\partial p},
\label{ecmwf21}
\ee
\medskip
\noindent where $F=F_{LW}$ is the net total longwave flux.

%

Assuming a non-scattering atmosphere in local thermodynamic equilibrium,
$F \equiv F^{\!\uparrow\!\downarrow}$ is given by:

%(2.2)
\medskip
\be
F^{\!\uparrow\!\downarrow} = \int_{-1}^{+1}\!{\mu\,d\mu} \int_0^\infty{\bigl[L_{\nu}(p_s,\mu) \, t_{\nu}(p_s,p,\mu) + \int_{p_s}^0{L_{\nu}(p',\mu) \, dt_{\nu}}\bigr]\,d\nu}
\label{ecmwf22}
\ee
\medskip

where $L_{\nu}(p_s,\mu)$ is the monochromatic radiance of wavenumber
$\nu$ at level $p$ propagating in a direction such as $\mu =
cos\vartheta $ is the cosine of the angle
$\vartheta $ that this direction makes with the vertical and
$t_{\nu}(p,p',\mu)$ is the monochromatic transmission through a layer
whose limits are at $p$ and $p'$, seen under the same angle.

After separating the upward and downward components and integrating by parts,
we obtain the radiation transfer equation as it is actually estimated in the
radiation code

\medskip
\begin{eqnarray}\label{ecmwf23}
F_{\nu}^{\!\uparrow}(p) &=&
\bigl[B_{\nu}(T_s) - B_{\nu}(T_{0^{+}})\bigr] \, t_{\nu}(p_s,p;r)
+ B_{\nu}(T_p) + \int_{p_s}^p \! {t_{\nu}(p,p';r) \, dB_{\nu}}, \cr
F_{\nu}^{\!\downarrow}(p) &=&
\bigl[B_{\nu}(T_t) - B_{\nu}(T_{\infty})\bigr] \, t_{\nu}(p,0;r)
- B_{\nu}(T_p) - \int_{p}^0 \! {t_{\nu}(p',p;r) \, dB_{\nu}}.
\end{eqnarray}
\medskip

Taking benefit of the isotropic nature of the longwave radiations,
the radiance $L_{\nu}$ of (\ref{ecmwf22}) is replaced by the Planck function
$B_{\nu}(T)$ in unit of flux $W m^{-2}$ (subroutine LWB) (hereafter
$B_{\nu}$ always includes the $\pi$ factor).  Notice that $T_s$ is the surface
temperature (in fact a radiative surface temperature {\tt PTSRAD} issuing from
a surface process scheme such as ISBA) and that $T_{0^{+}}$ is the temperature
of the air just above the surface. $T_{p}$ is the air temperature (PT) at the
{\bf M\'eso-NH} mandatoy levels where the atmospherique pressure $p$ needs also
to be calculated. $T_t$ is the temperature at the top of the atmosphere
(standard atmosphere extension above the last atmospheric level) and
$B_{\nu}(T_{\infty}$ is set to zero. The transmission $t_{\nu}$ is evaluated as
the radiance transmission in a direction $\vartheta $ to the vertical such
that $r = sec \, \vartheta $ is the diffusivity factor (Elsasser, 1942).
Such an approximation for the integration over the angle is usual in
radiative transfer calculations and tests on the validity of this
approximation have been presented by Rodgers and Walshaw (1966) among
others.  The use of the diffusivity factor gives cooling rates within
2\% of those obtained with a 4-point Gaussian quadrature.

%
\subsection{Vertical integration}
\label{subsection21}
%

Integrals in (\ref{ecmwf23}) are evaluated numerically, after discretization over the vertical grid, considering the atmosphere as a pile of homogeneous layers (subroutine LWV). As the cooling rate is strongly dependent on local conditions of temperature and pressure and energy is mainly exchanged with the layers adjacent to the level where fluxes are calculated, the contribution of the distant layers is simply computed using a trapezoidal rule integration (subroutine LWVD), but the contribution of the adjacent layers is evaluated with a 2-point Gaussian quadrature (subroutine LWVN; see common YOMLW), thus

%(2.4)
\medskip
\be
\int_{p_{s}}^{p_{i}} \! {t_{\nu}(p,p';r)} = \sum_{l=1}^{2} \, w_l \, t_{\nu}(p_i,p_l;r) \, dB_{\nu}(l) + \frac{1}{2} \sum_{j=1}^{i-2} \, \bigl[t_{\nu}(p_i,p_j;r) + t_{\nu}(p_i,p_{j-1};r)\bigr] \, dB_{\nu}(j)
\label{ecmwf24}
\ee
\medskip

\noindent where $p_l$ and $w_l$ are the pressure corresponding to the gaussian root and the gaussian weight, respectively. $dB_{\nu}(j)$ (PDBDT) and $dB_{\nu}(l)$ (PDBSL) are the Planck function gradients calculated between two interfaces and between mid-layer and interface, respectively.

%
\subsection{Spectral integration}
%

The integration over wavenumber $\nu$ is performed using a band emissivity method, as first discussed by Rodgers (1967). The longwave spectrum is divided into six spectral regions:

\begin{center}
\begin{tabular}{c||r|c|l}
1       &          0 -  350 $cm^{-1}$   &+&     1450 - 1880 $cm^{-1}$   \\
2       &        500 -  800 $cm^{-1}$   & &                             \\
3       &        800 -  970 $cm^{-1}$   &+&     1110 - 1250 $cm^{-1}$   \\
4       &        970 - 1110 $cm^{-1}$   & &                             \\
5       &        350 -  500 $cm^{-1}$   & &                             \\
6       &       1250 - 1450 $cm^{-1}$   &+&     1880 - 2820 $cm^{-1}$
\end{tabular}
\label{intervalles}
\end{center}

\noindent corresponding to the centers of the rotation and vibration-rotation bands of $H_2O$, the 15 micron band of $CO_2$, the atmospheric window, the 9.6 $\mu m$ band of $O_3$, the 25 $\mu m$ "window" region and the wings of the vibration-rotation band of $H_2O$, respectively. Over these spectral regions, band fluxes are evaluated with the help of band transmissivities precalculated from the narrow-band model of Morcrette and Fouquart (1985) -- See Appendix of Morcrette {\em et al.} (1986) for details.\\

Integration of (\ref{ecmwf23}) over wavenumber $\nu$ within the wide $k^{th}$
spectral region gives the upward and downward fluxes as

\medskip
\begin{eqnarray}\label{ecmwf25}
F_{k}^{\!\uparrow}(p) &=&
\bigl[B_{k}(T_s) - B_{k}(T_{0^{+}})\bigr] \, t_{B_k}(ru(p_s,p),T_u(p_s,p))
+ B_{k}(T_p) + \int_{p_s}^p{t_{dB_k}(ru(p,p'),T_u(p,p')) \, dB_{k}}, \cr
F_{k}^{\!\downarrow}(p) &=&
\bigl[B_{k}(T_0) - B_{k}(T_{\infty})\bigr] \, t_{B_k}(ru(p,0),T_u(p,0))
- B_{k}(T_p) - \int_{p}^0{t_{dB_k}(ru(p',p),T_u(p',p)) \, dB_{k}}.
\end{eqnarray}
\medskip

The formulation accounts for the different temperature dependences involved in atmospheric flux calculations, namely that on $T_p$, the temperature at the level where fluxes are calculated and that on $T_u$, the temperature that governs the transmission through the temperature dependence of the intensities and half-widths of the lines absorbing in the concerned spectral region. The band transmissivities are non-isothermal accounting for the temperature dependence that arises from the wavenumber integration of the product of the monochromatic absorption and the Planck function. Two normalized band transmissivities are used for each absorber in a given spectral region (subroutine SWTT1): the first one $t_B(\overline{up},T_p,T_u)$ for calculating the first r.h.s. term in (\ref{ecmwf23}) involving the boundaries; it corresponds to the weighted average of the transmission function by the Planck function

%(2.6a)
\medskip
\be
t_B(\overline{up},T_p,T_u) = \frac{\int_{\nu_1}^{\nu_2}\!{B_{\nu}(T_p) \, t_{\nu}(\overline{up},T_u) \, d\nu}}{\int_{\nu_1}^{\nu_2}\!{B_{\nu}(T_p) \, d\nu}}
\label{ecmwf26a}
\ee
\medskip

\noindent and the second one $t_{dB}(\overline{up},T_p,T_u)$ for calculating the integral terms in (\ref{ecmwf23}) is the weighted average of the transmission function by the derivative of the Planck function

%(2.6b)
\medskip
\be
t_{dB}(\overline{up},T_p,T_u) = \frac{\int_{\nu_1}^{\nu_2}{dB_{\nu}(T_p) / dT} \, \, t_{\nu}(\overline{up},T_u) \, d\nu}{\int_{\nu_1}^{\nu_2}{dB_{\nu}(T_p) / dT} \, \, d\nu}
\label{ecmwf26b}
\ee
\medskip

\noindent where $\overline{up}$ is the pressure weighted amount of absorber (computed in SWU).


In the scheme, the actual dependence on $T_p$ is carried out explicity in the Planck functions integrated over the spectral regions. Although normalized relative to $B(T_p)$ (or $dB(T_p)/dT$), the transmissivities still depend on $T_u$ both through Wien's displacement of the maximum of the Planck function with temperature and through the temperature dependence of the absorption coefficients.
$O_3$ transmissivity is obtained using the Malkmus band model and $CH_4,\,N_2O,\,CFC-11$ and $CFC-12$ transmissivities with a statistical model.
For computational efficiency, $H_2O$ and $CO_2$ transmissivities have been developed into Pad\'e approximants


%(2.7)
\medskip
\be
t(\overline{up},T_u) = \frac{\sum\limits_{i=0}^{2}{C_i \, u_{eff}^{i / 2}}}{\sum\limits_{j=0}^{2}{D_j \, u_{eff}^{j / 2}}},
\label{ecmwf27}
\ee
\medskip

\noindent where $u_{eff} = r \, \overline{up} \, f(T_u,\overline{up})$ is an effective amount of absorber which incorporates the diffusivity factor $r$, the weighting of the absorber amount by pressure, $\overline{up}$ and the temperature dependence of the absorption coefficients $f$, with

%(2.8)
\medskip
\be
f(T_u,\overline{up}) = exp [a(\overline{up}) \, (T_u - 250) + b(\overline{up}) \, (T_u - 250)^2].
\label{ecmwf28}
\ee
\medskip

The temperature dependence due to Wien's law is incorporated although there is no explicit variation of the coefficients $C_i$ and $D_j$ with temperature. These coefficients have been computed for temperatures between 187.5 and 312.5~K with a 12.5 K step and transmissivities corresponding to the reference temperature the closest to the pressure weighted temperature $T_u$ are actually used in the scheme.

%
\subsection{Incorporation of the effects of clouds}
%

The incorporation of the effects of clouds on the longwave fluxes follows the treatment discussed by Washington and Williamson (1977). Whatever the state of cloudiness of the atmosphere, the scheme starts by calculating the clear-sky fluxes and stores the terms representing exchanges of energy between the levels (integrals in (\ref{ecmwf23})).

Let $F_0{^{\!\uparrow}}(i)$ and $F_0{^{\!\downarrow}}(i)$ be the upward and downward clear-sky fluxes (PFUP and PFDN in LWC). For any cloud layer actually present in the atmosphere, the scheme then evaluates the fluxes assuming a unique overcast cloud of unity emissivity. Let $F_n{^{\!\uparrow}}(i)$ and $F_n{^{\!\downarrow}}(i)$ the upward and downward fluxes when such a cloud is present in the $n^{th}$ layer of the atmosphere. Downward fluxes above the cloud and upward fluxes below it have kept their clear-sky values that is:

%(2.9)
\medskip
\begin{eqnarray}\label{ecmwf29}
F_n{^{\!\uparrow}}(i) &=& F_0{^{\!\uparrow}}(i) \qquad for \qquad i \leq n, \cr
F_n{^{\!\downarrow}}(i) &=& F_0{^{\!\downarrow}}(i) \qquad for \qquad i > n.
\end{eqnarray}
\medskip

Upward fluxes above the cloud ($F_n{^{\!\uparrow}}(k)$ for $k~\geq~n+1$) and downward fluxes below it ($F_n{^{\!\downarrow}}(k)$ for $k~<~n$) can be expressed with expressions similar to (\ref{ecmwf23}) provided the boundary terms are now replaced by terms corresponding to possible temperature discontinuities between the cloud and the surrounding air

%(2.10)
\medskip
\begin{eqnarray}\label{ecmwf210}
F_{n}^{\!\uparrow}(k) &=& \bigl[F_{cld}^{\!\uparrow} - B(n+1)\bigr] \, t(p_k,p_{n+1};r) + B(k) + \int_{p_{n+1}}^{p_k}\!{t(p_k,p';r) \, dB}, \cr
F_{n}^{\!\downarrow}(k) &=& \bigl[F_{cld}^{\!\downarrow} - B(n)\bigr] \, t(p_k,p_n;r) + B(k) + \int_{p_{k}}^{p_n}\!{t(p',p_k;r) \, dB}.
\end{eqnarray}
\medskip

\noindent where $B(i)$ is now the total Planck function (integrated over the whole
longwave spectrum) at level $i$ and $F_{cld}^{\!\uparrow}$ and
$F_{cld}^{\!\downarrow}$ are the fluxes at the upper and lower
boundaries of the cloud.  Terms under the integrals correspond to
exchange of energy between layers in the clear-sky atmosphere and have
already been computed in the first step of the calculationm a
top-hat function applied to the prognostic cloud water mixing ratio $r_c$.\\

Let $N$ be the index of the layer containing the highest cloud.  $C_i$
the fractional cloud cover in layer $i$, with $C_0 = 1$ for the upward
flux at the surface and with $C_{N+1} = 1$ and $F_{N+1}^{\!\downarrow}
= F_0^{\!\downarrow}$ to have the right boundary condition for downward
fluxes above the highest cloud.  The cloudy upward ($F^{\!\uparrow}$)
and downward ($F^{\!\downarrow}$) fluxes are obtained, with the hypothesis
m a
top-hat function applied to the prognostic cloud water mixing ratio $r_c$.\\

Let $N$ be the index of the layer containing the highest cloud.  $C_i$
the fractional cloud cover in layer $i$, with $C_0 = 1$ for the upward
flux at the surface and with $C_{N+1} = 1$ and $F_{N+1}^{\!\downarrow}
= F_0^{\!\downarrow}$ to have the right boundary condition for downward
fluxes above the highest cloud.  The cloudy upward ($F^{\!\uparrow}$)
and downward ($F^{\!\downarrow}$) fluxes are obtained, with the hypothesis
of a random covering of clouds, as:

%(2.11)
\medskip
\[
F^{\!\uparrow}(i) = F_0^{\!\uparrow}(i) \qquad for \, i=1
\]
\be
F^{\!\uparrow}(i) = C_{i-1} F_{i-1}^{\!\uparrow}(i) + \sum_{n=0}^{i-2}{C_n F_n^{\!\uparrow}(i) \, \prod_{l=n+1}^{i-1}{(1 - C_l)}}       \qquad for \, 2\leq i \leq N+1 \\
\label{ecmwf211}
\ee
\[
F^{\!\uparrow}(i) = C_{N} F_{N}^{\!\uparrow}(i) + \sum_{n=0}^{N-1}{C_n F_n^{\!\uparrow}(i) \, \prod_{l=n+1}^{N}{(1 - C_l)}}     \qquad for \, i \geq N+2
\]
\medskip

In case of semi-transparent clouds, the fractional cloudiness entering the calculations is an effective cloud cover equal to the product of the emissivity by the horizontal coverage of the cloud layer, with the emissivity related to the condensed water amount by:

%(2.12)
\medskip
\be
\epsilon_{i} = 1 - exp( - K^{L}_{abs} \, u_{LWP}- K^{I}_{abs} \, u_{IWP})
\label{ecmwf212}
\ee
\medskip

\noindent where $K^{L}_{abs}=158$ is the condensed water mass absorption
coefficient (in $m^{2}kg^{-1}$) and $K^{I}_{abs}=113$, the corresponding one
for ice phase\footnotemark
%
\footnotetext{Note that an advanced version of the code will contain absorption
coefficients depending whether ice or liquid water cloud is present and
whether upward or downward radiations are considered. Following Smith and
Shi (1992), the recommended values for $K^{L,I}_{abs}$ will be (in $m^{2}kg^{-1}$
unit):
\begin{center}
\begin{tabular}{cccc}
$K^{\!\uparrow\;L}_{abs} = 130$ & $K^{\!\downarrow\;L}_{abs} = 158$ & $K^{\!\uparrow\;I}_{abs} = 93$ & $K^{\!\downarrow\;I}_{abs} = 113$
\end{tabular}
\end{center}
}
.

\section[Solar radiation]{Solar radiation\footnotemark}
%
\footnotetext{Solar or shortwave radiation...}
\label{section3}

%
\subsection{First glance}
%

The rate of atmospheric warming by absorption and scattering of shortwave radiation is:

%

%(3.1)
\medskip
\be
\frac{\partial T}{\partial t} = \frac{g}{C_{ph}} \frac{\partial F}{\partial p}
\label{ecmwf31}
\ee
\medskip

\noindent where $F=F_{SW}$ is the net total shortwave flux, expressed in
$W m^{-2}$ and positive when downward:

%(3.2)
\medskip
\be
F = \int_0^\infty\!{d\nu} \int_0^\pi\!{d\phi}\int_{-1}^{+1}\!{\mu \, L_{\nu}(\delta,\mu,\phi) \, d\mu \, d\phi}
\label{ecmwf32}
\ee
\medskip

$L_{\nu}$ is the diffuse radiance at wavenumber $\nu $, in a direction given by $\phi$, the azimuth angle and $\vartheta $ the zenith angle such as $\mu = cos\vartheta $. Notice that $\phi$ ({\tt PAZIMSOL}) and $\mu$ ({\tt PMU0}) are
computed in routine SUNPOS described in the last subsection. In (\ref{ecmwf32}),
we assume a plane parallel atmosphere with the optical depth $\delta$, as
a convenient vertical coordinate when the energy source is outside the medium
%(3.3)
\medskip
\be
\delta(p) = \int_p^0\!{\beta_{\nu}(p) \, dp}
\label{ecmwf33}
\ee
\medskip
\noindent where $\beta_{\nu}^{ext}(p)$ is the extinction coefficient equal to
the sum of the scattering coefficient $\beta_{\nu}^{sca}$ of the aerosol and
cloud particle absorption coefficient $\beta_{\nu}^{abs}$ and of the purely molecular absorption coefficient $k_{\nu}$. The diffuse radiance $L_{\nu}$ is governed by the radiation transfer equation

%(3.4)
\medskip
\begin{eqnarray}\label{ecmwf34}
\mu \, \frac{dL_{\nu}(\delta,\mu,\phi)}{d\delta} &=&
L_{\nu}(\delta,\mu,\phi) - \frac{\overline{\omega}_{\nu}(\delta)}{4} \, P_{\nu}(\delta,\mu,\phi,\mu_0,\phi_0) \, E_{\nu}^0 e^{-\frac{\delta}{\mu_0}} \cr
                                                 & &
\mbox{}- \frac{\overline{\omega}_{\nu}(\delta)}{4} \int_0^{2\pi}\!\!\int_{-1}^{+1}{P_{\nu}(\delta,\mu,\phi,\mu',\phi') \, L_{\nu}(\delta,\mu',\phi') \, d\mu' \, d\phi'}.
\end{eqnarray}
\medskip

$E_{\nu}^0$ is the incident solar irradiance in the direction $\mu_0 = cos\vartheta _0$, $\overline{\omega}_{\nu}$ is the single scattering albedo ($= \beta_{\nu}^{sca} / k_\nu$) and $P_{\nu}(\delta,\mu,\phi,\mu',\phi')$ is the scattering phase function which defines the probability that radiation coming from direction $(\mu',\phi')$ is scattered in direction $(\mu,\phi)$.
The shortwave part of the scheme, originally developed by Fouquart and Bonnel (1980) solves the radiation transfer equation and integrates the fluxes over the whole shortwave spectrum between 0.2 and 4 $\mu m$. Upward and downward fluxes are obtained from the reflectances and transmittances of the layers, and the photon path distribution method allows to separate the parametrization of the scattering processes from that of the molecular absorption.

%
\subsection{Spectral integration}
%

Solar radiation is attenuated by absorbing gases, mainly water vapor, uniformly mixed gases ($O_2$, $CO_2$)\footnotemark
%
\footnotetext{Absorption by $CH_4$ and by $N_{2}O$ is accounted for in a recent
version of the code}
and ozone, and scattered by molecules (Rayleigh scattering), aerosols and cloud particles. Since scattering and molecular absorption occur simultaneously, the
exact amount of absorber along the photon path length is unknown, and band
models of the transmission function cannot be used directly as in the shortwave
radiation transfer (see ~\ref{subsection21}). The approach of the photon path
distribution method is to calculate the probability $p(U) \, dU$ that a photon
contributing to the flux $F_c$ in the conservative case ({\em i.e.} no
absorption, $\overline{\omega}_{\nu} = 1, k_{\nu} = 0$) has encountered an
absorber amount between $U$ and $U + dU$. With this distribution, the radiative flux at wavenumber $\nu $ is related to $F_c$ by

%(3.5)
\medskip
\be
F_{\nu} = F_c \int_0^{\infty}\!{p(U) \, exp( - k_{\nu} U) \, dU}
\label{ecmwf35}
\ee
\medskip

\noindent and the flux averaged over the spectral interval $\Delta\nu$ can then be calculated with the help of any band model of the transmission function $t_{\Delta\nu}$

%(3.6)
\medskip
\be
F = \frac{1}{\Delta\nu} \int_{\Delta\nu}\!{F_{\nu} \, d\nu} = F_c \int_0^{\infty}\!{p(U) \, t_{\Delta\nu}(U) \, d\nu}.
\label{ecmwf36}
\ee
\medskip

To find the distribution function $p(U)$, the scattering problem is solved first, by any method, for a set of arbitrarily fixed absorption coefficients $k_l$, thus giving a set of simulated fluxes $F_{k_l}$. An inverse Laplace transform is then performed on (\ref{ecmwf35}) to get $p(U)$ (Fouquart, 1974). The main advantage of the method is that the actual distribution $p(U)$ is smooth enough that (\ref{ecmwf35}) gives accurate results even if $p(U)$ itself is not known accurately. In fact, $p(U)(U)$ needs not be calculated explicitly as the spectrally integrated fluxes are, in the two limiting cases of weak and strong absorption:

%(3.7)
\medskip
\begin{eqnarray}\label{ecmwf37}
F = F_c \, t_{\Delta\nu}(<U>)   \qquad &{\rm where}& \qquad <U> \, = \, \int_0^{\infty} \, p(U) \, U \, dU \cr
F = F_c \, t_{\Delta\nu}(<U^{\frac{1}{2}}>) \qquad &{\rm where}& \qquad <U^{\frac{1}{2}}> \, = \, \int_0^{\infty} \, p(U) \, U^{\frac{1}{2}} \, dU
\end{eqnarray}
\medskip

\noindent respectively. The atmospheric absorption in the water vapor bands is generally strong and the
scheme determines an effective absorber amount $U_e$ between $<U>$ and
$<U^{\frac{1}{2}}>$ derived from

%(3.8)
\medskip
\be
U_e = \frac{1}{k_e} \, ln(\frac{F_{k_e}}{F_c})
\label{ecmwf38}
\ee
\medskip

\noindent where $k_e$ is an absorption coefficient chosen to approximate the spectrally averaged transmission of the clear-sky atmosphere:

%(3.9)
\medskip
\be
k_e = (\frac{U_{tot}}{\mu_0})^{-1} \, ln(t_{\Delta\nu} \frac{U_{tot}}{\mu_0})
\label{ecmwf39}
\ee
\medskip

\noindent with $U_{tot}$ the total amount of absorber in a vertical column and
$\mu_0 = cos\vartheta _0$ (PMU0 computed in routine SUNPOS). Once the effective
absorber amounts of $H_{2}O$ and uniformly mixed gases are found, the
transmission functions are computed using Pad\'e approximants:

%(3.10)
\medskip
\be
t_{\Delta\nu}(U) = \frac{\sum\limits_{i=0}^N{a_i U^{i-1}}}{\sum\limits_{j=0}^N{b_j U^{j-1}}}.
\label{ecmwf310}
\ee
\medskip

Absorption by ozone is also taken into account, but since ozone is located at
low pressure levels for which molecular scattering is small and Mie scattering
is negligible, interactions between scattering processes and ozone absorption
are neglected. Transmission through ozone is computed using (\ref{ecmwf310})
where the amount of ozone $U_{O_3}$ is (POZON is the concentration in ozone
(Pa/Pa) taken from a standard atmosphere):

%
\medskip
\[
U_{O_3}^d = M \, \int_{p}^{0}{dU_{O_3}}
\]
for the downward transmission of the direct solar beam, and:
\[
U_{O_3}^u = r \, \int_{p_s}^{p}{dU_{O_3}} + U_{O_3}^d(p_s)
\]
for the upward transmission of the diffuse radiation with $r = 1,66$ the
diffusivity factor and $M$ (PSEC in SWU), the magnification factor
(Rodgers, 1967) used instead of $\mu_0$ to account for the sphericity of the
atmosphere at very small solar elevations:

%(3.11)
\medskip
\be
M = \frac{35}{\sqrt{1224 \, \mu_0^2 + 1}}.
\label{ecmwf311}
\ee
\medskip

To perform the spectral integration, it is convenient to discretize the solar
spectral interval into subintervals in which the surface reflectance can be
considered as constant. Since the main cause of the important spectral variation
of the surface albedo is the sharp increase in the reflectivity of the
vegetation in the near infrared and since water vapor does not absorb below
$0.68 \mu m$, the shortwave scheme considers two spectral intervals, one for the
visible ($0,2 - 0,68 \mu m$, subroutine SW1S) containing a fraction of 0.441676
the incoming solar energy and second one for the near infrared
($0,68 - 4,0 \mu m$, subroutine SW2S) for the 0.558324 remaining part of the
solar spectrum. This cut-off at $0,68 \mu m$ also makes the scheme more
computationally efficient, in as much as the interactions between gaseous
absorption (by water vapor and uniformly mixed gases) and scattering processes
are accounted for only in the near-infrared interval.

%
\subsection{Vertical integration}
%

Considering an atmosphere where a fraction $f$ (as seen from the surface or the top of the atmosphere) is covered by clouds (the fraction $f$ depends on which cloud overlap assumption is assumed for the calculations), the final fluxes are given as

%(3.110)
\medskip
\be
F^{\!\downarrow} = f \, F^{\!\downarrow}_{cloudy} \, + \, (1-f) \, F^{\!\downarrow}_{clear}
\label{ecmwf3110}
\ee
\medskip

\noindent with a similar expression holding for the upward flux. Contrarily to
the scheme of Geleyn and Hollingsworth (1979), the fluxes are not obtained
through the solution of a system of linear equations in a matrix form. Rather,
assuming an atmosphere divided into $N$ homogeneous layers, the upward and
downward fluxes at a given interface $j$ are given by:

%(3.12)
\medskip
\begin{eqnarray}\label{ecmwf312}
F^{\!\downarrow}(j)&=&  F_0 \prod_{k=j}^N{T_b(k)}, \cr
F^{\!\uparrow}(j)  &=&  F^{\!\downarrow}(j) \, R_t(j-1),
\end{eqnarray}
\medskip

\noindent where $R_t(j)$ and $T_b(j)$ are the reflectance at the top and the
transmittance at the bottom of the $j^{th}$ layer. Computations of $R_t$'s
start at the surface and work upward, whereas those of $T_b$'s start at the top
of the atmosphere and work downward. $R_t$ and $T_b$ account for the presence of
cloud in the layer:

%(3.13)
\medskip
\begin{eqnarray}\label{ecmwf313}
R_t &=& C \, R_{cdy} + (1 - C) \, R_{clr}, \cr
T_b &=& C \, T_{cdy} + (1 - C) \, T_{clr}.
\end{eqnarray}
\medskip

The subscripts $_{clr}$ and $_{cdy}$ respectively refer to the clear-sky and
cloudy fractions of the layer while $C$ is the cloud fractional.\\


%
\subsubsection{Cloudy fraction of the layers}
\label{subsubsection321}
%

$R_{t_{cdy}}$ and $T_{b_{cdy}}$ are the reflectance at the top and transmittance
at the bottom of the cloudy fraction of the layer calculated with the
Delta-Eddington Approximation. Given $\delta_c$ (PTAU), $\delta_a$ (PAER) and
$\delta_g$, the optical thicknesses for the cloud, the aerosol and the molecular
absorption ($= k_e U$), $g_c$ (PCG) and $g_a$ (CGA) the cloud and aerosol
asymetry factors, $R_{t_{cdy}}$ and $T_{b_{cdy}}$ are calculated as functions of:

\begin{itemize}
\item{the total optical thickness of the layer $\delta^{*}$:}

%(3.14)
\[
\delta^{*} = \delta_c + \delta_a + \delta_g
\]

\item{the total single scattering albedo:}

%(3.14)
\be
\omega^{*} = \frac{\delta_c + \delta_a}{\delta_c + \delta_a + \delta_g}
\label{ecmwf314}
\ee

\item{the total asymetry factor:}

%(3.14)
\[
g^{*} = \frac{\delta_c}{\delta_c + \delta_a} g_c + \frac{\delta_a}{\delta_c + \delta_a} g_a
\]
\end{itemize}


\noindent of the reflectance $R\_$ of the underlying medium (surface or layers
below the $j^{th}$ interface) and of the effective solar zenith angle $\mu_e(j)$
which accounts for the decrease of the direct solar beam and the corresponding
increase of the diffuse part of the downward radiation by the upper scattering
layers:

%(3.15)
\medskip
\be
\mu_e(j) = \biggl[\frac{(1 - C^{al}(j))}{\mu} + r \, C^{al}(j)\biggr]^{-1},
\label{ecmwf315}
\ee

\noindent with:

\[
C^{al}(j) = 1 - \prod_{i=j+1}^N{(1 - C(i) \, E(i))}
\]

\noindent and

%(3.16)
\be
E(i) = 1 - exp\biggl[ - \frac{(1 - \omega_c(i) \, g_c(i)^2)}{\mu}\biggr].
\label{ecmwf316}
\ee
\medskip

$\delta_c(i)$, $\omega_c(i)$ and $g_c(i)$ are the optical thickness, single scattering albedo and asymetry factor of the cloud in the $\, i^{th}$ layer, and $r$ is the diffusivity factor. The scheme follows the Eddington approximation, first proposed by Shettle and Weiman (1970), then modified by Joseph {\em et al.} (1976) to account more accurately for the large fraction of radiation directly transmitted in the forward scattering peak in case of highly asymetric phase functions. Eddington's approximation assumes that, in a scattering medium of optical thickness $\delta^{*}$, of single scattering albedo $\omega$, and of asymetry factor $g$, the radiance $L$ entering (\ref{ecmwf34}) can be written as:

%(3.17)
\medskip
\be
L(\delta,\mu) = L_0(\delta) + \mu \, L_1(\delta).
\label{ecmwf317}
\ee
\medskip

In that case, when the phase function is expanded as a series of associated Legendre functions, all terms of order greater than one vanish when (\ref{ecmwf34}) is integrated over $\mu$ and $\phi$. The phase function is therefore given by

\medskip
\[
P(\theta) = 1 + \beta_1(\theta) \, cos\theta,
\]
\medskip

\noindent where $\theta$ is the angle between incident and scattered radiances. The integral in (\ref{ecmwf34}) thus becomes

%(3.18)
\medskip
\be
\int_0^{2\pi}\!\!\int_{-1}^{+1}{P(\mu,\phi,\mu',\phi') \, L(\mu',\phi') \, d\mu' \, d\phi'} = 4\pi \, (L_0 + \pi L_1)
\label{ecmwf318}
\ee
\medskip

\noindent where $g$, the asymetry factor identifies as

\medskip
\[
g = \frac{\beta}{3} = \frac{1}{2} \int_{-1}^{+1}\!{P(\theta) \, cos\theta \, d(cos\theta)}.
\]
\medskip

Using (\ref{ecmwf318}) in (\ref{ecmwf34}) after integrating over $\mu$ and dividing by $2\pi$, we get

%(3.19)
\medskip
\be
\mu \, \frac{d(L_0+\mu L_1)}{d\delta} = - (L_0+\mu L_1) + \omega \, (L_0+g \mu L_1) + \frac{1}{4} \omega \, F_0 \, exp(\frac{-\delta}{\mu_0}) \, (1+3 g \mu_0 \, \mu).
\label{ecmwf319}
\ee
\medskip

We obtain a pair of equations for $L_0$ and $L_1$ by integrating (\ref{ecmwf319}) over $\mu$:

%(3.20)
\medskip
\be
\frac{d(L_0)}{d\delta} = -3 (1 - \omega) \, L_0 + \frac{3}{4} \omega \, F_0 \, exp(\frac{-\delta}{\mu_0}),
\label{ecmwf320}
\ee

%(3.21)
\be
\frac{d(L_1)}{d\delta} = - (1 - \omega g) \, L_1 + \frac{3}{4} \omega \, g \, \mu_0 \, F_0 \, exp(\frac{-\delta}{\mu_0}).
\label{ecmwf321}
\ee
\medskip

For the cloudy layer assumed non-conservative ($\omega < 1$), the solutions to
(\ref{ecmwf320}) and (\ref{ecmwf321}) are, in the range
$0 \leq \delta \leq \delta^{*}$:

%(3.22)
\medskip
\begin{eqnarray}\label{ecmwf322}
L_0(\delta) &=& C_1 \, exp(-k\delta) + C_2 \, exp(+k\delta) - \alpha \, exp(\frac{-\delta}{\mu_0}), \cr
L_1(\delta) &=& p (C_1 \, exp(-k\delta) - C_2 \, exp(+k\delta)) - \beta \, exp(\frac{-\delta}{\mu_0}),
\end{eqnarray}
\medskip

\noindent where

\medskip
$$
\begin{array}{rcl}
k      &=& {\bigl[ 3 \, (1 - \omega) \,  (1 - \omega g)\bigr]}^{\frac{1}{2}}\cr
p      &=& {\bigl[ 3 \, (1 - \omega) / (1 - \omega g)\bigr]}^{\frac{1}{2}}\cr
\alpha &=& 3 \, \omega \, F_0 \, \mu_0^2 \, \frac{\displaystyle{\bigl[ 1 + g \, (1 - \omega) \bigr]}}{\displaystyle{4 \, (1 - k^2 \mu_0^2)}} \cr
\beta  &=& 3 \, \omega \, F_0 \, \mu_0 \, \frac{\displaystyle{\bigl[ 1 + 3 \, g \, (1 - \omega) \, \mu_0^2 \bigr]}}{\displaystyle{4 \, (1 - k^2 \mu_0^2)}}.
\end{array}
$$
\medskip

The two boundary conditions allow to solve the system for $C_1$ and $C_2$.
First, the downward directed diffuse flux at the top of the layer is zero

\medskip
\[
F^{\!\downarrow}(0) = \bigl[ L_0(0) + \frac{2}{3} L_1(0) \bigr] = 0,
\]

\noindent which translates into

%(3.23)
\be
(1 + \frac{2 p}{3}) \, C_1 + (1 - \frac{2 p}{3}) \, C_2 = \alpha + \frac{2 \beta}{3}.
\label{ecmwf323}
\ee
\medskip

\noindent For the second condition, one assumes that the upward directed flux
at the bottom of the layer is equal to the product of the downward directed
diffuse and direct fluxes by the corresponding diffuse and direct reflectances
($R_d$ and $R\_$, respectively) of the underlying medium

\medskip
\[
F^{\!\uparrow}(\delta^{*}) = \bigl[ L_0(\delta^{*}) - \frac{2}{3} \, L_1(\delta^{*}) \bigr] = R\_ \, \bigl[ L_0(\delta^{*}) + \frac{2}{3} \, L_1(\delta^{*}) \bigr] + R_d \, \mu_0 \, F_0 \, exp(\frac{-\delta^{*}}{\mu_0}),
\]
\medskip

\noindent which translates into

%(3.24)
\medskip
\begin{eqnarray}\label{ecmwf324}
(1 - R\_ - \frac{2p}{3} \, (1 + R\_) ) \, C_1 \, exp(-k \, \delta^{*}) +
(1 - R\_ + \frac{2p}{3} \, (1 + R\_) ) \, C_2 \, &exp(+k \, \delta^{*})& \cr
\mbox{} = ( (1 - R\_)\alpha - \frac{2}{3} \, (1 + R\_)\beta +
R_d \, \mu_0 \, F_0 ) \, &exp(\frac{-\delta^{*}}{\mu_0})&.
\end{eqnarray}
\medskip

In the Delta-Eddington approximation, the phase function is approximated by a
Dirac delta function (forward scatter peak) and a two-term expansion of the
phase function

\medskip
\[
P(\theta) = 2 f \, (1 - cos\theta) + (1 - f) \, (1 + 3g' \, cos\theta),
\]

\noindent where $f$ is the fractional scattering into the forward peak and $g'$ the asymetry factor of the truncated phase function. As shown by Joseph {\em et al.} (1976), these parameters are:

%(3.25)
\medskip
\begin{eqnarray}\label{ecmwf325}
f  &=& g^2, \cr
g' &=& \frac{g}{1+g}.
\end{eqnarray}
\medskip

The solution of the Eddington's equations remains the same provided that the total optical thickness, single scattering albedo and asymetry factor entering (\ref{ecmwf319})-(\ref{ecmwf324}) take their transformed values:

%(3.26)
\medskip
\begin{eqnarray}\label{ecmwf326}
\delta^{*'} &=& (1 - \omega f) \, \delta^{*}, \cr
\omega^{'}  &=& \frac{(1 - f) \, \omega}{1 - \omega f}.
\end{eqnarray}

Practically, the optical thickness, single scattering albedo, asymmetry factor,
and solar zenith angle entering (\ref{ecmwf323})-(\ref{ecmwf326}) are $\delta^{*}$, $\omega^{*}$, $g^{*}$ and $u_e$ defined in (\ref{ecmwf314}) and (\ref{ecmwf315}).

%
\subsubsection{Clear-sky fraction of the layers}
%


In the clear-sky fraction of the layers, the shortwave scheme accounts for scattering and absorption by molecules and aerosols. As optical thickness for both Rayleigh and aerosol scattering is small, $R_{clr}(j-1)$ and $T_{clr}(j)$ the reflectance at the top and transmittance at the bottom of the $j^{th}$ layer can be calculated using respectively a first and a second-order expansion of the analytical solutions of the two-stream equations similar to that of Coakley and Chylek (1975). For Rayleigh scattering (subroutine SW1S), the optical thickness, single scattering albedo and asymmetry factor are respectively $\delta_R$, $\omega_R = 1$ and $g_R = 0$, so that

%(3.27)
\medskip
\begin{eqnarray}\label{ecmwf327}
R_R &=& \frac{\delta_R}{2 \, \mu + \delta_R}, \cr
T_R &=& \frac{2 \mu}{2 \, \mu + \delta_R}.
\end{eqnarray}
\medskip

The optical thickness $\delta_R$ of an atmospheric layer is simply:

\medskip
\[
\delta_R = \delta_R^{*} \frac{(p(j) - p(j-1))}{p_{surf}},
\]
\medskip

\noindent where $\delta_R^{*}$ is the Rayleigh optical thickness of the whole
atmosphere parameterized as a function of solar zenith angle
(Deschamps {\em et al.} 1983):

\[
\delta_R^{*} = \sum_{i=0}^5{a_i \, \mu_0^{i-1}}.
\]
\medskip

For aerosol scattering and absorption, the optical thickness, single scattering
albedo and asymmetry factor are respectively $\delta_a$, $\omega_a$ (with $1-\omega_a \ll 1$) and $g_a$ so that:

%(3.28)
\medskip
\[
den = 1 + ( 1 - \omega_a + back(\mu_e) \, \omega_a ) \, \frac{\delta_a}{\mu_e} + ( 1 - \omega_a ) \, (1 - \omega_a + 2 \,  back(\mu_e) \, \omega_a ) \, \frac{{\delta_a}^2}{{\mu_e}^2}
\]
\be
R(\mu_e) = \frac{back(\mu_e) \, \omega_a \, \delta_a/\mu_e} {den}
\label{ecmwf328}
\ee
\[
T(\mu_e) = \frac{1}{den}
\]
\medskip

\noindent where $back(\mu_e) = (2 - 3 \mu_e g_a)/4$ is the
backscattering factor.

Practically, $R_{clr}$ and $T_{clr}$ are computed using (\ref{ecmwf328}) and the
combined effect of aerosol and Rayleigh scattering comes from using modified
parameters corresponding to the addition of the two scatters with provision for
the highly asymmetric aerosol phase function through a Delta-Eddington
approximation of the forward scattering peak
(as in (\ref{ecmwf325})-(\ref{ecmwf326})):

%(3.29)
\medskip
\[
\delta^{+} = \delta_R + \delta_a \, (1 - \omega_a \, g_a^2)
\]
\be
g^{+} = \frac{g_a}{1 + g_a} \, \bigl(\frac{\delta_a}{\delta_R + \delta_a}\bigr)
\label{ecmwf329}
\ee
\[
\omega^{+} = \frac{\delta_R}{\delta_R + \delta_a} \, \omega_R + \frac{\delta_a}{\delta_R + \delta_a} \frac{\omega_a \, (1 - g_a^2)}{1 - \omega_a \, g_a^2}.
\]
\medskip

As for their cloudy counterparts, $R_{clr}$ and $T_{clr}$  must account for the multiple reflections due to the layers underneath:

%(3.30)
\medskip
\begin{eqnarray}\label{ecmwf330}
R_{clr} &=& R(\mu_e) + \frac{T(\mu_e)}{1 - R^{*} \, R\_} \, R\_, \cr
T_{clr} &=& \frac{T(\mu_e)}{1 - R^{*} \, R\_},
\end{eqnarray}
\medskip

\noindent with $R^{*} = R ({1}/{r})$, $T^{*} = T ({1}/{r})$, $R\_ = R_t(j-1)$
is the reflectance of the underlying medium and $r$ is the diffusivity factor.

Since interactions between molecular absorption and Rayleigh and aerosol
scattering are negligible, the radiative fluxes in a clear-sky atmosphere are
simply those calculated from (\ref{ecmwf312}) and (\ref{ecmwf330}) attenuated
by the gaseous transmissions (\ref{ecmwf310})(subroutine SW2S).\\

These calculations are practically done twice, the first time for the clear-sky
fraction ($1-f$) of the atmospheric column with $\mu$ equal to $\mu_0$, simply
modified for the effect of Rayleigh and aerosol scattering (subroutine SWR),
the second time for the clear-sky fraction of each individual layer within the
fraction $f$ of the atmospheric column with $\mu$ equal to $\mu_e$
(subroutine SWR).

%
\subsection{Multiple reflections between layers}
%


To deal properly with the multiple reflections between the surface and the
cloud layers, it should be necessary to separate the contribution of each
individual reflecting surface to the layer reflectances and transmittances
inasmuch as each such surface gives rise to a particular distribution of
absorber amount. In case of an atmosphere including $N$ cloud layers, the
reflected light above the highest cloud consists of photons directly reflected
by the highest cloud without interaction with the underlying atmosphere and of
photons that have passed through this cloud layer and undergone at least one
reflection on the underlying atmosphere. In fact, (\ref{ecmwf36}) should be
written

%(3.31)
\medskip
\be
F = \sum_{l=0}^N{F_{cl}} \int_0^{\infty}\!{p_l(U) \, t_{\Delta\nu}(U) \, d\nu},
\label{ecmwf331}
\ee
\medskip

\noindent where $F_{cl}$ and $p_l(U)$ are the conservative fluxes and the
distributions of absorber amount corresponding to the different reflecting
surfaces.

Fouquart and Bonnel (1980) have shown that a very good approximation to this
problem is obtained by evaluating the reflectance and transmittance of each
layer (using (\ref{ecmwf324}) and (\ref{ecmwf330})), assuming successively a
non-reflecting underlying medium ($R\_ = 0$), then a reflecting underlying
medium ($R\_ \ne 0$). First calculations provide the contribution to reflectance
and transmittance of those photons interacting only with the layer into
consideration, whereas the second ones give the contribution of the photons
with interactions also outside the layer itself.

From these two sets of layer reflectances and transmittances ($R_{t_0},T_{t_0}$)
and ($R_{t_{\not{=}}},T_{t_{\not{=}}}$) respectively, effective absorber amounts
to be applied to computing the transmission functions for upward and downward
fluxes are then derived using (\ref{ecmwf38}) and starting from the surface and
working the formulas upward:

%(3.32)
\medskip
\[
U_{e_0}^{\!\downarrow} = \frac{1}{k_e} ln(\frac{T_{b_0}}{T_{b_c}}),
\]
\[
U_{e_{\not{=}}}^{\!\downarrow} = \frac{1}{k_e} ln(\frac{T_{b_{\not{=}}}}{T_{b_c}}),
\]
\be
\label{ecmwf332}
\ee
\[
U_{e_0}^{\!\uparrow} = \frac{1}{k_e} ln(\frac{R_{t_0}}{R_{t_c}}),
\]
\[
U_{e_{\not{=}}}^{\!\uparrow} = \frac{1}{k_e} ln(\frac{R_{t_{\not{=}}}}{R_{t_c}}),
\]
\medskip

\noindent where $R_{t_c}$ and $T_{b_c}$ are the layer reflectance and
transmittance corresponding to a conservative scattering medium.
Finally the upward and downward fluxes are obtained as:

%(3.33a)
\medskip
\be
F^{\!\uparrow}(j) = F_0 \, \biggl[R_{t_0} \, t_{\Delta\nu}(U_{e_0}^{\!\uparrow}) + (R_{t_{\not{=}}} - R_{t_0}) \, t_{\Delta\nu}U_{e_{\not{=}}}^{\!\uparrow})\biggr]
\label{ecmwf333a}
\ee
%(3.33b)
\medskip
\be
F^{\!\downarrow}(j) = F_0 \, \biggl[T_{b_0} \, t_{\Delta\nu}(U_{e_0}^{\!\downarrow}) + (T_{b_{\not{=}}} - T_{b_0}) \, t_{\Delta\nu}U_{e_{\not{=}}}^{\!\downarrow})\biggr]
\label{ecmwf333b}
\ee
\medskip

%
\subsection{Cloud shortwave optical properties}
%

As seen in section (\ref{subsubsection321}), the cloud radiative properties
depend on three different parameters: the optical thickness $\delta_c$, the
asymetry factor $g_c$ and the single scattering albedo $\omega_c$. Presently
the cloud optical properties are derived from Fouquart (1987) for the cloud
water droplets and from Sun and Shine (1995) for the cloud ice crystals
\footnotemark
%
\footnotetext{Ebert and Curry (1992) provided also another set of parameters
for the cloud containing ice particles}. In case of mixed phase cloud, one has
to consider the combination formula:

\medskip
\begin{eqnarray}\label{combination}
\delta_{c} &=& \delta_{L} + \delta_{I} \cr
\omega_{c} &=& (\omega_{L} \; \delta_{L} + \omega_{I} \; \delta_{I})/\delta_{c} \cr
g_c        &=& (\omega_{L} \; \delta_{L} \; g_{L} + \omega_{I} \; \delta_{I} \; g_{I})/\omega_{c}
\end{eqnarray}
\medskip

%
\subsubsection{Warm cloud case}
%

The optical thickness $\delta_{L}$ is related to the cloud liquid water path
$u_{LWP}$ by:

%(3.34)
\medskip
\be
\delta_{L} = 1.5 \times 10^3 \frac{u_{LWP}}{r_e},
\label{ecmwf334}
\ee
\medskip

\noindent where $r_e=11 \times 10^3(\rho r_c)+4$ is the mean effective radius of
the size distribution of the cloud droplets (in $\mu m$) and where the liquid
water path $u_{LWP}$ is defined by
$\int_{\Delta z} \rho r_c dz=\int_{\Delta p} r_c/g dp$. In the two spectral
intervals (1 for VIS and 2 for NIR) of the shortwave radiation scheme, $g_L$ is
kept constant to a value of 0.85 while $\omega_L$ is given as a function of
$\delta_L$, following to Fouquart (1987):

%(3.35)
\medskip
\begin{eqnarray}\label{ecmwf335}
\omega_{L_1} &=& 0.9999 - 5 \, 10^{-4} \, exp(-0.5 \, \delta_L), \cr
\omega_{L_2} &=& 0.9988 - 2.5 \, 10^{-3} \, exp(-0.05 \, \delta_L).
\end{eqnarray}
\medskip

\noindent These cloud shortwave radiative parameters have been fitted to
{\em in situ} measurements of stratocumulus clouds
(Bonnel {\em et al.}, 1983).\\

%
\subsubsection{Ice cloud case}
%

The optical properties of cirrus clouds from measurements during the ICE89
experiment have been taken in order to calibrate the scheme. Sun and Shine
(1995) indicate that it is important to account for the effect of small crystals
through a correcting factor (function of T expressed in Celsius degree), namely:

\medskip
\be\label{correct}
Corr(T) = 1.047 - 0.913 \times 10^{-4} T + 0.203 \times 10^{-3} T^2 - 0.106 \times 10^{-4} T^3.
\ee
\medskip

Using (\ref{correct}), the optical thickness $\delta_I$ has been adjusted to:

\medskip
\be
\delta_I = Corr(T) \frac{u_{IWP}}{30.6 + 254.8 \times 10^3 (\rho r_i)}.
\ee
\medskip

In the VISible part of the solar spectrum, ice crystals are non absorbing
($\omega_{I_1}=1$) while in the NIR region their single scattering albedo is
given by:

\medskip
\be
\omega_{I_2} = (1.0 - 0.086822 (10^3 \rho r_i)^{0.096277}) \times (1 +23.204 \times 10^{-3} \frac{Corr(T)-1}{Corr(T)}).
\ee
\medskip

Finally the asymmetry factors are given by

\medskip
\begin{eqnarray}
g_{I_1} &=& (0.8522 (10^3 \rho r_i)^{0.0162})\times (1 + 0.327 \times 10^{-1} \frac{Corr(T)-1}{Corr(T)}) \cr
g_{I_2} &=& (0.8819 (10^3 \rho r_i)^{0.0163})\times (1 + 0.418 \times 10^{-1} \frac{Corr(T)-1}{Corr(T)})
\end{eqnarray}
\medskip

%
\section{Additional features of the radiation computations}
\label{section4}
%

%
\subsection{Solar astronomy}
%

To run the shortwave radiative computations it is necessary to feed the code
with the time varying solar zenithal and azimuthal angles and
the daily solar constant, all derived by analytical formulae of astronomy
(Paltridge and Platt, 1976).
For instance, the daily solar constant $S_{sun}$, the solar declination angle
$\delta_{sun}$, the cosine of the solar zenithal angle $\mu$ and the azimuthal
solar angle $\beta$ are given by:
\begin{eqnarray}
S_{sun} &=& S_0(1.000110+0.034221cos(d_r)+0.001280sin(d_r) \cr
        & &              \mbox{} +0.000719cos(2d_r)+0.000077sin(2d_r)), \\
\delta_{sun} &=& 0.006918-0.399912 cos(d_r)+0.070257 sin(d_r) \cr
             & &          \mbox{} -0.006758 cos(2d_r)+0.000907 sin(2d_r) \cr
             & &          \mbox{} -0.002697cos(3d_r)+0.001480sin(3d_r), \\
\mu &=& cos(\phi)cos(\delta_{sun})cos(h_r)+sin(\phi)sin(\delta_{sun}), \\
\beta &=& sin^{-1}\displaystyle{\biggl(
        \displaystyle{cos(\delta_{sun}) sin(h_r) \over sin(arcos(\mu))}
                                \biggr)},
\end{eqnarray}
respectively where $S_0=1370$ W/m$^2$, $\phi$ is the latitude,
$d_r=2\pi n_{day}/365$ ($n_{day}$ is the day number) and
$h_r=2\pi {hour}/24$ with ${hour}$, the true hour of the day at the time $t$
and at the longitude $\lambda$, defined by:
\begin{eqnarray}
{hour} = mod(24+mod(t/3600,24),24) + \lambda (12/\pi) - t_{sideral}^{cor},
\end{eqnarray}
with $mod$, the modulo arithmetic function and the sideral time correction
$t_{sideral}^{cor}$ given by:
\begin{eqnarray}
t_{sideral}^{cor} &=& (7.67825 sin(1.00554*n_{day}-6.28306) + \cr
        & &              \mbox{} 10.09176 sin(1.93946*n_{day}+23.35089))/60.
\end{eqnarray}

\noindent All these calculations are performed in routine SUNPOS prior calling
the radiation code.

%
\subsection{Surface slope angles}
%

In case of non flat terrain, the surface radiative fluxes need to be corrected
to account for the mean terrain slopes (Kondratyev, 1969) (through the slope
angle $\alpha=tan^{-1}
\sqrt{(\partial{z_{surf}}/\partial{x})^2+(\partial{z_{surf}}/\partial{y})^2}$
and the azimuthal slope angle $\gamma=\pi/2-
tan^{-1}((\partial{z_{surf}}/\partial{y})/(\partial{z_{surf}}/\partial{x}))$)
that is the direct shortwave surface flux $F^{\!\downarrow}_{SW}$ is corrected
into $F^{\!\downarrow}_{{SW}^{sl}}$ while the isotropic longwave surface flux
$F^{\!\downarrow}_{LW}$ becomes $F^{\!\downarrow}_{{LW}^{sl}}$ with
\begin{eqnarray}
F^{\!\downarrow}_{{SW}^{sl}} &=& F^{\!\downarrow}_{SW} cos(i), \\
F^{\!\downarrow}_{{LW}^{sl}} &=& F^{\!\downarrow}_{LW} cos^2(\alpha/2)
\end{eqnarray}
and where
$$cos(i) = cos(\alpha)+sin(\alpha)tan(arcos(\mu))cos(\beta-\gamma).$$
No corrections are made to account for the mutual interaction of facing slopes
in the infra red.

\begin{figure}[hpbt]
\psfig{figure=\EPSDIR/kondratyev.eps,width=9cm}
\caption{Angle definition for the slope correction (from Kontratyev, 1969)}
\end{figure}

%
\subsection{Upper level extensions}
%

In order to allow for a better solar absorption by ozone and more realistic
infra red fluxes in the stratosphere and above, the vertical profiles of
temperature and humidity is completed aloft by using reference atmospheres
(Mc Clatchey et al., 1972). One among the five different standard atmospheres:

\begin{itemize}
\item tropical,
\item mid-latitude in summer and winter,
\item polar in summer and winter,
\end{itemize}

\noindent is selected to extend the relevant fields up to 50 km high.\\

%
\section{Intermittent radiation call}
\label{section5}
%

Due to its relatively high cost, the radiation code can be unreasonably
expensive if called at each model time step. Furthermore as the radiative time
scales are significantly larger than the time step involved in the integration
of a mesoscale model, it is recommended to refresh the radiative thermal
tendency and downward surface fluxes at an adapted rate. To run even faster,
some approximations of the radiative computations have been made available.\\

The basic time step to call the radiation computation is {\tt XDTRAD}. Il must
be a multiple of the model time step {\tt XTSTEP} and a default value is set to
900 s. Note that the radiative code operates on instantaneous fields but the
angular position of the sun is calculated at the current time plus
{\tt XDTRAD}/2. Actually, three parameters can be set to enable partial
radiative computations:

\begin{itemize}
\item {\tt LCLEAR\_SKY}: when .TRUE. means that radiative computations over
clear sky columns are made for the ensemble mean column only. This option is
meaningful if the terrain is flat.
\item {\tt XDTRAD\_CLONLY}: is the time step of refreshment of the radiative
fluxes and tendeny for the cloudy columns only\footnotemark
%
\footnotetext{it can be used together with {\tt LCLEAR\_SKY} == .TRUE.}
. Choosing {\tt XDTRAD\_CLONLY} as a divider of {\tt XDTRAD} allows a faster
update of the radiative transfer through cloudy columns as they are known to be
the most perturbating ones for the radiations.
\item {\tt NSPOT}\footnotemark
%
\footnotetext{if larger than 1, it cannot be used with the {\tt LCLEAR\_SKY}
option}: gives the stepping of a steady geometric sweep out of the air columns
in the computational domain. The radiative tendency and fluxes of the
non-selected air columns are bilinearly interpolated. Notice that no care is
taken to check if is air columns are cloudy or not.
\end{itemize}

Also taking the opportunity of more advanced radiation code release, it is
planned to give to the user the opportunity to perform the radiation
computations on instantaneous fields (as done actually) or on mean accumulated
fields between two successive calls as suggested recently by Xu and
Randall (1995).

%
\section{References}
\label{section5}
%
\decrefname
Bonnel, B., Y. Fouquart, J.-C. Vanhoutte, C. Fravalo, and R. Rosset, 1983:
      Radiative properties of some African and mid-latitude stratocumulus
      clouds.
      {\it Beitr. Phys. Atmosph.},
      {\bf 56},
      409-428.
\decrefname
Coakley, J.A., Jr., and P. Chylek, 1975:
      The two-stream approximation in radiation transfer: Including the angle
      of the incident radiation.
      {\it J. Atmos. Sci.},
      {\bf 32},
      409-418.
\decrefname
Dandin, Ph., 1995:
      The radiation code in ARPEGE-Climat AGCM.
      {\it Technical documentation},
      in preparation.
\decrefname
Deschamps, P.-Y., M. Herman, and D. Tanr\'e, 1983:
      Mod\'elisation du rayonnement solaire r\'efl\'echi par l'atmosph\`ere et
      la Terre, entre 0,35 et 4 microns.
      {\it Rapport ESA 4393/80/F/DD(SC)},
      156 pp.
\decrefname
Ebert, E. E., and J. A. Curry, 1992:
      A parameterization of ice cloud optical properties for climate models.
      {\it J. Geophys. Res.},
      {\bf 97D},
      3831-3836.
\decrefname
ECMWF, 1989:
      Physical Parametrization, ECMWF Forecast Model.
      {\it Research Manual RM-3},
      3rd ed.
\decrefname
Elsasser, W.M., 1942:
      Heat Transfer by infrared Radiation in the Atmosphere.
      {\it Harvard University Press},
      43 pp.
\decrefname
Fouquart, Y., 1974:
      Utilisation des approximants de Pad\'e pour l'\'etude des largeurs
      \'equivalentes des raies form\'ees en atmosph\`ere diffusante.
      {\it J. Quant. Spectrosc. Radiat. Transfer},
      {\bf 14},
      497-506.
\decrefname
Fouquart, Y., 1987:
      Radiative transfer in climate modeling.
      {\it NATO Advanced Study Institute on Physically-Based Modeling and
      Simulation of Climate and Climatic Changes}.
      Erice, Sicily, 11-23 May 1986. M.E. Schlesinger, Ed.,
      223-283.
\decrefname
Fouquart, Y., and B. Bonnel, 1980:
      Computations of solar heating of the earth's atmosphere: A new
      parametrization.
      {\it Beitr. Phys. Atmosph.},
      {\bf 53},
      35-62.
\decrefname
Geleyn, J.-F., and A. Hollingsworth, 1979:
      An economical analytical method for the computation of the interaction
      between scattering and line absorption of radiation.
      {\it Beitr. Phys. Atmosph.},
      {\bf 52},
      1-16.
\decrefname
Joseph, J.H., W.J. Wiscombe, and J.A. Weinman, 1976:
      The Delta-Eddington approximation for radiative flux transfer.
      {\it J. Atmos. Sci.},
      {\bf 33},
      2452-2459.
\decrefname
Kondratyev, J., 1969:
      {\it Radiations in the Atmosphere}.
      Academic Press.
\decrefname
Morcrette, J.-J., and Y. Fouquart, 1985:
      On systematic errors in parametrized calculations of longwave radiation
      transfer.
      {\it Quart. J. Roy. Meteor. Soc.},
      {\bf 111},
      691-708.
\decrefname
Morcrette, J.-J., L. Smith, and Y. Fouquart, 1986:
      Pressure and temperature dependence of the absorption in longwave
      radiation parametrizations.
      {\it Beitr. Phys. Atmosph.},
      {\bf 59},
      455-469.
\decrefname
Paltridge, G. W. and C. M. R. Platt, 1976:
      {\it Radiative Processes in Meteorology and Climatology}.
      Elsevier.
\decrefname
Rodgers, C.D., and C.D. Walshaw, 1966:
      The computation of infrared cooling rate in planetary atmospheres.
      {\it Quart. J. Roy. Meteor. Soc.},
      {\bf 92},
      67-92.
\decrefname
Rodgers, C.D., 1967:
      The radiative Heat Budget of the troposphere and lower stratosphere.
      {\it Report N$^0$12},
      A2, Planetary Circulation Project, Dept. of Meteorology, Mass. Instit.
      Technology, Cambridge, Mass.,
      99 pp.
\decrefname
Rothman, L.S., 1981:
      AFGL atmospheric absorption line parameters compilation: 1980 version.
      {\it Appl. Opt.},
      {\bf 21},
      791-795.
\decrefname
Shettle, E.P., and J.A. Weinman, 1970:
      The transfer of solar irradiance through inhomogeneous turbid atmospheres
      evaluated by Eddington's approximation.
      {\it J. Atmos. Sci.},
      {\bf 27},
      1048-1055.
\decrefname
Smith, E. A., and Lei Shi, 1992:
      Surface forcing of the infrared cooling profile over the Tibetan plateau.
      Part I: Influence of relative longwave radiative heating at high altitude.
      {\it J. Atmos. Sci.},
      {\bf 49},
      805-822.
\decrefname
Stephens, G.L., 1984:
      The parametrization of radiation for numerical weather prediction and
      climate models.
      {\it Mon. Wea. Rev.},
      {\bf 112},
      826-867.
\decrefname
Stephens, G.L., 1979:
      Optical properties of eight water cloud types.
      {\it CSIRO, Div. Atmos. Phys.},
      Technical Paper N$^0$12.36,
      Australia.
\decrefname
Stephens, G.L., 1978:
      Radiative properties of extended water clouds. Part II.
      {\it J. Atmos. Sci.},
      {\bf 35},
      2111-2132.
\decrefname
Sun, Z., and K. P. Shine, 1995:
      Parameterization of ice cloud radiative properties and its application to
      the potential climatic importance of mixed-phase clouds.
      {\it J. Climate},
      {\bf 8},
      1874-1888.
\decrefname
Tanr\'e D., J.-F. Geleyn, and J. Slingo, 1984:
      First results of the introduction of an advanced aerosol-radiation
      interaction in the ECMWF low resolution global model.
      {\it Aerosols and their Climatic Effects},
      H.E. Gerber and A. Deepak, Eds., A. Deepak Publishing, Hampton, Va.,
      133-177.
\decrefname
Washington, W.M., and D.L. Williamson, 1977:
      A description of the NCAR GCMs.
      {\it General Circulation Models of the Atmosphere},
      J. Chang, Ed., Methods in Computational Physics, vol. 17, Academic Press,
      111-172.
\decrefname
WMO-ICSU, 1984:
      Optical properties for the standard aerosols of the Radiation Commission,
      {\it WCP-55},
      World Climate Program, Geneva, Switzerland.
\decrefname
Xu, K.-M., and D.A. Randall, 1995:
      Impact of interactive radiative transfer on the macroscopic behavior of
      cumulus ensembles. Part I: Radiation parameterization ans sensitivity
      tests.
      {\it J. Atmos. Sci.},
      {\bf 52},
      785-799.

\vskip 1cm
\noindent Acknowledgements are due to Jean-Jacques Morcrette (ECMWF) and to
Philippe Dandin (CNRM) whose help was really appreciated.

%%%%%%%%%%%%%%%%%%%%%%%%%%%%%%%%%%%%%%%%%%%%%%%%%%%%%%%%%%%%%%%%%%%%%%%%%%%%%%%
%%%%%%%%%%%%%%%%%%%%%%% END OF RADIATION CHAPTER  %%%%%%%%%%%%%%%%%%%%%%%%%%%%%
%%%%%%%%%%%%%%%%%%%%%%%%%%%%%%%%%%%%%%%%%%%%%%%%%%%%%%%%%%%%%%%%%%%%%%%%%%%%%%%


%\end{document}

