\chapter{ORILAM aerosol scheme}
\minitoc

%\author{Pierre Tulet, CNRM/GMEI}

The prognostic evolution of the aerosol size distribution is determined by a 
general dynamical
equation \citep{Friedlander-1977}; \citep{Seinfeld-1997} without analytical 
solution: 

\begin{equation}
\frac{\partial n(r_p)}{\partial t} = f(n(r_p))
\label{eqanal}
\end{equation}
where $n$ is the function of aerosol size distribution ($ particles / cm^3$) 
and $r_p$ is the aerosol radius ($\mu m$).
This equation can be integrated to obtain an equation system such as: 
\begin{equation}
\frac{\partial M_k}{\partial t} = f(M_k)
\label{genmoment}
\end{equation}
where the $k^{th}$ moment is given by $M_k = \int_{0}^{\infty} r_{p}^{k} n(r_p) 
dr_p \ (\mu m^k / cm^3) $.
Using several assumptions (choice of the aerosol spectral distribution) the 
equation system \ref{genmoment}
can be closed giving $f(M_k)$ in terms of moments.
Three modes have been implemented; the first one to represent the new particles 
formed (nuclei mode); the second one
for bigger and evoluted particles (accumulation mode); the last one is devoted to 
dust particles and introduced as a passive mode (no interaction with the nuclei and 
accumulation mode). Each mode is represented by a log-normal distribution as:
\begin{equation}
 n(\ln D) = \frac{N}{\sqrt{2 \pi} \ln{\sigma_g}} \exp \left(- \frac{\ln^2 (\frac{D}{D_g})}{2 \ln^2(\sigma_g)}\right) \label{distribution}
\end{equation}
where $N$ is the particle number concentration (in $ particles /cm^3$), $D$ is the particle diameter ($\mu m$) and $D_g$
, $\sigma_g$ are respectively the number median diameter and the geometric 
standard deviation of the modal distribution.
The $k^{th}$ moment of the mode i is defined as :
\begin{equation}
M_{k,i} = \int_{0}^{\infty} r^k n_i(r) dr
\label{defmomentint}
\end{equation}
After integration (variable change as $ x = \frac{\ln (r/D_g)}{\ln(\sigma)}$) we 
obtain:

\begin{equation}
M_{k,i} =  N R^{k}_g \exp \left(\frac{k^2}{2} \ln^2 (\sigma)\right)
\label{defmoment}
\end{equation}

A simple combination with equation \ref{defmoment} gives a relationship between 
$M_k$ and the log-normal parameters $\sigma_g$, $R_g$ and $N$:
\begin{equation}
N = M_0 \\
\label{number}
\end{equation}
\begin{equation}
R_g = \left( \frac{M_3^4}{M_6 M_0^3} \right)^{1/6} \\
\label{radius}
\end{equation}
\begin{equation}
\sigma_g = \exp \left( \frac{1}{3} \sqrt{\ln \left( \frac{M_0 M_6}{M_3^2} 
\right)} \right)
\label{sigma}
\end{equation}


For aerosol with many monomers, it is possible to modelize the aerosol size 
distribution by 
a continuous function $n$ relative to mean radius $r_p$. The general dynamical 
equation is given by \citet{Friedlander-1977}:
\begin{equation}
\frac{\partial n(r_p)}{\partial t} = (f_{convection} + f_{diffusion} + 
f_{coagulation} +
f_{growth} + f_{sources/sink} + f_{external})(n(r_p))
\label{GDE}
\end{equation}
This equation can be integrated in terms of moments of the distribution as:
\begin{equation}
\frac{\partial M_k}{\partial t} = (f_{convection} + f_{diffusion} + 
f_{coagulation} +
f_{growth} + f_{sources/sink} + f_{external})(M_k)
\label{GDE_M}
\end{equation}

The aerosol dynamics are modeled  as described by \citet{Whitby-1991}; 
\citet{Binkowski-1995}; \citet{Ackermann-1998}; \citet{Binkowski-2003} 
with notable differences: (1) We chose to integrate 3 moments (0, 3 and 6th) as 
prognostic variables. This
procedure permits us to keep all parameters of the modal distribution variable.
(2) Different sets of parameterization of sulfate nucleation and inorganic 
chemistry solvers are given.
(3) The aerosol module is coupled on-line with meteorological fields and chemical 
species (organic condensation).
(4) Sedimentation is integrated analyticaly for all moments.
(5) Surface exchange is coupled to a mesoscale 
atmosphere/biosphere model (section 5).

One can note that the moments of order 0 and 3 are well known ; the integration 
of $M_k = \int_{0}^{\infty} r_{p}^{k} n(r_p) dr_p$ gives for $M_{0,i}=N_i$ where 
$N_i$ is the total concentration of particles for mode i and $M_{3,i}= 
\frac{3}{4 \Pi} V_i$  is a direct function of the total volume of mode i.

\section{Coagulation}
\subsection{General description}
Aerosol size distribution evolves by collision between particles, leading to 
coagulation process. Numerical cost of coagulation treatment is expensive but less expensive when
a log-normal approach is used. 

Several assumptions has been made to solve the binary coagulation:
(1) A collision between two particles forms a new particle
(2) The new particle is spherical
(3) The new volume is equal to the sum of both initial particle volumes.
Moments of the new particle formed after collision of particles of respective radius 
$r_{p1}$ and $r_{p2}$ is 
\begin{equation}
r_{p12}^k = (r_{p1}^3 + r_{p2}^3)^{\frac{k}{3}}
\label{rp12}
\end{equation}
It is necessary to consider coagulation as a transfer process  of particles in 
the lognormal distribution: to update the moment evolution due to coagulation, 
 we first consider the loss in moment due to extinction of 
both initial particles 
 $r_{p1}$ and $r_{p2}$ (term ($r_{p1}^k + r_{p2}^k$ )), and the supply of moment 
due to the
 creation of a new particule  $r_{p12}$ (term  $r_{p12}^k$). 
 The coagulation process can be integrated:
 \begin{eqnarray}
\frac{\partial M_k}{\partial t} = \frac{1}{2} \int_{0}^{\infty} 
\int_{0}^{\infty} r_{p12}^k
\beta(r_{p1},r_{p2}) n(r_{p1}) n(r_{p2}) dr_{p1} dr_{p2} \nonumber \\
- \frac{1}{2} \int_{0}^{\infty} \int_{0}^{\infty} (r_{p1}^k + r_{p2}^k)
\beta(r_{p1},r_{p2}) n(r_{p1}) n(r_{p2})  dr_{p1} dr_{p2}
\label{coag1}
\end{eqnarray}
with $\beta(r_{p1},r_{p2})$ representing the coagulation rate between particles 
$r_{p1}$ and 
$ r_{p2}$ in $cm^3.s^{-1}$.

For the particular case of $N=2$, different modes i and j  ($N > 2$ is an 
extrapolation of results below),  we 
obtain from \ref{coag1} and \ref{rp12}:

 \begin{eqnarray}
\frac{\partial M_k}{\partial t} = \frac{1}{2} \int_{0}^{\infty} 
\int_{0}^{\infty} (r_{p1}^3 + r_{p2}^3)^{\frac{k}{3}}
\beta(r_{p1},r_{p2}) (n_i + n_j)(r_{p1}) (n_i + n_j)(r_{p2}) dr_{p1} dr_{p2} 
\nonumber \\
- \frac{1}{2} \int_{0}^{\infty} \int_{0}^{\infty}
(r_{p1}^k + r_{p2}^k)  \beta(r_{p1},r_{p2}) (n_i + n_j)(r_{p1}) (n_i + 
n_j)(r_{p2}) dr_{p1} dr_{p2}
\label{coag2}
\end{eqnarray}
It is easy to solve terms of equation \ref{coag2} with the following convention:
(1) When two particles collide in the same mode (intra-modal coagulation), the 
new one stays in this mode.
(2) When two particles of different modes collide (inter-modal coagulation), the 
new one 
is in the mode with largest radius (here j).
The second convention implies for the inter-modal coagulation that each 
particle of the lowest
mode is transferred into the largest one.
Nevertheless, the intra-modal coagulation of the lowest mode (i) is able to 
reach beyond the largest mode (j). As a consequence, we use an hybrid approach 
discussed by \citet{Ackermann-1998}. First, the model resolves the intersection 
diameter ($d_{eq}$) of both modes, with the equation:
 \begin{equation}
 \ln \left(\frac{N_i \ln \sigma_i}{N_j \ln \sigma_j}\right) = \frac{(\ln d_{eq} 
- \ln d_{pi})^2}{2\ln^2 \sigma_{gi}} - \frac{(\ln d_{eq} - \ln d_{pj})^2}{2\ln^2 
\sigma_{gj}}
\label{coag3}
\end{equation}
At the end of the time-step, particles of the lowest mode (Aitken mode) with diameter greater than  $d_{eq}$
are transfered into the largest one (Accumulation mode).

\subsection{Coagulation rate}
Coagulation is represented by an harmonic function which is an
average between both coagulation limit regimes : 'free-
molecular' and 'near continuum' \citep{Whitby-1991}. Knudsen number defined by  $Kn = 
\frac{\lambda}{r_p}$ permits to define
the nature of the relationship between atmospheric gas and the particle. For $Kn < 
0.1$, the particle is in a continuous fluid (continuum regime), whereas for $Kn > 
10$  the particle moves as a gas molecule (free molecular regime).

\begin{itemize}
\item Free molecular regime $Kn > 10$

In this regime, \citet{Friedlander-1977} gave the expression of coagulation rate 
as:
\begin{equation}
\beta^{fm}(r_{p1},r_{p2}) = \left(\frac{6 k 
T}{\rho_p}\right)^{\frac{1}{2}}\left(\frac{1}{r^3_{p1}} 
+ \frac{1}{r^3_{p2}}\right)^{\frac{1}{2}} (r_{p1} + r_{p2})^2 
\label{coagfm}
\end{equation}
To integrate this equation, we need to make the same assumption as in 
\citet{Lee-1984}:
 \begin{equation}
\left(\frac{1}{r^3_{p1}} + \frac{1}{r^3_{p2}}\right)^{\frac{1}{2}} \approx 
\left(\frac{1}{r^{\frac{3}{2}_{p1}}} 
                                                               + 
\frac{1}{r^{\frac{3}{2}_{p2}}}\right)
\label{coag_ap}
\end{equation}

Now the equation \ref{coagfm} is written on the form:
\begin{equation}
\widetilde{\beta}^{fm}(r_{p1},r_{p2})
= \left(\frac{6 k_B T}{\rho_p}\right)^{\frac{1}{2}}
\left(r^{\frac{1}{2}}_{p1} + 2 \frac{r_{p2}}{r^{\frac{1}{2}}_{p1}} + 
\frac{r_{p2}^2}{r^{\frac{3}{2}}_{p1}} 
+ \frac{r_{p1}^2}{r^{\frac{3}{2}}_{p2}} +  2\frac{r_{p1}}{r^{\frac{1}{2}}_{p2}} 
+ r^{\frac{1}{2}}_{p2}\right)
\end{equation}

With this assumption, we can compute the variation of the moment due to 
coagulation process. But it is clear that this approximation is valid only in a 
short range of particle radius. To minimize the limitation of the 
approximation, a correction factor is definied 
as (for the mode i and $k^{th}$ moment) :


\begin{equation}
\label{coag_betafm}
b_{(6,i),intra} = \frac{\int_{0}^{\infty} \int_{0}^{\infty} r_{p1}^6 \beta^{fm} 
n_i(r_{p1}) n_i(r_{p2}) dr_{p1} dr_{p2}}
{\int_{0}^{\infty} \int_{0}^{\infty} r_{p1}^6 \widetilde{\beta}^{fm} n_i(r_{p1}) 
n_i(r_{p2}) dr_{p1} dr_{p2}}
\end{equation}

These factors are tabulated  in function of the log-normal 
parameters.
Finally, we obtain:

\begin{equation}
(\frac{\partial M_{6,i}}{\partial t})_{intra}  \approx - b_{6,i,intra} 
\int_{0}^{\infty} \int_{0}^{\infty} r_{p1}^6 \widetilde{\beta}^{fm} n_i(r_{p1}) 
n_i(r_{p2}) dr_{p1} dr_{p2}
\label{coag_betafm}
\end{equation}


\item Near-continuum regime $Kn < 1$

For particles larger than their free mean path $\lambda$, \citet{Friedlander-1977} suggested
the following  expression for the coagulation rate:
\begin{equation}
\widetilde{\beta}^{nc}(r_{p1},r_{p2}) = 4 \Pi (D_{p1} + D_{p2}) (r_{p1} + 
r_{p2})
\label{coagne}
\end{equation}

where $D_{p} = (k_B T Cc / 6 \Pi \mu r_p)$,\\
the Cunningham coefficient $Cc=1 + 
Kn(a+b \exp(-c/Kn))$, with  a=1.126; b=0.42 and c= 0.87. \\
Equation \ref{coagne} cannot be integrated analytically. If we consider only the 
continuum/near-continuum regime, we can approximate Cc as:
\begin{equation}
Cc \approx 1 + A^{nc'} Kn
\label{Cc}
\end{equation}

with $ A^{nc'} = 1.392 Kn^{0.0783}$. After substitution, equation \ref{coagne} 
becomes:
\begin{equation}
\widetilde{\beta}^{nc}(r_{p1},r_{p2}) = \frac{2 k_B T}{3 \mu}\left(2+\lambda 
A^{nc'}_i 
\left(\frac{1}{r_{p1}} + \frac{r_{p2}}{r_{p1}^2}\right) + \lambda A^{nc'}_j 
\left(\frac{1}{r_{p2}} + \frac{r_{p1}}{r_{p2}^2}\right) + \frac{r_{p1}}{r_{p2}}
+ \frac{r_{p2}}{r_{p1}}\right)
\label{coagne_cor}
\end{equation}
As previously, correction factor can be considered. Nevertheless, the 
approximation used is
precise enough in the continuum/near-continuum regime to be exempted.

\item Generalization of Brownien Coagulation:\\
\citet{Seinfeld-1997} developed a general formulation for the coagulation rate $
\beta(r_{p1},r_{p2})$ using \citet{Fuchs-1964} formulation and a Cunningham 
coefficient
due to \citet{Philips-1975}. But the expression is too complicated to be 
integrated in
a log-normal distribution approach.
That is why \citet{Whitby-1991} proposed an alternative solution to compute all 
coagulation coefficients averaging previously expression of free-molecule and 
near continuum regime as:

\begin{equation}
\frac{\partial M_k}{\partial t}  \approx \frac{ (\frac{\partial M_k}{\partial 
t})^{fm} (\frac{\partial M_k}{\partial t})^{nc}}
{(\frac{\partial M_k}{\partial t})^{fm} +  (\frac{\partial M_k}{\partial 
t})^{nc}}
\label{coag_betafm}
\end{equation}

\end{itemize}

\section{Gas-Particles conversion}

Pre-existing particles grow by gaseous transfer upon their surface. A second way 
of gaseous-particles transfer is related to the formation of new particles by 
nucleation.

Just after the emission of a combustion particle, some of the gaseous species 
fix on the aerosol surface as an adsorption process. When these atmospheric 
molecules are either in sufficient number or on aerosol site with low curve 
radius, a phase change appears. At this stage, adsorption classical formalism on
dry surface is not applicable. If the surface film is composed by an unique 
constituant in balance with the gas phase, when the atmospheric concentration of 
the constituant increases,
molecules condense upon the aerosol surface to restore
the thermodynamic balance.\\
The condensation process is discontinuous: the partial pressure needs to exceed 
a critical step to allow the phase change. In this case, the aerosol 
surface is crucial to transfer gaseous molecules into aerosol by condensation.
The absorption process needs to have a pre-existing liquid film at the aerosol 
surface.
The problem becomes different: as soon as a quantity of a species appears in 
gaseous
phase, some molecules are transfered into particle phase by thermodynamical 
balance. In this model, we assume the aerosol is old enough to have a short
liquid film at the surface. 
So, absorption has been retained as the dominant process
of aerosol growth.\\
Gaseous species that interact with aerosol phase are from two different 
categories: mineral and organic species. Mineral species are fundamental to 
predict the condensation of water $H_2O$ and thus the aerosol growth.
Organic species include a large number of different 
species with particular specificity of solubility, saturation vapor pressure and hearthless.
The organic aerosol fraction is able to modify the aerosol hygroscopic 
specificity. It is necessary to distinguish the organic matter issued from urban 
and rural areas. The first one  is mainly primary (emitted) and hydrophobic 
whereas the second one is mainly secondary (condensed) and hydrophilic 
\citep{Saxena-1995}.

\subsection{Growth Processing}
Several parameters, such as temperature, relative humidity, total aerosol surface and 
the condensation matter rate determine which one is the principal growth factor 
of the aerosol.  
\citet{Whitby-1991} gave the growth rate of the $k^{th}$ moment relative to $i$ 
mode as:
\begin{equation}
\frac{\partial M_{k,i}}{\partial t} = \frac{2 k}{\Pi} \int_{0}^{\infty} r_p^{k-
3} \Psi_p(r_p)n_i(r_p) dr_p
\label{growth_gen}
\end{equation}
where $\Psi_p$ is the condensation law on particles ($\mu m^3.s^{-1}$). $\Psi_p$ 
can be separated in a $\Psi_T$ and $\Psi$ respectively independant and dependant 
on particle size.
\begin{equation}
\Psi_p = \Psi_T . \Psi
\label{psi}
\end{equation}

Equation \ref{growth_gen} can be written as:
\begin{equation}
\frac{\partial M_{k,i}}{\partial t} = \frac{2 k}{\Pi} \Psi_T I_{k,i}
\label{growth_sec}
\end{equation}

with 
\begin{equation}
I_{k,i} = \int_{0}^{\infty} r_p^{k-3} \Psi(r_p) n_i(r_p) dr_p
\label{ik}
\end{equation}
and 
\begin{equation}
\Psi_T = \frac{m_w . (P_l - P_{surf,l})}{\rho_l R T}
\label{psit}
\end{equation}

where $m_w$ is the molar mass of species $l$, $P_l$ the partial pressure of 
species l, $P_{surf,l}$ the pressure of species l at the particle surface,  
$\rho_l$ the volumic mass of $l$ and $T$ the ambient temperature of 
the system (the aerosol is suposed to be in thermal equilibrium with its 
environment).
$\Psi(r_p)$ is the size contribution with two asymptotic forms:
\begin{itemize}
\item Free molecular regime;
\begin{equation}
\Psi^{fm} = \pi \alpha \overline{c} r_p^2
\label{psifm}
\end{equation}
where $\alpha$ is the accomodation coefficient, $\overline{c}$ the kinetic 
velocity of vapor molecules ($\overline{c} = \sqrt{8 RT/\pi m_w}$).
Integration of equation \ref{psifm} gives:
\begin{equation}
I^{fm}_{k,i} = \pi \alpha \overline{c} M_{k-1,i}
\label{psifmint}
\end{equation}
\item Near continuum regime;
\begin{equation}
\Psi^{nc} = 4 \pi D_v r_p
\label{psinc}
\end{equation}
where $D_v$ the diffusivity of species $l$ in the air.
Integration of equation \ref{psinc} gives:
\begin{equation}
I^{nc}_{k,i} = 4 \pi D_v M_{k-2,i}
\label{psincint}
\end{equation}
\end{itemize}

Finally, with the same average as for coagulation we can approximate the general
form of $I_{k,i}$ as:
\begin{equation}
I_{k,i} = \frac{I^{fm}_{k,i} I^{nc}_{k,i}}{I^{fm}_{k,i} + I^{nc}_{k,i}}
\label{ik_fin}
\end{equation}
\citet{Pratsini-1988} estimated that this kind of procedure to average 
growth process is a very good approximation in the transitional regime.
In our model, we assume thermodynamical equilibrium ($P_l = P_{surf,l}) $) in equation (28).
We calculate $\delta \Psi_p$ as a diagnostic at the end of the timestep for use in equation (24).

\subsection{Nucleation}

To activate the nucleation process of aerosol, it is necessary to 
have
the partial vapor pressure of gas species greater than associated saturated 
vapor pressure.
Nevertheless  there is few knowledge about nucleation of organic matter. In this 
model, only the sulfur nucleation is considered. We choose the \citet{Kulmala-1998} 
parameterization for its consistance with the classical theory of binary 
homegeneous
nucleation \citep{Wilemski-1984} and for taking into account the hydratation 
effect.
The nucleation rate is parameterized as: 
\begin{equation}
J = \exp \left( 25.1289 N_{sulf} - 4890.8 \frac{N_{sulf}}{T} -\frac{1743.3}{T} 
-2.2479 \delta N_{sulf} RH + 7643.4 \frac{x_{al}}{T} - 1.9712 \frac{x_{al}}{RH} 
\right)
\label{nucleation}
\end{equation}

with  
$x_{al}=1.2233 - \frac{0.0154 RA}{RA+RH} + 0.0102 \ln(N_{av})-0.0415\ln(N_{wv}) 
+ 0.0016 T$ the molar fraction of $H_2SO_4$ in the critical nucleus (stable);\\
$N_{av}$ and $N_{wv}$ respectively the concentration  of sulfuric acid vapor and water vapor in 
$cm^{-3}$; \\
$T$ the atmospheric temperature (K), RA and RH absolute and relative humidity. 
$N_{sulf} = \ln\left(\frac{N_a}{exp(-14.5125+0.1335T-10.5462 RH+1958.4(RH/T))} 
\right)$ 
is the logarithm ratio between $N_a$ (ambiant concentration of sulfur acid in 
$cm^{-3}$) and $N_{a,c}$ the concentration sulfur acid need to reach a nuclation 
rate of $J= 1 cm^{-3}.s^{-1}$; 
and $\delta = 1 + \frac{T - 273.15}{273.15}$.
$N_{a,c}$ can be given by:
\begin{equation}
N_{a,c} = \exp (-14.5125 + 0.1335 T - 10.5462 RH + 1958.4 (RH/T)) 
\label{nac}
\end{equation}

\subsection{Mineral Thermodynamic balance}
In this version, two sets of mineral thermodynamical equilibrium has been 
introduced for 
the prediction of the balance between aerosol and gas phases of the  system $NH_3$-
$SO_4$-$HNO_3$-$H_2O$.
The first parameterization is ARES, a revised version of MARS \citep{Saxena-1986}, 
developed by \citet{Binkowski-1995}.
The second parameterization introduced is ISORROPIA from \citet{Nenes-1998}. 

\subsection{Heterogenous chemistry}
The aerosol phase can modify the gaseous composition by heterogeneous and 
multiphase reactions
\citep{Ravishankara-1997}. Following \citet{Jacob-2000}, we introduced the 
minimal set of reactions which is presented here with their associated uptake 
coefficients $\lambda$ to improve the  ozone model:

\begin{equation}
HO_2 \rightarrow 0.5 H_2O_2 \qquad   \lambda = 0.2 
\label{hetero1}
\end{equation}
\begin{equation}
NO_3 \rightarrow HNO_3   \qquad   \lambda = 10^{-3} 
\label{hetero2}
\end{equation}
\begin{equation}
NO_2 \rightarrow 0.5 HNO_3 + 0.5 HONO  \qquad    \lambda = 10^{-4}  
\label{hetero3}
\end{equation}
\begin{equation}
N_2O_5 \rightarrow 2 HNO_3 \qquad    \lambda \in [0.01,1] 
\label{hetero4}
\end{equation}
The first order rate constant for gas heterogeneous loss onto particles is given 
by:\\
\begin{equation}
ka = \sum_{k}\left(\frac{d_k}{2D_g} +  \frac{4}{\nu \lambda} \right)^{-1} A_k
\label{ka}
\end{equation}
with $d_k$ the particle diameter ($m$), $D_g$ the reacting gas molecular 
diffusivity ($m^2.s^{-1}$), $\mu$ 
the mean molecular velocity ($m.s^{-1}$), $A_k$ the total surface area of mode 
$k$ and $\lambda$ the uptake coefficient of reactive species.

\subsection{Organic condensation}

In the troposphere, Volatil Organic Compounds (VOCs) are mainly oxidized by $OH$ 
radical, $NO_3$ and $O_3$. Some of these products have a very low 
saturated vapor pressure to be absorbed and form SOA (secondary organic aerosol).
To take into account correct condensation process we need to restore 
thermodynamical balance for all species. 
With this aim, some new chemical schemes, such as CACM \citep{Griffin-2002a} 
distinguish VOCs products in
accordance with their capability to condense on the aerosol phase.
The default scheme (ReLACS) do not allow these secondary organic aerosol parent:
in this case the use of ORILAM do not compute organic condensation.
Otherwise, using CACM scheme (or it's reduced version ReLACS2), one can active two
different set of organic thermodynamic balance which are MPMPO \citep{Griffin-2005} and the AER module of \citet{Pun-2002}.  The SOA precursor lumped groups are partitioned to distinguish hydrophobic, hydrophilic structural characteritics \citep{Pun-2002} together with sources (biogenic versus anthropogenic), volatility and potential dissociation  \citep{Griffin-2005},  chemical and structural characteristics defined by \citep{Pun-2002} in previous applications of CACM.  The predictions of lumped SOA precursors can be coupled with thermodynamic equilibrium modules for aerosol prediction.  SOA are group into ten differents aerosol class.
To active SOA, it is necessary to compile CACM.chf or ReLACS.chf using the preprocessor m9 (or m10).

\section{Sedimentation - Dry deposition}
Dry deposition and sedimentation of aerosols are driven by the Brownian 
diffusivity:
\begin{equation}
D_p = \left(\frac{k T}{6 \pi \nu \rho_{air} r_p} \right) C_c
\label{diff_bro}
\end{equation}
and by  the gravitational velocity:
\begin{equation}
V_g = \left(\frac{2 g}{9 \nu}\left(\frac{\rho_{p,i}}{\rho_{air}}\right) r_p^2 
\right) C_c
\label{vitesse_grav}
\end{equation}
where $k$ is the Bolzmann constant, $T$ the ambiant temperature, $\nu$ the air 
kinematic velocity, $\rho_{air}$ the air density, $g$ the gravitational 
acceleration, $\rho_{p,i}$ the aerosol density of mode $i$, and $C_c = 1 + 
1.246\frac{\lambda_{air}}{r_p}$ the gliding coefficient. 
These expressions need to be averaged on the $k^{th}$ moment and mode $i$ as:
\begin{equation}
\hat{X} = \frac{1}{M_{k,i}} \int_{-\infty}^{\infty} X r^k_p n_i(\ln r_p)d(\ln r_p)
\label{moyenne}
\end{equation}
where $X$ represents either $D_p$ or $v_g$.
After integration, we obtain for  brownian diffusivity:
\begin{equation}
\hat{D}_{p_{k,i}} = \tilde{D}_{p_{g,i}}\left[\exp\left(\frac{-2k+1}{2} 
\ln^2\sigma_{g,i} \right) + 1.246 Kn_g 
\exp \left(\frac{-4k+4}{2} \ln^2\sigma_{g,i}\right) \right]
\label{diff_brow}
\end{equation}
with $\tilde{D}_{p_{g,i}} = \left(\frac{kT}{6 \pi \nu \rho_{air} R_{g,i}} 
\right)$

and for gravitational velocity: 
\begin{equation}
\hat{Vg}_{p_{k,i}} = \tilde{Vg}_{p_{g,i}}\left[\exp\left(\frac{4k+4}{2} 
\ln^2\sigma_{g,i} \right) + 1.246 Kn_g \exp \left(\frac{2k+4}{2} 
\ln^2\sigma_{g,i}\right) \right]
\label{gravi_vel}
\end{equation}
with $\tilde{Vg}_{p_{g,i}} = \left(\frac{2g \rho_{p,i}}{9 \nu \rho_{air}} 
R_{g,i}^2 \right)$

\subsection{Dry Deposition}
According to \citet{Seinfeld-1997} and using the resistance concept of 
\citet{Wesely-1989}, aerosol dry deposition
velocity for the $k^{th}$ moment and mode $i$ is:
\begin{equation}
\hat{v}_{d_{k,i}} = ( r_a + \hat{r}_{d_{k,i}} + r_a  \hat{r}_{d_{k,i}} 
\hat{Vg}_{p_{k,i}})^{-1} + \hat{Vg}_{p_{k,i}}
\label{gravi_vel}
\end{equation}
where surface resistance $\hat{r}_{d_{k,i}}$ is given by
\begin{equation}
\hat{r}_{d_{k,i}} = \left[(\hat{Sc}_{k,i}^{-2/3} + 10^{-3/\hat{St}_{k,i}}) 
\left(1+ 0.24 \frac{w_*^2}{u_*^2} \right) u_* \right]^{-1}
\label{surfres}
\end{equation}
Schmidt and Stockes number are respectively equal to $\hat{Sc}_{k,i} = \nu / 
\hat{D}_{p_{k,i}}$ and
$\hat{St}_{k,i}= (u_*^2/g\nu)\hat{v}_{d_{k,i}}$.
One can observe that the friction velocity $u_*$ and the convective velocity 
$w_*$ depend on meteorological
and surface conditions. 

\subsection{Sedimentation}
For sedimentation process, we can use the above parameterization of 
$\hat{Vg}_{p_{k,i}}$. When
vertical resolution is high, it is necessary to use 
a classical time splitting to compute sedimentation fluxes.
It can be noted that the sedimentation / deposition processing modifies the 
particle distribution
with  an important loss of large particles in comparison to the small ones. 
After integration 
of the three moments, the distribution does not preserve the log-normal shape. 
If we consider after 
sedimentation processing the distribution as log-normal, the  
reconstruction of
log-normal parameters $\sigma$, $R_g$ induces a decrease of $\sigma$ and an 
increase of $R_g$. 
The variation of $\sigma$ is stronger than the $R_g$ one. Nevertheless, in nature
sedimentation process must decrease simultaneously  $\sigma$ and $R_g$.
 Therefore, we cannot consider the integration of the 
three moments to solve this problem. Two choices are possible:\\
\begin{itemize}
\item Sedimentation process with $\sigma$ fixed :
$M_{6,i}$ can be computed by maintaining $\sigma$ equal to the previous values:
\begin{equation}
M_{6,i} = M_{0,i} \left(\frac{M_{3,i}}{M_{0,i}} \right)^{1/3}  \exp \left( -3/2 
\log(\sigma_g)^2\right)^6  \exp\left(18 \log(\sigma_g)^2\right)
\label{sigmafix}
\end{equation}

\item Sedimentation process with $R_g$ fixed :
A simple combination of $M_{k,i}$ gives $M_{6,i}$ in function of $M_{0,i}$, 
$M_{3,i}$, and $R_{g,i}$ as:

\begin{equation}
M_{6,i} = \frac{M_{3,i}^4}{R_{g,i}^6 M_{0,i}^3}
\label{rgfix}
\end{equation}

\end{itemize}

To decrease $\sigma$ and $R_g$, a solution is to consider alternatively both treatment of $M_{6,i}$ for each 
time step.

\bibliographystyle{agu}
%\bibliography{bibtot}
\section{References}
%\begin{thebibliography}{21}
\expandafter\ifx\csname natexlab\endcsname\relax\def\natexlab#1{#1}\fi

\bibitem[{{\it Ackermann et~al.\/}(1998){\it Ackermann, Hass, Memmesheimer,
  Ebel, Binkowski, and Shankar\/}}]{Ackermann-1998}
Ackermann, I., H.~Hass, M.~Memmesheimer, A.~Ebel, F.~Binkowski, and U.~Shankar,
  Modal aerosol dynamics model for {E}urope : development and first
  applications., {\it Atmospheric Environment\/}, {\it 32(17)\/}, 2981--2999,
  1998.

\bibitem[{{\it Binkowski and Roselle\/}(2003)}]{Binkowski-2003}
Binkowski, F., and S.~Roselle, Models-3 community multiscale air quality (cmaq)
  model aerosol component. 1. {M}odel description, {\it J. Geophys. Res.\/},
  {\it 108\/}, D6, 2003.

\bibitem[{{\it Binkowski and Shankar\/}(1995)}]{Binkowski-1995}
Binkowski, F., and U.~Shankar, The regional particulate model 1. {M}odel
  description and preliminary results, {\it J. Geophys. Res.\/}, {\it 100\/},
  26,191--26,209, 1995.

\bibitem[{{\it Friedlander\/}(1977)}]{Friedlander-1977}
Friedlander, S., {S}moke, {D}ust, and {H}aze, {F}undamentals of aerosols
  dynamics, {\it Oxford University Press\/}, {\it 2nd ed\/}, 1977.

\bibitem[{{\it Fuchs\/}(1964)}]{Fuchs-1964}
Fuchs, N., The mechanism of aerosols, {\it Pergamon Press, Oxford\/}, 1964.

\bibitem[{{\it Griffin et~al.\/}(2002a){\it Griffin, Dabdub, and
  Seinfeld\/}}]{Griffin-2002a}
Griffin, R., D.~Dabdub, and J.~Seinfeld, Secondary organic aerosol. 1.
  {A}tmospheric chemical mechanism for production of molecular constituents.,
  {\it J. Geophys. Res.\/}, {\it 107(D17), 4332\/}, doi:10.1029/2001JD000,541,
  2002a.

\bibitem[{{\it Griffin et~al.\/}(2005){\it Griffin, Dabdub, and
  Seinfeld\/}}]{Griffin-2005}
Griffin, R., D.~Dabdub, and J.~Seinfeld, Development and initial evaluation of
  a dynamic species-resolved model for gas-phase chemistry and suze-resolved
  gas/particle partitioning associated with secondary organic aerosol
  formation, {\it J. Geophys. Res.\/}, {\it in press\/}, 2004JD005,219, 2005.

\bibitem[{{\it Jacob\/}(2000)}]{Jacob-2000}
Jacob, D., Heterogenous chemistry and tropospheric ozone, {\it Atmospheric
  Environment\/}, {\it 34\/}, 2131--2159, 2000.

\bibitem[{{\it Kulmala et~al.\/}(1998){\it Kulmala, Laaksonen, and
  Pirjola\/}}]{Kulmala-1998}
Kulmala, M., A.~Laaksonen, and L.~Pirjola, Parametrization for sulfuric acid /
  water nucleation rates., {\it J. Geophys. Res.\/}, {\it 103, D7\/},
  8301--8307, 1998.

\bibitem[{{\it Lee et~al.\/}(1984){\it Lee, Chen, and Gieseke\/}}]{Lee-1984}
Lee, K., H.~Chen, and J.~Gieseke, Log-normally preserving size distribution for
  brownian coagulation in the free-molecule regime, {\it Aerosol Sci.
  Technol.\/}, {\it 3\/}, 1984.

\bibitem[{{\it Nenes et~al.\/}(1998){\it Nenes, Pilinis, and
  Pandis\/}}]{Nenes-1998}
Nenes, A., C.~Pilinis, and S.~Pandis, {ISORROPIA} : {A} new thermodynamic model
  for inorganic multicomponent atmospheric aerosols., {\it Aquatic
  Geochemistry\/}, {\it 4\/}, 123--152, 1998.

\bibitem[{{\it Philips\/}(1975)}]{Philips-1975}
Philips, W., Drag on small sphere moving throught a gas, {\it Phys. Fluids\/},
  {\it 18\/}, 1089--1093, 1975.

\bibitem[{{\it Pratsini\/}(1988)}]{Pratsini-1988}
Pratsini, S.~E., Simultaneous aerosol nucleation, codensation, and coagulation
  in aerosol reactors, {\it J. Colloid Interface Sci.\/}, {\it 124\/},
  416--417, 1988.

\bibitem[{{\it Pun et~al.\/}(2002){\it Pun, Griffin, Seigneur, and
  Seinfeld\/}}]{Pun-2002}
Pun, B., R.~Griffin, C.~Seigneur, and J.~Seinfeld, Secondary organic aerosol.
  2. {T}hermodynamic model for gas/partitioning of molecular constituents.,
  {\it J. Geophys. Res.\/}, {\it 107(D17), 4333\/}, doi:10.1029/2001JD000,542,
  2002.

\bibitem[{{\it Ravishankara\/}(1997)}]{Ravishankara-1997}
Ravishankara, A., Heterogeneous and multiphase chemistry in the troposphere.,
  {\it Science\/}, {\it 276\/}, 1058--1065, 1997.

\bibitem[{{\it Saxena et~al.\/}(1986){\it Saxena, Hudischewskyj, Seigneur, and
  Seinfeld\/}}]{Saxena-1986}
Saxena, P., A.~Hudischewskyj, C.~Seigneur, and J.~Seinfeld, A comparative study
  of equilibrium approaches to the chemical characterization of secondary
  aerosols, {\it Atmospheric Environment\/}, {\it 20\/}, 1471--1483, 1986.

\bibitem[{{\it Saxena et~al.\/}(1995){\it Saxena, Hildemann, and
  Seinfeld\/}}]{Saxena-1995}
Saxena, P., L.~Hildemann, and J.~Seinfeld, Organics alter hygroscopic behavior
  of atmospheric particles, {\it J. Geophys. Res.\/}, {\it 100\/},
  18,755--18,770, 1995.

\bibitem[{{\it Seinfeld and Pandis\/}(1997)}]{Seinfeld-1997}
Seinfeld, J., and S.~Pandis, Atmospheric {C}hemistry and {P}hysics, {\it Wiley
  interscience pub\/}, 1997.

\bibitem[{{\it Wesely\/}(1989)}]{Wesely-1989}
Wesely, M., Parametrizations of surface resistance to gaseous dry deposition in
  regional scale, numerical models, {\it Atmos. Environ.\/}, {\it 23\/},
  1293--1304, 1989.

\bibitem[{{\it Whitby et~al.\/}(1991){\it Whitby, McMurry, Shankar, and
  Binkowski\/}}]{Whitby-1991}
Whitby, E., P.~McMurry, U.~Shankar, and F.~Binkowski, Modal aerosol dynamics
  modeling., {\it Atm. Res. and Exposure Asses. Lab., U.S. Environ. Prot.
  Agency, Research Triangle Park, N.C.\/}, 1991.

\bibitem[{{\it Wilemski\/}(1984)}]{Wilemski-1984}
Wilemski, G., Composition of the critical nucleus in multicomponent vapor
  nucleation., {\it J. Chem. Phys.\/}, {\it 80\/}, 1370--1372, 1984.

%\end{thebibliography}

%\end{article}

%\end{document}

