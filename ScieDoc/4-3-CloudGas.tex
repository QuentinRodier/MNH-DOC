\chapter{Clouds and chemistry}
%%%%%%%%%%%%%%%%%%%%%%%%%%%%%%%%%%%%%
% author: Celine Mari, CNRS/LA (20/03/08)
%%%%%%%%%%%%%%%%%%%%%%%%%%%%%%%%%%%%%%
\minitoc
%
Clouds play an important role in atmospheric chemistry.
Convective clouds transport pollutants vertically from the boundary layer to 
the upper troposphere. Clouds and associated precipitations scavenge pollutants 
from the air. Once inside the cloud or rain water, some compounds 
dissociate into ions and/or react with the one another through aqueous 
chemistry.
Another important role for clouds is the removal of pollutants trapped in 
rain water and its deposition onto the ground. Clouds can also affect 
gas-phase chemistry by attenuating solar radiation below the cloud base,
which has a significant impact on the photolysis reactions. 
The model currently incorporates parameterizations for sub-grid convective 
clouds (precipitating and non-precipitating). Parameterization for grid-scale 
resolved clouds is not yet available. 
%-------------------------------------
\section{Model description}
%-------------------------------------
The cloud scheme can be divided into two main components: the 
{\bf sub-grid} cloud
model and the {\bf resolved} cloud model. 
For large horizontal grid resolutions, the grid size will be larger than the 
size of a typical convective cloud, requiring a parameterization for these
sub-grid clouds. The sub-grid cloud scheme simulates convective 
precipitating and non-precipitation clouds (see Chapter:Convection Scheme).
The second component of the cloud model considers clouds which occupy the 
entire grid cell and have been "resolved" by the model. The rate of change in 
pollutant concentrations ($s_i$) due to cloud processes is given by:
$$
\frac{\partial C_i}{\partial t} = \frac{\partial C_i}{\partial t}_{subcld} + \frac{\partial C_i}{\partial t}_{rescld}
$$
%-------------------------------------
\section{Subgrid convective cloud scheme}
%-------------------------------------
%-------------------------------------
\subsection{Wet scavenging}
%-------------------------------------

Scavenging by subgrid wet convective updrafts is applied within the 
convective mass transport algorithm (see Chapter:Convection Scheme) in 
order to prevent soluble tracers from being transported to the top of the 
convective updraft and then dispersed on the grid scale. The transport model 
provide wet convective air mass fluxes through each grid level in the 
updraft. As air is lifted a distance $\Delta z$ from one level to the next, 
it loses a fraction $\rm F_i$ of soluble tracer $i$ to scavenging. This 
fraction depends on (1) the rate constant $k$ ($\rm s^{-1}$) for conversion 
of cloud condensate (including liquid and ice) to precipitation; 
(2) the fraction $f_{i,L}$ of tracer present in the liquid cloud condensate; 
(3) the fraction $f_{i,I}$ of tracer present in the ice cloud condensate; 
and (4) the retention efficiency $R_i$ of tracer in the liquid cloud condensate
as it is converted to precipitation ($R_i <$ 1 accounts for volatilization 
during riming). Thus the rate constant $k_i$ ($\rm s^{-1}$) for loss of 
tracer from the updraft is given by \citet{Mari2000}:
$$
k_i = (R_i f_{i,L} + f_{i,I}) k
$$
and the fraction $F_i$ of tracer scavenged as the air is lifted by $\Delta z$ 
is
$$
F_i = 1 - \exp\left[ -k_i \frac{\Delta z}{w} \right]
$$
where $w$ is the updraft velocity. The scavenged tracer is directly deposited 
to the surface, there can be no re-evaporation.
\subsubsection*{$\rm HNO_3$} 
$\rm HNO_3$ is 100\% in the cloud condensate phase ($\rm f_{i,L} + f_{i,I}=1$) 
and we assume $R_i = 1$, therefore $k_i = k$ \citep{Mari2000,Liu2001}.
\subsubsection*{Gases other than $\rm HNO_3$} 
For gases other than $\rm HNO_3$, a significant fraction of tracer may be in 
the gas phase so that $k_i < k$. The phase partitioning of the tracer depends 
on the cloud liquid water content, L ($\rm cm^3$water$\rm cm^{-3}$air), and the 
cloud ice water content W ($\rm cm^3$ice$\rm cm^{-3}$ air). 

Let $C_{i,G}, C_{i,L}, C_{i,I}, C_{i,T}$ represent the atmospheric mixing 
ratios of tracer in the gas, liquid, cloud condensate, ice cloud condensate, 
and all phases, respectively, so that 
$$
 C_{i,T} = C_{i,G} + C_{i,L} + C_{i,I}
$$
$$
f_{i,L} = \frac{C_{i,L}}{C_{i,T}} = \frac{\frac{C_{i,L}}{C_{i,G}}}{1 + \frac{C_{i,L}}{C_{i,G}} + \frac{C_{i,I}}{C_{i,G}}}
$$
$$
f_{i,I} = \frac{C_{i,I}}{C_{i,T}} = \frac{\frac{C_{i,I}}{C_{i,G}}}{1 + \frac{C_{i,L}}{C_{i,G}} + \frac{C_{i,I}}{C_{i,G}}}
$$
The ratio $C_{i,L}/C_{i,G}$ is obtained from Henry's law:
$$
\frac{C_{i,L}}{C_{i,G}} = K_i^* LRT
$$
where $K_i^*$ (M atm$^{-1}$) is the effective Henry's law constant including 
contributions from dissociated species in fast equilibrium with the dissolved 
tracer, $R = 8.32\times 10^{-2}$ atm M$^{-1}$ K$^{-1}$ is the ideal gas constant, and T is the local temperature in K. We calculate $K_i^*$ from the van't Hoff
equation: 
$$
K_i^* = K_{i 298}^* \exp\left[ -\frac{\Delta H^0_{i 298}}{R}\left(\frac{1}{T} - \frac{1}{T_0} \right) \right]
$$
where $T_0 = 298 K$. 

The retention efficiency $R_i$ is unity for all gases in a warm cloud
(T$\ge$268 K). Values of $R_i$ in a mixed cloud 
($\rm 248 < T < 268$ K) are 0.02 for $\rm CH_3OOH$ and $\rm CH_2O$ and 0.05 for $\rm H_2O_2$ \citep{Mari2000}.

The ratio $C_{i,I}/C_{i,G}$ for $\rm H_2O_2$ is obtained by assuming scavenging
by co-condensation: 
$$
\frac{C_{i,I}}{C_{i,G}}=
\frac{W}{C_{H_2O}}
\left( \frac{\alpha_{H_2O_2}}{\alpha_{H_2O}} \right) 
\left( \frac{M_{H_2O_2}}{M_{H_2O}} \right) ^{\frac{1}{2}}
$$
where $C_{H_2O}$ is the water vapor mixing ratio (to be calculated from 
saturation over ice at the local temperature), $\alpha_{H_2O_2}/\alpha_{H_2O}$=0.6 is the ratio of sticking coefficients on the ice surface, and $M_{H_2O_2}/M_{H_2O}$=1.9 is the ratio of molecular weights. 

For $\rm CH_3OOH$ and $\rm CH_2O$ scavenging by co-condensation is inefficient
and we assume $C_{i,I}/C_{i,G}$=0.

%-------------------------------------
\subsection{Lightning produced NOx}
%-------------------------------------
The mass flux formalism applied to the convective transport of a chemical 
compound $\overline{C}$, writes :
%
\begin{equation}\label{transport_convectif}
{\frac {\partial \overline C}{\partial t}}\bigg\vert_{convection} =
-\frac{1}{\rho A} \frac {\partial (MC)}{\partial z} 
- \overline{w} {\frac {\partial \overline C}{\partial z}}
\end{equation}
%
where $A$ is the grid mesh area, $\rho$ is the air density, $M$ is the mass flux
(in kg/s) and $\overline{w}$ is the environmental subsidence to compensate the
upward mass flux. $C$ is the mixing ratio of chemical compound in the 
convective cells. The mass flux term of Eq. (\ref{transport_convectif}) is
further decomposed into:
%
\begin{equation}\label{decomposition}
\frac {\partial (MC)}{\partial z} = \frac {\partial (M^u C^u)}{\partial z} + \frac {\partial (M^d C^d)}{\partial z}
\end{equation}
%
where the superscripts $u$ and $d$ refer to the "updrafts" and to the 
"downdrafts" components, respectively. The different mass flux divergences are 
expressed 
as:
%
\begin{equation}\label{div_mass_fluxes_u}
\frac{\partial}{\partial z} (M^u C^u) = \epsilon^u\overline{C}-\delta^u C^u
\end{equation}
\begin{equation}\label{div_mass_fluxes_d}
\frac{\partial}{\partial z} (M^d C^d) = \epsilon^d\overline{C}-\delta^d C^d
\end{equation}
%
where $\epsilon$ and $\delta$ are the parameterized entrainment and detrainment
rates, respectively. Selecting $C$ as the mixing ratio of nitrogen monoxide, 
$[NO]$, Eqs (\ref{div_mass_fluxes_u})-(\ref{div_mass_fluxes_d}) are modified to
include the internal LiNOx production rates \citep{Mari2006}:
%
\begin{equation}\label{div_flux_NO_u}
\frac{\partial}{\partial z} (M^u[NO]^u) = \epsilon^u\overline{[NO]}-\delta^u[NO]^u + (\overline{\rho} A) \frac{\partial [NO]^u}{\partial t}\bigg\vert_{LiNOx}
\end{equation}
\begin{equation}\label{div_flux_NO_d}
\frac{\partial}{\partial z} (M^d[NO]^d) = \epsilon^d\overline{[NO]}-\delta^d[NO]^d
\end{equation}
%\begin{equation}\label{div_flux_NO_d}
%{\partial\over\partial z} (M^d[NO]^d) = \epsilon^d\overline{[NO]}-\delta^d[NO]^d + (\overline{\rho} A) \frac{\partial [NO]^d}{\partial t}\bigg\vert_{LiNOx}
%\end{equation}
%
The two terms on the right hand side of 
Eqs (\ref{div_flux_NO_u}) and (\ref{div_flux_NO_d})
represent the subgrid scale
transport of NO.
Transport of NO 
is assumed to take place instantaneously during each model timestep. 
The third term is the LiNOx term to be parameterized.
It is worth noting that 
no a-priori vertical placement of LiNOx is necessary with this approach.
Once produced inside the convective column, NO molecules are redistributed 
by upward and downward transport and detrained in the environment. The 
vertical placement of LiNOx is a direct consequence of the redistribution by 
mass fluxes inside the convective scheme. 

The electrical activity in the thunderstorms is related to the vertical 
extension of the glaciated region where ice-ice particle rebounding collisions 
are efficient enough to explain the charging mechanisms 
\citep{Reynolds1957,Takahashi1978,Saunders1992}. 
A growing electrical 
field then results from the organization of dipolar, tripolar or even 
more complex charge 
structures at storm scale \citep{Rust2002,Rust1996,Stolzenburg2002,Barthe2005}. 
The electrical field is broken down by a partial 
neutralization of the electrical charges. This is realized by a repetitive 
triggering of intra-cloud (IC) and cloud-to-ground (CG) flashes. The flashes 
lead to the formation of NO in the lightning channels after dissociation
of air molecules at high temperature followed by a rapid cooling. 
According to 
the statistical regression formula of \citet{Price1992}, 
the total lightning
frequency over land and ocean, $f_f$ can be grossly estimated from 
mean cloud morphological parameters: 
%
\begin{equation} \label{eqn : fréquence}
f_f=3.44 \times 10^{-5}{H_{ct}}^{4.9} \quad
\end{equation}
over land and 
\begin{equation} \label{eqn : fréquence2}
f_f=6.40 \times 10^{-4}{H_{ct}}^{1.73} \quad
\end{equation}
over ocean.
$H_{ct}$ is the cloud top height of the convective cells (in km).

\citet{Price1993} proposed the following polynomial relationship
between the thickness of the cold icy cloud ($H_{fr}$ in km) and the 
IC/CG ratio, $\beta$:
\begin{equation} \label{eqn : rapport}
\beta =0.021H_{fr}^4-0.648H_{fr}^3+7.493H_{fr}^2-36.54H_{fr}+63.09 \quad
\end{equation}
with $\rm 1<\beta<50$.
%
A scaling factor 
$c_{pr}=0.97241 \; \exp(0.048203\times \Delta lat \Delta lon)$ is introduced 
by \citet{Price1994} to adapt $f_f$ to different mesh sizes in 
interval of latitude ($\Delta lat$) and longitude ($\Delta lon$) given in 
degree. 

The combination of Eqs (\ref{eqn : fréquence})-(\ref{eqn : rapport}), leads to
the final expression of the LiNOx production rates to be inserted in
Eq (\ref{div_flux_NO_u}). It can be written in condensed form :
\begin{equation}\label{LiNOx_rate_ud}
{\frac {\partial (NO)^{u}}{\partial t}} \bigg\vert_{LiNOx} = \frac{\beta f_f}{1+\beta}\times P(IC) + \frac{f_f}{1+\beta}\times P(CG)
\end{equation}
where the value of the mean production rate per CG and IC namely, 
$P(CG) = 6.7\times 10^{26}$ of NO molecules, and 
$P(IC) = 6.7\times 10^{25}$ of NO molecules, of
\citet{Price1997} have been retained. 
It is worth noting that recent studies based on airborne observations and 
cloud scale modeling found that intracloud flashes are likely to be as 
effective 
in producing NO as cloud-to-ground flashes 
\citep{DeCaria2000,Fehr2004,Ridley2005}. 
It is also important to note that the estimates for the lightning 
NOx production based on \citet{Price1992,Price1994,Price1997}
are on the high end of current estimates in the  literature
\citep{Labrador2005}. 

The practical implementation of the LiNOx parameterization is based on critical 
vertical levels defined in the deep convection scheme. IC flashes are equally 
distributed between the cloud top and the freezing levels. The CGs are located 
between the -10$^{\circ}$C level (or the level of free sink if below) and the 
ground. The NO production in flashes 
is assumed to be proportional to air density following 
\citet{Goldenbaum1993}. 
Recent studies have shown that unimodal or bimodal distributions would be more 
realistic than uniform distributions as discussed in 
\citet{MacGorman1998,DeCaria2000,DeCaria2005}. 

%-------------------------------------
\section{Resolved Cloud Scheme}
%-------------------------------------
 \subsection{Aqueous chemistry extensions}
To be implemented
 \subsection{Wet scavenging}
To be implemented
 \subsection{Lightning produced NOx at cloud scale}
To be implemented
%\end{document}

%%%%%%%%%%%%%%%%%%%%%%%%%%%% BIBLIOGRAPHY %%%%%%%%%%%%%%%%%%%%%%%%
\begin{btSect}{4-3-CloudGas}
\section{References}
\btPrintCited
\end{btSect}
%%%%%%%%%%%%%%%%%%%%%%%%%%%% BIBLIOGRAPHY %%%%%%%%%%%%%%%%%%%%%%%%
