\chapter{Passive Dust transport and direct effects in MESONH}
\minitoc

%\author{Alf Grini, CNRM/GMEI}

\section{General theory}

Dust is mobilized from dry desert surfaces when the wind friction 
speed reaches a threshold wind friction speed of approximately
0.2 m/s. Dust is an important aerosol with annual global
 emissions ranging from 1000 to 3000 $Tg~yr^{-1}$ and average global
load around 10-30 $Tg$ \cite[]{zender04}.

Dust aerosols both scatter and absorb solar radiation, and it absorbs and
re-emitts terrestrial radiation. Since the dust is emitted to the atmosphere
when there are large winds, dust emission often occur in the form of
{\it dust storms} and optical depths near the source can be very large.


\section{Sources and transport}


The dust is produced depending on the wind speed. The externalized surface
of MESONH takes care of the mobilization, and sends moments of the size distribution
to the atmospheric model. The size distribution of dust is assumed to consist
of lognormal modes, and these modes are described by their $0^{th}$, $3^{rd}$ and
$6^{th}$ moment. The moments can be transformed to diagnostic variables of 
the size distribution ($\sigma$ (dispersion coefficient), $r$ (median radius) and
$N$ (number concentration) following \cite{tulet05})
Dust is currently lost through sedimentation and rain out in convective clouds.
The sedimentation takes into account the size of the aerosol where large particles
have larger sedimentation velocity (see Sedimentation - Dry deposition in section ORILAM aerosol scheme).
The convective rainout assumes that the 
aerosols are completely soluble in rain water. (Henry's law coefficient similar to $HNO_3$).

\section{Radiation and Optical properties}

The dust influences the radiation scheme of MESONH. Optical properties in the short wave
are calculated using the Mie code SHDOM \cite{evans98} {\it http://nit.colorado.edu/~evans/shdom.html}
They are calculated off line, assuming a Refractive index of $1.48 -0.0014$ for all 
wavelengths in the shortwave, similar to what was used by \cite{myhre03}. The results of the
Mie calculations are stored in a look up table as {\it assymetry factor $[no~unit]$}, 
{\it single scattering albedo $[no~unit]$} and {\it extinction coefficient $[m^2~g^{-1}]$}.
These values are looked up as a function of wavelength, number median radius and dispersion coefficient
of the aerosols. If the dust scalars are activated, the dust code calls 

The long wave code is not modified for special treatment of dust. There is a default {\it Longwave
absorption coeffient} defined in {\it yoesw.f90} of the radiation code. The coeffcient is used
to calculated absorbtion of terrestrial radiation from optical depth at 550 ${nm}$. Although
this treatment could be refined, the longwave effect of dust is supposed to be smaller than the
shortwave effect \cite[]{myhre03}. 


%% ------------------------------------------------------ %%
\bibliographystyle{agu}
%NAME OF BIBTEX FILE (It is in the repository)
%\bibliography{bibtot}
%\begin{thebibliography}{4}
\section{References}
\expandafter\ifx\csname natexlab\endcsname\relax\def\natexlab#1{#1}\fi

\bibitem[{{\it Evans\/}(1998)}]{evans98}
Evans, K., The spherical harmonic discrete ordinate method for
  three-dimensional atmospheric radiative transfer, {\it J. Atmos. Sci.\/},
  {\it 55\/}, 429--446, 1998.

\bibitem[{{\it Myhre et~al.\/}(2003){\it Myhre, Grini, Haywood, Stordal,
  Chatenet, Tanre, Sundet, and Isaksen\/}}]{myhre03}
Myhre, G., A.~Grini, J.~Haywood, F.~Stordal, B.~Chatenet, D.~Tanre, J.~Sundet,
  and I.~Isaksen, Modeling the radiative impact of mineral dust during the
  saharan dust experiment ({SHADE}) campaign, {\it \jgr\/}, {\it 108\/}, 8579,
  2003.

\bibitem[{{\it Tulet et~al.\/}(2005){\it Tulet, Crassier, Cousin, Suhre, and
  Rosset\/}}]{tulet05}
Tulet, P., V.~Crassier, F.~Cousin, K.~Suhre, and R.~Rosset, {ORILAM}, a three
  moment lognormal aerosol scheme for mesoscale atmospheric model. on-line
  coupling into the {Meso-NH-C} model and validation on the {Escompte}
  campaign, {\it \jgr\/}, 2005, in press.

\bibitem[{{\it Zender et~al.\/}(2004){\it Zender, Miller, and
  Tegen\/}}]{zender04}
Zender, C., R.~Miller, and I.~Tegen, Quantifying mineral dust mass budgets:
  Terminology, constraints, and current estimate, {\it {Eos Trans}\/}, {\it
  85\/}, 509--512, 2004.

%\end{thebibliography}
%-----------------------------------------------------
